% **************************************************************************** %
\chapter{Berechnungen zur Modellentwicklung}
\label{app:models:calcs}
% **************************************************************************** %

Dieses Kapitel beinhaltet Berechnungen, welche zur Modellierung benutzt werden
und aus Gr\"unden der \"Ubersichtlichkeit nicht im Hauptteil zu finden sind.


% ---------------------------------------------------------------------------- %
\section{Modellierung einer PV-Zelle}
\label{app:models:develop:cell}
% ---------------------------------------------------------------------------- %

Zur  Herleitung  der  Zellenparameter  werden vier  Quellen  herangezogen,  um
ein  einigermassen  gut  abgest\"utztes Ergebnis  zu  erhalten. Die  gesuchten
Parameter sollen  f\"ur am Markt  erh\"altliche Module g\"ultig  sein, weshalb
Datenbl\"atter von Solar\emph{modulen} und nicht Zellen verwendet werden.

Zuerst  werden  Zellenstrom  und Zellenspannung  bestimmt,  anschliessend  die
Fl\"ache einer Zelle,  um damit auf die im Modell  verwendete Kapazit\"at, den
Shunt-Widerstand und den Seriewiderstand schliessen zu k\"onnen.


% ---------------------------------------------------------------------------- %
\subsection{Zellenstrom und Zellenspannung}
\label{app:subsec:cell:UI}
% ---------------------------------------------------------------------------- %

Tabelle   \ref{tab:moduleData:IU}    auf   Seite   \pageref{tab:moduleData:IU}
enth\"alt die  Daten zu  Kurzschlussstr\"omen und Leerlaufspannungen  von vier
Modulen. Die  Spannung  $U_{\mathrm{OC,  Zelle}}$ pro  Zelle  (letzte  Spalte)
errechnet sich gem\"ass:

\begin{equation}
    \label{eq:voltagePerCell}
    U_{\mathrm{OC, Zelle}} = \frac{U_{\mathrm{OC, Strang}}}{\text{Anzahl Zellen pro Strang}}
\end{equation}

\begin{table}
    \centering
    \small
    \caption{%
        Daten   f\"ur   Solarmodule.  \textbf{pk}:   polykristallines   Panel,
        \textbf{mk}:     monokristallines     Panel.     \emph{Anmerkung}: Die
        Konfiguration  der Module  (wieviele  Zellen in  Serie  und wie  viele
        Str\"ange  parallel)  ist  mit   Ausnahme  des  Solarex  MSX-60  nicht
        angegeben. Es  ist  aber  bekannt,  in  welcher  Gr\"ossenordnung  die
        Spannung  pro  Zelle  ungef\"ahr  liegen sollte,  womit  man  aus  den
        angegebenen  Leerlaufspannungen  und  der Gesamtzahl  Zellen  auf  die
        Konfiguration eines Modules schliessen kann.%
    }
    \label{tab:moduleData:IU}
    \begin{tabular}{lp{20mm}lllll}
        \toprule
          \rotatebox{70}{\pbox{25mm}{Quelle}}
        & \rotatebox{70}{\pbox{25mm}{Modell}}
        & \rotatebox{70}{\pbox{25mm}{Kurzschluss-\\strom $I_{\mathrm{SC}}$}}
        & \rotatebox{70}{\pbox{25mm}{Leerlauf-\\spannung $V_{\mathrm{OC}}$}}
        & \rotatebox{70}{\pbox{25mm}{Anzahl Zellen \\(total)}}
        & \rotatebox{70}{\pbox{25mm}{Anzahl Zellen \\(Strang)}}
        & \rotatebox{70}{\pbox{25mm}{Leerlaufspan-\\nung pro Zelle}} \\
        \midrule

          \cite{ref:solar:bonkoungou}
        & Solarex MSX-60
        & \SI{3.8}{\ampere}
        & \SI{21.1}{\volt}
        & \num{36}
        & \num{36}
        & \SI{586}{\milli\volt}
        \\

          \cite{ref:solar:px85}
        & Sunset PX85 (\textbf{pk})
        & \SI{5.5}{\ampere}
        & \SI{21.5}{\volt}
        & \num{76}
        & \num{38}
        & \SI{566}{\milli\volt}
        \\

          \cite{ref:solar:as150}
        & Sunset Solargenerator AS150 (\textbf{mk})
        & \SI{8.7}{\ampere}
        & \SI{22.3}{\volt}
        & \num{36}
        & \num{36}
        & \SI{620}{\milli\volt}
        \\

          \cite{ref:solar:sunmodulePro}
        & Sunmodule Pro-Series XL SW320 (\textbf{mk})
        & \SI{9.41}{\ampere}
        & \SI{45.9}{\volt}
        & \num{72}
        & \num{72}
        & \SI{638}{\milli\volt}
        \\

        \bottomrule
    \end{tabular}
\end{table}

Wir verwenden f\"ur  die Simulation einer Zelle den  gerundeten Mittelwert der
Zellenspannungen aus der letzten Spalte von Tabelle \ref{tab:moduleData:IU}:

\begin{equation}
    \label{eq:cell:UOC}
    \underline{\underline{U_{\mathrm{OC, Zelle, Simu}} = \SI{600}{\milli\volt}}}
\end{equation}

Polykristalline  Zellen  liefern  bedeutend  kleinere  Kurzschlusstr\"ome  als
monokristalline  Zellen.  Jedoch  werden bei  monokristallinen Zellen  weniger
Str\"ange parallel  geschaltet, womit  der Gesamtstrom  des Moduls  immer noch
unter \SI{10}{\ampere}  bleibt. Unabh\"angig vom  genauen Aufbau  eines Moduls
gehen wir  daher davon aus,  dass es  nicht mehr als  \SI{10}{\ampere} liefern
wird.


% ---------------------------------------------------------------------------- %
\subsection{Bestimmung der Zellenfl\"ache}
\label{app:subsec:cell:surface}
% ---------------------------------------------------------------------------- %

Das    PX-85-Modul   aus    \cite{ref:solar:px85}    verwendet   76    Zellen,
angeordnet  in   einer  $4  \times  19$   -  Konfiguration. Seine  Abmessungen
betragen   $\SI{1477}{\milli\meter}    \times   \SI{660}{\milli\meter}$,   was
sich   herunterrechnen  l\"asst   auf  eine   ungef\"ahre  Modulgr\"osse   von
$\SI{165}{\milli\meter}   \times   \SI{75}{\milli\meter}$. Dabei  werden   die
Abmessungen   des   Rahmens   und   die  Abst\"ande   zwischen   den   Modulen
vernachl\"assigt.

Die Fl\"ache  des AS-150-Moduls wird analog  aus Quelle~\cite{ref:solar:as150}
zu  $\SI{155}{\milli\meter}  \times   \SI{164}{\milli\meter}$  bestimmt.   Das
XL-320-Modul   aus   \cite{ref:solar:sunmodulePro}    hat   die   Zellgr\"osse
direkt    angegeben,    sie     betr\"agt    $\SI{156}{\milli\meter}    \times
\SI{156}{\milli\meter}$. Es ist naheliegend dass aufgrund von Standardisierung
das AS-150-Modul die gleiche Zelldimension hat wie das XL-320-Modul, n\"amlich
den verbreiteten 6-Zoll-Formfaktor.

Da  eine gr\"ossere  Zelle eine  gr\"ossere Kapazit\"at  und somit  gr\"ossere
Probleme   im  Falle   der  Kurzschlussvariante   bedeutet,  wird   mit  einer
Zellgr\"osse   von   $\SI{156}{\milli\meter}  \times   \SI{156}{\milli\meter}$
gerechnet, womit sich die Fl\"ache der Zelle bestimmt zu:

\begin{equation}
    \label{eq:cell:surface}
    \underline{\underline{A_{\mathrm{Zelle}} = \SI{156}{\milli\meter} \times \SI{156}{\milli\meter} = \SI{243.36}{\centi\meter\squared}}}
\end{equation}

Dies     entspricht     ungef\"ahr     der     600-fachen     Fl\"ache     des
\SI{0.43}{\centi\meter\squared}-Modules aus  Quelle \cite{ref:solar:scofield}.

