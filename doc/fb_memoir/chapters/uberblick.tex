% **************************************************************************** %
\chapter{\"Uberblick}
\label{chap:uberblick}
% **************************************************************************** %

Dieses Kapitel  beschreibt zuerst die grobe  Idee unsers L\"osungskonzepts. Es
wird  dargelegt, wie  unser System  in  eine Solaranlage  (bestehend oder  neu
aufgebaut) integriert wird, wie das System mit seiner Umgebung interagiert und
wie es zu bedienen ist.


% ---------------------------------------------------------------------------- %
\section{Aufbau einer Photovoltaikanlage}
\label{sec:solaranlage:aufbau}
% ---------------------------------------------------------------------------- %

\begin{wrapfigure}{r}{0.45\textwidth}
    \centering
    \includegraphics[width=0.4\textwidth]{images/solar-facility/cell--400px.png}
    \caption{Solarzelle, Frontalansicht \cite{ref:pvcell:wikipedia}}
    \label{fig:pvcell:front}
\end{wrapfigure}

Die   Grundbausteine    einer   Photovoltaikanlage   legen    die   PV-Zellen,
welche   aus   verschiedenen   Halbleitermaterialien   bestehen. Die   meisten
heutzutage  verwendeten  Zellen  werden aus  dem  Halbleitermaterial  Silizium
hergestellt. Siliziumzellen  sind in  verschiedenen Formfaktoren  verf\"ugbar;
g\"angige   Gr\"ossen    haben   ca.   zwischen    \SI{10}{\centi\meter}   und
\SI{15}{\centi\meter}  Kantenl\"ange.   Zum  mechanischen  Schutz  der  Zellen
sind  diese mit  einer durchsichtigen  Antireflexschicht \"uberzogen.   Die an
der  PV-Zelle  abgreifbare  Spannung betr\"agt  zwischen  \SI{0.5}{\volt}  und
\SI{0.8}{\volt} DC,  wobei die  Klemmenspannung einer  voll funktionsf\"ahigen
Zelle nur schwach von  der Lichteinstrahlung abh\"angig ist. Die Stromst\"arke
hingegen   ist  sehr   stark  von   der  Beleuchtungsst\"arke   abh\"angig. Je
nach    Sonneneinstrahlung   erreicht    eine   \SI{100}{\centi\meter\squared}
grosse   Siliziumzelle  eine   Stromst\"arke   von   bis  zu   \SI{2}{\ampere}
\cite{ref:pv:gesellschaftFuerSonnenenergie}. \fref{fig:pvcell:front} zeigt die
Frontansicht einer PV-Zelle.

\clearpage
\begin{wrapfigure}{l}{0.45\textwidth}
    \centering
    \includegraphics[width=0.4\textwidth]{images/solar-facility/pvmodule.jpeg}
    \caption{
        Solarmodul,  zusammengesetzt  aus  72 Solarzellen  gem\"ass  Abbildung
        \ref{fig:pvcell:front}%
    }
    \label{fig:pvmodule}
    \vspace*{-1em}
\end{wrapfigure}


Die  in der  Praxis  \"ubliche  Baugruppe ist  nicht  die Solarzelle,  sondern
das  Photovoltaikmodul.  Um  die Ausgangsspannung  und/oder den  Ausgangsstrom
zu  erh\"ohen  und um  diese  in  der  Praxis  besser einsetzen  zu  k\"onnen,
werden mehrere  Zellen in  verschiedenen Konfigurationen miteinander  zu einem
PV-Modul  verschaltet. Das Konzept  ist schematisch  dargestellt in  Abbildung
\ref{fig:pvmodule}, die Eckdaten einiger PV-Module  sind als Beispiele sind in
Anhang \ref{app:commercial:modules}  ab Seite \pageref{app:commercial:modules}
aufgef\"uhrt.

Zur  Erh\"ohung  des Ausgangsstroms  werden  Zellen  parallel geschaltet,  zur
Erh\"ohung  der  Ausgangsspannung werden  sie  in  Serie verbunden.   Je  nach
gew\"unschten Spezifikationen eines Moduls werden diese Ans\"atze einzeln oder
kombiniert angewandt.  In der Praxis \"ublich  sind Module mit 36, 72 oder 144
Zellen und  einer Ausgangsspannung  von \SI{12}{\volt}  bis \SI{60}{\volt}. In
Photovoltaikanlagen werden nur solche  ganzen Module eingesetzt; die einzelnen
Zellen  sind  f\"ur  Reparaturen  nicht  mehr  zug\"anglich. Die  elektrischen
Anschl\"usse befinden sich in kleinen  Kunststoffdosen auf der R\"uckseite der
Module (Beispiel in Abbildung \ref{fig:pvJunctionBox}) \cite{ref:pv:baunetz}.


Mehrere  identische  Module  werden  \"ublicherweise  mittels  Reihenschaltung
zu  einem  Modulstrang  verschaltet  (vereinfacht  dargestellt\hfill  in\hfill
Abbildung\hfill    \ref{fig:pvarray:gak:inverter}).    \hfill    Dadurch\hfill
wird\hfill eine\hfill h\"ohere\hfill Ausgangsspannung

\begin{wrapfigure}{r}{0.35\textwidth}
    \centering
    \includegraphics[width=0.3\textwidth]{images/solar-facility/pvJunctionBox.jpeg}
    \caption{
        Anschlussbox f\"ur Solarmodul, wird  normalerweise auf der R\"uckseite
        des Moduls montiert. \cite{ref:junctionBox}%
    }
    \label{fig:pvJunctionBox}
\end{wrapfigure}

\noindent erzeugt,  was die  Umwandlung des  Gleichstroms in  Wechselstrom und
dessen  Einspeisung  ins  Wechselstromnetz  erleichtert. Die  Ausgangsspannung
eines  Modulstrangs   darf  gem\"ass  Vorschrift  \SI{1000}{\volt}   DC  nicht
\"uberschreiten,  was die  Anzahl der  in einem  Strang in  Serie geschalteten
Module beschr\"ankt.  Module, die  an unterschiedlichen Dachneigungen montiert
sind  oder die  unterschiedliche  Ausrichtungen haben,  sollten  nie zu  einem
Modulstrang  zusammengeschaltet  werden,  da  sie  nicht  den  gleichen  Strom
produzieren  werden,  was  die  Effizienz  des  Strangs  stark  reduziert. Ein
einzelnes   beschattetes  oder   nicht   einwandfrei  funktionierendes   Modul
beeintr\"achtigt die vom gesamten Modulstrang abgegebene Leistung stark (siehe
auch n\"achster Abschnitt).

\todo{Freilaufdioden}

\begin{figure}[h!t]
    \centering
    \begin{tikzpicture}
        \begin{scope}[x={(0mm,\textwidth)},y={(0mm,-70mm)}]
            \node[inner sep=0pt,anchor=north west] at (0mm,0mm) {%
                \includegraphics[width=\textwidth]{images/solar-facility/pvarray.jpeg}%
            };
            \node at (132mm,-55mm) {\small{GAK}};
            \node at (120mm,0mm) {\small{Wechselrichter}};
            \node at (55mm,-5mm) {\small{Modulstrang}};
        \end{scope}
    \end{tikzpicture}
    \caption{%
        Modulstrang aus 16 Modulen  gem\"ass Abbildung \ref{fig:pvmodule}, GAK
        und Wechselrichter. Bild des GAK  von \cite{ref:gak:gantner}, Bild des
        Wechselrichters von \cite{ref:inverter:sunnyboy}.%
    }
    \label{fig:pvarray:gak:inverter}
\end{figure}

Die    Umwandlung    der    Gleichspannung    in    Wechselspannung    erfolgt
mittels    Wechselrichter   (dargestellt    auf   der    rechten   Seite    in
Abbildung    \ref{fig:pvarray:gak:inverter}). Je     nach    Anwendungsbereich
werden  unterschiedliche  Wechselrichter  eingesetzt:\todo{Anwendungsbereiche,
Vor-Nachteile spezifizieren}

\begin{symbols}
    \firmlist
    \item[\textbf{Modulwechselrichter}:]
        Ein  Wechselrichter,  welcher  den  Strom eines  einzelnen  Moduls  in
        Wechselstrom konvertiert.
    \item[\textbf{Strangwechselrichter}:]
        Konvertiert den Strom eines Modulstrangs zu Wechselstrom.
    \item[\textbf{Multistrangwechselrichter}:]
        Fasst   die   Gleichstr\"ome   mehrer  Modulstr\"ange   zusammen   und
        konvertiert diese zu Wechselstrom.
    \item[\textbf{Zentralwechselrichter}:]
        Ein  einzelner Wechselrichter,  welcher den  Gleichstrom einer  ganzen
        Photovoltaikanlage in Wechselstrom umwandelt. Kommt haupts\"achlich in
        Grossanlagen zum Einsatz, bei denen alle Str\"ange die gleiche Neigung
        aufweisen und gleich ausgerichtet sind \cite{ref:pv:ratgeber}.
\end{symbols}

Als Leitungsschutz gegen \"Uberstrom und Kurzschluss befindet sich unmittelbar
vor  dem   Wechselrichter  im  Gleichstromnetz   ein  Generatoranschlusskasten
(GAK). Dieser   dient   auch   zur    Trennung   der   gesamten   Anlage   vom
Netz. \"Ublicherweise  werden  GAK  und   Wechselrichter  am  selben  Standort
platziert.


% ---------------------------------------------------------------------------- %
\clearpage
\section{Leistungseinbr\"uche}
\label{sec:shadedCells}
% ---------------------------------------------------------------------------- %

\todo{Referenzen auf Simu-Circuits}

\begin{figure}[h!tb]
    \input{images/iv-curves/module-3d.tex}
    \begin{tikzpicture}
   \begin{scope}[x={(0mm,0mm)},y={(90mm,0.9\textwidth)}]
       \begin{axis}[%
               height=50mm,
               width=\textwidth,
               at={(0,50mm)},
               grid=both,
               xlabel=Spannung (\si{\volt}),
               ylabel=Strom (\si{\ampere}),
               colormap/hot2,
               %axis y line*=left,
               %x unit=u,
               %change x base=true,
               %line width = 1pt,
               %thick,
               %x SI prefix=micro,
           ]
           \addplot[-,purple]  table[x=V(out), y=I(R3)] {data/iv-curves/generic-module/iv-generic--1a.dat};
           \addplot[-,teal]    table[x=V(out), y=I(R3)] {data/iv-curves/generic-module/iv-generic--4a.dat};
           \addplot[-,magenta] table[x=V(out), y=I(R3)] {data/iv-curves/generic-module/iv-generic--7a.dat};
           \addplot[-,blue]    table[x=V(out), y=I(R3)] {data/iv-curves/generic-module/iv-generic--9a.dat};
            %\addplot[-,color=blue] table {data/iv-curves/module-72cells-series--reference--all-ok.dat};
            %\addplot[-,color=teal] table {data/iv-curves/module-72cells-series--reference--ifail-5A.dat};
            %\addplot[-,color=magenta] table {data/iv-curves/module-72cells-series--reference--ifail-1A.dat};
        \end{axis}
        \begin{axis}[%
               height=50mm,
               width=\textwidth,
               at={(0,0)},
               grid=both,
               xlabel=Spannung (\si{\volt}),
               ylabel=Leistung (\si{\watt}),
               %axis y line*=left,
               %x unit=u,
               %change x base=true,
               %line width = 1pt,
               %thick,
               %x SI prefix=micro,
           ]
           \addplot[-,purple]  table[x=V(out), y=power] {data/iv-curves/generic-module/iv-generic--1a.dat};
           \addplot[-,teal]    table[x=V(out), y=power] {data/iv-curves/generic-module/iv-generic--4a.dat};
           \addplot[-,magenta] table[x=V(out), y=power] {data/iv-curves/generic-module/iv-generic--7a.dat};
           \addplot[-,blue]    table[x=V(out), y=power] {data/iv-curves/generic-module/iv-generic--9a.dat};
        \end{axis}
    \end{scope}
\end{tikzpicture}

    \caption{
        Verhalten   eines    Moduls   bei    Reduktion   des    Stroms   einer
        einzelnen   Zelle.   Die   zuge\"orige  \code{LTspice}-Schaltung   ist
        in    Abbildung    \ref{fig:ltspice:iv:generic:module}    auf    Seite
        \pageref{fig:ltspice:iv:generic:module} zu finden.%
    }
    \label{fig:simu:iv-curves:module:generic}
\end{figure}

\begin{figure}[h!tb]
    %\input{images/iv-curves/array-3d.tex}
    \input{images/iv-curves/array-2d.tex}
    \caption{
        Verhalten       eines      Modulstrangs       ohne      Freilaufdioden
        bei     Reduktion     des     Stroms    einer     einzelnen     Zelle.
        Die      zugeh\"origen     \code{LTspice}-Schaltungen      sind     im
        Anhang       in       Abbildung       \ref{fig:ltspice:string:ivCurve}
        auf     Seite      \pageref{fig:ltspice:string:ivCurve}     und     in
        Abbildung      \ref{fig:ltspice:jacModule:NoDiode}      auf      Seite
        \pageref{fig:ltspice:jacModule:NoDiode} dokumentiert.%
    }
    \label{fig:simu:iv-curves:array:generic}
\end{figure}

\begin{figure}[h!tb]
    %\input{images/iv-curves/array-bypass-3d.tex}
    \input{images/iv-curves/array-bypass-2d.tex}
    \caption{
        Verhalten   eines    Modulstrangs   bei   einer    Freilaufdiode   pro
        Modul. Es    werden     zwei    Zellen    in     zwei    verschiedenen
        Modulen   im   gesamten   Modulstrang   auf   reduzierte   Kapazit\"at
        gesetzt,   Die   Prozentangaben   in   der   Legende   beziehen   sich
        auf   die    Stromabgabe   der   zwei   Zellen    bezogen   auf   ihre
        maximale  Kapazit\"at. Die   zugeh\"origen  \code{LTspice}-Schaltungen
        sind   im   Anhang   in   Abbildung   \ref{fig:ltspice:string:ivCurve}
        auf     Seite      \pageref{fig:ltspice:string:ivCurve}     und     in
        Abbildung       \ref{fig:ltspice:jacModule:Diode}      auf       Seite
        \pageref{fig:ltspice:jacModule:Diode} dokumentiert.%
    }
    \label{fig:simu:iv-curves:array:generic:bypass}
\end{figure}


% ---------------------------------------------------------------------------- %
\clearpage
\section{Unser System}
\label{sec:ourSystem}
% ---------------------------------------------------------------------------- %

Unser   System  bietet   neben   einer   benutzerfreundlichen  Bedienung   und
Inbetriebnahme,  so wie  einer  einfachen Integration  in bestehende  Anlagen,
ebenfalls  eine  kosteng\"unstige  und energieeffiziente  L\"osung  f\"ur  die
\"Uberwachung von Photovoltaikanlagen. Das System setzt sich zusammen aus zwei
Hauptkomponenten, dem Master-Ger\"at und den jeweiligen Sensoren.

Die Installation  des Master-Ger\"ates erfolgt im  zugeh\"origen Schaltschrank
des  Wechselrichters. Das  schlichte  Hutschienengeh\"ause  erm\"oglicht  eine
einfache   Installation   in   bestehende  oder   neue   Schaltschr\"anke. Die
Energieversorgung,  wie  auch  die  Ankopplung  an  die  DC-Leitungen  erfolgt
intern   des   Schaltschrankes   und  wird   von   einem   Elektroinstallateur
ausgef\"uhrt. Neben dem  Klemmenanschluss f\"ur die Energieversorgung  und den
zwei Relaisausg\"angen  f\"ur externe Alarmger\"ate, sieht  das Master-Ger\"at
die  Ankopplung  der  DC-Leitungen  von  bis  zu  drei  Modulstrangs  vor. Ein
internes 3.5“ Touch-Display erm\"oglicht eine einfache Inbetriebnahme und eine
benutzerfreundliche  Bedienung des  Systems. Alle  Messresultate der  Sensoren
werden  vom Master-Ger\"at  ausgewertet  und dem  Kunden  in einer  grafischen
Benutzeroberfl\"ache  dargestellt. Tritt ein  Fehler an  der Anlage  auf, wird
eine entsprechende Meldung auf dem Display angezeigt, die beiden Relais werden
bet\"atigt  und  eine  Nachricht  mittels  SMS  wird  versendet. Dadurch  wird
sichergestellt, dass der Kunde m\"oglichst schnell \"uber defekte PV-Module in
Kenntnis gesetzt wird.

Als Fehler werden folgende Ereignisse definiert:
\begin{itemize}
    \firmlist
    \item
        Eine defekte Zelle
    \item
        Eine dauerhaft verschmutzte Zelle
    \item
        Eine defekte Leitung
\end{itemize}

Kein Fehler soll in folgenden Situationen ausgel\"ost werden.
\begin{itemize}
    \firmlist
    \item
        Kurzzeitige Abschattungen (z.B. Vogel auf Zelle)
    \item
        Regelm\"assige Abschattungen,  die zu den  Umweltbedingungen geh\"oren
        (z.B. Baum, der t\"aglich abschattet)
    \item
        Nacht-/Schlechtwetter
    \item
        Anlage absichtlich ausser Betrieb genommen (z.B. Unterhaltsarbeiten)
\end{itemize}

F\"ur  die  Spannungsmessung  und  die \"Ubertragung  der  Messresultate  sind
die  Sensoren  zust\"andig. Die  Installation  der  Sensoren  erfolgt  in  den
Anschlussdosen  auf der  R\"uckseite der  PV-Module. F\"ur jedes  PV-Modul ist
jeweils  ein  Sensor  zu  installieren. Eine  zweiadrige  Verbindung  von  der
Sensorplatine  zu den  Klemmen  der Anschlussdose  ist  zust\"andig f\"ur  die
Energieversorgung des Sensors und die  Ankopplung an die DC-Leitung, f\"ur die
\"Ubertragung der Messresultate an das Master-Ger\"at.

% ---------------------------------------------------------------------------- %
\clearpage
\section{Kommunikation \"uber DC-Leitung}
\label{sec:commDCLine}
% ---------------------------------------------------------------------------- %

Die Kommunikation zwischen  \Sensor und \Master \"uber die  DC-Leitung ist das
Herzst\"uck des Systems und das zu l\"osende Kernproblem des Projekts. Es sind
im Rahmen  des Projekts  zwei grunds\"atzliche  Ans\"atze verfolgt  worden, um
Daten  zwischen  \Sensor und  \Master  \"uber  die  DC-Leitung zu  senden  und
empfangen.

\textbf{Frequency-shift keying}: Bei  der FSK (Frequenzumtastung  auf Deutsch)
wird  dem in  der Leitung  fliessenden Gleichstrom  ein (verh\"altnism\"assig)
kleines  Signal aufmoduliert,  welches  die  zu \"ubertragenden  Informationen
enth\"alt. Die  Frequenz   des  aufmodulierten   Anteils  wird   in  diskreten
Schritten variiert  und jeweils  einem Symbol zugeordnet. Bei  einer bin\"aren
Umsetzung  werden  zwei  Frequenzen  benutzt; eine  f\"ur  \code{0}  und  eine
f\"ur  \code{1}.   Das  Verfahren ist  schematisch  in  \fref{fig:fsk:concept}
dargestellt.

\begin{figure}[h!tb]
    \centering
    %% Creator: Matplotlib, PGF backend
%%
%% To include the figure in your LaTeX document, write
%%   \input{<filename>.pgf}
%%
%% Make sure the required packages are loaded in your preamble
%%   \usepackage{pgf}
%%
%% Figures using additional raster images can only be included by \input if
%% they are in the same directory as the main LaTeX file. For loading figures
%% from other directories you can use the `import` package
%%   \usepackage{import}
%% and then include the figures with
%%   \import{<path to file>}{<filename>.pgf}
%%
%% Matplotlib used the following preamble
%%   \usepackage{fontspec}
%%   \setmainfont{Bitstream Vera Serif}
%%   \setsansfont{Bitstream Vera Sans}
%%   \setmonofont{Bitstream Vera Sans Mono}
%%
\begingroup%
\makeatletter%
\begin{pgfpicture}%
\pgfpathrectangle{\pgfpointorigin}{\pgfqpoint{4.500000in}{3.500000in}}%
\pgfusepath{use as bounding box, clip}%
\begin{pgfscope}%
\pgfsetbuttcap%
\pgfsetmiterjoin%
\pgfsetlinewidth{0.000000pt}%
\definecolor{currentstroke}{rgb}{0.000000,0.000000,0.000000}%
\pgfsetstrokecolor{currentstroke}%
\pgfsetstrokeopacity{0.000000}%
\pgfsetdash{}{0pt}%
\pgfpathmoveto{\pgfqpoint{0.000000in}{0.000000in}}%
\pgfpathlineto{\pgfqpoint{4.500000in}{0.000000in}}%
\pgfpathlineto{\pgfqpoint{4.500000in}{3.500000in}}%
\pgfpathlineto{\pgfqpoint{0.000000in}{3.500000in}}%
\pgfpathclose%
\pgfusepath{}%
\end{pgfscope}%
\begin{pgfscope}%
\pgfsetbuttcap%
\pgfsetmiterjoin%
\pgfsetlinewidth{0.000000pt}%
\definecolor{currentstroke}{rgb}{0.000000,0.000000,0.000000}%
\pgfsetstrokecolor{currentstroke}%
\pgfsetstrokeopacity{0.000000}%
\pgfsetdash{}{0pt}%
\pgfpathmoveto{\pgfqpoint{0.225000in}{2.065000in}}%
\pgfpathlineto{\pgfqpoint{4.275000in}{2.065000in}}%
\pgfpathlineto{\pgfqpoint{4.275000in}{3.325000in}}%
\pgfpathlineto{\pgfqpoint{0.225000in}{3.325000in}}%
\pgfpathclose%
\pgfusepath{}%
\end{pgfscope}%
\begin{pgfscope}%
\pgfpathrectangle{\pgfqpoint{0.225000in}{2.065000in}}{\pgfqpoint{4.050000in}{1.260000in}} %
\pgfusepath{clip}%
\pgfsetrectcap%
\pgfsetroundjoin%
\pgfsetlinewidth{1.003750pt}%
\definecolor{currentstroke}{rgb}{0.000000,0.000000,1.000000}%
\pgfsetstrokecolor{currentstroke}%
\pgfsetdash{}{0pt}%
\pgfpathmoveto{\pgfqpoint{0.225000in}{2.170000in}}%
\pgfpathlineto{\pgfqpoint{1.188898in}{2.170000in}}%
\pgfusepath{stroke}%
\end{pgfscope}%
\begin{pgfscope}%
\pgfpathrectangle{\pgfqpoint{0.225000in}{2.065000in}}{\pgfqpoint{4.050000in}{1.260000in}} %
\pgfusepath{clip}%
\pgfsetrectcap%
\pgfsetroundjoin%
\pgfsetlinewidth{1.003750pt}%
\definecolor{currentstroke}{rgb}{0.501961,0.501961,0.501961}%
\pgfsetstrokecolor{currentstroke}%
\pgfsetdash{}{0pt}%
\pgfpathmoveto{\pgfqpoint{1.188898in}{2.170000in}}%
\pgfpathlineto{\pgfqpoint{1.188898in}{3.220000in}}%
\pgfusepath{stroke}%
\end{pgfscope}%
\begin{pgfscope}%
\pgfpathrectangle{\pgfqpoint{0.225000in}{2.065000in}}{\pgfqpoint{4.050000in}{1.260000in}} %
\pgfusepath{clip}%
\pgfsetrectcap%
\pgfsetroundjoin%
\pgfsetlinewidth{1.003750pt}%
\definecolor{currentstroke}{rgb}{1.000000,0.000000,1.000000}%
\pgfsetstrokecolor{currentstroke}%
\pgfsetdash{}{0pt}%
\pgfpathmoveto{\pgfqpoint{1.188898in}{3.220000in}}%
\pgfpathlineto{\pgfqpoint{2.152795in}{3.220000in}}%
\pgfusepath{stroke}%
\end{pgfscope}%
\begin{pgfscope}%
\pgfpathrectangle{\pgfqpoint{0.225000in}{2.065000in}}{\pgfqpoint{4.050000in}{1.260000in}} %
\pgfusepath{clip}%
\pgfsetrectcap%
\pgfsetroundjoin%
\pgfsetlinewidth{1.003750pt}%
\definecolor{currentstroke}{rgb}{0.501961,0.501961,0.501961}%
\pgfsetstrokecolor{currentstroke}%
\pgfsetdash{}{0pt}%
\pgfpathmoveto{\pgfqpoint{2.152795in}{3.220000in}}%
\pgfpathlineto{\pgfqpoint{2.152795in}{2.170000in}}%
\pgfusepath{stroke}%
\end{pgfscope}%
\begin{pgfscope}%
\pgfpathrectangle{\pgfqpoint{0.225000in}{2.065000in}}{\pgfqpoint{4.050000in}{1.260000in}} %
\pgfusepath{clip}%
\pgfsetrectcap%
\pgfsetroundjoin%
\pgfsetlinewidth{1.003750pt}%
\definecolor{currentstroke}{rgb}{0.000000,0.000000,1.000000}%
\pgfsetstrokecolor{currentstroke}%
\pgfsetdash{}{0pt}%
\pgfpathmoveto{\pgfqpoint{2.152795in}{2.170000in}}%
\pgfpathlineto{\pgfqpoint{3.116693in}{2.170000in}}%
\pgfusepath{stroke}%
\end{pgfscope}%
\begin{pgfscope}%
\pgfpathrectangle{\pgfqpoint{0.225000in}{2.065000in}}{\pgfqpoint{4.050000in}{1.260000in}} %
\pgfusepath{clip}%
\pgfsetrectcap%
\pgfsetroundjoin%
\pgfsetlinewidth{1.003750pt}%
\definecolor{currentstroke}{rgb}{0.501961,0.501961,0.501961}%
\pgfsetstrokecolor{currentstroke}%
\pgfsetdash{}{0pt}%
\pgfpathmoveto{\pgfqpoint{3.116693in}{2.170000in}}%
\pgfpathlineto{\pgfqpoint{3.116693in}{3.220000in}}%
\pgfusepath{stroke}%
\end{pgfscope}%
\begin{pgfscope}%
\pgfpathrectangle{\pgfqpoint{0.225000in}{2.065000in}}{\pgfqpoint{4.050000in}{1.260000in}} %
\pgfusepath{clip}%
\pgfsetrectcap%
\pgfsetroundjoin%
\pgfsetlinewidth{1.003750pt}%
\definecolor{currentstroke}{rgb}{1.000000,0.000000,1.000000}%
\pgfsetstrokecolor{currentstroke}%
\pgfsetdash{}{0pt}%
\pgfpathmoveto{\pgfqpoint{3.116693in}{3.220000in}}%
\pgfpathlineto{\pgfqpoint{4.080591in}{3.220000in}}%
\pgfusepath{stroke}%
\end{pgfscope}%
\begin{pgfscope}%
\pgfsetrectcap%
\pgfsetmiterjoin%
\pgfsetlinewidth{1.003750pt}%
\definecolor{currentstroke}{rgb}{0.000000,0.000000,0.000000}%
\pgfsetstrokecolor{currentstroke}%
\pgfsetdash{}{0pt}%
\pgfpathmoveto{\pgfqpoint{4.275000in}{2.065000in}}%
\pgfpathlineto{\pgfqpoint{4.275000in}{3.325000in}}%
\pgfusepath{stroke}%
\end{pgfscope}%
\begin{pgfscope}%
\pgfsetrectcap%
\pgfsetmiterjoin%
\pgfsetlinewidth{1.003750pt}%
\definecolor{currentstroke}{rgb}{0.000000,0.000000,0.000000}%
\pgfsetstrokecolor{currentstroke}%
\pgfsetdash{}{0pt}%
\pgfpathmoveto{\pgfqpoint{0.225000in}{2.065000in}}%
\pgfpathlineto{\pgfqpoint{4.275000in}{2.065000in}}%
\pgfusepath{stroke}%
\end{pgfscope}%
\begin{pgfscope}%
\pgfsetrectcap%
\pgfsetmiterjoin%
\pgfsetlinewidth{1.003750pt}%
\definecolor{currentstroke}{rgb}{0.000000,0.000000,0.000000}%
\pgfsetstrokecolor{currentstroke}%
\pgfsetdash{}{0pt}%
\pgfpathmoveto{\pgfqpoint{0.225000in}{3.325000in}}%
\pgfpathlineto{\pgfqpoint{4.275000in}{3.325000in}}%
\pgfusepath{stroke}%
\end{pgfscope}%
\begin{pgfscope}%
\pgfsetrectcap%
\pgfsetmiterjoin%
\pgfsetlinewidth{1.003750pt}%
\definecolor{currentstroke}{rgb}{0.000000,0.000000,0.000000}%
\pgfsetstrokecolor{currentstroke}%
\pgfsetdash{}{0pt}%
\pgfpathmoveto{\pgfqpoint{0.225000in}{2.065000in}}%
\pgfpathlineto{\pgfqpoint{0.225000in}{3.325000in}}%
\pgfusepath{stroke}%
\end{pgfscope}%
\begin{pgfscope}%
\pgfsetbuttcap%
\pgfsetroundjoin%
\definecolor{currentfill}{rgb}{0.000000,0.000000,0.000000}%
\pgfsetfillcolor{currentfill}%
\pgfsetlinewidth{0.501875pt}%
\definecolor{currentstroke}{rgb}{0.000000,0.000000,0.000000}%
\pgfsetstrokecolor{currentstroke}%
\pgfsetdash{}{0pt}%
\pgfsys@defobject{currentmarker}{\pgfqpoint{0.000000in}{0.000000in}}{\pgfqpoint{0.000000in}{0.055556in}}{%
\pgfpathmoveto{\pgfqpoint{0.000000in}{0.000000in}}%
\pgfpathlineto{\pgfqpoint{0.000000in}{0.055556in}}%
\pgfusepath{stroke,fill}%
}%
\begin{pgfscope}%
\pgfsys@transformshift{0.225000in}{2.065000in}%
\pgfsys@useobject{currentmarker}{}%
\end{pgfscope}%
\end{pgfscope}%
\begin{pgfscope}%
\pgfsetbuttcap%
\pgfsetroundjoin%
\definecolor{currentfill}{rgb}{0.000000,0.000000,0.000000}%
\pgfsetfillcolor{currentfill}%
\pgfsetlinewidth{0.501875pt}%
\definecolor{currentstroke}{rgb}{0.000000,0.000000,0.000000}%
\pgfsetstrokecolor{currentstroke}%
\pgfsetdash{}{0pt}%
\pgfsys@defobject{currentmarker}{\pgfqpoint{0.000000in}{-0.055556in}}{\pgfqpoint{0.000000in}{0.000000in}}{%
\pgfpathmoveto{\pgfqpoint{0.000000in}{0.000000in}}%
\pgfpathlineto{\pgfqpoint{0.000000in}{-0.055556in}}%
\pgfusepath{stroke,fill}%
}%
\begin{pgfscope}%
\pgfsys@transformshift{0.225000in}{3.325000in}%
\pgfsys@useobject{currentmarker}{}%
\end{pgfscope}%
\end{pgfscope}%
\begin{pgfscope}%
\pgftext[x=0.225000in,y=2.009444in,,top]{\rmfamily\fontsize{9.000000}{10.800000}\selectfont \(\displaystyle 0.0000\)}%
\end{pgfscope}%
\begin{pgfscope}%
\pgfsetbuttcap%
\pgfsetroundjoin%
\definecolor{currentfill}{rgb}{0.000000,0.000000,0.000000}%
\pgfsetfillcolor{currentfill}%
\pgfsetlinewidth{0.501875pt}%
\definecolor{currentstroke}{rgb}{0.000000,0.000000,0.000000}%
\pgfsetstrokecolor{currentstroke}%
\pgfsetdash{}{0pt}%
\pgfsys@defobject{currentmarker}{\pgfqpoint{0.000000in}{0.000000in}}{\pgfqpoint{0.000000in}{0.055556in}}{%
\pgfpathmoveto{\pgfqpoint{0.000000in}{0.000000in}}%
\pgfpathlineto{\pgfqpoint{0.000000in}{0.055556in}}%
\pgfusepath{stroke,fill}%
}%
\begin{pgfscope}%
\pgfsys@transformshift{0.731250in}{2.065000in}%
\pgfsys@useobject{currentmarker}{}%
\end{pgfscope}%
\end{pgfscope}%
\begin{pgfscope}%
\pgfsetbuttcap%
\pgfsetroundjoin%
\definecolor{currentfill}{rgb}{0.000000,0.000000,0.000000}%
\pgfsetfillcolor{currentfill}%
\pgfsetlinewidth{0.501875pt}%
\definecolor{currentstroke}{rgb}{0.000000,0.000000,0.000000}%
\pgfsetstrokecolor{currentstroke}%
\pgfsetdash{}{0pt}%
\pgfsys@defobject{currentmarker}{\pgfqpoint{0.000000in}{-0.055556in}}{\pgfqpoint{0.000000in}{0.000000in}}{%
\pgfpathmoveto{\pgfqpoint{0.000000in}{0.000000in}}%
\pgfpathlineto{\pgfqpoint{0.000000in}{-0.055556in}}%
\pgfusepath{stroke,fill}%
}%
\begin{pgfscope}%
\pgfsys@transformshift{0.731250in}{3.325000in}%
\pgfsys@useobject{currentmarker}{}%
\end{pgfscope}%
\end{pgfscope}%
\begin{pgfscope}%
\pgftext[x=0.731250in,y=2.009444in,,top]{\rmfamily\fontsize{9.000000}{10.800000}\selectfont \(\displaystyle 0.0002\)}%
\end{pgfscope}%
\begin{pgfscope}%
\pgfsetbuttcap%
\pgfsetroundjoin%
\definecolor{currentfill}{rgb}{0.000000,0.000000,0.000000}%
\pgfsetfillcolor{currentfill}%
\pgfsetlinewidth{0.501875pt}%
\definecolor{currentstroke}{rgb}{0.000000,0.000000,0.000000}%
\pgfsetstrokecolor{currentstroke}%
\pgfsetdash{}{0pt}%
\pgfsys@defobject{currentmarker}{\pgfqpoint{0.000000in}{0.000000in}}{\pgfqpoint{0.000000in}{0.055556in}}{%
\pgfpathmoveto{\pgfqpoint{0.000000in}{0.000000in}}%
\pgfpathlineto{\pgfqpoint{0.000000in}{0.055556in}}%
\pgfusepath{stroke,fill}%
}%
\begin{pgfscope}%
\pgfsys@transformshift{1.237500in}{2.065000in}%
\pgfsys@useobject{currentmarker}{}%
\end{pgfscope}%
\end{pgfscope}%
\begin{pgfscope}%
\pgfsetbuttcap%
\pgfsetroundjoin%
\definecolor{currentfill}{rgb}{0.000000,0.000000,0.000000}%
\pgfsetfillcolor{currentfill}%
\pgfsetlinewidth{0.501875pt}%
\definecolor{currentstroke}{rgb}{0.000000,0.000000,0.000000}%
\pgfsetstrokecolor{currentstroke}%
\pgfsetdash{}{0pt}%
\pgfsys@defobject{currentmarker}{\pgfqpoint{0.000000in}{-0.055556in}}{\pgfqpoint{0.000000in}{0.000000in}}{%
\pgfpathmoveto{\pgfqpoint{0.000000in}{0.000000in}}%
\pgfpathlineto{\pgfqpoint{0.000000in}{-0.055556in}}%
\pgfusepath{stroke,fill}%
}%
\begin{pgfscope}%
\pgfsys@transformshift{1.237500in}{3.325000in}%
\pgfsys@useobject{currentmarker}{}%
\end{pgfscope}%
\end{pgfscope}%
\begin{pgfscope}%
\pgftext[x=1.237500in,y=2.009444in,,top]{\rmfamily\fontsize{9.000000}{10.800000}\selectfont \(\displaystyle 0.0004\)}%
\end{pgfscope}%
\begin{pgfscope}%
\pgfsetbuttcap%
\pgfsetroundjoin%
\definecolor{currentfill}{rgb}{0.000000,0.000000,0.000000}%
\pgfsetfillcolor{currentfill}%
\pgfsetlinewidth{0.501875pt}%
\definecolor{currentstroke}{rgb}{0.000000,0.000000,0.000000}%
\pgfsetstrokecolor{currentstroke}%
\pgfsetdash{}{0pt}%
\pgfsys@defobject{currentmarker}{\pgfqpoint{0.000000in}{0.000000in}}{\pgfqpoint{0.000000in}{0.055556in}}{%
\pgfpathmoveto{\pgfqpoint{0.000000in}{0.000000in}}%
\pgfpathlineto{\pgfqpoint{0.000000in}{0.055556in}}%
\pgfusepath{stroke,fill}%
}%
\begin{pgfscope}%
\pgfsys@transformshift{1.743750in}{2.065000in}%
\pgfsys@useobject{currentmarker}{}%
\end{pgfscope}%
\end{pgfscope}%
\begin{pgfscope}%
\pgfsetbuttcap%
\pgfsetroundjoin%
\definecolor{currentfill}{rgb}{0.000000,0.000000,0.000000}%
\pgfsetfillcolor{currentfill}%
\pgfsetlinewidth{0.501875pt}%
\definecolor{currentstroke}{rgb}{0.000000,0.000000,0.000000}%
\pgfsetstrokecolor{currentstroke}%
\pgfsetdash{}{0pt}%
\pgfsys@defobject{currentmarker}{\pgfqpoint{0.000000in}{-0.055556in}}{\pgfqpoint{0.000000in}{0.000000in}}{%
\pgfpathmoveto{\pgfqpoint{0.000000in}{0.000000in}}%
\pgfpathlineto{\pgfqpoint{0.000000in}{-0.055556in}}%
\pgfusepath{stroke,fill}%
}%
\begin{pgfscope}%
\pgfsys@transformshift{1.743750in}{3.325000in}%
\pgfsys@useobject{currentmarker}{}%
\end{pgfscope}%
\end{pgfscope}%
\begin{pgfscope}%
\pgftext[x=1.743750in,y=2.009444in,,top]{\rmfamily\fontsize{9.000000}{10.800000}\selectfont \(\displaystyle 0.0006\)}%
\end{pgfscope}%
\begin{pgfscope}%
\pgfsetbuttcap%
\pgfsetroundjoin%
\definecolor{currentfill}{rgb}{0.000000,0.000000,0.000000}%
\pgfsetfillcolor{currentfill}%
\pgfsetlinewidth{0.501875pt}%
\definecolor{currentstroke}{rgb}{0.000000,0.000000,0.000000}%
\pgfsetstrokecolor{currentstroke}%
\pgfsetdash{}{0pt}%
\pgfsys@defobject{currentmarker}{\pgfqpoint{0.000000in}{0.000000in}}{\pgfqpoint{0.000000in}{0.055556in}}{%
\pgfpathmoveto{\pgfqpoint{0.000000in}{0.000000in}}%
\pgfpathlineto{\pgfqpoint{0.000000in}{0.055556in}}%
\pgfusepath{stroke,fill}%
}%
\begin{pgfscope}%
\pgfsys@transformshift{2.250000in}{2.065000in}%
\pgfsys@useobject{currentmarker}{}%
\end{pgfscope}%
\end{pgfscope}%
\begin{pgfscope}%
\pgfsetbuttcap%
\pgfsetroundjoin%
\definecolor{currentfill}{rgb}{0.000000,0.000000,0.000000}%
\pgfsetfillcolor{currentfill}%
\pgfsetlinewidth{0.501875pt}%
\definecolor{currentstroke}{rgb}{0.000000,0.000000,0.000000}%
\pgfsetstrokecolor{currentstroke}%
\pgfsetdash{}{0pt}%
\pgfsys@defobject{currentmarker}{\pgfqpoint{0.000000in}{-0.055556in}}{\pgfqpoint{0.000000in}{0.000000in}}{%
\pgfpathmoveto{\pgfqpoint{0.000000in}{0.000000in}}%
\pgfpathlineto{\pgfqpoint{0.000000in}{-0.055556in}}%
\pgfusepath{stroke,fill}%
}%
\begin{pgfscope}%
\pgfsys@transformshift{2.250000in}{3.325000in}%
\pgfsys@useobject{currentmarker}{}%
\end{pgfscope}%
\end{pgfscope}%
\begin{pgfscope}%
\pgftext[x=2.250000in,y=2.009444in,,top]{\rmfamily\fontsize{9.000000}{10.800000}\selectfont \(\displaystyle 0.0008\)}%
\end{pgfscope}%
\begin{pgfscope}%
\pgfsetbuttcap%
\pgfsetroundjoin%
\definecolor{currentfill}{rgb}{0.000000,0.000000,0.000000}%
\pgfsetfillcolor{currentfill}%
\pgfsetlinewidth{0.501875pt}%
\definecolor{currentstroke}{rgb}{0.000000,0.000000,0.000000}%
\pgfsetstrokecolor{currentstroke}%
\pgfsetdash{}{0pt}%
\pgfsys@defobject{currentmarker}{\pgfqpoint{0.000000in}{0.000000in}}{\pgfqpoint{0.000000in}{0.055556in}}{%
\pgfpathmoveto{\pgfqpoint{0.000000in}{0.000000in}}%
\pgfpathlineto{\pgfqpoint{0.000000in}{0.055556in}}%
\pgfusepath{stroke,fill}%
}%
\begin{pgfscope}%
\pgfsys@transformshift{2.756250in}{2.065000in}%
\pgfsys@useobject{currentmarker}{}%
\end{pgfscope}%
\end{pgfscope}%
\begin{pgfscope}%
\pgfsetbuttcap%
\pgfsetroundjoin%
\definecolor{currentfill}{rgb}{0.000000,0.000000,0.000000}%
\pgfsetfillcolor{currentfill}%
\pgfsetlinewidth{0.501875pt}%
\definecolor{currentstroke}{rgb}{0.000000,0.000000,0.000000}%
\pgfsetstrokecolor{currentstroke}%
\pgfsetdash{}{0pt}%
\pgfsys@defobject{currentmarker}{\pgfqpoint{0.000000in}{-0.055556in}}{\pgfqpoint{0.000000in}{0.000000in}}{%
\pgfpathmoveto{\pgfqpoint{0.000000in}{0.000000in}}%
\pgfpathlineto{\pgfqpoint{0.000000in}{-0.055556in}}%
\pgfusepath{stroke,fill}%
}%
\begin{pgfscope}%
\pgfsys@transformshift{2.756250in}{3.325000in}%
\pgfsys@useobject{currentmarker}{}%
\end{pgfscope}%
\end{pgfscope}%
\begin{pgfscope}%
\pgftext[x=2.756250in,y=2.009444in,,top]{\rmfamily\fontsize{9.000000}{10.800000}\selectfont \(\displaystyle 0.0010\)}%
\end{pgfscope}%
\begin{pgfscope}%
\pgfsetbuttcap%
\pgfsetroundjoin%
\definecolor{currentfill}{rgb}{0.000000,0.000000,0.000000}%
\pgfsetfillcolor{currentfill}%
\pgfsetlinewidth{0.501875pt}%
\definecolor{currentstroke}{rgb}{0.000000,0.000000,0.000000}%
\pgfsetstrokecolor{currentstroke}%
\pgfsetdash{}{0pt}%
\pgfsys@defobject{currentmarker}{\pgfqpoint{0.000000in}{0.000000in}}{\pgfqpoint{0.000000in}{0.055556in}}{%
\pgfpathmoveto{\pgfqpoint{0.000000in}{0.000000in}}%
\pgfpathlineto{\pgfqpoint{0.000000in}{0.055556in}}%
\pgfusepath{stroke,fill}%
}%
\begin{pgfscope}%
\pgfsys@transformshift{3.262500in}{2.065000in}%
\pgfsys@useobject{currentmarker}{}%
\end{pgfscope}%
\end{pgfscope}%
\begin{pgfscope}%
\pgfsetbuttcap%
\pgfsetroundjoin%
\definecolor{currentfill}{rgb}{0.000000,0.000000,0.000000}%
\pgfsetfillcolor{currentfill}%
\pgfsetlinewidth{0.501875pt}%
\definecolor{currentstroke}{rgb}{0.000000,0.000000,0.000000}%
\pgfsetstrokecolor{currentstroke}%
\pgfsetdash{}{0pt}%
\pgfsys@defobject{currentmarker}{\pgfqpoint{0.000000in}{-0.055556in}}{\pgfqpoint{0.000000in}{0.000000in}}{%
\pgfpathmoveto{\pgfqpoint{0.000000in}{0.000000in}}%
\pgfpathlineto{\pgfqpoint{0.000000in}{-0.055556in}}%
\pgfusepath{stroke,fill}%
}%
\begin{pgfscope}%
\pgfsys@transformshift{3.262500in}{3.325000in}%
\pgfsys@useobject{currentmarker}{}%
\end{pgfscope}%
\end{pgfscope}%
\begin{pgfscope}%
\pgftext[x=3.262500in,y=2.009444in,,top]{\rmfamily\fontsize{9.000000}{10.800000}\selectfont \(\displaystyle 0.0012\)}%
\end{pgfscope}%
\begin{pgfscope}%
\pgfsetbuttcap%
\pgfsetroundjoin%
\definecolor{currentfill}{rgb}{0.000000,0.000000,0.000000}%
\pgfsetfillcolor{currentfill}%
\pgfsetlinewidth{0.501875pt}%
\definecolor{currentstroke}{rgb}{0.000000,0.000000,0.000000}%
\pgfsetstrokecolor{currentstroke}%
\pgfsetdash{}{0pt}%
\pgfsys@defobject{currentmarker}{\pgfqpoint{0.000000in}{0.000000in}}{\pgfqpoint{0.000000in}{0.055556in}}{%
\pgfpathmoveto{\pgfqpoint{0.000000in}{0.000000in}}%
\pgfpathlineto{\pgfqpoint{0.000000in}{0.055556in}}%
\pgfusepath{stroke,fill}%
}%
\begin{pgfscope}%
\pgfsys@transformshift{3.768750in}{2.065000in}%
\pgfsys@useobject{currentmarker}{}%
\end{pgfscope}%
\end{pgfscope}%
\begin{pgfscope}%
\pgfsetbuttcap%
\pgfsetroundjoin%
\definecolor{currentfill}{rgb}{0.000000,0.000000,0.000000}%
\pgfsetfillcolor{currentfill}%
\pgfsetlinewidth{0.501875pt}%
\definecolor{currentstroke}{rgb}{0.000000,0.000000,0.000000}%
\pgfsetstrokecolor{currentstroke}%
\pgfsetdash{}{0pt}%
\pgfsys@defobject{currentmarker}{\pgfqpoint{0.000000in}{-0.055556in}}{\pgfqpoint{0.000000in}{0.000000in}}{%
\pgfpathmoveto{\pgfqpoint{0.000000in}{0.000000in}}%
\pgfpathlineto{\pgfqpoint{0.000000in}{-0.055556in}}%
\pgfusepath{stroke,fill}%
}%
\begin{pgfscope}%
\pgfsys@transformshift{3.768750in}{3.325000in}%
\pgfsys@useobject{currentmarker}{}%
\end{pgfscope}%
\end{pgfscope}%
\begin{pgfscope}%
\pgftext[x=3.768750in,y=2.009444in,,top]{\rmfamily\fontsize{9.000000}{10.800000}\selectfont \(\displaystyle 0.0014\)}%
\end{pgfscope}%
\begin{pgfscope}%
\pgfsetbuttcap%
\pgfsetroundjoin%
\definecolor{currentfill}{rgb}{0.000000,0.000000,0.000000}%
\pgfsetfillcolor{currentfill}%
\pgfsetlinewidth{0.501875pt}%
\definecolor{currentstroke}{rgb}{0.000000,0.000000,0.000000}%
\pgfsetstrokecolor{currentstroke}%
\pgfsetdash{}{0pt}%
\pgfsys@defobject{currentmarker}{\pgfqpoint{0.000000in}{0.000000in}}{\pgfqpoint{0.000000in}{0.055556in}}{%
\pgfpathmoveto{\pgfqpoint{0.000000in}{0.000000in}}%
\pgfpathlineto{\pgfqpoint{0.000000in}{0.055556in}}%
\pgfusepath{stroke,fill}%
}%
\begin{pgfscope}%
\pgfsys@transformshift{4.275000in}{2.065000in}%
\pgfsys@useobject{currentmarker}{}%
\end{pgfscope}%
\end{pgfscope}%
\begin{pgfscope}%
\pgfsetbuttcap%
\pgfsetroundjoin%
\definecolor{currentfill}{rgb}{0.000000,0.000000,0.000000}%
\pgfsetfillcolor{currentfill}%
\pgfsetlinewidth{0.501875pt}%
\definecolor{currentstroke}{rgb}{0.000000,0.000000,0.000000}%
\pgfsetstrokecolor{currentstroke}%
\pgfsetdash{}{0pt}%
\pgfsys@defobject{currentmarker}{\pgfqpoint{0.000000in}{-0.055556in}}{\pgfqpoint{0.000000in}{0.000000in}}{%
\pgfpathmoveto{\pgfqpoint{0.000000in}{0.000000in}}%
\pgfpathlineto{\pgfqpoint{0.000000in}{-0.055556in}}%
\pgfusepath{stroke,fill}%
}%
\begin{pgfscope}%
\pgfsys@transformshift{4.275000in}{3.325000in}%
\pgfsys@useobject{currentmarker}{}%
\end{pgfscope}%
\end{pgfscope}%
\begin{pgfscope}%
\pgftext[x=4.275000in,y=2.009444in,,top]{\rmfamily\fontsize{9.000000}{10.800000}\selectfont \(\displaystyle 0.0016\)}%
\end{pgfscope}%
\begin{pgfscope}%
\pgfsetbuttcap%
\pgfsetroundjoin%
\definecolor{currentfill}{rgb}{0.000000,0.000000,0.000000}%
\pgfsetfillcolor{currentfill}%
\pgfsetlinewidth{0.501875pt}%
\definecolor{currentstroke}{rgb}{0.000000,0.000000,0.000000}%
\pgfsetstrokecolor{currentstroke}%
\pgfsetdash{}{0pt}%
\pgfsys@defobject{currentmarker}{\pgfqpoint{0.000000in}{0.000000in}}{\pgfqpoint{0.055556in}{0.000000in}}{%
\pgfpathmoveto{\pgfqpoint{0.000000in}{0.000000in}}%
\pgfpathlineto{\pgfqpoint{0.055556in}{0.000000in}}%
\pgfusepath{stroke,fill}%
}%
\begin{pgfscope}%
\pgfsys@transformshift{0.225000in}{2.170000in}%
\pgfsys@useobject{currentmarker}{}%
\end{pgfscope}%
\end{pgfscope}%
\begin{pgfscope}%
\pgfsetbuttcap%
\pgfsetroundjoin%
\definecolor{currentfill}{rgb}{0.000000,0.000000,0.000000}%
\pgfsetfillcolor{currentfill}%
\pgfsetlinewidth{0.501875pt}%
\definecolor{currentstroke}{rgb}{0.000000,0.000000,0.000000}%
\pgfsetstrokecolor{currentstroke}%
\pgfsetdash{}{0pt}%
\pgfsys@defobject{currentmarker}{\pgfqpoint{-0.055556in}{0.000000in}}{\pgfqpoint{0.000000in}{0.000000in}}{%
\pgfpathmoveto{\pgfqpoint{0.000000in}{0.000000in}}%
\pgfpathlineto{\pgfqpoint{-0.055556in}{0.000000in}}%
\pgfusepath{stroke,fill}%
}%
\begin{pgfscope}%
\pgfsys@transformshift{4.275000in}{2.170000in}%
\pgfsys@useobject{currentmarker}{}%
\end{pgfscope}%
\end{pgfscope}%
\begin{pgfscope}%
\pgftext[x=0.169444in,y=2.170000in,right,]{\rmfamily\fontsize{9.000000}{10.800000}\selectfont \(\displaystyle 0.0\)}%
\end{pgfscope}%
\begin{pgfscope}%
\pgfsetbuttcap%
\pgfsetroundjoin%
\definecolor{currentfill}{rgb}{0.000000,0.000000,0.000000}%
\pgfsetfillcolor{currentfill}%
\pgfsetlinewidth{0.501875pt}%
\definecolor{currentstroke}{rgb}{0.000000,0.000000,0.000000}%
\pgfsetstrokecolor{currentstroke}%
\pgfsetdash{}{0pt}%
\pgfsys@defobject{currentmarker}{\pgfqpoint{0.000000in}{0.000000in}}{\pgfqpoint{0.055556in}{0.000000in}}{%
\pgfpathmoveto{\pgfqpoint{0.000000in}{0.000000in}}%
\pgfpathlineto{\pgfqpoint{0.055556in}{0.000000in}}%
\pgfusepath{stroke,fill}%
}%
\begin{pgfscope}%
\pgfsys@transformshift{0.225000in}{2.380000in}%
\pgfsys@useobject{currentmarker}{}%
\end{pgfscope}%
\end{pgfscope}%
\begin{pgfscope}%
\pgfsetbuttcap%
\pgfsetroundjoin%
\definecolor{currentfill}{rgb}{0.000000,0.000000,0.000000}%
\pgfsetfillcolor{currentfill}%
\pgfsetlinewidth{0.501875pt}%
\definecolor{currentstroke}{rgb}{0.000000,0.000000,0.000000}%
\pgfsetstrokecolor{currentstroke}%
\pgfsetdash{}{0pt}%
\pgfsys@defobject{currentmarker}{\pgfqpoint{-0.055556in}{0.000000in}}{\pgfqpoint{0.000000in}{0.000000in}}{%
\pgfpathmoveto{\pgfqpoint{0.000000in}{0.000000in}}%
\pgfpathlineto{\pgfqpoint{-0.055556in}{0.000000in}}%
\pgfusepath{stroke,fill}%
}%
\begin{pgfscope}%
\pgfsys@transformshift{4.275000in}{2.380000in}%
\pgfsys@useobject{currentmarker}{}%
\end{pgfscope}%
\end{pgfscope}%
\begin{pgfscope}%
\pgftext[x=0.169444in,y=2.380000in,right,]{\rmfamily\fontsize{9.000000}{10.800000}\selectfont \(\displaystyle 0.2\)}%
\end{pgfscope}%
\begin{pgfscope}%
\pgfsetbuttcap%
\pgfsetroundjoin%
\definecolor{currentfill}{rgb}{0.000000,0.000000,0.000000}%
\pgfsetfillcolor{currentfill}%
\pgfsetlinewidth{0.501875pt}%
\definecolor{currentstroke}{rgb}{0.000000,0.000000,0.000000}%
\pgfsetstrokecolor{currentstroke}%
\pgfsetdash{}{0pt}%
\pgfsys@defobject{currentmarker}{\pgfqpoint{0.000000in}{0.000000in}}{\pgfqpoint{0.055556in}{0.000000in}}{%
\pgfpathmoveto{\pgfqpoint{0.000000in}{0.000000in}}%
\pgfpathlineto{\pgfqpoint{0.055556in}{0.000000in}}%
\pgfusepath{stroke,fill}%
}%
\begin{pgfscope}%
\pgfsys@transformshift{0.225000in}{2.590000in}%
\pgfsys@useobject{currentmarker}{}%
\end{pgfscope}%
\end{pgfscope}%
\begin{pgfscope}%
\pgfsetbuttcap%
\pgfsetroundjoin%
\definecolor{currentfill}{rgb}{0.000000,0.000000,0.000000}%
\pgfsetfillcolor{currentfill}%
\pgfsetlinewidth{0.501875pt}%
\definecolor{currentstroke}{rgb}{0.000000,0.000000,0.000000}%
\pgfsetstrokecolor{currentstroke}%
\pgfsetdash{}{0pt}%
\pgfsys@defobject{currentmarker}{\pgfqpoint{-0.055556in}{0.000000in}}{\pgfqpoint{0.000000in}{0.000000in}}{%
\pgfpathmoveto{\pgfqpoint{0.000000in}{0.000000in}}%
\pgfpathlineto{\pgfqpoint{-0.055556in}{0.000000in}}%
\pgfusepath{stroke,fill}%
}%
\begin{pgfscope}%
\pgfsys@transformshift{4.275000in}{2.590000in}%
\pgfsys@useobject{currentmarker}{}%
\end{pgfscope}%
\end{pgfscope}%
\begin{pgfscope}%
\pgftext[x=0.169444in,y=2.590000in,right,]{\rmfamily\fontsize{9.000000}{10.800000}\selectfont \(\displaystyle 0.4\)}%
\end{pgfscope}%
\begin{pgfscope}%
\pgfsetbuttcap%
\pgfsetroundjoin%
\definecolor{currentfill}{rgb}{0.000000,0.000000,0.000000}%
\pgfsetfillcolor{currentfill}%
\pgfsetlinewidth{0.501875pt}%
\definecolor{currentstroke}{rgb}{0.000000,0.000000,0.000000}%
\pgfsetstrokecolor{currentstroke}%
\pgfsetdash{}{0pt}%
\pgfsys@defobject{currentmarker}{\pgfqpoint{0.000000in}{0.000000in}}{\pgfqpoint{0.055556in}{0.000000in}}{%
\pgfpathmoveto{\pgfqpoint{0.000000in}{0.000000in}}%
\pgfpathlineto{\pgfqpoint{0.055556in}{0.000000in}}%
\pgfusepath{stroke,fill}%
}%
\begin{pgfscope}%
\pgfsys@transformshift{0.225000in}{2.800000in}%
\pgfsys@useobject{currentmarker}{}%
\end{pgfscope}%
\end{pgfscope}%
\begin{pgfscope}%
\pgfsetbuttcap%
\pgfsetroundjoin%
\definecolor{currentfill}{rgb}{0.000000,0.000000,0.000000}%
\pgfsetfillcolor{currentfill}%
\pgfsetlinewidth{0.501875pt}%
\definecolor{currentstroke}{rgb}{0.000000,0.000000,0.000000}%
\pgfsetstrokecolor{currentstroke}%
\pgfsetdash{}{0pt}%
\pgfsys@defobject{currentmarker}{\pgfqpoint{-0.055556in}{0.000000in}}{\pgfqpoint{0.000000in}{0.000000in}}{%
\pgfpathmoveto{\pgfqpoint{0.000000in}{0.000000in}}%
\pgfpathlineto{\pgfqpoint{-0.055556in}{0.000000in}}%
\pgfusepath{stroke,fill}%
}%
\begin{pgfscope}%
\pgfsys@transformshift{4.275000in}{2.800000in}%
\pgfsys@useobject{currentmarker}{}%
\end{pgfscope}%
\end{pgfscope}%
\begin{pgfscope}%
\pgftext[x=0.169444in,y=2.800000in,right,]{\rmfamily\fontsize{9.000000}{10.800000}\selectfont \(\displaystyle 0.6\)}%
\end{pgfscope}%
\begin{pgfscope}%
\pgfsetbuttcap%
\pgfsetroundjoin%
\definecolor{currentfill}{rgb}{0.000000,0.000000,0.000000}%
\pgfsetfillcolor{currentfill}%
\pgfsetlinewidth{0.501875pt}%
\definecolor{currentstroke}{rgb}{0.000000,0.000000,0.000000}%
\pgfsetstrokecolor{currentstroke}%
\pgfsetdash{}{0pt}%
\pgfsys@defobject{currentmarker}{\pgfqpoint{0.000000in}{0.000000in}}{\pgfqpoint{0.055556in}{0.000000in}}{%
\pgfpathmoveto{\pgfqpoint{0.000000in}{0.000000in}}%
\pgfpathlineto{\pgfqpoint{0.055556in}{0.000000in}}%
\pgfusepath{stroke,fill}%
}%
\begin{pgfscope}%
\pgfsys@transformshift{0.225000in}{3.010000in}%
\pgfsys@useobject{currentmarker}{}%
\end{pgfscope}%
\end{pgfscope}%
\begin{pgfscope}%
\pgfsetbuttcap%
\pgfsetroundjoin%
\definecolor{currentfill}{rgb}{0.000000,0.000000,0.000000}%
\pgfsetfillcolor{currentfill}%
\pgfsetlinewidth{0.501875pt}%
\definecolor{currentstroke}{rgb}{0.000000,0.000000,0.000000}%
\pgfsetstrokecolor{currentstroke}%
\pgfsetdash{}{0pt}%
\pgfsys@defobject{currentmarker}{\pgfqpoint{-0.055556in}{0.000000in}}{\pgfqpoint{0.000000in}{0.000000in}}{%
\pgfpathmoveto{\pgfqpoint{0.000000in}{0.000000in}}%
\pgfpathlineto{\pgfqpoint{-0.055556in}{0.000000in}}%
\pgfusepath{stroke,fill}%
}%
\begin{pgfscope}%
\pgfsys@transformshift{4.275000in}{3.010000in}%
\pgfsys@useobject{currentmarker}{}%
\end{pgfscope}%
\end{pgfscope}%
\begin{pgfscope}%
\pgftext[x=0.169444in,y=3.010000in,right,]{\rmfamily\fontsize{9.000000}{10.800000}\selectfont \(\displaystyle 0.8\)}%
\end{pgfscope}%
\begin{pgfscope}%
\pgfsetbuttcap%
\pgfsetroundjoin%
\definecolor{currentfill}{rgb}{0.000000,0.000000,0.000000}%
\pgfsetfillcolor{currentfill}%
\pgfsetlinewidth{0.501875pt}%
\definecolor{currentstroke}{rgb}{0.000000,0.000000,0.000000}%
\pgfsetstrokecolor{currentstroke}%
\pgfsetdash{}{0pt}%
\pgfsys@defobject{currentmarker}{\pgfqpoint{0.000000in}{0.000000in}}{\pgfqpoint{0.055556in}{0.000000in}}{%
\pgfpathmoveto{\pgfqpoint{0.000000in}{0.000000in}}%
\pgfpathlineto{\pgfqpoint{0.055556in}{0.000000in}}%
\pgfusepath{stroke,fill}%
}%
\begin{pgfscope}%
\pgfsys@transformshift{0.225000in}{3.220000in}%
\pgfsys@useobject{currentmarker}{}%
\end{pgfscope}%
\end{pgfscope}%
\begin{pgfscope}%
\pgfsetbuttcap%
\pgfsetroundjoin%
\definecolor{currentfill}{rgb}{0.000000,0.000000,0.000000}%
\pgfsetfillcolor{currentfill}%
\pgfsetlinewidth{0.501875pt}%
\definecolor{currentstroke}{rgb}{0.000000,0.000000,0.000000}%
\pgfsetstrokecolor{currentstroke}%
\pgfsetdash{}{0pt}%
\pgfsys@defobject{currentmarker}{\pgfqpoint{-0.055556in}{0.000000in}}{\pgfqpoint{0.000000in}{0.000000in}}{%
\pgfpathmoveto{\pgfqpoint{0.000000in}{0.000000in}}%
\pgfpathlineto{\pgfqpoint{-0.055556in}{0.000000in}}%
\pgfusepath{stroke,fill}%
}%
\begin{pgfscope}%
\pgfsys@transformshift{4.275000in}{3.220000in}%
\pgfsys@useobject{currentmarker}{}%
\end{pgfscope}%
\end{pgfscope}%
\begin{pgfscope}%
\pgftext[x=0.169444in,y=3.220000in,right,]{\rmfamily\fontsize{9.000000}{10.800000}\selectfont \(\displaystyle 1.0\)}%
\end{pgfscope}%
\begin{pgfscope}%
\pgftext[x=2.250000in,y=3.394444in,,base]{\rmfamily\fontsize{11.000000}{13.200000}\selectfont Daten}%
\end{pgfscope}%
\begin{pgfscope}%
\pgfsetbuttcap%
\pgfsetmiterjoin%
\pgfsetlinewidth{0.000000pt}%
\definecolor{currentstroke}{rgb}{0.000000,0.000000,0.000000}%
\pgfsetstrokecolor{currentstroke}%
\pgfsetstrokeopacity{0.000000}%
\pgfsetdash{}{0pt}%
\pgfpathmoveto{\pgfqpoint{0.225000in}{0.175000in}}%
\pgfpathlineto{\pgfqpoint{4.275000in}{0.175000in}}%
\pgfpathlineto{\pgfqpoint{4.275000in}{1.435000in}}%
\pgfpathlineto{\pgfqpoint{0.225000in}{1.435000in}}%
\pgfpathclose%
\pgfusepath{}%
\end{pgfscope}%
\begin{pgfscope}%
\pgfpathrectangle{\pgfqpoint{0.225000in}{0.175000in}}{\pgfqpoint{4.050000in}{1.260000in}} %
\pgfusepath{clip}%
\pgfsetrectcap%
\pgfsetroundjoin%
\pgfsetlinewidth{1.003750pt}%
\definecolor{currentstroke}{rgb}{0.000000,0.000000,1.000000}%
\pgfsetstrokecolor{currentstroke}%
\pgfsetdash{}{0pt}%
\pgfpathmoveto{\pgfqpoint{0.225000in}{0.232273in}}%
\pgfpathlineto{\pgfqpoint{0.229824in}{0.233405in}}%
\pgfpathlineto{\pgfqpoint{0.234649in}{0.236798in}}%
\pgfpathlineto{\pgfqpoint{0.240438in}{0.243833in}}%
\pgfpathlineto{\pgfqpoint{0.247192in}{0.256076in}}%
\pgfpathlineto{\pgfqpoint{0.254911in}{0.275268in}}%
\pgfpathlineto{\pgfqpoint{0.263595in}{0.303254in}}%
\pgfpathlineto{\pgfqpoint{0.273243in}{0.341866in}}%
\pgfpathlineto{\pgfqpoint{0.284821in}{0.397798in}}%
\pgfpathlineto{\pgfqpoint{0.298330in}{0.474655in}}%
\pgfpathlineto{\pgfqpoint{0.314732in}{0.581466in}}%
\pgfpathlineto{\pgfqpoint{0.337889in}{0.748359in}}%
\pgfpathlineto{\pgfqpoint{0.381308in}{1.062844in}}%
\pgfpathlineto{\pgfqpoint{0.398675in}{1.171068in}}%
\pgfpathlineto{\pgfqpoint{0.412183in}{1.242575in}}%
\pgfpathlineto{\pgfqpoint{0.423762in}{1.293166in}}%
\pgfpathlineto{\pgfqpoint{0.434375in}{1.329812in}}%
\pgfpathlineto{\pgfqpoint{0.443059in}{1.352358in}}%
\pgfpathlineto{\pgfqpoint{0.450778in}{1.366524in}}%
\pgfpathlineto{\pgfqpoint{0.457532in}{1.374262in}}%
\pgfpathlineto{\pgfqpoint{0.463321in}{1.377385in}}%
\pgfpathlineto{\pgfqpoint{0.468145in}{1.377498in}}%
\pgfpathlineto{\pgfqpoint{0.472970in}{1.375347in}}%
\pgfpathlineto{\pgfqpoint{0.478759in}{1.369791in}}%
\pgfpathlineto{\pgfqpoint{0.484548in}{1.361018in}}%
\pgfpathlineto{\pgfqpoint{0.491302in}{1.346788in}}%
\pgfpathlineto{\pgfqpoint{0.499021in}{1.325391in}}%
\pgfpathlineto{\pgfqpoint{0.508670in}{1.291272in}}%
\pgfpathlineto{\pgfqpoint{0.519283in}{1.244890in}}%
\pgfpathlineto{\pgfqpoint{0.531826in}{1.179314in}}%
\pgfpathlineto{\pgfqpoint{0.546299in}{1.091364in}}%
\pgfpathlineto{\pgfqpoint{0.565597in}{0.958879in}}%
\pgfpathlineto{\pgfqpoint{0.635067in}{0.465887in}}%
\pgfpathlineto{\pgfqpoint{0.649540in}{0.385336in}}%
\pgfpathlineto{\pgfqpoint{0.662083in}{0.327485in}}%
\pgfpathlineto{\pgfqpoint{0.672696in}{0.288434in}}%
\pgfpathlineto{\pgfqpoint{0.681380in}{0.263799in}}%
\pgfpathlineto{\pgfqpoint{0.689099in}{0.247711in}}%
\pgfpathlineto{\pgfqpoint{0.695853in}{0.238255in}}%
\pgfpathlineto{\pgfqpoint{0.701642in}{0.233643in}}%
\pgfpathlineto{\pgfqpoint{0.706466in}{0.232284in}}%
\pgfpathlineto{\pgfqpoint{0.711291in}{0.233190in}}%
\pgfpathlineto{\pgfqpoint{0.716115in}{0.236357in}}%
\pgfpathlineto{\pgfqpoint{0.721904in}{0.243124in}}%
\pgfpathlineto{\pgfqpoint{0.728658in}{0.255059in}}%
\pgfpathlineto{\pgfqpoint{0.736377in}{0.273909in}}%
\pgfpathlineto{\pgfqpoint{0.745061in}{0.301527in}}%
\pgfpathlineto{\pgfqpoint{0.754710in}{0.339756in}}%
\pgfpathlineto{\pgfqpoint{0.766288in}{0.395273in}}%
\pgfpathlineto{\pgfqpoint{0.779796in}{0.471719in}}%
\pgfpathlineto{\pgfqpoint{0.796199in}{0.578154in}}%
\pgfpathlineto{\pgfqpoint{0.819355in}{0.744775in}}%
\pgfpathlineto{\pgfqpoint{0.862774in}{1.059622in}}%
\pgfpathlineto{\pgfqpoint{0.880142in}{1.168291in}}%
\pgfpathlineto{\pgfqpoint{0.893650in}{1.240242in}}%
\pgfpathlineto{\pgfqpoint{0.905228in}{1.291272in}}%
\pgfpathlineto{\pgfqpoint{0.915842in}{1.328359in}}%
\pgfpathlineto{\pgfqpoint{0.924525in}{1.351287in}}%
\pgfpathlineto{\pgfqpoint{0.932244in}{1.365804in}}%
\pgfpathlineto{\pgfqpoint{0.938998in}{1.373855in}}%
\pgfpathlineto{\pgfqpoint{0.944788in}{1.377249in}}%
\pgfpathlineto{\pgfqpoint{0.949612in}{1.377589in}}%
\pgfpathlineto{\pgfqpoint{0.954436in}{1.375664in}}%
\pgfpathlineto{\pgfqpoint{0.959260in}{1.371483in}}%
\pgfpathlineto{\pgfqpoint{0.965050in}{1.363511in}}%
\pgfpathlineto{\pgfqpoint{0.971804in}{1.350195in}}%
\pgfpathlineto{\pgfqpoint{0.979523in}{1.329812in}}%
\pgfpathlineto{\pgfqpoint{0.988206in}{1.300545in}}%
\pgfpathlineto{\pgfqpoint{0.998820in}{1.256204in}}%
\pgfpathlineto{\pgfqpoint{1.010398in}{1.198029in}}%
\pgfpathlineto{\pgfqpoint{1.024871in}{1.112916in}}%
\pgfpathlineto{\pgfqpoint{1.042239in}{0.996647in}}%
\pgfpathlineto{\pgfqpoint{1.070220in}{0.791493in}}%
\pgfpathlineto{\pgfqpoint{1.102060in}{0.561731in}}%
\pgfpathlineto{\pgfqpoint{1.119428in}{0.451543in}}%
\pgfpathlineto{\pgfqpoint{1.133901in}{0.373289in}}%
\pgfpathlineto{\pgfqpoint{1.145479in}{0.321604in}}%
\pgfpathlineto{\pgfqpoint{1.156092in}{0.283859in}}%
\pgfpathlineto{\pgfqpoint{1.164776in}{0.260360in}}%
\pgfpathlineto{\pgfqpoint{1.172495in}{0.245318in}}%
\pgfpathlineto{\pgfqpoint{1.179249in}{0.236798in}}%
\pgfpathlineto{\pgfqpoint{1.185038in}{0.232998in}}%
\pgfpathlineto{\pgfqpoint{1.188898in}{0.232273in}}%
\pgfpathlineto{\pgfqpoint{1.188898in}{0.232273in}}%
\pgfusepath{stroke}%
\end{pgfscope}%
\begin{pgfscope}%
\pgfpathrectangle{\pgfqpoint{0.225000in}{0.175000in}}{\pgfqpoint{4.050000in}{1.260000in}} %
\pgfusepath{clip}%
\pgfsetrectcap%
\pgfsetroundjoin%
\pgfsetlinewidth{1.003750pt}%
\definecolor{currentstroke}{rgb}{1.000000,0.000000,1.000000}%
\pgfsetstrokecolor{currentstroke}%
\pgfsetdash{}{0pt}%
\pgfpathmoveto{\pgfqpoint{1.188898in}{0.232273in}}%
\pgfpathlineto{\pgfqpoint{1.190827in}{0.233903in}}%
\pgfpathlineto{\pgfqpoint{1.193722in}{0.242438in}}%
\pgfpathlineto{\pgfqpoint{1.197582in}{0.264988in}}%
\pgfpathlineto{\pgfqpoint{1.202406in}{0.310360in}}%
\pgfpathlineto{\pgfqpoint{1.208195in}{0.387796in}}%
\pgfpathlineto{\pgfqpoint{1.215914in}{0.523328in}}%
\pgfpathlineto{\pgfqpoint{1.228457in}{0.791493in}}%
\pgfpathlineto{\pgfqpoint{1.244860in}{1.136814in}}%
\pgfpathlineto{\pgfqpoint{1.253544in}{1.273375in}}%
\pgfpathlineto{\pgfqpoint{1.259333in}{1.335414in}}%
\pgfpathlineto{\pgfqpoint{1.264157in}{1.366524in}}%
\pgfpathlineto{\pgfqpoint{1.267052in}{1.375664in}}%
\pgfpathlineto{\pgfqpoint{1.268981in}{1.377702in}}%
\pgfpathlineto{\pgfqpoint{1.270911in}{1.376479in}}%
\pgfpathlineto{\pgfqpoint{1.272841in}{1.372002in}}%
\pgfpathlineto{\pgfqpoint{1.275735in}{1.359247in}}%
\pgfpathlineto{\pgfqpoint{1.279595in}{1.331243in}}%
\pgfpathlineto{\pgfqpoint{1.284419in}{1.279510in}}%
\pgfpathlineto{\pgfqpoint{1.291173in}{1.179314in}}%
\pgfpathlineto{\pgfqpoint{1.299857in}{1.013522in}}%
\pgfpathlineto{\pgfqpoint{1.331697in}{0.366265in}}%
\pgfpathlineto{\pgfqpoint{1.338451in}{0.285363in}}%
\pgfpathlineto{\pgfqpoint{1.343276in}{0.249416in}}%
\pgfpathlineto{\pgfqpoint{1.347135in}{0.234820in}}%
\pgfpathlineto{\pgfqpoint{1.349065in}{0.232375in}}%
\pgfpathlineto{\pgfqpoint{1.350995in}{0.233190in}}%
\pgfpathlineto{\pgfqpoint{1.352924in}{0.237261in}}%
\pgfpathlineto{\pgfqpoint{1.355819in}{0.249416in}}%
\pgfpathlineto{\pgfqpoint{1.359678in}{0.276648in}}%
\pgfpathlineto{\pgfqpoint{1.364503in}{0.327485in}}%
\pgfpathlineto{\pgfqpoint{1.371257in}{0.426614in}}%
\pgfpathlineto{\pgfqpoint{1.379941in}{0.591455in}}%
\pgfpathlineto{\pgfqpoint{1.411781in}{1.240242in}}%
\pgfpathlineto{\pgfqpoint{1.418535in}{1.322342in}}%
\pgfpathlineto{\pgfqpoint{1.423359in}{1.359247in}}%
\pgfpathlineto{\pgfqpoint{1.427219in}{1.374646in}}%
\pgfpathlineto{\pgfqpoint{1.429149in}{1.377498in}}%
\pgfpathlineto{\pgfqpoint{1.431078in}{1.377090in}}%
\pgfpathlineto{\pgfqpoint{1.433008in}{1.373425in}}%
\pgfpathlineto{\pgfqpoint{1.435903in}{1.361871in}}%
\pgfpathlineto{\pgfqpoint{1.439762in}{1.335414in}}%
\pgfpathlineto{\pgfqpoint{1.444586in}{1.285477in}}%
\pgfpathlineto{\pgfqpoint{1.451340in}{1.187426in}}%
\pgfpathlineto{\pgfqpoint{1.460024in}{1.023549in}}%
\pgfpathlineto{\pgfqpoint{1.492829in}{0.359398in}}%
\pgfpathlineto{\pgfqpoint{1.499584in}{0.280912in}}%
\pgfpathlineto{\pgfqpoint{1.504408in}{0.246891in}}%
\pgfpathlineto{\pgfqpoint{1.508267in}{0.233903in}}%
\pgfpathlineto{\pgfqpoint{1.510197in}{0.232273in}}%
\pgfpathlineto{\pgfqpoint{1.512127in}{0.233903in}}%
\pgfpathlineto{\pgfqpoint{1.515021in}{0.242438in}}%
\pgfpathlineto{\pgfqpoint{1.518881in}{0.264988in}}%
\pgfpathlineto{\pgfqpoint{1.523705in}{0.310360in}}%
\pgfpathlineto{\pgfqpoint{1.529494in}{0.387796in}}%
\pgfpathlineto{\pgfqpoint{1.537213in}{0.523328in}}%
\pgfpathlineto{\pgfqpoint{1.549756in}{0.791493in}}%
\pgfpathlineto{\pgfqpoint{1.566159in}{1.136814in}}%
\pgfpathlineto{\pgfqpoint{1.574843in}{1.273375in}}%
\pgfpathlineto{\pgfqpoint{1.580632in}{1.335414in}}%
\pgfpathlineto{\pgfqpoint{1.585456in}{1.366524in}}%
\pgfpathlineto{\pgfqpoint{1.588351in}{1.375664in}}%
\pgfpathlineto{\pgfqpoint{1.590281in}{1.377702in}}%
\pgfpathlineto{\pgfqpoint{1.592210in}{1.376479in}}%
\pgfpathlineto{\pgfqpoint{1.594140in}{1.372002in}}%
\pgfpathlineto{\pgfqpoint{1.597035in}{1.359247in}}%
\pgfpathlineto{\pgfqpoint{1.600894in}{1.331243in}}%
\pgfpathlineto{\pgfqpoint{1.605718in}{1.279510in}}%
\pgfpathlineto{\pgfqpoint{1.612472in}{1.179314in}}%
\pgfpathlineto{\pgfqpoint{1.621156in}{1.013522in}}%
\pgfpathlineto{\pgfqpoint{1.652997in}{0.366265in}}%
\pgfpathlineto{\pgfqpoint{1.659751in}{0.285363in}}%
\pgfpathlineto{\pgfqpoint{1.664575in}{0.249416in}}%
\pgfpathlineto{\pgfqpoint{1.668434in}{0.234820in}}%
\pgfpathlineto{\pgfqpoint{1.670364in}{0.232375in}}%
\pgfpathlineto{\pgfqpoint{1.672294in}{0.233190in}}%
\pgfpathlineto{\pgfqpoint{1.674224in}{0.237261in}}%
\pgfpathlineto{\pgfqpoint{1.677118in}{0.249416in}}%
\pgfpathlineto{\pgfqpoint{1.680978in}{0.276648in}}%
\pgfpathlineto{\pgfqpoint{1.685802in}{0.327485in}}%
\pgfpathlineto{\pgfqpoint{1.692556in}{0.426614in}}%
\pgfpathlineto{\pgfqpoint{1.701240in}{0.591455in}}%
\pgfpathlineto{\pgfqpoint{1.733080in}{1.240242in}}%
\pgfpathlineto{\pgfqpoint{1.739834in}{1.322342in}}%
\pgfpathlineto{\pgfqpoint{1.744659in}{1.359247in}}%
\pgfpathlineto{\pgfqpoint{1.748518in}{1.374646in}}%
\pgfpathlineto{\pgfqpoint{1.750448in}{1.377498in}}%
\pgfpathlineto{\pgfqpoint{1.752378in}{1.377090in}}%
\pgfpathlineto{\pgfqpoint{1.754307in}{1.373425in}}%
\pgfpathlineto{\pgfqpoint{1.757202in}{1.361871in}}%
\pgfpathlineto{\pgfqpoint{1.761061in}{1.335414in}}%
\pgfpathlineto{\pgfqpoint{1.765886in}{1.285477in}}%
\pgfpathlineto{\pgfqpoint{1.772640in}{1.187426in}}%
\pgfpathlineto{\pgfqpoint{1.781323in}{1.023549in}}%
\pgfpathlineto{\pgfqpoint{1.814129in}{0.359398in}}%
\pgfpathlineto{\pgfqpoint{1.820883in}{0.280912in}}%
\pgfpathlineto{\pgfqpoint{1.825707in}{0.246891in}}%
\pgfpathlineto{\pgfqpoint{1.829567in}{0.233903in}}%
\pgfpathlineto{\pgfqpoint{1.831496in}{0.232273in}}%
\pgfpathlineto{\pgfqpoint{1.833426in}{0.233903in}}%
\pgfpathlineto{\pgfqpoint{1.836321in}{0.242438in}}%
\pgfpathlineto{\pgfqpoint{1.840180in}{0.264988in}}%
\pgfpathlineto{\pgfqpoint{1.845004in}{0.310360in}}%
\pgfpathlineto{\pgfqpoint{1.850793in}{0.387796in}}%
\pgfpathlineto{\pgfqpoint{1.858512in}{0.523328in}}%
\pgfpathlineto{\pgfqpoint{1.871056in}{0.791493in}}%
\pgfpathlineto{\pgfqpoint{1.887458in}{1.136814in}}%
\pgfpathlineto{\pgfqpoint{1.896142in}{1.273375in}}%
\pgfpathlineto{\pgfqpoint{1.901931in}{1.335414in}}%
\pgfpathlineto{\pgfqpoint{1.906756in}{1.366524in}}%
\pgfpathlineto{\pgfqpoint{1.909650in}{1.375664in}}%
\pgfpathlineto{\pgfqpoint{1.911580in}{1.377702in}}%
\pgfpathlineto{\pgfqpoint{1.913510in}{1.376479in}}%
\pgfpathlineto{\pgfqpoint{1.915439in}{1.372002in}}%
\pgfpathlineto{\pgfqpoint{1.918334in}{1.359247in}}%
\pgfpathlineto{\pgfqpoint{1.922193in}{1.331243in}}%
\pgfpathlineto{\pgfqpoint{1.927018in}{1.279510in}}%
\pgfpathlineto{\pgfqpoint{1.933772in}{1.179314in}}%
\pgfpathlineto{\pgfqpoint{1.942455in}{1.013522in}}%
\pgfpathlineto{\pgfqpoint{1.974296in}{0.366265in}}%
\pgfpathlineto{\pgfqpoint{1.981050in}{0.285363in}}%
\pgfpathlineto{\pgfqpoint{1.985874in}{0.249416in}}%
\pgfpathlineto{\pgfqpoint{1.989734in}{0.234820in}}%
\pgfpathlineto{\pgfqpoint{1.991663in}{0.232375in}}%
\pgfpathlineto{\pgfqpoint{1.993593in}{0.233190in}}%
\pgfpathlineto{\pgfqpoint{1.995523in}{0.237261in}}%
\pgfpathlineto{\pgfqpoint{1.998417in}{0.249416in}}%
\pgfpathlineto{\pgfqpoint{2.002277in}{0.276648in}}%
\pgfpathlineto{\pgfqpoint{2.007101in}{0.327485in}}%
\pgfpathlineto{\pgfqpoint{2.013855in}{0.426614in}}%
\pgfpathlineto{\pgfqpoint{2.022539in}{0.591455in}}%
\pgfpathlineto{\pgfqpoint{2.054380in}{1.240242in}}%
\pgfpathlineto{\pgfqpoint{2.061134in}{1.322342in}}%
\pgfpathlineto{\pgfqpoint{2.065958in}{1.359247in}}%
\pgfpathlineto{\pgfqpoint{2.069817in}{1.374646in}}%
\pgfpathlineto{\pgfqpoint{2.071747in}{1.377498in}}%
\pgfpathlineto{\pgfqpoint{2.073677in}{1.377090in}}%
\pgfpathlineto{\pgfqpoint{2.075606in}{1.373425in}}%
\pgfpathlineto{\pgfqpoint{2.078501in}{1.361871in}}%
\pgfpathlineto{\pgfqpoint{2.082361in}{1.335414in}}%
\pgfpathlineto{\pgfqpoint{2.087185in}{1.285477in}}%
\pgfpathlineto{\pgfqpoint{2.093939in}{1.187426in}}%
\pgfpathlineto{\pgfqpoint{2.102623in}{1.023549in}}%
\pgfpathlineto{\pgfqpoint{2.135428in}{0.359398in}}%
\pgfpathlineto{\pgfqpoint{2.142182in}{0.280912in}}%
\pgfpathlineto{\pgfqpoint{2.147006in}{0.246891in}}%
\pgfpathlineto{\pgfqpoint{2.150866in}{0.233903in}}%
\pgfpathlineto{\pgfqpoint{2.152795in}{0.232273in}}%
\pgfpathlineto{\pgfqpoint{2.152795in}{0.232273in}}%
\pgfusepath{stroke}%
\end{pgfscope}%
\begin{pgfscope}%
\pgfpathrectangle{\pgfqpoint{0.225000in}{0.175000in}}{\pgfqpoint{4.050000in}{1.260000in}} %
\pgfusepath{clip}%
\pgfsetrectcap%
\pgfsetroundjoin%
\pgfsetlinewidth{1.003750pt}%
\definecolor{currentstroke}{rgb}{0.000000,0.000000,1.000000}%
\pgfsetstrokecolor{currentstroke}%
\pgfsetdash{}{0pt}%
\pgfpathmoveto{\pgfqpoint{2.152795in}{0.232273in}}%
\pgfpathlineto{\pgfqpoint{2.157620in}{0.233405in}}%
\pgfpathlineto{\pgfqpoint{2.162444in}{0.236798in}}%
\pgfpathlineto{\pgfqpoint{2.168233in}{0.243833in}}%
\pgfpathlineto{\pgfqpoint{2.174987in}{0.256076in}}%
\pgfpathlineto{\pgfqpoint{2.182706in}{0.275268in}}%
\pgfpathlineto{\pgfqpoint{2.191390in}{0.303254in}}%
\pgfpathlineto{\pgfqpoint{2.201039in}{0.341866in}}%
\pgfpathlineto{\pgfqpoint{2.212617in}{0.397798in}}%
\pgfpathlineto{\pgfqpoint{2.226125in}{0.474655in}}%
\pgfpathlineto{\pgfqpoint{2.242528in}{0.581466in}}%
\pgfpathlineto{\pgfqpoint{2.265684in}{0.748359in}}%
\pgfpathlineto{\pgfqpoint{2.309103in}{1.062844in}}%
\pgfpathlineto{\pgfqpoint{2.326471in}{1.171068in}}%
\pgfpathlineto{\pgfqpoint{2.339979in}{1.242575in}}%
\pgfpathlineto{\pgfqpoint{2.351557in}{1.293166in}}%
\pgfpathlineto{\pgfqpoint{2.362171in}{1.329812in}}%
\pgfpathlineto{\pgfqpoint{2.370854in}{1.352358in}}%
\pgfpathlineto{\pgfqpoint{2.378573in}{1.366524in}}%
\pgfpathlineto{\pgfqpoint{2.385327in}{1.374262in}}%
\pgfpathlineto{\pgfqpoint{2.391117in}{1.377385in}}%
\pgfpathlineto{\pgfqpoint{2.395941in}{1.377498in}}%
\pgfpathlineto{\pgfqpoint{2.400765in}{1.375347in}}%
\pgfpathlineto{\pgfqpoint{2.406554in}{1.369791in}}%
\pgfpathlineto{\pgfqpoint{2.412344in}{1.361018in}}%
\pgfpathlineto{\pgfqpoint{2.419098in}{1.346788in}}%
\pgfpathlineto{\pgfqpoint{2.426816in}{1.325391in}}%
\pgfpathlineto{\pgfqpoint{2.436465in}{1.291272in}}%
\pgfpathlineto{\pgfqpoint{2.447079in}{1.244890in}}%
\pgfpathlineto{\pgfqpoint{2.459622in}{1.179314in}}%
\pgfpathlineto{\pgfqpoint{2.474095in}{1.091364in}}%
\pgfpathlineto{\pgfqpoint{2.493392in}{0.958879in}}%
\pgfpathlineto{\pgfqpoint{2.562862in}{0.465887in}}%
\pgfpathlineto{\pgfqpoint{2.577335in}{0.385336in}}%
\pgfpathlineto{\pgfqpoint{2.589878in}{0.327485in}}%
\pgfpathlineto{\pgfqpoint{2.600492in}{0.288434in}}%
\pgfpathlineto{\pgfqpoint{2.609176in}{0.263799in}}%
\pgfpathlineto{\pgfqpoint{2.616894in}{0.247711in}}%
\pgfpathlineto{\pgfqpoint{2.623648in}{0.238255in}}%
\pgfpathlineto{\pgfqpoint{2.629438in}{0.233643in}}%
\pgfpathlineto{\pgfqpoint{2.634262in}{0.232284in}}%
\pgfpathlineto{\pgfqpoint{2.639086in}{0.233190in}}%
\pgfpathlineto{\pgfqpoint{2.643911in}{0.236357in}}%
\pgfpathlineto{\pgfqpoint{2.649700in}{0.243124in}}%
\pgfpathlineto{\pgfqpoint{2.656454in}{0.255059in}}%
\pgfpathlineto{\pgfqpoint{2.664173in}{0.273909in}}%
\pgfpathlineto{\pgfqpoint{2.672856in}{0.301527in}}%
\pgfpathlineto{\pgfqpoint{2.682505in}{0.339756in}}%
\pgfpathlineto{\pgfqpoint{2.694083in}{0.395273in}}%
\pgfpathlineto{\pgfqpoint{2.707591in}{0.471719in}}%
\pgfpathlineto{\pgfqpoint{2.723994in}{0.578154in}}%
\pgfpathlineto{\pgfqpoint{2.747151in}{0.744775in}}%
\pgfpathlineto{\pgfqpoint{2.790570in}{1.059622in}}%
\pgfpathlineto{\pgfqpoint{2.807937in}{1.168291in}}%
\pgfpathlineto{\pgfqpoint{2.821445in}{1.240242in}}%
\pgfpathlineto{\pgfqpoint{2.833024in}{1.291272in}}%
\pgfpathlineto{\pgfqpoint{2.843637in}{1.328359in}}%
\pgfpathlineto{\pgfqpoint{2.852321in}{1.351287in}}%
\pgfpathlineto{\pgfqpoint{2.860040in}{1.365804in}}%
\pgfpathlineto{\pgfqpoint{2.866794in}{1.373855in}}%
\pgfpathlineto{\pgfqpoint{2.872583in}{1.377249in}}%
\pgfpathlineto{\pgfqpoint{2.877407in}{1.377589in}}%
\pgfpathlineto{\pgfqpoint{2.882232in}{1.375664in}}%
\pgfpathlineto{\pgfqpoint{2.887056in}{1.371483in}}%
\pgfpathlineto{\pgfqpoint{2.892845in}{1.363511in}}%
\pgfpathlineto{\pgfqpoint{2.899599in}{1.350195in}}%
\pgfpathlineto{\pgfqpoint{2.907318in}{1.329812in}}%
\pgfpathlineto{\pgfqpoint{2.916002in}{1.300545in}}%
\pgfpathlineto{\pgfqpoint{2.926615in}{1.256204in}}%
\pgfpathlineto{\pgfqpoint{2.938194in}{1.198029in}}%
\pgfpathlineto{\pgfqpoint{2.952667in}{1.112916in}}%
\pgfpathlineto{\pgfqpoint{2.970034in}{0.996647in}}%
\pgfpathlineto{\pgfqpoint{2.998015in}{0.791493in}}%
\pgfpathlineto{\pgfqpoint{3.029856in}{0.561731in}}%
\pgfpathlineto{\pgfqpoint{3.047223in}{0.451543in}}%
\pgfpathlineto{\pgfqpoint{3.061696in}{0.373289in}}%
\pgfpathlineto{\pgfqpoint{3.073274in}{0.321604in}}%
\pgfpathlineto{\pgfqpoint{3.083888in}{0.283859in}}%
\pgfpathlineto{\pgfqpoint{3.092572in}{0.260360in}}%
\pgfpathlineto{\pgfqpoint{3.100291in}{0.245318in}}%
\pgfpathlineto{\pgfqpoint{3.107045in}{0.236798in}}%
\pgfpathlineto{\pgfqpoint{3.112834in}{0.232998in}}%
\pgfpathlineto{\pgfqpoint{3.116693in}{0.232273in}}%
\pgfpathlineto{\pgfqpoint{3.116693in}{0.232273in}}%
\pgfusepath{stroke}%
\end{pgfscope}%
\begin{pgfscope}%
\pgfpathrectangle{\pgfqpoint{0.225000in}{0.175000in}}{\pgfqpoint{4.050000in}{1.260000in}} %
\pgfusepath{clip}%
\pgfsetrectcap%
\pgfsetroundjoin%
\pgfsetlinewidth{1.003750pt}%
\definecolor{currentstroke}{rgb}{1.000000,0.000000,1.000000}%
\pgfsetstrokecolor{currentstroke}%
\pgfsetdash{}{0pt}%
\pgfpathmoveto{\pgfqpoint{3.116693in}{0.232273in}}%
\pgfpathlineto{\pgfqpoint{3.118623in}{0.233903in}}%
\pgfpathlineto{\pgfqpoint{3.121518in}{0.242438in}}%
\pgfpathlineto{\pgfqpoint{3.125377in}{0.264988in}}%
\pgfpathlineto{\pgfqpoint{3.130201in}{0.310360in}}%
\pgfpathlineto{\pgfqpoint{3.135990in}{0.387796in}}%
\pgfpathlineto{\pgfqpoint{3.143709in}{0.523328in}}%
\pgfpathlineto{\pgfqpoint{3.156253in}{0.791493in}}%
\pgfpathlineto{\pgfqpoint{3.172655in}{1.136814in}}%
\pgfpathlineto{\pgfqpoint{3.181339in}{1.273375in}}%
\pgfpathlineto{\pgfqpoint{3.187128in}{1.335414in}}%
\pgfpathlineto{\pgfqpoint{3.191953in}{1.366524in}}%
\pgfpathlineto{\pgfqpoint{3.194847in}{1.375664in}}%
\pgfpathlineto{\pgfqpoint{3.196777in}{1.377702in}}%
\pgfpathlineto{\pgfqpoint{3.198707in}{1.376479in}}%
\pgfpathlineto{\pgfqpoint{3.200636in}{1.372002in}}%
\pgfpathlineto{\pgfqpoint{3.203531in}{1.359247in}}%
\pgfpathlineto{\pgfqpoint{3.207390in}{1.331243in}}%
\pgfpathlineto{\pgfqpoint{3.212215in}{1.279510in}}%
\pgfpathlineto{\pgfqpoint{3.218969in}{1.179314in}}%
\pgfpathlineto{\pgfqpoint{3.227652in}{1.013522in}}%
\pgfpathlineto{\pgfqpoint{3.259493in}{0.366265in}}%
\pgfpathlineto{\pgfqpoint{3.266247in}{0.285363in}}%
\pgfpathlineto{\pgfqpoint{3.271071in}{0.249416in}}%
\pgfpathlineto{\pgfqpoint{3.274931in}{0.234820in}}%
\pgfpathlineto{\pgfqpoint{3.276860in}{0.232375in}}%
\pgfpathlineto{\pgfqpoint{3.278790in}{0.233190in}}%
\pgfpathlineto{\pgfqpoint{3.280720in}{0.237261in}}%
\pgfpathlineto{\pgfqpoint{3.283614in}{0.249416in}}%
\pgfpathlineto{\pgfqpoint{3.287474in}{0.276648in}}%
\pgfpathlineto{\pgfqpoint{3.292298in}{0.327485in}}%
\pgfpathlineto{\pgfqpoint{3.299052in}{0.426614in}}%
\pgfpathlineto{\pgfqpoint{3.307736in}{0.591455in}}%
\pgfpathlineto{\pgfqpoint{3.339577in}{1.240242in}}%
\pgfpathlineto{\pgfqpoint{3.346331in}{1.322342in}}%
\pgfpathlineto{\pgfqpoint{3.351155in}{1.359247in}}%
\pgfpathlineto{\pgfqpoint{3.355014in}{1.374646in}}%
\pgfpathlineto{\pgfqpoint{3.356944in}{1.377498in}}%
\pgfpathlineto{\pgfqpoint{3.358874in}{1.377090in}}%
\pgfpathlineto{\pgfqpoint{3.360803in}{1.373425in}}%
\pgfpathlineto{\pgfqpoint{3.363698in}{1.361871in}}%
\pgfpathlineto{\pgfqpoint{3.367558in}{1.335414in}}%
\pgfpathlineto{\pgfqpoint{3.372382in}{1.285477in}}%
\pgfpathlineto{\pgfqpoint{3.379136in}{1.187426in}}%
\pgfpathlineto{\pgfqpoint{3.387820in}{1.023549in}}%
\pgfpathlineto{\pgfqpoint{3.420625in}{0.359398in}}%
\pgfpathlineto{\pgfqpoint{3.427379in}{0.280912in}}%
\pgfpathlineto{\pgfqpoint{3.432203in}{0.246891in}}%
\pgfpathlineto{\pgfqpoint{3.436063in}{0.233903in}}%
\pgfpathlineto{\pgfqpoint{3.437992in}{0.232273in}}%
\pgfpathlineto{\pgfqpoint{3.439922in}{0.233903in}}%
\pgfpathlineto{\pgfqpoint{3.442817in}{0.242438in}}%
\pgfpathlineto{\pgfqpoint{3.446676in}{0.264988in}}%
\pgfpathlineto{\pgfqpoint{3.451501in}{0.310360in}}%
\pgfpathlineto{\pgfqpoint{3.457290in}{0.387796in}}%
\pgfpathlineto{\pgfqpoint{3.465009in}{0.523328in}}%
\pgfpathlineto{\pgfqpoint{3.477552in}{0.791493in}}%
\pgfpathlineto{\pgfqpoint{3.493955in}{1.136814in}}%
\pgfpathlineto{\pgfqpoint{3.502638in}{1.273375in}}%
\pgfpathlineto{\pgfqpoint{3.508427in}{1.335414in}}%
\pgfpathlineto{\pgfqpoint{3.513252in}{1.366524in}}%
\pgfpathlineto{\pgfqpoint{3.516146in}{1.375664in}}%
\pgfpathlineto{\pgfqpoint{3.518076in}{1.377702in}}%
\pgfpathlineto{\pgfqpoint{3.520006in}{1.376479in}}%
\pgfpathlineto{\pgfqpoint{3.521936in}{1.372002in}}%
\pgfpathlineto{\pgfqpoint{3.524830in}{1.359247in}}%
\pgfpathlineto{\pgfqpoint{3.528690in}{1.331243in}}%
\pgfpathlineto{\pgfqpoint{3.533514in}{1.279510in}}%
\pgfpathlineto{\pgfqpoint{3.540268in}{1.179314in}}%
\pgfpathlineto{\pgfqpoint{3.548952in}{1.013522in}}%
\pgfpathlineto{\pgfqpoint{3.580792in}{0.366265in}}%
\pgfpathlineto{\pgfqpoint{3.587546in}{0.285363in}}%
\pgfpathlineto{\pgfqpoint{3.592371in}{0.249416in}}%
\pgfpathlineto{\pgfqpoint{3.596230in}{0.234820in}}%
\pgfpathlineto{\pgfqpoint{3.598160in}{0.232375in}}%
\pgfpathlineto{\pgfqpoint{3.600089in}{0.233190in}}%
\pgfpathlineto{\pgfqpoint{3.602019in}{0.237261in}}%
\pgfpathlineto{\pgfqpoint{3.604914in}{0.249416in}}%
\pgfpathlineto{\pgfqpoint{3.608773in}{0.276648in}}%
\pgfpathlineto{\pgfqpoint{3.613597in}{0.327485in}}%
\pgfpathlineto{\pgfqpoint{3.620352in}{0.426614in}}%
\pgfpathlineto{\pgfqpoint{3.629035in}{0.591455in}}%
\pgfpathlineto{\pgfqpoint{3.660876in}{1.240242in}}%
\pgfpathlineto{\pgfqpoint{3.667630in}{1.322342in}}%
\pgfpathlineto{\pgfqpoint{3.672454in}{1.359247in}}%
\pgfpathlineto{\pgfqpoint{3.676314in}{1.374646in}}%
\pgfpathlineto{\pgfqpoint{3.678243in}{1.377498in}}%
\pgfpathlineto{\pgfqpoint{3.680173in}{1.377090in}}%
\pgfpathlineto{\pgfqpoint{3.682103in}{1.373425in}}%
\pgfpathlineto{\pgfqpoint{3.684997in}{1.361871in}}%
\pgfpathlineto{\pgfqpoint{3.688857in}{1.335414in}}%
\pgfpathlineto{\pgfqpoint{3.693681in}{1.285477in}}%
\pgfpathlineto{\pgfqpoint{3.700435in}{1.187426in}}%
\pgfpathlineto{\pgfqpoint{3.709119in}{1.023549in}}%
\pgfpathlineto{\pgfqpoint{3.741924in}{0.359398in}}%
\pgfpathlineto{\pgfqpoint{3.748678in}{0.280912in}}%
\pgfpathlineto{\pgfqpoint{3.753503in}{0.246891in}}%
\pgfpathlineto{\pgfqpoint{3.757362in}{0.233903in}}%
\pgfpathlineto{\pgfqpoint{3.759292in}{0.232273in}}%
\pgfpathlineto{\pgfqpoint{3.761221in}{0.233903in}}%
\pgfpathlineto{\pgfqpoint{3.764116in}{0.242438in}}%
\pgfpathlineto{\pgfqpoint{3.767975in}{0.264988in}}%
\pgfpathlineto{\pgfqpoint{3.772800in}{0.310360in}}%
\pgfpathlineto{\pgfqpoint{3.778589in}{0.387796in}}%
\pgfpathlineto{\pgfqpoint{3.786308in}{0.523328in}}%
\pgfpathlineto{\pgfqpoint{3.798851in}{0.791493in}}%
\pgfpathlineto{\pgfqpoint{3.815254in}{1.136814in}}%
\pgfpathlineto{\pgfqpoint{3.823938in}{1.273375in}}%
\pgfpathlineto{\pgfqpoint{3.829727in}{1.335414in}}%
\pgfpathlineto{\pgfqpoint{3.834551in}{1.366524in}}%
\pgfpathlineto{\pgfqpoint{3.837446in}{1.375664in}}%
\pgfpathlineto{\pgfqpoint{3.839375in}{1.377702in}}%
\pgfpathlineto{\pgfqpoint{3.841305in}{1.376479in}}%
\pgfpathlineto{\pgfqpoint{3.843235in}{1.372002in}}%
\pgfpathlineto{\pgfqpoint{3.846129in}{1.359247in}}%
\pgfpathlineto{\pgfqpoint{3.849989in}{1.331243in}}%
\pgfpathlineto{\pgfqpoint{3.854813in}{1.279510in}}%
\pgfpathlineto{\pgfqpoint{3.861567in}{1.179314in}}%
\pgfpathlineto{\pgfqpoint{3.870251in}{1.013522in}}%
\pgfpathlineto{\pgfqpoint{3.902091in}{0.366265in}}%
\pgfpathlineto{\pgfqpoint{3.908845in}{0.285363in}}%
\pgfpathlineto{\pgfqpoint{3.913670in}{0.249416in}}%
\pgfpathlineto{\pgfqpoint{3.917529in}{0.234820in}}%
\pgfpathlineto{\pgfqpoint{3.919459in}{0.232375in}}%
\pgfpathlineto{\pgfqpoint{3.921389in}{0.233190in}}%
\pgfpathlineto{\pgfqpoint{3.923318in}{0.237261in}}%
\pgfpathlineto{\pgfqpoint{3.926213in}{0.249416in}}%
\pgfpathlineto{\pgfqpoint{3.930072in}{0.276648in}}%
\pgfpathlineto{\pgfqpoint{3.934897in}{0.327485in}}%
\pgfpathlineto{\pgfqpoint{3.941651in}{0.426614in}}%
\pgfpathlineto{\pgfqpoint{3.950335in}{0.591455in}}%
\pgfpathlineto{\pgfqpoint{3.982175in}{1.240242in}}%
\pgfpathlineto{\pgfqpoint{3.988929in}{1.322342in}}%
\pgfpathlineto{\pgfqpoint{3.993753in}{1.359247in}}%
\pgfpathlineto{\pgfqpoint{3.997613in}{1.374646in}}%
\pgfpathlineto{\pgfqpoint{3.999543in}{1.377498in}}%
\pgfpathlineto{\pgfqpoint{4.001472in}{1.377090in}}%
\pgfpathlineto{\pgfqpoint{4.003402in}{1.373425in}}%
\pgfpathlineto{\pgfqpoint{4.006297in}{1.361871in}}%
\pgfpathlineto{\pgfqpoint{4.010156in}{1.335414in}}%
\pgfpathlineto{\pgfqpoint{4.014980in}{1.285477in}}%
\pgfpathlineto{\pgfqpoint{4.021734in}{1.187426in}}%
\pgfpathlineto{\pgfqpoint{4.030418in}{1.023549in}}%
\pgfpathlineto{\pgfqpoint{4.063223in}{0.359398in}}%
\pgfpathlineto{\pgfqpoint{4.069977in}{0.280912in}}%
\pgfpathlineto{\pgfqpoint{4.074802in}{0.246891in}}%
\pgfpathlineto{\pgfqpoint{4.078661in}{0.233903in}}%
\pgfpathlineto{\pgfqpoint{4.080591in}{0.232273in}}%
\pgfpathlineto{\pgfqpoint{4.080591in}{0.232273in}}%
\pgfusepath{stroke}%
\end{pgfscope}%
\begin{pgfscope}%
\pgfsetrectcap%
\pgfsetmiterjoin%
\pgfsetlinewidth{1.003750pt}%
\definecolor{currentstroke}{rgb}{0.000000,0.000000,0.000000}%
\pgfsetstrokecolor{currentstroke}%
\pgfsetdash{}{0pt}%
\pgfpathmoveto{\pgfqpoint{4.275000in}{0.175000in}}%
\pgfpathlineto{\pgfqpoint{4.275000in}{1.435000in}}%
\pgfusepath{stroke}%
\end{pgfscope}%
\begin{pgfscope}%
\pgfsetrectcap%
\pgfsetmiterjoin%
\pgfsetlinewidth{1.003750pt}%
\definecolor{currentstroke}{rgb}{0.000000,0.000000,0.000000}%
\pgfsetstrokecolor{currentstroke}%
\pgfsetdash{}{0pt}%
\pgfpathmoveto{\pgfqpoint{0.225000in}{0.175000in}}%
\pgfpathlineto{\pgfqpoint{4.275000in}{0.175000in}}%
\pgfusepath{stroke}%
\end{pgfscope}%
\begin{pgfscope}%
\pgfsetrectcap%
\pgfsetmiterjoin%
\pgfsetlinewidth{1.003750pt}%
\definecolor{currentstroke}{rgb}{0.000000,0.000000,0.000000}%
\pgfsetstrokecolor{currentstroke}%
\pgfsetdash{}{0pt}%
\pgfpathmoveto{\pgfqpoint{0.225000in}{1.435000in}}%
\pgfpathlineto{\pgfqpoint{4.275000in}{1.435000in}}%
\pgfusepath{stroke}%
\end{pgfscope}%
\begin{pgfscope}%
\pgfsetrectcap%
\pgfsetmiterjoin%
\pgfsetlinewidth{1.003750pt}%
\definecolor{currentstroke}{rgb}{0.000000,0.000000,0.000000}%
\pgfsetstrokecolor{currentstroke}%
\pgfsetdash{}{0pt}%
\pgfpathmoveto{\pgfqpoint{0.225000in}{0.175000in}}%
\pgfpathlineto{\pgfqpoint{0.225000in}{1.435000in}}%
\pgfusepath{stroke}%
\end{pgfscope}%
\begin{pgfscope}%
\pgfsetbuttcap%
\pgfsetroundjoin%
\definecolor{currentfill}{rgb}{0.000000,0.000000,0.000000}%
\pgfsetfillcolor{currentfill}%
\pgfsetlinewidth{0.501875pt}%
\definecolor{currentstroke}{rgb}{0.000000,0.000000,0.000000}%
\pgfsetstrokecolor{currentstroke}%
\pgfsetdash{}{0pt}%
\pgfsys@defobject{currentmarker}{\pgfqpoint{0.000000in}{0.000000in}}{\pgfqpoint{0.000000in}{0.055556in}}{%
\pgfpathmoveto{\pgfqpoint{0.000000in}{0.000000in}}%
\pgfpathlineto{\pgfqpoint{0.000000in}{0.055556in}}%
\pgfusepath{stroke,fill}%
}%
\begin{pgfscope}%
\pgfsys@transformshift{0.225000in}{0.175000in}%
\pgfsys@useobject{currentmarker}{}%
\end{pgfscope}%
\end{pgfscope}%
\begin{pgfscope}%
\pgfsetbuttcap%
\pgfsetroundjoin%
\definecolor{currentfill}{rgb}{0.000000,0.000000,0.000000}%
\pgfsetfillcolor{currentfill}%
\pgfsetlinewidth{0.501875pt}%
\definecolor{currentstroke}{rgb}{0.000000,0.000000,0.000000}%
\pgfsetstrokecolor{currentstroke}%
\pgfsetdash{}{0pt}%
\pgfsys@defobject{currentmarker}{\pgfqpoint{0.000000in}{-0.055556in}}{\pgfqpoint{0.000000in}{0.000000in}}{%
\pgfpathmoveto{\pgfqpoint{0.000000in}{0.000000in}}%
\pgfpathlineto{\pgfqpoint{0.000000in}{-0.055556in}}%
\pgfusepath{stroke,fill}%
}%
\begin{pgfscope}%
\pgfsys@transformshift{0.225000in}{1.435000in}%
\pgfsys@useobject{currentmarker}{}%
\end{pgfscope}%
\end{pgfscope}%
\begin{pgfscope}%
\pgftext[x=0.225000in,y=0.119444in,,top]{\rmfamily\fontsize{9.000000}{10.800000}\selectfont \(\displaystyle 0.0000\)}%
\end{pgfscope}%
\begin{pgfscope}%
\pgfsetbuttcap%
\pgfsetroundjoin%
\definecolor{currentfill}{rgb}{0.000000,0.000000,0.000000}%
\pgfsetfillcolor{currentfill}%
\pgfsetlinewidth{0.501875pt}%
\definecolor{currentstroke}{rgb}{0.000000,0.000000,0.000000}%
\pgfsetstrokecolor{currentstroke}%
\pgfsetdash{}{0pt}%
\pgfsys@defobject{currentmarker}{\pgfqpoint{0.000000in}{0.000000in}}{\pgfqpoint{0.000000in}{0.055556in}}{%
\pgfpathmoveto{\pgfqpoint{0.000000in}{0.000000in}}%
\pgfpathlineto{\pgfqpoint{0.000000in}{0.055556in}}%
\pgfusepath{stroke,fill}%
}%
\begin{pgfscope}%
\pgfsys@transformshift{0.731250in}{0.175000in}%
\pgfsys@useobject{currentmarker}{}%
\end{pgfscope}%
\end{pgfscope}%
\begin{pgfscope}%
\pgfsetbuttcap%
\pgfsetroundjoin%
\definecolor{currentfill}{rgb}{0.000000,0.000000,0.000000}%
\pgfsetfillcolor{currentfill}%
\pgfsetlinewidth{0.501875pt}%
\definecolor{currentstroke}{rgb}{0.000000,0.000000,0.000000}%
\pgfsetstrokecolor{currentstroke}%
\pgfsetdash{}{0pt}%
\pgfsys@defobject{currentmarker}{\pgfqpoint{0.000000in}{-0.055556in}}{\pgfqpoint{0.000000in}{0.000000in}}{%
\pgfpathmoveto{\pgfqpoint{0.000000in}{0.000000in}}%
\pgfpathlineto{\pgfqpoint{0.000000in}{-0.055556in}}%
\pgfusepath{stroke,fill}%
}%
\begin{pgfscope}%
\pgfsys@transformshift{0.731250in}{1.435000in}%
\pgfsys@useobject{currentmarker}{}%
\end{pgfscope}%
\end{pgfscope}%
\begin{pgfscope}%
\pgftext[x=0.731250in,y=0.119444in,,top]{\rmfamily\fontsize{9.000000}{10.800000}\selectfont \(\displaystyle 0.0002\)}%
\end{pgfscope}%
\begin{pgfscope}%
\pgfsetbuttcap%
\pgfsetroundjoin%
\definecolor{currentfill}{rgb}{0.000000,0.000000,0.000000}%
\pgfsetfillcolor{currentfill}%
\pgfsetlinewidth{0.501875pt}%
\definecolor{currentstroke}{rgb}{0.000000,0.000000,0.000000}%
\pgfsetstrokecolor{currentstroke}%
\pgfsetdash{}{0pt}%
\pgfsys@defobject{currentmarker}{\pgfqpoint{0.000000in}{0.000000in}}{\pgfqpoint{0.000000in}{0.055556in}}{%
\pgfpathmoveto{\pgfqpoint{0.000000in}{0.000000in}}%
\pgfpathlineto{\pgfqpoint{0.000000in}{0.055556in}}%
\pgfusepath{stroke,fill}%
}%
\begin{pgfscope}%
\pgfsys@transformshift{1.237500in}{0.175000in}%
\pgfsys@useobject{currentmarker}{}%
\end{pgfscope}%
\end{pgfscope}%
\begin{pgfscope}%
\pgfsetbuttcap%
\pgfsetroundjoin%
\definecolor{currentfill}{rgb}{0.000000,0.000000,0.000000}%
\pgfsetfillcolor{currentfill}%
\pgfsetlinewidth{0.501875pt}%
\definecolor{currentstroke}{rgb}{0.000000,0.000000,0.000000}%
\pgfsetstrokecolor{currentstroke}%
\pgfsetdash{}{0pt}%
\pgfsys@defobject{currentmarker}{\pgfqpoint{0.000000in}{-0.055556in}}{\pgfqpoint{0.000000in}{0.000000in}}{%
\pgfpathmoveto{\pgfqpoint{0.000000in}{0.000000in}}%
\pgfpathlineto{\pgfqpoint{0.000000in}{-0.055556in}}%
\pgfusepath{stroke,fill}%
}%
\begin{pgfscope}%
\pgfsys@transformshift{1.237500in}{1.435000in}%
\pgfsys@useobject{currentmarker}{}%
\end{pgfscope}%
\end{pgfscope}%
\begin{pgfscope}%
\pgftext[x=1.237500in,y=0.119444in,,top]{\rmfamily\fontsize{9.000000}{10.800000}\selectfont \(\displaystyle 0.0004\)}%
\end{pgfscope}%
\begin{pgfscope}%
\pgfsetbuttcap%
\pgfsetroundjoin%
\definecolor{currentfill}{rgb}{0.000000,0.000000,0.000000}%
\pgfsetfillcolor{currentfill}%
\pgfsetlinewidth{0.501875pt}%
\definecolor{currentstroke}{rgb}{0.000000,0.000000,0.000000}%
\pgfsetstrokecolor{currentstroke}%
\pgfsetdash{}{0pt}%
\pgfsys@defobject{currentmarker}{\pgfqpoint{0.000000in}{0.000000in}}{\pgfqpoint{0.000000in}{0.055556in}}{%
\pgfpathmoveto{\pgfqpoint{0.000000in}{0.000000in}}%
\pgfpathlineto{\pgfqpoint{0.000000in}{0.055556in}}%
\pgfusepath{stroke,fill}%
}%
\begin{pgfscope}%
\pgfsys@transformshift{1.743750in}{0.175000in}%
\pgfsys@useobject{currentmarker}{}%
\end{pgfscope}%
\end{pgfscope}%
\begin{pgfscope}%
\pgfsetbuttcap%
\pgfsetroundjoin%
\definecolor{currentfill}{rgb}{0.000000,0.000000,0.000000}%
\pgfsetfillcolor{currentfill}%
\pgfsetlinewidth{0.501875pt}%
\definecolor{currentstroke}{rgb}{0.000000,0.000000,0.000000}%
\pgfsetstrokecolor{currentstroke}%
\pgfsetdash{}{0pt}%
\pgfsys@defobject{currentmarker}{\pgfqpoint{0.000000in}{-0.055556in}}{\pgfqpoint{0.000000in}{0.000000in}}{%
\pgfpathmoveto{\pgfqpoint{0.000000in}{0.000000in}}%
\pgfpathlineto{\pgfqpoint{0.000000in}{-0.055556in}}%
\pgfusepath{stroke,fill}%
}%
\begin{pgfscope}%
\pgfsys@transformshift{1.743750in}{1.435000in}%
\pgfsys@useobject{currentmarker}{}%
\end{pgfscope}%
\end{pgfscope}%
\begin{pgfscope}%
\pgftext[x=1.743750in,y=0.119444in,,top]{\rmfamily\fontsize{9.000000}{10.800000}\selectfont \(\displaystyle 0.0006\)}%
\end{pgfscope}%
\begin{pgfscope}%
\pgfsetbuttcap%
\pgfsetroundjoin%
\definecolor{currentfill}{rgb}{0.000000,0.000000,0.000000}%
\pgfsetfillcolor{currentfill}%
\pgfsetlinewidth{0.501875pt}%
\definecolor{currentstroke}{rgb}{0.000000,0.000000,0.000000}%
\pgfsetstrokecolor{currentstroke}%
\pgfsetdash{}{0pt}%
\pgfsys@defobject{currentmarker}{\pgfqpoint{0.000000in}{0.000000in}}{\pgfqpoint{0.000000in}{0.055556in}}{%
\pgfpathmoveto{\pgfqpoint{0.000000in}{0.000000in}}%
\pgfpathlineto{\pgfqpoint{0.000000in}{0.055556in}}%
\pgfusepath{stroke,fill}%
}%
\begin{pgfscope}%
\pgfsys@transformshift{2.250000in}{0.175000in}%
\pgfsys@useobject{currentmarker}{}%
\end{pgfscope}%
\end{pgfscope}%
\begin{pgfscope}%
\pgfsetbuttcap%
\pgfsetroundjoin%
\definecolor{currentfill}{rgb}{0.000000,0.000000,0.000000}%
\pgfsetfillcolor{currentfill}%
\pgfsetlinewidth{0.501875pt}%
\definecolor{currentstroke}{rgb}{0.000000,0.000000,0.000000}%
\pgfsetstrokecolor{currentstroke}%
\pgfsetdash{}{0pt}%
\pgfsys@defobject{currentmarker}{\pgfqpoint{0.000000in}{-0.055556in}}{\pgfqpoint{0.000000in}{0.000000in}}{%
\pgfpathmoveto{\pgfqpoint{0.000000in}{0.000000in}}%
\pgfpathlineto{\pgfqpoint{0.000000in}{-0.055556in}}%
\pgfusepath{stroke,fill}%
}%
\begin{pgfscope}%
\pgfsys@transformshift{2.250000in}{1.435000in}%
\pgfsys@useobject{currentmarker}{}%
\end{pgfscope}%
\end{pgfscope}%
\begin{pgfscope}%
\pgftext[x=2.250000in,y=0.119444in,,top]{\rmfamily\fontsize{9.000000}{10.800000}\selectfont \(\displaystyle 0.0008\)}%
\end{pgfscope}%
\begin{pgfscope}%
\pgfsetbuttcap%
\pgfsetroundjoin%
\definecolor{currentfill}{rgb}{0.000000,0.000000,0.000000}%
\pgfsetfillcolor{currentfill}%
\pgfsetlinewidth{0.501875pt}%
\definecolor{currentstroke}{rgb}{0.000000,0.000000,0.000000}%
\pgfsetstrokecolor{currentstroke}%
\pgfsetdash{}{0pt}%
\pgfsys@defobject{currentmarker}{\pgfqpoint{0.000000in}{0.000000in}}{\pgfqpoint{0.000000in}{0.055556in}}{%
\pgfpathmoveto{\pgfqpoint{0.000000in}{0.000000in}}%
\pgfpathlineto{\pgfqpoint{0.000000in}{0.055556in}}%
\pgfusepath{stroke,fill}%
}%
\begin{pgfscope}%
\pgfsys@transformshift{2.756250in}{0.175000in}%
\pgfsys@useobject{currentmarker}{}%
\end{pgfscope}%
\end{pgfscope}%
\begin{pgfscope}%
\pgfsetbuttcap%
\pgfsetroundjoin%
\definecolor{currentfill}{rgb}{0.000000,0.000000,0.000000}%
\pgfsetfillcolor{currentfill}%
\pgfsetlinewidth{0.501875pt}%
\definecolor{currentstroke}{rgb}{0.000000,0.000000,0.000000}%
\pgfsetstrokecolor{currentstroke}%
\pgfsetdash{}{0pt}%
\pgfsys@defobject{currentmarker}{\pgfqpoint{0.000000in}{-0.055556in}}{\pgfqpoint{0.000000in}{0.000000in}}{%
\pgfpathmoveto{\pgfqpoint{0.000000in}{0.000000in}}%
\pgfpathlineto{\pgfqpoint{0.000000in}{-0.055556in}}%
\pgfusepath{stroke,fill}%
}%
\begin{pgfscope}%
\pgfsys@transformshift{2.756250in}{1.435000in}%
\pgfsys@useobject{currentmarker}{}%
\end{pgfscope}%
\end{pgfscope}%
\begin{pgfscope}%
\pgftext[x=2.756250in,y=0.119444in,,top]{\rmfamily\fontsize{9.000000}{10.800000}\selectfont \(\displaystyle 0.0010\)}%
\end{pgfscope}%
\begin{pgfscope}%
\pgfsetbuttcap%
\pgfsetroundjoin%
\definecolor{currentfill}{rgb}{0.000000,0.000000,0.000000}%
\pgfsetfillcolor{currentfill}%
\pgfsetlinewidth{0.501875pt}%
\definecolor{currentstroke}{rgb}{0.000000,0.000000,0.000000}%
\pgfsetstrokecolor{currentstroke}%
\pgfsetdash{}{0pt}%
\pgfsys@defobject{currentmarker}{\pgfqpoint{0.000000in}{0.000000in}}{\pgfqpoint{0.000000in}{0.055556in}}{%
\pgfpathmoveto{\pgfqpoint{0.000000in}{0.000000in}}%
\pgfpathlineto{\pgfqpoint{0.000000in}{0.055556in}}%
\pgfusepath{stroke,fill}%
}%
\begin{pgfscope}%
\pgfsys@transformshift{3.262500in}{0.175000in}%
\pgfsys@useobject{currentmarker}{}%
\end{pgfscope}%
\end{pgfscope}%
\begin{pgfscope}%
\pgfsetbuttcap%
\pgfsetroundjoin%
\definecolor{currentfill}{rgb}{0.000000,0.000000,0.000000}%
\pgfsetfillcolor{currentfill}%
\pgfsetlinewidth{0.501875pt}%
\definecolor{currentstroke}{rgb}{0.000000,0.000000,0.000000}%
\pgfsetstrokecolor{currentstroke}%
\pgfsetdash{}{0pt}%
\pgfsys@defobject{currentmarker}{\pgfqpoint{0.000000in}{-0.055556in}}{\pgfqpoint{0.000000in}{0.000000in}}{%
\pgfpathmoveto{\pgfqpoint{0.000000in}{0.000000in}}%
\pgfpathlineto{\pgfqpoint{0.000000in}{-0.055556in}}%
\pgfusepath{stroke,fill}%
}%
\begin{pgfscope}%
\pgfsys@transformshift{3.262500in}{1.435000in}%
\pgfsys@useobject{currentmarker}{}%
\end{pgfscope}%
\end{pgfscope}%
\begin{pgfscope}%
\pgftext[x=3.262500in,y=0.119444in,,top]{\rmfamily\fontsize{9.000000}{10.800000}\selectfont \(\displaystyle 0.0012\)}%
\end{pgfscope}%
\begin{pgfscope}%
\pgfsetbuttcap%
\pgfsetroundjoin%
\definecolor{currentfill}{rgb}{0.000000,0.000000,0.000000}%
\pgfsetfillcolor{currentfill}%
\pgfsetlinewidth{0.501875pt}%
\definecolor{currentstroke}{rgb}{0.000000,0.000000,0.000000}%
\pgfsetstrokecolor{currentstroke}%
\pgfsetdash{}{0pt}%
\pgfsys@defobject{currentmarker}{\pgfqpoint{0.000000in}{0.000000in}}{\pgfqpoint{0.000000in}{0.055556in}}{%
\pgfpathmoveto{\pgfqpoint{0.000000in}{0.000000in}}%
\pgfpathlineto{\pgfqpoint{0.000000in}{0.055556in}}%
\pgfusepath{stroke,fill}%
}%
\begin{pgfscope}%
\pgfsys@transformshift{3.768750in}{0.175000in}%
\pgfsys@useobject{currentmarker}{}%
\end{pgfscope}%
\end{pgfscope}%
\begin{pgfscope}%
\pgfsetbuttcap%
\pgfsetroundjoin%
\definecolor{currentfill}{rgb}{0.000000,0.000000,0.000000}%
\pgfsetfillcolor{currentfill}%
\pgfsetlinewidth{0.501875pt}%
\definecolor{currentstroke}{rgb}{0.000000,0.000000,0.000000}%
\pgfsetstrokecolor{currentstroke}%
\pgfsetdash{}{0pt}%
\pgfsys@defobject{currentmarker}{\pgfqpoint{0.000000in}{-0.055556in}}{\pgfqpoint{0.000000in}{0.000000in}}{%
\pgfpathmoveto{\pgfqpoint{0.000000in}{0.000000in}}%
\pgfpathlineto{\pgfqpoint{0.000000in}{-0.055556in}}%
\pgfusepath{stroke,fill}%
}%
\begin{pgfscope}%
\pgfsys@transformshift{3.768750in}{1.435000in}%
\pgfsys@useobject{currentmarker}{}%
\end{pgfscope}%
\end{pgfscope}%
\begin{pgfscope}%
\pgftext[x=3.768750in,y=0.119444in,,top]{\rmfamily\fontsize{9.000000}{10.800000}\selectfont \(\displaystyle 0.0014\)}%
\end{pgfscope}%
\begin{pgfscope}%
\pgfsetbuttcap%
\pgfsetroundjoin%
\definecolor{currentfill}{rgb}{0.000000,0.000000,0.000000}%
\pgfsetfillcolor{currentfill}%
\pgfsetlinewidth{0.501875pt}%
\definecolor{currentstroke}{rgb}{0.000000,0.000000,0.000000}%
\pgfsetstrokecolor{currentstroke}%
\pgfsetdash{}{0pt}%
\pgfsys@defobject{currentmarker}{\pgfqpoint{0.000000in}{0.000000in}}{\pgfqpoint{0.000000in}{0.055556in}}{%
\pgfpathmoveto{\pgfqpoint{0.000000in}{0.000000in}}%
\pgfpathlineto{\pgfqpoint{0.000000in}{0.055556in}}%
\pgfusepath{stroke,fill}%
}%
\begin{pgfscope}%
\pgfsys@transformshift{4.275000in}{0.175000in}%
\pgfsys@useobject{currentmarker}{}%
\end{pgfscope}%
\end{pgfscope}%
\begin{pgfscope}%
\pgfsetbuttcap%
\pgfsetroundjoin%
\definecolor{currentfill}{rgb}{0.000000,0.000000,0.000000}%
\pgfsetfillcolor{currentfill}%
\pgfsetlinewidth{0.501875pt}%
\definecolor{currentstroke}{rgb}{0.000000,0.000000,0.000000}%
\pgfsetstrokecolor{currentstroke}%
\pgfsetdash{}{0pt}%
\pgfsys@defobject{currentmarker}{\pgfqpoint{0.000000in}{-0.055556in}}{\pgfqpoint{0.000000in}{0.000000in}}{%
\pgfpathmoveto{\pgfqpoint{0.000000in}{0.000000in}}%
\pgfpathlineto{\pgfqpoint{0.000000in}{-0.055556in}}%
\pgfusepath{stroke,fill}%
}%
\begin{pgfscope}%
\pgfsys@transformshift{4.275000in}{1.435000in}%
\pgfsys@useobject{currentmarker}{}%
\end{pgfscope}%
\end{pgfscope}%
\begin{pgfscope}%
\pgftext[x=4.275000in,y=0.119444in,,top]{\rmfamily\fontsize{9.000000}{10.800000}\selectfont \(\displaystyle 0.0016\)}%
\end{pgfscope}%
\begin{pgfscope}%
\pgfsetbuttcap%
\pgfsetroundjoin%
\definecolor{currentfill}{rgb}{0.000000,0.000000,0.000000}%
\pgfsetfillcolor{currentfill}%
\pgfsetlinewidth{0.501875pt}%
\definecolor{currentstroke}{rgb}{0.000000,0.000000,0.000000}%
\pgfsetstrokecolor{currentstroke}%
\pgfsetdash{}{0pt}%
\pgfsys@defobject{currentmarker}{\pgfqpoint{0.000000in}{0.000000in}}{\pgfqpoint{0.055556in}{0.000000in}}{%
\pgfpathmoveto{\pgfqpoint{0.000000in}{0.000000in}}%
\pgfpathlineto{\pgfqpoint{0.055556in}{0.000000in}}%
\pgfusepath{stroke,fill}%
}%
\begin{pgfscope}%
\pgfsys@transformshift{0.225000in}{0.232273in}%
\pgfsys@useobject{currentmarker}{}%
\end{pgfscope}%
\end{pgfscope}%
\begin{pgfscope}%
\pgfsetbuttcap%
\pgfsetroundjoin%
\definecolor{currentfill}{rgb}{0.000000,0.000000,0.000000}%
\pgfsetfillcolor{currentfill}%
\pgfsetlinewidth{0.501875pt}%
\definecolor{currentstroke}{rgb}{0.000000,0.000000,0.000000}%
\pgfsetstrokecolor{currentstroke}%
\pgfsetdash{}{0pt}%
\pgfsys@defobject{currentmarker}{\pgfqpoint{-0.055556in}{0.000000in}}{\pgfqpoint{0.000000in}{0.000000in}}{%
\pgfpathmoveto{\pgfqpoint{0.000000in}{0.000000in}}%
\pgfpathlineto{\pgfqpoint{-0.055556in}{0.000000in}}%
\pgfusepath{stroke,fill}%
}%
\begin{pgfscope}%
\pgfsys@transformshift{4.275000in}{0.232273in}%
\pgfsys@useobject{currentmarker}{}%
\end{pgfscope}%
\end{pgfscope}%
\begin{pgfscope}%
\pgftext[x=0.169444in,y=0.232273in,right,]{\rmfamily\fontsize{9.000000}{10.800000}\selectfont \(\displaystyle -1.5\)}%
\end{pgfscope}%
\begin{pgfscope}%
\pgfsetbuttcap%
\pgfsetroundjoin%
\definecolor{currentfill}{rgb}{0.000000,0.000000,0.000000}%
\pgfsetfillcolor{currentfill}%
\pgfsetlinewidth{0.501875pt}%
\definecolor{currentstroke}{rgb}{0.000000,0.000000,0.000000}%
\pgfsetstrokecolor{currentstroke}%
\pgfsetdash{}{0pt}%
\pgfsys@defobject{currentmarker}{\pgfqpoint{0.000000in}{0.000000in}}{\pgfqpoint{0.055556in}{0.000000in}}{%
\pgfpathmoveto{\pgfqpoint{0.000000in}{0.000000in}}%
\pgfpathlineto{\pgfqpoint{0.055556in}{0.000000in}}%
\pgfusepath{stroke,fill}%
}%
\begin{pgfscope}%
\pgfsys@transformshift{0.225000in}{0.423182in}%
\pgfsys@useobject{currentmarker}{}%
\end{pgfscope}%
\end{pgfscope}%
\begin{pgfscope}%
\pgfsetbuttcap%
\pgfsetroundjoin%
\definecolor{currentfill}{rgb}{0.000000,0.000000,0.000000}%
\pgfsetfillcolor{currentfill}%
\pgfsetlinewidth{0.501875pt}%
\definecolor{currentstroke}{rgb}{0.000000,0.000000,0.000000}%
\pgfsetstrokecolor{currentstroke}%
\pgfsetdash{}{0pt}%
\pgfsys@defobject{currentmarker}{\pgfqpoint{-0.055556in}{0.000000in}}{\pgfqpoint{0.000000in}{0.000000in}}{%
\pgfpathmoveto{\pgfqpoint{0.000000in}{0.000000in}}%
\pgfpathlineto{\pgfqpoint{-0.055556in}{0.000000in}}%
\pgfusepath{stroke,fill}%
}%
\begin{pgfscope}%
\pgfsys@transformshift{4.275000in}{0.423182in}%
\pgfsys@useobject{currentmarker}{}%
\end{pgfscope}%
\end{pgfscope}%
\begin{pgfscope}%
\pgftext[x=0.169444in,y=0.423182in,right,]{\rmfamily\fontsize{9.000000}{10.800000}\selectfont \(\displaystyle -1.0\)}%
\end{pgfscope}%
\begin{pgfscope}%
\pgfsetbuttcap%
\pgfsetroundjoin%
\definecolor{currentfill}{rgb}{0.000000,0.000000,0.000000}%
\pgfsetfillcolor{currentfill}%
\pgfsetlinewidth{0.501875pt}%
\definecolor{currentstroke}{rgb}{0.000000,0.000000,0.000000}%
\pgfsetstrokecolor{currentstroke}%
\pgfsetdash{}{0pt}%
\pgfsys@defobject{currentmarker}{\pgfqpoint{0.000000in}{0.000000in}}{\pgfqpoint{0.055556in}{0.000000in}}{%
\pgfpathmoveto{\pgfqpoint{0.000000in}{0.000000in}}%
\pgfpathlineto{\pgfqpoint{0.055556in}{0.000000in}}%
\pgfusepath{stroke,fill}%
}%
\begin{pgfscope}%
\pgfsys@transformshift{0.225000in}{0.614091in}%
\pgfsys@useobject{currentmarker}{}%
\end{pgfscope}%
\end{pgfscope}%
\begin{pgfscope}%
\pgfsetbuttcap%
\pgfsetroundjoin%
\definecolor{currentfill}{rgb}{0.000000,0.000000,0.000000}%
\pgfsetfillcolor{currentfill}%
\pgfsetlinewidth{0.501875pt}%
\definecolor{currentstroke}{rgb}{0.000000,0.000000,0.000000}%
\pgfsetstrokecolor{currentstroke}%
\pgfsetdash{}{0pt}%
\pgfsys@defobject{currentmarker}{\pgfqpoint{-0.055556in}{0.000000in}}{\pgfqpoint{0.000000in}{0.000000in}}{%
\pgfpathmoveto{\pgfqpoint{0.000000in}{0.000000in}}%
\pgfpathlineto{\pgfqpoint{-0.055556in}{0.000000in}}%
\pgfusepath{stroke,fill}%
}%
\begin{pgfscope}%
\pgfsys@transformshift{4.275000in}{0.614091in}%
\pgfsys@useobject{currentmarker}{}%
\end{pgfscope}%
\end{pgfscope}%
\begin{pgfscope}%
\pgftext[x=0.169444in,y=0.614091in,right,]{\rmfamily\fontsize{9.000000}{10.800000}\selectfont \(\displaystyle -0.5\)}%
\end{pgfscope}%
\begin{pgfscope}%
\pgfsetbuttcap%
\pgfsetroundjoin%
\definecolor{currentfill}{rgb}{0.000000,0.000000,0.000000}%
\pgfsetfillcolor{currentfill}%
\pgfsetlinewidth{0.501875pt}%
\definecolor{currentstroke}{rgb}{0.000000,0.000000,0.000000}%
\pgfsetstrokecolor{currentstroke}%
\pgfsetdash{}{0pt}%
\pgfsys@defobject{currentmarker}{\pgfqpoint{0.000000in}{0.000000in}}{\pgfqpoint{0.055556in}{0.000000in}}{%
\pgfpathmoveto{\pgfqpoint{0.000000in}{0.000000in}}%
\pgfpathlineto{\pgfqpoint{0.055556in}{0.000000in}}%
\pgfusepath{stroke,fill}%
}%
\begin{pgfscope}%
\pgfsys@transformshift{0.225000in}{0.805000in}%
\pgfsys@useobject{currentmarker}{}%
\end{pgfscope}%
\end{pgfscope}%
\begin{pgfscope}%
\pgfsetbuttcap%
\pgfsetroundjoin%
\definecolor{currentfill}{rgb}{0.000000,0.000000,0.000000}%
\pgfsetfillcolor{currentfill}%
\pgfsetlinewidth{0.501875pt}%
\definecolor{currentstroke}{rgb}{0.000000,0.000000,0.000000}%
\pgfsetstrokecolor{currentstroke}%
\pgfsetdash{}{0pt}%
\pgfsys@defobject{currentmarker}{\pgfqpoint{-0.055556in}{0.000000in}}{\pgfqpoint{0.000000in}{0.000000in}}{%
\pgfpathmoveto{\pgfqpoint{0.000000in}{0.000000in}}%
\pgfpathlineto{\pgfqpoint{-0.055556in}{0.000000in}}%
\pgfusepath{stroke,fill}%
}%
\begin{pgfscope}%
\pgfsys@transformshift{4.275000in}{0.805000in}%
\pgfsys@useobject{currentmarker}{}%
\end{pgfscope}%
\end{pgfscope}%
\begin{pgfscope}%
\pgftext[x=0.169444in,y=0.805000in,right,]{\rmfamily\fontsize{9.000000}{10.800000}\selectfont \(\displaystyle 0.0\)}%
\end{pgfscope}%
\begin{pgfscope}%
\pgfsetbuttcap%
\pgfsetroundjoin%
\definecolor{currentfill}{rgb}{0.000000,0.000000,0.000000}%
\pgfsetfillcolor{currentfill}%
\pgfsetlinewidth{0.501875pt}%
\definecolor{currentstroke}{rgb}{0.000000,0.000000,0.000000}%
\pgfsetstrokecolor{currentstroke}%
\pgfsetdash{}{0pt}%
\pgfsys@defobject{currentmarker}{\pgfqpoint{0.000000in}{0.000000in}}{\pgfqpoint{0.055556in}{0.000000in}}{%
\pgfpathmoveto{\pgfqpoint{0.000000in}{0.000000in}}%
\pgfpathlineto{\pgfqpoint{0.055556in}{0.000000in}}%
\pgfusepath{stroke,fill}%
}%
\begin{pgfscope}%
\pgfsys@transformshift{0.225000in}{0.995909in}%
\pgfsys@useobject{currentmarker}{}%
\end{pgfscope}%
\end{pgfscope}%
\begin{pgfscope}%
\pgfsetbuttcap%
\pgfsetroundjoin%
\definecolor{currentfill}{rgb}{0.000000,0.000000,0.000000}%
\pgfsetfillcolor{currentfill}%
\pgfsetlinewidth{0.501875pt}%
\definecolor{currentstroke}{rgb}{0.000000,0.000000,0.000000}%
\pgfsetstrokecolor{currentstroke}%
\pgfsetdash{}{0pt}%
\pgfsys@defobject{currentmarker}{\pgfqpoint{-0.055556in}{0.000000in}}{\pgfqpoint{0.000000in}{0.000000in}}{%
\pgfpathmoveto{\pgfqpoint{0.000000in}{0.000000in}}%
\pgfpathlineto{\pgfqpoint{-0.055556in}{0.000000in}}%
\pgfusepath{stroke,fill}%
}%
\begin{pgfscope}%
\pgfsys@transformshift{4.275000in}{0.995909in}%
\pgfsys@useobject{currentmarker}{}%
\end{pgfscope}%
\end{pgfscope}%
\begin{pgfscope}%
\pgftext[x=0.169444in,y=0.995909in,right,]{\rmfamily\fontsize{9.000000}{10.800000}\selectfont \(\displaystyle 0.5\)}%
\end{pgfscope}%
\begin{pgfscope}%
\pgfsetbuttcap%
\pgfsetroundjoin%
\definecolor{currentfill}{rgb}{0.000000,0.000000,0.000000}%
\pgfsetfillcolor{currentfill}%
\pgfsetlinewidth{0.501875pt}%
\definecolor{currentstroke}{rgb}{0.000000,0.000000,0.000000}%
\pgfsetstrokecolor{currentstroke}%
\pgfsetdash{}{0pt}%
\pgfsys@defobject{currentmarker}{\pgfqpoint{0.000000in}{0.000000in}}{\pgfqpoint{0.055556in}{0.000000in}}{%
\pgfpathmoveto{\pgfqpoint{0.000000in}{0.000000in}}%
\pgfpathlineto{\pgfqpoint{0.055556in}{0.000000in}}%
\pgfusepath{stroke,fill}%
}%
\begin{pgfscope}%
\pgfsys@transformshift{0.225000in}{1.186818in}%
\pgfsys@useobject{currentmarker}{}%
\end{pgfscope}%
\end{pgfscope}%
\begin{pgfscope}%
\pgfsetbuttcap%
\pgfsetroundjoin%
\definecolor{currentfill}{rgb}{0.000000,0.000000,0.000000}%
\pgfsetfillcolor{currentfill}%
\pgfsetlinewidth{0.501875pt}%
\definecolor{currentstroke}{rgb}{0.000000,0.000000,0.000000}%
\pgfsetstrokecolor{currentstroke}%
\pgfsetdash{}{0pt}%
\pgfsys@defobject{currentmarker}{\pgfqpoint{-0.055556in}{0.000000in}}{\pgfqpoint{0.000000in}{0.000000in}}{%
\pgfpathmoveto{\pgfqpoint{0.000000in}{0.000000in}}%
\pgfpathlineto{\pgfqpoint{-0.055556in}{0.000000in}}%
\pgfusepath{stroke,fill}%
}%
\begin{pgfscope}%
\pgfsys@transformshift{4.275000in}{1.186818in}%
\pgfsys@useobject{currentmarker}{}%
\end{pgfscope}%
\end{pgfscope}%
\begin{pgfscope}%
\pgftext[x=0.169444in,y=1.186818in,right,]{\rmfamily\fontsize{9.000000}{10.800000}\selectfont \(\displaystyle 1.0\)}%
\end{pgfscope}%
\begin{pgfscope}%
\pgfsetbuttcap%
\pgfsetroundjoin%
\definecolor{currentfill}{rgb}{0.000000,0.000000,0.000000}%
\pgfsetfillcolor{currentfill}%
\pgfsetlinewidth{0.501875pt}%
\definecolor{currentstroke}{rgb}{0.000000,0.000000,0.000000}%
\pgfsetstrokecolor{currentstroke}%
\pgfsetdash{}{0pt}%
\pgfsys@defobject{currentmarker}{\pgfqpoint{0.000000in}{0.000000in}}{\pgfqpoint{0.055556in}{0.000000in}}{%
\pgfpathmoveto{\pgfqpoint{0.000000in}{0.000000in}}%
\pgfpathlineto{\pgfqpoint{0.055556in}{0.000000in}}%
\pgfusepath{stroke,fill}%
}%
\begin{pgfscope}%
\pgfsys@transformshift{0.225000in}{1.377727in}%
\pgfsys@useobject{currentmarker}{}%
\end{pgfscope}%
\end{pgfscope}%
\begin{pgfscope}%
\pgfsetbuttcap%
\pgfsetroundjoin%
\definecolor{currentfill}{rgb}{0.000000,0.000000,0.000000}%
\pgfsetfillcolor{currentfill}%
\pgfsetlinewidth{0.501875pt}%
\definecolor{currentstroke}{rgb}{0.000000,0.000000,0.000000}%
\pgfsetstrokecolor{currentstroke}%
\pgfsetdash{}{0pt}%
\pgfsys@defobject{currentmarker}{\pgfqpoint{-0.055556in}{0.000000in}}{\pgfqpoint{0.000000in}{0.000000in}}{%
\pgfpathmoveto{\pgfqpoint{0.000000in}{0.000000in}}%
\pgfpathlineto{\pgfqpoint{-0.055556in}{0.000000in}}%
\pgfusepath{stroke,fill}%
}%
\begin{pgfscope}%
\pgfsys@transformshift{4.275000in}{1.377727in}%
\pgfsys@useobject{currentmarker}{}%
\end{pgfscope}%
\end{pgfscope}%
\begin{pgfscope}%
\pgftext[x=0.169444in,y=1.377727in,right,]{\rmfamily\fontsize{9.000000}{10.800000}\selectfont \(\displaystyle 1.5\)}%
\end{pgfscope}%
\begin{pgfscope}%
\pgftext[x=2.250000in,y=1.504444in,,base]{\rmfamily\fontsize{11.000000}{13.200000}\selectfont Moduliertes Signal}%
\end{pgfscope}%
\end{pgfpicture}%
\makeatother%
\endgroup%

    \caption{%
        \emph{Frequency-shift  keying}: Oben   sind  die   zu  \"ubertragenden
        digitalen  Daten  als  \code{1}  und \code{0}  abgebildet,  unten  das
        zugeh\"orige Verhalten des  modulierten Signals.%
    }
    \label{fig:fsk:concept}
\end{figure}


\textbf{Amplitude-shift  keying}: Die  ASK (Amplitudenumtastung  auf  Deutsch)
benutzt statt verschiedenen Frequenzen unterschiedliche Amplituden, um Symbole
zu  codieren.  \fref{fig:ask:concept}  stellt  das  grundlegende  Konzept  des
Verfahrens schematisch dar.

Unsere  Umsetzung unterscheidet  sich  von  einer <<normalen>> \todo{<<text>> oder ``text'' ?}  Implementation
(mittlere   Graphik  in   \fref{fig:ask:concept})  dadurch,   dass  nicht   im
eigentlichen Sinne  ein zus\"atzliches  Signal auf den  DC-Anteil aufmoduliert
wird. Stattdessen   wird  ein   Solarmodul  mit   einer  bestimmten   Frequenz
kurzgeschlossen,  was die  Spannung auf  einem Strang  von Modulen  kurzzeitig
in  Intervallen  einbrechen  l\"asst. Diese  Spannungseinbr\"uche  werden  zur
Datencodierung  benutzt   (unterster  Plot   in  \fref{fig:ask:concept}). Dies
entspricht  im  Prinzip  einer  ASK  mit Amplituden  null  und  dem  Wert  des
Spannungseinbruchs mit einem Offset des halben Spannungseinbruchs.

\todo{Zahlen in Ticks auf Achsen weglassen?}

\begin{figure}[h!tb]
    \centering
    %% Creator: Matplotlib, PGF backend
%%
%% To include the figure in your LaTeX document, write
%%   \input{<filename>.pgf}
%%
%% Make sure the required packages are loaded in your preamble
%%   \usepackage{pgf}
%%
%% Figures using additional raster images can only be included by \input if
%% they are in the same directory as the main LaTeX file. For loading figures
%% from other directories you can use the `import` package
%%   \usepackage{import}
%% and then include the figures with
%%   \import{<path to file>}{<filename>.pgf}
%%
%% Matplotlib used the following preamble
%%   \usepackage{fontspec}
%%   \setmainfont{Bitstream Vera Serif}
%%   \setsansfont{Bitstream Vera Sans}
%%   \setmonofont{Bitstream Vera Sans Mono}
%%
\begingroup%
\makeatletter%
\begin{pgfpicture}%
\pgfpathrectangle{\pgfpointorigin}{\pgfqpoint{4.500000in}{3.500000in}}%
\pgfusepath{use as bounding box, clip}%
\begin{pgfscope}%
\pgfsetbuttcap%
\pgfsetmiterjoin%
\pgfsetlinewidth{0.000000pt}%
\definecolor{currentstroke}{rgb}{0.000000,0.000000,0.000000}%
\pgfsetstrokecolor{currentstroke}%
\pgfsetstrokeopacity{0.000000}%
\pgfsetdash{}{0pt}%
\pgfpathmoveto{\pgfqpoint{0.000000in}{0.000000in}}%
\pgfpathlineto{\pgfqpoint{4.500000in}{0.000000in}}%
\pgfpathlineto{\pgfqpoint{4.500000in}{3.500000in}}%
\pgfpathlineto{\pgfqpoint{0.000000in}{3.500000in}}%
\pgfpathclose%
\pgfusepath{}%
\end{pgfscope}%
\begin{pgfscope}%
\pgfsetbuttcap%
\pgfsetmiterjoin%
\pgfsetlinewidth{0.000000pt}%
\definecolor{currentstroke}{rgb}{0.000000,0.000000,0.000000}%
\pgfsetstrokecolor{currentstroke}%
\pgfsetstrokeopacity{0.000000}%
\pgfsetdash{}{0pt}%
\pgfpathmoveto{\pgfqpoint{0.225000in}{2.065000in}}%
\pgfpathlineto{\pgfqpoint{4.275000in}{2.065000in}}%
\pgfpathlineto{\pgfqpoint{4.275000in}{3.325000in}}%
\pgfpathlineto{\pgfqpoint{0.225000in}{3.325000in}}%
\pgfpathclose%
\pgfusepath{}%
\end{pgfscope}%
\begin{pgfscope}%
\pgfpathrectangle{\pgfqpoint{0.225000in}{2.065000in}}{\pgfqpoint{4.050000in}{1.260000in}} %
\pgfusepath{clip}%
\pgfsetrectcap%
\pgfsetroundjoin%
\pgfsetlinewidth{1.003750pt}%
\definecolor{currentstroke}{rgb}{0.000000,0.000000,1.000000}%
\pgfsetstrokecolor{currentstroke}%
\pgfsetdash{}{0pt}%
\pgfpathmoveto{\pgfqpoint{0.225000in}{2.170000in}}%
\pgfpathlineto{\pgfqpoint{1.188898in}{2.170000in}}%
\pgfusepath{stroke}%
\end{pgfscope}%
\begin{pgfscope}%
\pgfpathrectangle{\pgfqpoint{0.225000in}{2.065000in}}{\pgfqpoint{4.050000in}{1.260000in}} %
\pgfusepath{clip}%
\pgfsetrectcap%
\pgfsetroundjoin%
\pgfsetlinewidth{1.003750pt}%
\definecolor{currentstroke}{rgb}{0.501961,0.501961,0.501961}%
\pgfsetstrokecolor{currentstroke}%
\pgfsetdash{}{0pt}%
\pgfpathmoveto{\pgfqpoint{1.188898in}{2.170000in}}%
\pgfpathlineto{\pgfqpoint{1.188898in}{3.220000in}}%
\pgfusepath{stroke}%
\end{pgfscope}%
\begin{pgfscope}%
\pgfpathrectangle{\pgfqpoint{0.225000in}{2.065000in}}{\pgfqpoint{4.050000in}{1.260000in}} %
\pgfusepath{clip}%
\pgfsetrectcap%
\pgfsetroundjoin%
\pgfsetlinewidth{1.003750pt}%
\definecolor{currentstroke}{rgb}{1.000000,0.000000,1.000000}%
\pgfsetstrokecolor{currentstroke}%
\pgfsetdash{}{0pt}%
\pgfpathmoveto{\pgfqpoint{1.188898in}{3.220000in}}%
\pgfpathlineto{\pgfqpoint{2.152795in}{3.220000in}}%
\pgfusepath{stroke}%
\end{pgfscope}%
\begin{pgfscope}%
\pgfpathrectangle{\pgfqpoint{0.225000in}{2.065000in}}{\pgfqpoint{4.050000in}{1.260000in}} %
\pgfusepath{clip}%
\pgfsetrectcap%
\pgfsetroundjoin%
\pgfsetlinewidth{1.003750pt}%
\definecolor{currentstroke}{rgb}{0.501961,0.501961,0.501961}%
\pgfsetstrokecolor{currentstroke}%
\pgfsetdash{}{0pt}%
\pgfpathmoveto{\pgfqpoint{2.152795in}{3.220000in}}%
\pgfpathlineto{\pgfqpoint{2.152795in}{2.170000in}}%
\pgfusepath{stroke}%
\end{pgfscope}%
\begin{pgfscope}%
\pgfpathrectangle{\pgfqpoint{0.225000in}{2.065000in}}{\pgfqpoint{4.050000in}{1.260000in}} %
\pgfusepath{clip}%
\pgfsetrectcap%
\pgfsetroundjoin%
\pgfsetlinewidth{1.003750pt}%
\definecolor{currentstroke}{rgb}{0.000000,0.000000,1.000000}%
\pgfsetstrokecolor{currentstroke}%
\pgfsetdash{}{0pt}%
\pgfpathmoveto{\pgfqpoint{2.152795in}{2.170000in}}%
\pgfpathlineto{\pgfqpoint{3.116693in}{2.170000in}}%
\pgfusepath{stroke}%
\end{pgfscope}%
\begin{pgfscope}%
\pgfpathrectangle{\pgfqpoint{0.225000in}{2.065000in}}{\pgfqpoint{4.050000in}{1.260000in}} %
\pgfusepath{clip}%
\pgfsetrectcap%
\pgfsetroundjoin%
\pgfsetlinewidth{1.003750pt}%
\definecolor{currentstroke}{rgb}{0.501961,0.501961,0.501961}%
\pgfsetstrokecolor{currentstroke}%
\pgfsetdash{}{0pt}%
\pgfpathmoveto{\pgfqpoint{3.116693in}{2.170000in}}%
\pgfpathlineto{\pgfqpoint{3.116693in}{3.220000in}}%
\pgfusepath{stroke}%
\end{pgfscope}%
\begin{pgfscope}%
\pgfpathrectangle{\pgfqpoint{0.225000in}{2.065000in}}{\pgfqpoint{4.050000in}{1.260000in}} %
\pgfusepath{clip}%
\pgfsetrectcap%
\pgfsetroundjoin%
\pgfsetlinewidth{1.003750pt}%
\definecolor{currentstroke}{rgb}{1.000000,0.000000,1.000000}%
\pgfsetstrokecolor{currentstroke}%
\pgfsetdash{}{0pt}%
\pgfpathmoveto{\pgfqpoint{3.116693in}{3.220000in}}%
\pgfpathlineto{\pgfqpoint{4.080591in}{3.220000in}}%
\pgfusepath{stroke}%
\end{pgfscope}%
\begin{pgfscope}%
\pgfsetrectcap%
\pgfsetmiterjoin%
\pgfsetlinewidth{1.003750pt}%
\definecolor{currentstroke}{rgb}{0.000000,0.000000,0.000000}%
\pgfsetstrokecolor{currentstroke}%
\pgfsetdash{}{0pt}%
\pgfpathmoveto{\pgfqpoint{0.225000in}{2.065000in}}%
\pgfpathlineto{\pgfqpoint{0.225000in}{3.325000in}}%
\pgfusepath{stroke}%
\end{pgfscope}%
\begin{pgfscope}%
\pgfsetrectcap%
\pgfsetmiterjoin%
\pgfsetlinewidth{1.003750pt}%
\definecolor{currentstroke}{rgb}{0.000000,0.000000,0.000000}%
\pgfsetstrokecolor{currentstroke}%
\pgfsetdash{}{0pt}%
\pgfpathmoveto{\pgfqpoint{0.225000in}{3.325000in}}%
\pgfpathlineto{\pgfqpoint{4.275000in}{3.325000in}}%
\pgfusepath{stroke}%
\end{pgfscope}%
\begin{pgfscope}%
\pgfsetrectcap%
\pgfsetmiterjoin%
\pgfsetlinewidth{1.003750pt}%
\definecolor{currentstroke}{rgb}{0.000000,0.000000,0.000000}%
\pgfsetstrokecolor{currentstroke}%
\pgfsetdash{}{0pt}%
\pgfpathmoveto{\pgfqpoint{4.275000in}{2.065000in}}%
\pgfpathlineto{\pgfqpoint{4.275000in}{3.325000in}}%
\pgfusepath{stroke}%
\end{pgfscope}%
\begin{pgfscope}%
\pgfsetrectcap%
\pgfsetmiterjoin%
\pgfsetlinewidth{1.003750pt}%
\definecolor{currentstroke}{rgb}{0.000000,0.000000,0.000000}%
\pgfsetstrokecolor{currentstroke}%
\pgfsetdash{}{0pt}%
\pgfpathmoveto{\pgfqpoint{0.225000in}{2.065000in}}%
\pgfpathlineto{\pgfqpoint{4.275000in}{2.065000in}}%
\pgfusepath{stroke}%
\end{pgfscope}%
\begin{pgfscope}%
\pgfsetbuttcap%
\pgfsetroundjoin%
\definecolor{currentfill}{rgb}{0.000000,0.000000,0.000000}%
\pgfsetfillcolor{currentfill}%
\pgfsetlinewidth{0.501875pt}%
\definecolor{currentstroke}{rgb}{0.000000,0.000000,0.000000}%
\pgfsetstrokecolor{currentstroke}%
\pgfsetdash{}{0pt}%
\pgfsys@defobject{currentmarker}{\pgfqpoint{0.000000in}{0.000000in}}{\pgfqpoint{0.000000in}{0.055556in}}{%
\pgfpathmoveto{\pgfqpoint{0.000000in}{0.000000in}}%
\pgfpathlineto{\pgfqpoint{0.000000in}{0.055556in}}%
\pgfusepath{stroke,fill}%
}%
\begin{pgfscope}%
\pgfsys@transformshift{0.225000in}{2.065000in}%
\pgfsys@useobject{currentmarker}{}%
\end{pgfscope}%
\end{pgfscope}%
\begin{pgfscope}%
\pgfsetbuttcap%
\pgfsetroundjoin%
\definecolor{currentfill}{rgb}{0.000000,0.000000,0.000000}%
\pgfsetfillcolor{currentfill}%
\pgfsetlinewidth{0.501875pt}%
\definecolor{currentstroke}{rgb}{0.000000,0.000000,0.000000}%
\pgfsetstrokecolor{currentstroke}%
\pgfsetdash{}{0pt}%
\pgfsys@defobject{currentmarker}{\pgfqpoint{0.000000in}{-0.055556in}}{\pgfqpoint{0.000000in}{0.000000in}}{%
\pgfpathmoveto{\pgfqpoint{0.000000in}{0.000000in}}%
\pgfpathlineto{\pgfqpoint{0.000000in}{-0.055556in}}%
\pgfusepath{stroke,fill}%
}%
\begin{pgfscope}%
\pgfsys@transformshift{0.225000in}{3.325000in}%
\pgfsys@useobject{currentmarker}{}%
\end{pgfscope}%
\end{pgfscope}%
\begin{pgfscope}%
\pgftext[x=0.225000in,y=2.009444in,,top]{\rmfamily\fontsize{9.000000}{10.800000}\selectfont \(\displaystyle 0.0000\)}%
\end{pgfscope}%
\begin{pgfscope}%
\pgfsetbuttcap%
\pgfsetroundjoin%
\definecolor{currentfill}{rgb}{0.000000,0.000000,0.000000}%
\pgfsetfillcolor{currentfill}%
\pgfsetlinewidth{0.501875pt}%
\definecolor{currentstroke}{rgb}{0.000000,0.000000,0.000000}%
\pgfsetstrokecolor{currentstroke}%
\pgfsetdash{}{0pt}%
\pgfsys@defobject{currentmarker}{\pgfqpoint{0.000000in}{0.000000in}}{\pgfqpoint{0.000000in}{0.055556in}}{%
\pgfpathmoveto{\pgfqpoint{0.000000in}{0.000000in}}%
\pgfpathlineto{\pgfqpoint{0.000000in}{0.055556in}}%
\pgfusepath{stroke,fill}%
}%
\begin{pgfscope}%
\pgfsys@transformshift{0.731250in}{2.065000in}%
\pgfsys@useobject{currentmarker}{}%
\end{pgfscope}%
\end{pgfscope}%
\begin{pgfscope}%
\pgfsetbuttcap%
\pgfsetroundjoin%
\definecolor{currentfill}{rgb}{0.000000,0.000000,0.000000}%
\pgfsetfillcolor{currentfill}%
\pgfsetlinewidth{0.501875pt}%
\definecolor{currentstroke}{rgb}{0.000000,0.000000,0.000000}%
\pgfsetstrokecolor{currentstroke}%
\pgfsetdash{}{0pt}%
\pgfsys@defobject{currentmarker}{\pgfqpoint{0.000000in}{-0.055556in}}{\pgfqpoint{0.000000in}{0.000000in}}{%
\pgfpathmoveto{\pgfqpoint{0.000000in}{0.000000in}}%
\pgfpathlineto{\pgfqpoint{0.000000in}{-0.055556in}}%
\pgfusepath{stroke,fill}%
}%
\begin{pgfscope}%
\pgfsys@transformshift{0.731250in}{3.325000in}%
\pgfsys@useobject{currentmarker}{}%
\end{pgfscope}%
\end{pgfscope}%
\begin{pgfscope}%
\pgftext[x=0.731250in,y=2.009444in,,top]{\rmfamily\fontsize{9.000000}{10.800000}\selectfont \(\displaystyle 0.0002\)}%
\end{pgfscope}%
\begin{pgfscope}%
\pgfsetbuttcap%
\pgfsetroundjoin%
\definecolor{currentfill}{rgb}{0.000000,0.000000,0.000000}%
\pgfsetfillcolor{currentfill}%
\pgfsetlinewidth{0.501875pt}%
\definecolor{currentstroke}{rgb}{0.000000,0.000000,0.000000}%
\pgfsetstrokecolor{currentstroke}%
\pgfsetdash{}{0pt}%
\pgfsys@defobject{currentmarker}{\pgfqpoint{0.000000in}{0.000000in}}{\pgfqpoint{0.000000in}{0.055556in}}{%
\pgfpathmoveto{\pgfqpoint{0.000000in}{0.000000in}}%
\pgfpathlineto{\pgfqpoint{0.000000in}{0.055556in}}%
\pgfusepath{stroke,fill}%
}%
\begin{pgfscope}%
\pgfsys@transformshift{1.237500in}{2.065000in}%
\pgfsys@useobject{currentmarker}{}%
\end{pgfscope}%
\end{pgfscope}%
\begin{pgfscope}%
\pgfsetbuttcap%
\pgfsetroundjoin%
\definecolor{currentfill}{rgb}{0.000000,0.000000,0.000000}%
\pgfsetfillcolor{currentfill}%
\pgfsetlinewidth{0.501875pt}%
\definecolor{currentstroke}{rgb}{0.000000,0.000000,0.000000}%
\pgfsetstrokecolor{currentstroke}%
\pgfsetdash{}{0pt}%
\pgfsys@defobject{currentmarker}{\pgfqpoint{0.000000in}{-0.055556in}}{\pgfqpoint{0.000000in}{0.000000in}}{%
\pgfpathmoveto{\pgfqpoint{0.000000in}{0.000000in}}%
\pgfpathlineto{\pgfqpoint{0.000000in}{-0.055556in}}%
\pgfusepath{stroke,fill}%
}%
\begin{pgfscope}%
\pgfsys@transformshift{1.237500in}{3.325000in}%
\pgfsys@useobject{currentmarker}{}%
\end{pgfscope}%
\end{pgfscope}%
\begin{pgfscope}%
\pgftext[x=1.237500in,y=2.009444in,,top]{\rmfamily\fontsize{9.000000}{10.800000}\selectfont \(\displaystyle 0.0004\)}%
\end{pgfscope}%
\begin{pgfscope}%
\pgfsetbuttcap%
\pgfsetroundjoin%
\definecolor{currentfill}{rgb}{0.000000,0.000000,0.000000}%
\pgfsetfillcolor{currentfill}%
\pgfsetlinewidth{0.501875pt}%
\definecolor{currentstroke}{rgb}{0.000000,0.000000,0.000000}%
\pgfsetstrokecolor{currentstroke}%
\pgfsetdash{}{0pt}%
\pgfsys@defobject{currentmarker}{\pgfqpoint{0.000000in}{0.000000in}}{\pgfqpoint{0.000000in}{0.055556in}}{%
\pgfpathmoveto{\pgfqpoint{0.000000in}{0.000000in}}%
\pgfpathlineto{\pgfqpoint{0.000000in}{0.055556in}}%
\pgfusepath{stroke,fill}%
}%
\begin{pgfscope}%
\pgfsys@transformshift{1.743750in}{2.065000in}%
\pgfsys@useobject{currentmarker}{}%
\end{pgfscope}%
\end{pgfscope}%
\begin{pgfscope}%
\pgfsetbuttcap%
\pgfsetroundjoin%
\definecolor{currentfill}{rgb}{0.000000,0.000000,0.000000}%
\pgfsetfillcolor{currentfill}%
\pgfsetlinewidth{0.501875pt}%
\definecolor{currentstroke}{rgb}{0.000000,0.000000,0.000000}%
\pgfsetstrokecolor{currentstroke}%
\pgfsetdash{}{0pt}%
\pgfsys@defobject{currentmarker}{\pgfqpoint{0.000000in}{-0.055556in}}{\pgfqpoint{0.000000in}{0.000000in}}{%
\pgfpathmoveto{\pgfqpoint{0.000000in}{0.000000in}}%
\pgfpathlineto{\pgfqpoint{0.000000in}{-0.055556in}}%
\pgfusepath{stroke,fill}%
}%
\begin{pgfscope}%
\pgfsys@transformshift{1.743750in}{3.325000in}%
\pgfsys@useobject{currentmarker}{}%
\end{pgfscope}%
\end{pgfscope}%
\begin{pgfscope}%
\pgftext[x=1.743750in,y=2.009444in,,top]{\rmfamily\fontsize{9.000000}{10.800000}\selectfont \(\displaystyle 0.0006\)}%
\end{pgfscope}%
\begin{pgfscope}%
\pgfsetbuttcap%
\pgfsetroundjoin%
\definecolor{currentfill}{rgb}{0.000000,0.000000,0.000000}%
\pgfsetfillcolor{currentfill}%
\pgfsetlinewidth{0.501875pt}%
\definecolor{currentstroke}{rgb}{0.000000,0.000000,0.000000}%
\pgfsetstrokecolor{currentstroke}%
\pgfsetdash{}{0pt}%
\pgfsys@defobject{currentmarker}{\pgfqpoint{0.000000in}{0.000000in}}{\pgfqpoint{0.000000in}{0.055556in}}{%
\pgfpathmoveto{\pgfqpoint{0.000000in}{0.000000in}}%
\pgfpathlineto{\pgfqpoint{0.000000in}{0.055556in}}%
\pgfusepath{stroke,fill}%
}%
\begin{pgfscope}%
\pgfsys@transformshift{2.250000in}{2.065000in}%
\pgfsys@useobject{currentmarker}{}%
\end{pgfscope}%
\end{pgfscope}%
\begin{pgfscope}%
\pgfsetbuttcap%
\pgfsetroundjoin%
\definecolor{currentfill}{rgb}{0.000000,0.000000,0.000000}%
\pgfsetfillcolor{currentfill}%
\pgfsetlinewidth{0.501875pt}%
\definecolor{currentstroke}{rgb}{0.000000,0.000000,0.000000}%
\pgfsetstrokecolor{currentstroke}%
\pgfsetdash{}{0pt}%
\pgfsys@defobject{currentmarker}{\pgfqpoint{0.000000in}{-0.055556in}}{\pgfqpoint{0.000000in}{0.000000in}}{%
\pgfpathmoveto{\pgfqpoint{0.000000in}{0.000000in}}%
\pgfpathlineto{\pgfqpoint{0.000000in}{-0.055556in}}%
\pgfusepath{stroke,fill}%
}%
\begin{pgfscope}%
\pgfsys@transformshift{2.250000in}{3.325000in}%
\pgfsys@useobject{currentmarker}{}%
\end{pgfscope}%
\end{pgfscope}%
\begin{pgfscope}%
\pgftext[x=2.250000in,y=2.009444in,,top]{\rmfamily\fontsize{9.000000}{10.800000}\selectfont \(\displaystyle 0.0008\)}%
\end{pgfscope}%
\begin{pgfscope}%
\pgfsetbuttcap%
\pgfsetroundjoin%
\definecolor{currentfill}{rgb}{0.000000,0.000000,0.000000}%
\pgfsetfillcolor{currentfill}%
\pgfsetlinewidth{0.501875pt}%
\definecolor{currentstroke}{rgb}{0.000000,0.000000,0.000000}%
\pgfsetstrokecolor{currentstroke}%
\pgfsetdash{}{0pt}%
\pgfsys@defobject{currentmarker}{\pgfqpoint{0.000000in}{0.000000in}}{\pgfqpoint{0.000000in}{0.055556in}}{%
\pgfpathmoveto{\pgfqpoint{0.000000in}{0.000000in}}%
\pgfpathlineto{\pgfqpoint{0.000000in}{0.055556in}}%
\pgfusepath{stroke,fill}%
}%
\begin{pgfscope}%
\pgfsys@transformshift{2.756250in}{2.065000in}%
\pgfsys@useobject{currentmarker}{}%
\end{pgfscope}%
\end{pgfscope}%
\begin{pgfscope}%
\pgfsetbuttcap%
\pgfsetroundjoin%
\definecolor{currentfill}{rgb}{0.000000,0.000000,0.000000}%
\pgfsetfillcolor{currentfill}%
\pgfsetlinewidth{0.501875pt}%
\definecolor{currentstroke}{rgb}{0.000000,0.000000,0.000000}%
\pgfsetstrokecolor{currentstroke}%
\pgfsetdash{}{0pt}%
\pgfsys@defobject{currentmarker}{\pgfqpoint{0.000000in}{-0.055556in}}{\pgfqpoint{0.000000in}{0.000000in}}{%
\pgfpathmoveto{\pgfqpoint{0.000000in}{0.000000in}}%
\pgfpathlineto{\pgfqpoint{0.000000in}{-0.055556in}}%
\pgfusepath{stroke,fill}%
}%
\begin{pgfscope}%
\pgfsys@transformshift{2.756250in}{3.325000in}%
\pgfsys@useobject{currentmarker}{}%
\end{pgfscope}%
\end{pgfscope}%
\begin{pgfscope}%
\pgftext[x=2.756250in,y=2.009444in,,top]{\rmfamily\fontsize{9.000000}{10.800000}\selectfont \(\displaystyle 0.0010\)}%
\end{pgfscope}%
\begin{pgfscope}%
\pgfsetbuttcap%
\pgfsetroundjoin%
\definecolor{currentfill}{rgb}{0.000000,0.000000,0.000000}%
\pgfsetfillcolor{currentfill}%
\pgfsetlinewidth{0.501875pt}%
\definecolor{currentstroke}{rgb}{0.000000,0.000000,0.000000}%
\pgfsetstrokecolor{currentstroke}%
\pgfsetdash{}{0pt}%
\pgfsys@defobject{currentmarker}{\pgfqpoint{0.000000in}{0.000000in}}{\pgfqpoint{0.000000in}{0.055556in}}{%
\pgfpathmoveto{\pgfqpoint{0.000000in}{0.000000in}}%
\pgfpathlineto{\pgfqpoint{0.000000in}{0.055556in}}%
\pgfusepath{stroke,fill}%
}%
\begin{pgfscope}%
\pgfsys@transformshift{3.262500in}{2.065000in}%
\pgfsys@useobject{currentmarker}{}%
\end{pgfscope}%
\end{pgfscope}%
\begin{pgfscope}%
\pgfsetbuttcap%
\pgfsetroundjoin%
\definecolor{currentfill}{rgb}{0.000000,0.000000,0.000000}%
\pgfsetfillcolor{currentfill}%
\pgfsetlinewidth{0.501875pt}%
\definecolor{currentstroke}{rgb}{0.000000,0.000000,0.000000}%
\pgfsetstrokecolor{currentstroke}%
\pgfsetdash{}{0pt}%
\pgfsys@defobject{currentmarker}{\pgfqpoint{0.000000in}{-0.055556in}}{\pgfqpoint{0.000000in}{0.000000in}}{%
\pgfpathmoveto{\pgfqpoint{0.000000in}{0.000000in}}%
\pgfpathlineto{\pgfqpoint{0.000000in}{-0.055556in}}%
\pgfusepath{stroke,fill}%
}%
\begin{pgfscope}%
\pgfsys@transformshift{3.262500in}{3.325000in}%
\pgfsys@useobject{currentmarker}{}%
\end{pgfscope}%
\end{pgfscope}%
\begin{pgfscope}%
\pgftext[x=3.262500in,y=2.009444in,,top]{\rmfamily\fontsize{9.000000}{10.800000}\selectfont \(\displaystyle 0.0012\)}%
\end{pgfscope}%
\begin{pgfscope}%
\pgfsetbuttcap%
\pgfsetroundjoin%
\definecolor{currentfill}{rgb}{0.000000,0.000000,0.000000}%
\pgfsetfillcolor{currentfill}%
\pgfsetlinewidth{0.501875pt}%
\definecolor{currentstroke}{rgb}{0.000000,0.000000,0.000000}%
\pgfsetstrokecolor{currentstroke}%
\pgfsetdash{}{0pt}%
\pgfsys@defobject{currentmarker}{\pgfqpoint{0.000000in}{0.000000in}}{\pgfqpoint{0.000000in}{0.055556in}}{%
\pgfpathmoveto{\pgfqpoint{0.000000in}{0.000000in}}%
\pgfpathlineto{\pgfqpoint{0.000000in}{0.055556in}}%
\pgfusepath{stroke,fill}%
}%
\begin{pgfscope}%
\pgfsys@transformshift{3.768750in}{2.065000in}%
\pgfsys@useobject{currentmarker}{}%
\end{pgfscope}%
\end{pgfscope}%
\begin{pgfscope}%
\pgfsetbuttcap%
\pgfsetroundjoin%
\definecolor{currentfill}{rgb}{0.000000,0.000000,0.000000}%
\pgfsetfillcolor{currentfill}%
\pgfsetlinewidth{0.501875pt}%
\definecolor{currentstroke}{rgb}{0.000000,0.000000,0.000000}%
\pgfsetstrokecolor{currentstroke}%
\pgfsetdash{}{0pt}%
\pgfsys@defobject{currentmarker}{\pgfqpoint{0.000000in}{-0.055556in}}{\pgfqpoint{0.000000in}{0.000000in}}{%
\pgfpathmoveto{\pgfqpoint{0.000000in}{0.000000in}}%
\pgfpathlineto{\pgfqpoint{0.000000in}{-0.055556in}}%
\pgfusepath{stroke,fill}%
}%
\begin{pgfscope}%
\pgfsys@transformshift{3.768750in}{3.325000in}%
\pgfsys@useobject{currentmarker}{}%
\end{pgfscope}%
\end{pgfscope}%
\begin{pgfscope}%
\pgftext[x=3.768750in,y=2.009444in,,top]{\rmfamily\fontsize{9.000000}{10.800000}\selectfont \(\displaystyle 0.0014\)}%
\end{pgfscope}%
\begin{pgfscope}%
\pgfsetbuttcap%
\pgfsetroundjoin%
\definecolor{currentfill}{rgb}{0.000000,0.000000,0.000000}%
\pgfsetfillcolor{currentfill}%
\pgfsetlinewidth{0.501875pt}%
\definecolor{currentstroke}{rgb}{0.000000,0.000000,0.000000}%
\pgfsetstrokecolor{currentstroke}%
\pgfsetdash{}{0pt}%
\pgfsys@defobject{currentmarker}{\pgfqpoint{0.000000in}{0.000000in}}{\pgfqpoint{0.000000in}{0.055556in}}{%
\pgfpathmoveto{\pgfqpoint{0.000000in}{0.000000in}}%
\pgfpathlineto{\pgfqpoint{0.000000in}{0.055556in}}%
\pgfusepath{stroke,fill}%
}%
\begin{pgfscope}%
\pgfsys@transformshift{4.275000in}{2.065000in}%
\pgfsys@useobject{currentmarker}{}%
\end{pgfscope}%
\end{pgfscope}%
\begin{pgfscope}%
\pgfsetbuttcap%
\pgfsetroundjoin%
\definecolor{currentfill}{rgb}{0.000000,0.000000,0.000000}%
\pgfsetfillcolor{currentfill}%
\pgfsetlinewidth{0.501875pt}%
\definecolor{currentstroke}{rgb}{0.000000,0.000000,0.000000}%
\pgfsetstrokecolor{currentstroke}%
\pgfsetdash{}{0pt}%
\pgfsys@defobject{currentmarker}{\pgfqpoint{0.000000in}{-0.055556in}}{\pgfqpoint{0.000000in}{0.000000in}}{%
\pgfpathmoveto{\pgfqpoint{0.000000in}{0.000000in}}%
\pgfpathlineto{\pgfqpoint{0.000000in}{-0.055556in}}%
\pgfusepath{stroke,fill}%
}%
\begin{pgfscope}%
\pgfsys@transformshift{4.275000in}{3.325000in}%
\pgfsys@useobject{currentmarker}{}%
\end{pgfscope}%
\end{pgfscope}%
\begin{pgfscope}%
\pgftext[x=4.275000in,y=2.009444in,,top]{\rmfamily\fontsize{9.000000}{10.800000}\selectfont \(\displaystyle 0.0016\)}%
\end{pgfscope}%
\begin{pgfscope}%
\pgfsetbuttcap%
\pgfsetroundjoin%
\definecolor{currentfill}{rgb}{0.000000,0.000000,0.000000}%
\pgfsetfillcolor{currentfill}%
\pgfsetlinewidth{0.501875pt}%
\definecolor{currentstroke}{rgb}{0.000000,0.000000,0.000000}%
\pgfsetstrokecolor{currentstroke}%
\pgfsetdash{}{0pt}%
\pgfsys@defobject{currentmarker}{\pgfqpoint{0.000000in}{0.000000in}}{\pgfqpoint{0.055556in}{0.000000in}}{%
\pgfpathmoveto{\pgfqpoint{0.000000in}{0.000000in}}%
\pgfpathlineto{\pgfqpoint{0.055556in}{0.000000in}}%
\pgfusepath{stroke,fill}%
}%
\begin{pgfscope}%
\pgfsys@transformshift{0.225000in}{2.170000in}%
\pgfsys@useobject{currentmarker}{}%
\end{pgfscope}%
\end{pgfscope}%
\begin{pgfscope}%
\pgfsetbuttcap%
\pgfsetroundjoin%
\definecolor{currentfill}{rgb}{0.000000,0.000000,0.000000}%
\pgfsetfillcolor{currentfill}%
\pgfsetlinewidth{0.501875pt}%
\definecolor{currentstroke}{rgb}{0.000000,0.000000,0.000000}%
\pgfsetstrokecolor{currentstroke}%
\pgfsetdash{}{0pt}%
\pgfsys@defobject{currentmarker}{\pgfqpoint{-0.055556in}{0.000000in}}{\pgfqpoint{0.000000in}{0.000000in}}{%
\pgfpathmoveto{\pgfqpoint{0.000000in}{0.000000in}}%
\pgfpathlineto{\pgfqpoint{-0.055556in}{0.000000in}}%
\pgfusepath{stroke,fill}%
}%
\begin{pgfscope}%
\pgfsys@transformshift{4.275000in}{2.170000in}%
\pgfsys@useobject{currentmarker}{}%
\end{pgfscope}%
\end{pgfscope}%
\begin{pgfscope}%
\pgftext[x=0.169444in,y=2.170000in,right,]{\rmfamily\fontsize{9.000000}{10.800000}\selectfont \(\displaystyle 0.0\)}%
\end{pgfscope}%
\begin{pgfscope}%
\pgfsetbuttcap%
\pgfsetroundjoin%
\definecolor{currentfill}{rgb}{0.000000,0.000000,0.000000}%
\pgfsetfillcolor{currentfill}%
\pgfsetlinewidth{0.501875pt}%
\definecolor{currentstroke}{rgb}{0.000000,0.000000,0.000000}%
\pgfsetstrokecolor{currentstroke}%
\pgfsetdash{}{0pt}%
\pgfsys@defobject{currentmarker}{\pgfqpoint{0.000000in}{0.000000in}}{\pgfqpoint{0.055556in}{0.000000in}}{%
\pgfpathmoveto{\pgfqpoint{0.000000in}{0.000000in}}%
\pgfpathlineto{\pgfqpoint{0.055556in}{0.000000in}}%
\pgfusepath{stroke,fill}%
}%
\begin{pgfscope}%
\pgfsys@transformshift{0.225000in}{2.380000in}%
\pgfsys@useobject{currentmarker}{}%
\end{pgfscope}%
\end{pgfscope}%
\begin{pgfscope}%
\pgfsetbuttcap%
\pgfsetroundjoin%
\definecolor{currentfill}{rgb}{0.000000,0.000000,0.000000}%
\pgfsetfillcolor{currentfill}%
\pgfsetlinewidth{0.501875pt}%
\definecolor{currentstroke}{rgb}{0.000000,0.000000,0.000000}%
\pgfsetstrokecolor{currentstroke}%
\pgfsetdash{}{0pt}%
\pgfsys@defobject{currentmarker}{\pgfqpoint{-0.055556in}{0.000000in}}{\pgfqpoint{0.000000in}{0.000000in}}{%
\pgfpathmoveto{\pgfqpoint{0.000000in}{0.000000in}}%
\pgfpathlineto{\pgfqpoint{-0.055556in}{0.000000in}}%
\pgfusepath{stroke,fill}%
}%
\begin{pgfscope}%
\pgfsys@transformshift{4.275000in}{2.380000in}%
\pgfsys@useobject{currentmarker}{}%
\end{pgfscope}%
\end{pgfscope}%
\begin{pgfscope}%
\pgftext[x=0.169444in,y=2.380000in,right,]{\rmfamily\fontsize{9.000000}{10.800000}\selectfont \(\displaystyle 0.2\)}%
\end{pgfscope}%
\begin{pgfscope}%
\pgfsetbuttcap%
\pgfsetroundjoin%
\definecolor{currentfill}{rgb}{0.000000,0.000000,0.000000}%
\pgfsetfillcolor{currentfill}%
\pgfsetlinewidth{0.501875pt}%
\definecolor{currentstroke}{rgb}{0.000000,0.000000,0.000000}%
\pgfsetstrokecolor{currentstroke}%
\pgfsetdash{}{0pt}%
\pgfsys@defobject{currentmarker}{\pgfqpoint{0.000000in}{0.000000in}}{\pgfqpoint{0.055556in}{0.000000in}}{%
\pgfpathmoveto{\pgfqpoint{0.000000in}{0.000000in}}%
\pgfpathlineto{\pgfqpoint{0.055556in}{0.000000in}}%
\pgfusepath{stroke,fill}%
}%
\begin{pgfscope}%
\pgfsys@transformshift{0.225000in}{2.590000in}%
\pgfsys@useobject{currentmarker}{}%
\end{pgfscope}%
\end{pgfscope}%
\begin{pgfscope}%
\pgfsetbuttcap%
\pgfsetroundjoin%
\definecolor{currentfill}{rgb}{0.000000,0.000000,0.000000}%
\pgfsetfillcolor{currentfill}%
\pgfsetlinewidth{0.501875pt}%
\definecolor{currentstroke}{rgb}{0.000000,0.000000,0.000000}%
\pgfsetstrokecolor{currentstroke}%
\pgfsetdash{}{0pt}%
\pgfsys@defobject{currentmarker}{\pgfqpoint{-0.055556in}{0.000000in}}{\pgfqpoint{0.000000in}{0.000000in}}{%
\pgfpathmoveto{\pgfqpoint{0.000000in}{0.000000in}}%
\pgfpathlineto{\pgfqpoint{-0.055556in}{0.000000in}}%
\pgfusepath{stroke,fill}%
}%
\begin{pgfscope}%
\pgfsys@transformshift{4.275000in}{2.590000in}%
\pgfsys@useobject{currentmarker}{}%
\end{pgfscope}%
\end{pgfscope}%
\begin{pgfscope}%
\pgftext[x=0.169444in,y=2.590000in,right,]{\rmfamily\fontsize{9.000000}{10.800000}\selectfont \(\displaystyle 0.4\)}%
\end{pgfscope}%
\begin{pgfscope}%
\pgfsetbuttcap%
\pgfsetroundjoin%
\definecolor{currentfill}{rgb}{0.000000,0.000000,0.000000}%
\pgfsetfillcolor{currentfill}%
\pgfsetlinewidth{0.501875pt}%
\definecolor{currentstroke}{rgb}{0.000000,0.000000,0.000000}%
\pgfsetstrokecolor{currentstroke}%
\pgfsetdash{}{0pt}%
\pgfsys@defobject{currentmarker}{\pgfqpoint{0.000000in}{0.000000in}}{\pgfqpoint{0.055556in}{0.000000in}}{%
\pgfpathmoveto{\pgfqpoint{0.000000in}{0.000000in}}%
\pgfpathlineto{\pgfqpoint{0.055556in}{0.000000in}}%
\pgfusepath{stroke,fill}%
}%
\begin{pgfscope}%
\pgfsys@transformshift{0.225000in}{2.800000in}%
\pgfsys@useobject{currentmarker}{}%
\end{pgfscope}%
\end{pgfscope}%
\begin{pgfscope}%
\pgfsetbuttcap%
\pgfsetroundjoin%
\definecolor{currentfill}{rgb}{0.000000,0.000000,0.000000}%
\pgfsetfillcolor{currentfill}%
\pgfsetlinewidth{0.501875pt}%
\definecolor{currentstroke}{rgb}{0.000000,0.000000,0.000000}%
\pgfsetstrokecolor{currentstroke}%
\pgfsetdash{}{0pt}%
\pgfsys@defobject{currentmarker}{\pgfqpoint{-0.055556in}{0.000000in}}{\pgfqpoint{0.000000in}{0.000000in}}{%
\pgfpathmoveto{\pgfqpoint{0.000000in}{0.000000in}}%
\pgfpathlineto{\pgfqpoint{-0.055556in}{0.000000in}}%
\pgfusepath{stroke,fill}%
}%
\begin{pgfscope}%
\pgfsys@transformshift{4.275000in}{2.800000in}%
\pgfsys@useobject{currentmarker}{}%
\end{pgfscope}%
\end{pgfscope}%
\begin{pgfscope}%
\pgftext[x=0.169444in,y=2.800000in,right,]{\rmfamily\fontsize{9.000000}{10.800000}\selectfont \(\displaystyle 0.6\)}%
\end{pgfscope}%
\begin{pgfscope}%
\pgfsetbuttcap%
\pgfsetroundjoin%
\definecolor{currentfill}{rgb}{0.000000,0.000000,0.000000}%
\pgfsetfillcolor{currentfill}%
\pgfsetlinewidth{0.501875pt}%
\definecolor{currentstroke}{rgb}{0.000000,0.000000,0.000000}%
\pgfsetstrokecolor{currentstroke}%
\pgfsetdash{}{0pt}%
\pgfsys@defobject{currentmarker}{\pgfqpoint{0.000000in}{0.000000in}}{\pgfqpoint{0.055556in}{0.000000in}}{%
\pgfpathmoveto{\pgfqpoint{0.000000in}{0.000000in}}%
\pgfpathlineto{\pgfqpoint{0.055556in}{0.000000in}}%
\pgfusepath{stroke,fill}%
}%
\begin{pgfscope}%
\pgfsys@transformshift{0.225000in}{3.010000in}%
\pgfsys@useobject{currentmarker}{}%
\end{pgfscope}%
\end{pgfscope}%
\begin{pgfscope}%
\pgfsetbuttcap%
\pgfsetroundjoin%
\definecolor{currentfill}{rgb}{0.000000,0.000000,0.000000}%
\pgfsetfillcolor{currentfill}%
\pgfsetlinewidth{0.501875pt}%
\definecolor{currentstroke}{rgb}{0.000000,0.000000,0.000000}%
\pgfsetstrokecolor{currentstroke}%
\pgfsetdash{}{0pt}%
\pgfsys@defobject{currentmarker}{\pgfqpoint{-0.055556in}{0.000000in}}{\pgfqpoint{0.000000in}{0.000000in}}{%
\pgfpathmoveto{\pgfqpoint{0.000000in}{0.000000in}}%
\pgfpathlineto{\pgfqpoint{-0.055556in}{0.000000in}}%
\pgfusepath{stroke,fill}%
}%
\begin{pgfscope}%
\pgfsys@transformshift{4.275000in}{3.010000in}%
\pgfsys@useobject{currentmarker}{}%
\end{pgfscope}%
\end{pgfscope}%
\begin{pgfscope}%
\pgftext[x=0.169444in,y=3.010000in,right,]{\rmfamily\fontsize{9.000000}{10.800000}\selectfont \(\displaystyle 0.8\)}%
\end{pgfscope}%
\begin{pgfscope}%
\pgfsetbuttcap%
\pgfsetroundjoin%
\definecolor{currentfill}{rgb}{0.000000,0.000000,0.000000}%
\pgfsetfillcolor{currentfill}%
\pgfsetlinewidth{0.501875pt}%
\definecolor{currentstroke}{rgb}{0.000000,0.000000,0.000000}%
\pgfsetstrokecolor{currentstroke}%
\pgfsetdash{}{0pt}%
\pgfsys@defobject{currentmarker}{\pgfqpoint{0.000000in}{0.000000in}}{\pgfqpoint{0.055556in}{0.000000in}}{%
\pgfpathmoveto{\pgfqpoint{0.000000in}{0.000000in}}%
\pgfpathlineto{\pgfqpoint{0.055556in}{0.000000in}}%
\pgfusepath{stroke,fill}%
}%
\begin{pgfscope}%
\pgfsys@transformshift{0.225000in}{3.220000in}%
\pgfsys@useobject{currentmarker}{}%
\end{pgfscope}%
\end{pgfscope}%
\begin{pgfscope}%
\pgfsetbuttcap%
\pgfsetroundjoin%
\definecolor{currentfill}{rgb}{0.000000,0.000000,0.000000}%
\pgfsetfillcolor{currentfill}%
\pgfsetlinewidth{0.501875pt}%
\definecolor{currentstroke}{rgb}{0.000000,0.000000,0.000000}%
\pgfsetstrokecolor{currentstroke}%
\pgfsetdash{}{0pt}%
\pgfsys@defobject{currentmarker}{\pgfqpoint{-0.055556in}{0.000000in}}{\pgfqpoint{0.000000in}{0.000000in}}{%
\pgfpathmoveto{\pgfqpoint{0.000000in}{0.000000in}}%
\pgfpathlineto{\pgfqpoint{-0.055556in}{0.000000in}}%
\pgfusepath{stroke,fill}%
}%
\begin{pgfscope}%
\pgfsys@transformshift{4.275000in}{3.220000in}%
\pgfsys@useobject{currentmarker}{}%
\end{pgfscope}%
\end{pgfscope}%
\begin{pgfscope}%
\pgftext[x=0.169444in,y=3.220000in,right,]{\rmfamily\fontsize{9.000000}{10.800000}\selectfont \(\displaystyle 1.0\)}%
\end{pgfscope}%
\begin{pgfscope}%
\pgftext[x=2.250000in,y=3.394444in,,base]{\rmfamily\fontsize{11.000000}{13.200000}\selectfont Daten}%
\end{pgfscope}%
\begin{pgfscope}%
\pgfsetbuttcap%
\pgfsetmiterjoin%
\pgfsetlinewidth{0.000000pt}%
\definecolor{currentstroke}{rgb}{0.000000,0.000000,0.000000}%
\pgfsetstrokecolor{currentstroke}%
\pgfsetstrokeopacity{0.000000}%
\pgfsetdash{}{0pt}%
\pgfpathmoveto{\pgfqpoint{0.225000in}{0.175000in}}%
\pgfpathlineto{\pgfqpoint{4.275000in}{0.175000in}}%
\pgfpathlineto{\pgfqpoint{4.275000in}{1.435000in}}%
\pgfpathlineto{\pgfqpoint{0.225000in}{1.435000in}}%
\pgfpathclose%
\pgfusepath{}%
\end{pgfscope}%
\begin{pgfscope}%
\pgfpathrectangle{\pgfqpoint{0.225000in}{0.175000in}}{\pgfqpoint{4.050000in}{1.260000in}} %
\pgfusepath{clip}%
\pgfsetrectcap%
\pgfsetroundjoin%
\pgfsetlinewidth{1.003750pt}%
\definecolor{currentstroke}{rgb}{0.000000,0.000000,1.000000}%
\pgfsetstrokecolor{currentstroke}%
\pgfsetdash{}{0pt}%
\pgfpathmoveto{\pgfqpoint{0.225000in}{0.805000in}}%
\pgfpathlineto{\pgfqpoint{0.246227in}{0.855175in}}%
\pgfpathlineto{\pgfqpoint{0.257805in}{0.877050in}}%
\pgfpathlineto{\pgfqpoint{0.266489in}{0.889267in}}%
\pgfpathlineto{\pgfqpoint{0.274208in}{0.896530in}}%
\pgfpathlineto{\pgfqpoint{0.280962in}{0.899860in}}%
\pgfpathlineto{\pgfqpoint{0.286751in}{0.900381in}}%
\pgfpathlineto{\pgfqpoint{0.292540in}{0.898732in}}%
\pgfpathlineto{\pgfqpoint{0.299294in}{0.894120in}}%
\pgfpathlineto{\pgfqpoint{0.306048in}{0.886751in}}%
\pgfpathlineto{\pgfqpoint{0.314732in}{0.873602in}}%
\pgfpathlineto{\pgfqpoint{0.325346in}{0.852857in}}%
\pgfpathlineto{\pgfqpoint{0.339819in}{0.819057in}}%
\pgfpathlineto{\pgfqpoint{0.368765in}{0.750557in}}%
\pgfpathlineto{\pgfqpoint{0.379378in}{0.731207in}}%
\pgfpathlineto{\pgfqpoint{0.388062in}{0.719498in}}%
\pgfpathlineto{\pgfqpoint{0.395781in}{0.712740in}}%
\pgfpathlineto{\pgfqpoint{0.402535in}{0.709877in}}%
\pgfpathlineto{\pgfqpoint{0.408324in}{0.709764in}}%
\pgfpathlineto{\pgfqpoint{0.414113in}{0.711816in}}%
\pgfpathlineto{\pgfqpoint{0.420867in}{0.716883in}}%
\pgfpathlineto{\pgfqpoint{0.428586in}{0.725999in}}%
\pgfpathlineto{\pgfqpoint{0.437270in}{0.740043in}}%
\pgfpathlineto{\pgfqpoint{0.447883in}{0.761624in}}%
\pgfpathlineto{\pgfqpoint{0.464286in}{0.800799in}}%
\pgfpathlineto{\pgfqpoint{0.487443in}{0.855685in}}%
\pgfpathlineto{\pgfqpoint{0.499021in}{0.877443in}}%
\pgfpathlineto{\pgfqpoint{0.507705in}{0.889548in}}%
\pgfpathlineto{\pgfqpoint{0.515424in}{0.896699in}}%
\pgfpathlineto{\pgfqpoint{0.522178in}{0.899925in}}%
\pgfpathlineto{\pgfqpoint{0.527967in}{0.900355in}}%
\pgfpathlineto{\pgfqpoint{0.533756in}{0.898617in}}%
\pgfpathlineto{\pgfqpoint{0.540510in}{0.893903in}}%
\pgfpathlineto{\pgfqpoint{0.547264in}{0.886439in}}%
\pgfpathlineto{\pgfqpoint{0.555948in}{0.873183in}}%
\pgfpathlineto{\pgfqpoint{0.566561in}{0.852337in}}%
\pgfpathlineto{\pgfqpoint{0.581999in}{0.816082in}}%
\pgfpathlineto{\pgfqpoint{0.609015in}{0.752046in}}%
\pgfpathlineto{\pgfqpoint{0.619629in}{0.732362in}}%
\pgfpathlineto{\pgfqpoint{0.628313in}{0.720313in}}%
\pgfpathlineto{\pgfqpoint{0.636031in}{0.713218in}}%
\pgfpathlineto{\pgfqpoint{0.642786in}{0.710044in}}%
\pgfpathlineto{\pgfqpoint{0.648575in}{0.709659in}}%
\pgfpathlineto{\pgfqpoint{0.654364in}{0.711442in}}%
\pgfpathlineto{\pgfqpoint{0.661118in}{0.716207in}}%
\pgfpathlineto{\pgfqpoint{0.668837in}{0.725002in}}%
\pgfpathlineto{\pgfqpoint{0.677521in}{0.738735in}}%
\pgfpathlineto{\pgfqpoint{0.688134in}{0.760028in}}%
\pgfpathlineto{\pgfqpoint{0.703572in}{0.796606in}}%
\pgfpathlineto{\pgfqpoint{0.728658in}{0.856193in}}%
\pgfpathlineto{\pgfqpoint{0.740237in}{0.877832in}}%
\pgfpathlineto{\pgfqpoint{0.748920in}{0.889825in}}%
\pgfpathlineto{\pgfqpoint{0.756639in}{0.896864in}}%
\pgfpathlineto{\pgfqpoint{0.763393in}{0.899987in}}%
\pgfpathlineto{\pgfqpoint{0.769183in}{0.900326in}}%
\pgfpathlineto{\pgfqpoint{0.774972in}{0.898498in}}%
\pgfpathlineto{\pgfqpoint{0.781726in}{0.893683in}}%
\pgfpathlineto{\pgfqpoint{0.789445in}{0.884833in}}%
\pgfpathlineto{\pgfqpoint{0.798128in}{0.871049in}}%
\pgfpathlineto{\pgfqpoint{0.808742in}{0.849707in}}%
\pgfpathlineto{\pgfqpoint{0.824180in}{0.813095in}}%
\pgfpathlineto{\pgfqpoint{0.849266in}{0.753554in}}%
\pgfpathlineto{\pgfqpoint{0.860844in}{0.731974in}}%
\pgfpathlineto{\pgfqpoint{0.869528in}{0.720038in}}%
\pgfpathlineto{\pgfqpoint{0.877247in}{0.713055in}}%
\pgfpathlineto{\pgfqpoint{0.884001in}{0.709984in}}%
\pgfpathlineto{\pgfqpoint{0.889790in}{0.709690in}}%
\pgfpathlineto{\pgfqpoint{0.895580in}{0.711563in}}%
\pgfpathlineto{\pgfqpoint{0.902334in}{0.716429in}}%
\pgfpathlineto{\pgfqpoint{0.910052in}{0.725331in}}%
\pgfpathlineto{\pgfqpoint{0.918736in}{0.739168in}}%
\pgfpathlineto{\pgfqpoint{0.929350in}{0.760558in}}%
\pgfpathlineto{\pgfqpoint{0.945752in}{0.799600in}}%
\pgfpathlineto{\pgfqpoint{0.969874in}{0.856698in}}%
\pgfpathlineto{\pgfqpoint{0.981452in}{0.878219in}}%
\pgfpathlineto{\pgfqpoint{0.990136in}{0.890098in}}%
\pgfpathlineto{\pgfqpoint{0.997855in}{0.897025in}}%
\pgfpathlineto{\pgfqpoint{1.004609in}{0.900044in}}%
\pgfpathlineto{\pgfqpoint{1.010398in}{0.900293in}}%
\pgfpathlineto{\pgfqpoint{1.016187in}{0.898375in}}%
\pgfpathlineto{\pgfqpoint{1.022941in}{0.893459in}}%
\pgfpathlineto{\pgfqpoint{1.030660in}{0.884503in}}%
\pgfpathlineto{\pgfqpoint{1.039344in}{0.870614in}}%
\pgfpathlineto{\pgfqpoint{1.049958in}{0.849176in}}%
\pgfpathlineto{\pgfqpoint{1.066360in}{0.810101in}}%
\pgfpathlineto{\pgfqpoint{1.090482in}{0.753049in}}%
\pgfpathlineto{\pgfqpoint{1.102060in}{0.731589in}}%
\pgfpathlineto{\pgfqpoint{1.110744in}{0.719766in}}%
\pgfpathlineto{\pgfqpoint{1.118463in}{0.712896in}}%
\pgfpathlineto{\pgfqpoint{1.125217in}{0.709929in}}%
\pgfpathlineto{\pgfqpoint{1.131006in}{0.709725in}}%
\pgfpathlineto{\pgfqpoint{1.136795in}{0.711688in}}%
\pgfpathlineto{\pgfqpoint{1.143549in}{0.716654in}}%
\pgfpathlineto{\pgfqpoint{1.151268in}{0.725664in}}%
\pgfpathlineto{\pgfqpoint{1.159952in}{0.739604in}}%
\pgfpathlineto{\pgfqpoint{1.170565in}{0.761090in}}%
\pgfpathlineto{\pgfqpoint{1.186968in}{0.800199in}}%
\pgfpathlineto{\pgfqpoint{1.188898in}{0.805000in}}%
\pgfpathlineto{\pgfqpoint{1.188898in}{0.805000in}}%
\pgfusepath{stroke}%
\end{pgfscope}%
\begin{pgfscope}%
\pgfpathrectangle{\pgfqpoint{0.225000in}{0.175000in}}{\pgfqpoint{4.050000in}{1.260000in}} %
\pgfusepath{clip}%
\pgfsetrectcap%
\pgfsetroundjoin%
\pgfsetlinewidth{1.003750pt}%
\definecolor{currentstroke}{rgb}{1.000000,0.000000,1.000000}%
\pgfsetstrokecolor{currentstroke}%
\pgfsetdash{}{0pt}%
\pgfpathmoveto{\pgfqpoint{1.188898in}{0.805000in}}%
\pgfpathlineto{\pgfqpoint{1.210125in}{1.106052in}}%
\pgfpathlineto{\pgfqpoint{1.220738in}{1.227715in}}%
\pgfpathlineto{\pgfqpoint{1.229422in}{1.303676in}}%
\pgfpathlineto{\pgfqpoint{1.236176in}{1.345311in}}%
\pgfpathlineto{\pgfqpoint{1.241965in}{1.367731in}}%
\pgfpathlineto{\pgfqpoint{1.245825in}{1.375587in}}%
\pgfpathlineto{\pgfqpoint{1.248719in}{1.377693in}}%
\pgfpathlineto{\pgfqpoint{1.251614in}{1.376538in}}%
\pgfpathlineto{\pgfqpoint{1.254508in}{1.372128in}}%
\pgfpathlineto{\pgfqpoint{1.258368in}{1.361234in}}%
\pgfpathlineto{\pgfqpoint{1.263192in}{1.339719in}}%
\pgfpathlineto{\pgfqpoint{1.268981in}{1.302788in}}%
\pgfpathlineto{\pgfqpoint{1.276700in}{1.236118in}}%
\pgfpathlineto{\pgfqpoint{1.285384in}{1.140474in}}%
\pgfpathlineto{\pgfqpoint{1.296962in}{0.987284in}}%
\pgfpathlineto{\pgfqpoint{1.338451in}{0.411316in}}%
\pgfpathlineto{\pgfqpoint{1.347135in}{0.327983in}}%
\pgfpathlineto{\pgfqpoint{1.354854in}{0.274247in}}%
\pgfpathlineto{\pgfqpoint{1.360643in}{0.247919in}}%
\pgfpathlineto{\pgfqpoint{1.365468in}{0.235640in}}%
\pgfpathlineto{\pgfqpoint{1.368362in}{0.232585in}}%
\pgfpathlineto{\pgfqpoint{1.371257in}{0.232789in}}%
\pgfpathlineto{\pgfqpoint{1.374151in}{0.236251in}}%
\pgfpathlineto{\pgfqpoint{1.378011in}{0.245897in}}%
\pgfpathlineto{\pgfqpoint{1.382835in}{0.265894in}}%
\pgfpathlineto{\pgfqpoint{1.388624in}{0.301099in}}%
\pgfpathlineto{\pgfqpoint{1.395378in}{0.356583in}}%
\pgfpathlineto{\pgfqpoint{1.404062in}{0.448011in}}%
\pgfpathlineto{\pgfqpoint{1.415640in}{0.597317in}}%
\pgfpathlineto{\pgfqpoint{1.437832in}{0.923019in}}%
\pgfpathlineto{\pgfqpoint{1.453270in}{1.133134in}}%
\pgfpathlineto{\pgfqpoint{1.463884in}{1.248900in}}%
\pgfpathlineto{\pgfqpoint{1.471602in}{1.312286in}}%
\pgfpathlineto{\pgfqpoint{1.478357in}{1.351016in}}%
\pgfpathlineto{\pgfqpoint{1.483181in}{1.368390in}}%
\pgfpathlineto{\pgfqpoint{1.487040in}{1.375887in}}%
\pgfpathlineto{\pgfqpoint{1.489935in}{1.377721in}}%
\pgfpathlineto{\pgfqpoint{1.492829in}{1.376294in}}%
\pgfpathlineto{\pgfqpoint{1.495724in}{1.371615in}}%
\pgfpathlineto{\pgfqpoint{1.499584in}{1.360364in}}%
\pgfpathlineto{\pgfqpoint{1.504408in}{1.338418in}}%
\pgfpathlineto{\pgfqpoint{1.510197in}{1.300996in}}%
\pgfpathlineto{\pgfqpoint{1.517916in}{1.233738in}}%
\pgfpathlineto{\pgfqpoint{1.527565in}{1.125713in}}%
\pgfpathlineto{\pgfqpoint{1.540108in}{0.956275in}}%
\pgfpathlineto{\pgfqpoint{1.576773in}{0.441013in}}%
\pgfpathlineto{\pgfqpoint{1.586421in}{0.342396in}}%
\pgfpathlineto{\pgfqpoint{1.594140in}{0.284233in}}%
\pgfpathlineto{\pgfqpoint{1.599929in}{0.254311in}}%
\pgfpathlineto{\pgfqpoint{1.604754in}{0.238921in}}%
\pgfpathlineto{\pgfqpoint{1.608613in}{0.233044in}}%
\pgfpathlineto{\pgfqpoint{1.611508in}{0.232432in}}%
\pgfpathlineto{\pgfqpoint{1.614402in}{0.235080in}}%
\pgfpathlineto{\pgfqpoint{1.618262in}{0.243654in}}%
\pgfpathlineto{\pgfqpoint{1.623086in}{0.262342in}}%
\pgfpathlineto{\pgfqpoint{1.628875in}{0.296052in}}%
\pgfpathlineto{\pgfqpoint{1.635629in}{0.349941in}}%
\pgfpathlineto{\pgfqpoint{1.644313in}{0.439625in}}%
\pgfpathlineto{\pgfqpoint{1.654926in}{0.574027in}}%
\pgfpathlineto{\pgfqpoint{1.673259in}{0.840998in}}%
\pgfpathlineto{\pgfqpoint{1.692556in}{1.112156in}}%
\pgfpathlineto{\pgfqpoint{1.703170in}{1.232542in}}%
\pgfpathlineto{\pgfqpoint{1.711853in}{1.307179in}}%
\pgfpathlineto{\pgfqpoint{1.718607in}{1.347658in}}%
\pgfpathlineto{\pgfqpoint{1.723432in}{1.366346in}}%
\pgfpathlineto{\pgfqpoint{1.727291in}{1.374920in}}%
\pgfpathlineto{\pgfqpoint{1.730186in}{1.377568in}}%
\pgfpathlineto{\pgfqpoint{1.733080in}{1.376956in}}%
\pgfpathlineto{\pgfqpoint{1.735975in}{1.373088in}}%
\pgfpathlineto{\pgfqpoint{1.739834in}{1.362906in}}%
\pgfpathlineto{\pgfqpoint{1.744659in}{1.342258in}}%
\pgfpathlineto{\pgfqpoint{1.750448in}{1.306311in}}%
\pgfpathlineto{\pgfqpoint{1.757202in}{1.250036in}}%
\pgfpathlineto{\pgfqpoint{1.765886in}{1.157748in}}%
\pgfpathlineto{\pgfqpoint{1.777464in}{1.007638in}}%
\pgfpathlineto{\pgfqpoint{1.801586in}{0.653725in}}%
\pgfpathlineto{\pgfqpoint{1.816058in}{0.460828in}}%
\pgfpathlineto{\pgfqpoint{1.825707in}{0.357705in}}%
\pgfpathlineto{\pgfqpoint{1.833426in}{0.295229in}}%
\pgfpathlineto{\pgfqpoint{1.840180in}{0.257377in}}%
\pgfpathlineto{\pgfqpoint{1.845004in}{0.240664in}}%
\pgfpathlineto{\pgfqpoint{1.848864in}{0.233706in}}%
\pgfpathlineto{\pgfqpoint{1.851758in}{0.232279in}}%
\pgfpathlineto{\pgfqpoint{1.854653in}{0.234113in}}%
\pgfpathlineto{\pgfqpoint{1.857548in}{0.239198in}}%
\pgfpathlineto{\pgfqpoint{1.861407in}{0.250981in}}%
\pgfpathlineto{\pgfqpoint{1.866231in}{0.273573in}}%
\pgfpathlineto{\pgfqpoint{1.872020in}{0.311727in}}%
\pgfpathlineto{\pgfqpoint{1.879739in}{0.379863in}}%
\pgfpathlineto{\pgfqpoint{1.889388in}{0.488778in}}%
\pgfpathlineto{\pgfqpoint{1.901931in}{0.658943in}}%
\pgfpathlineto{\pgfqpoint{1.937631in}{1.161989in}}%
\pgfpathlineto{\pgfqpoint{1.947280in}{1.262238in}}%
\pgfpathlineto{\pgfqpoint{1.954999in}{1.321955in}}%
\pgfpathlineto{\pgfqpoint{1.960788in}{1.353147in}}%
\pgfpathlineto{\pgfqpoint{1.965612in}{1.369641in}}%
\pgfpathlineto{\pgfqpoint{1.969472in}{1.376419in}}%
\pgfpathlineto{\pgfqpoint{1.972366in}{1.377710in}}%
\pgfpathlineto{\pgfqpoint{1.975261in}{1.375740in}}%
\pgfpathlineto{\pgfqpoint{1.978155in}{1.370520in}}%
\pgfpathlineto{\pgfqpoint{1.982015in}{1.358560in}}%
\pgfpathlineto{\pgfqpoint{1.986839in}{1.335753in}}%
\pgfpathlineto{\pgfqpoint{1.992628in}{1.297355in}}%
\pgfpathlineto{\pgfqpoint{2.000347in}{1.228928in}}%
\pgfpathlineto{\pgfqpoint{2.009996in}{1.119719in}}%
\pgfpathlineto{\pgfqpoint{2.022539in}{0.949315in}}%
\pgfpathlineto{\pgfqpoint{2.058239in}{0.446604in}}%
\pgfpathlineto{\pgfqpoint{2.067888in}{0.346680in}}%
\pgfpathlineto{\pgfqpoint{2.075606in}{0.287273in}}%
\pgfpathlineto{\pgfqpoint{2.081396in}{0.256333in}}%
\pgfpathlineto{\pgfqpoint{2.086220in}{0.240061in}}%
\pgfpathlineto{\pgfqpoint{2.090079in}{0.233462in}}%
\pgfpathlineto{\pgfqpoint{2.092974in}{0.232307in}}%
\pgfpathlineto{\pgfqpoint{2.095869in}{0.234413in}}%
\pgfpathlineto{\pgfqpoint{2.098763in}{0.239767in}}%
\pgfpathlineto{\pgfqpoint{2.102623in}{0.251905in}}%
\pgfpathlineto{\pgfqpoint{2.107447in}{0.274926in}}%
\pgfpathlineto{\pgfqpoint{2.113236in}{0.313567in}}%
\pgfpathlineto{\pgfqpoint{2.120955in}{0.382285in}}%
\pgfpathlineto{\pgfqpoint{2.130604in}{0.491788in}}%
\pgfpathlineto{\pgfqpoint{2.143147in}{0.662429in}}%
\pgfpathlineto{\pgfqpoint{2.152795in}{0.805000in}}%
\pgfpathlineto{\pgfqpoint{2.152795in}{0.805000in}}%
\pgfusepath{stroke}%
\end{pgfscope}%
\begin{pgfscope}%
\pgfpathrectangle{\pgfqpoint{0.225000in}{0.175000in}}{\pgfqpoint{4.050000in}{1.260000in}} %
\pgfusepath{clip}%
\pgfsetrectcap%
\pgfsetroundjoin%
\pgfsetlinewidth{1.003750pt}%
\definecolor{currentstroke}{rgb}{0.000000,0.000000,1.000000}%
\pgfsetstrokecolor{currentstroke}%
\pgfsetdash{}{0pt}%
\pgfpathmoveto{\pgfqpoint{2.152795in}{0.805000in}}%
\pgfpathlineto{\pgfqpoint{2.174022in}{0.855175in}}%
\pgfpathlineto{\pgfqpoint{2.185601in}{0.877050in}}%
\pgfpathlineto{\pgfqpoint{2.194285in}{0.889267in}}%
\pgfpathlineto{\pgfqpoint{2.202003in}{0.896530in}}%
\pgfpathlineto{\pgfqpoint{2.208758in}{0.899860in}}%
\pgfpathlineto{\pgfqpoint{2.214547in}{0.900381in}}%
\pgfpathlineto{\pgfqpoint{2.220336in}{0.898732in}}%
\pgfpathlineto{\pgfqpoint{2.227090in}{0.894120in}}%
\pgfpathlineto{\pgfqpoint{2.233844in}{0.886751in}}%
\pgfpathlineto{\pgfqpoint{2.242528in}{0.873602in}}%
\pgfpathlineto{\pgfqpoint{2.253141in}{0.852857in}}%
\pgfpathlineto{\pgfqpoint{2.267614in}{0.819057in}}%
\pgfpathlineto{\pgfqpoint{2.296560in}{0.750557in}}%
\pgfpathlineto{\pgfqpoint{2.307174in}{0.731207in}}%
\pgfpathlineto{\pgfqpoint{2.315857in}{0.719498in}}%
\pgfpathlineto{\pgfqpoint{2.323576in}{0.712740in}}%
\pgfpathlineto{\pgfqpoint{2.330330in}{0.709877in}}%
\pgfpathlineto{\pgfqpoint{2.336119in}{0.709764in}}%
\pgfpathlineto{\pgfqpoint{2.341909in}{0.711816in}}%
\pgfpathlineto{\pgfqpoint{2.348663in}{0.716883in}}%
\pgfpathlineto{\pgfqpoint{2.356382in}{0.725999in}}%
\pgfpathlineto{\pgfqpoint{2.365065in}{0.740043in}}%
\pgfpathlineto{\pgfqpoint{2.375679in}{0.761624in}}%
\pgfpathlineto{\pgfqpoint{2.392081in}{0.800799in}}%
\pgfpathlineto{\pgfqpoint{2.415238in}{0.855685in}}%
\pgfpathlineto{\pgfqpoint{2.426816in}{0.877443in}}%
\pgfpathlineto{\pgfqpoint{2.435500in}{0.889548in}}%
\pgfpathlineto{\pgfqpoint{2.443219in}{0.896699in}}%
\pgfpathlineto{\pgfqpoint{2.449973in}{0.899925in}}%
\pgfpathlineto{\pgfqpoint{2.455762in}{0.900355in}}%
\pgfpathlineto{\pgfqpoint{2.461552in}{0.898617in}}%
\pgfpathlineto{\pgfqpoint{2.468306in}{0.893903in}}%
\pgfpathlineto{\pgfqpoint{2.475060in}{0.886439in}}%
\pgfpathlineto{\pgfqpoint{2.483743in}{0.873183in}}%
\pgfpathlineto{\pgfqpoint{2.494357in}{0.852337in}}%
\pgfpathlineto{\pgfqpoint{2.509795in}{0.816082in}}%
\pgfpathlineto{\pgfqpoint{2.536811in}{0.752046in}}%
\pgfpathlineto{\pgfqpoint{2.547424in}{0.732362in}}%
\pgfpathlineto{\pgfqpoint{2.556108in}{0.720313in}}%
\pgfpathlineto{\pgfqpoint{2.563827in}{0.713218in}}%
\pgfpathlineto{\pgfqpoint{2.570581in}{0.710044in}}%
\pgfpathlineto{\pgfqpoint{2.576370in}{0.709659in}}%
\pgfpathlineto{\pgfqpoint{2.582159in}{0.711442in}}%
\pgfpathlineto{\pgfqpoint{2.588913in}{0.716207in}}%
\pgfpathlineto{\pgfqpoint{2.596632in}{0.725002in}}%
\pgfpathlineto{\pgfqpoint{2.605316in}{0.738735in}}%
\pgfpathlineto{\pgfqpoint{2.615930in}{0.760028in}}%
\pgfpathlineto{\pgfqpoint{2.631367in}{0.796606in}}%
\pgfpathlineto{\pgfqpoint{2.656454in}{0.856193in}}%
\pgfpathlineto{\pgfqpoint{2.668032in}{0.877832in}}%
\pgfpathlineto{\pgfqpoint{2.676716in}{0.889825in}}%
\pgfpathlineto{\pgfqpoint{2.684435in}{0.896864in}}%
\pgfpathlineto{\pgfqpoint{2.691189in}{0.899987in}}%
\pgfpathlineto{\pgfqpoint{2.696978in}{0.900326in}}%
\pgfpathlineto{\pgfqpoint{2.702767in}{0.898498in}}%
\pgfpathlineto{\pgfqpoint{2.709521in}{0.893683in}}%
\pgfpathlineto{\pgfqpoint{2.717240in}{0.884833in}}%
\pgfpathlineto{\pgfqpoint{2.725924in}{0.871049in}}%
\pgfpathlineto{\pgfqpoint{2.736537in}{0.849707in}}%
\pgfpathlineto{\pgfqpoint{2.751975in}{0.813095in}}%
\pgfpathlineto{\pgfqpoint{2.777062in}{0.753554in}}%
\pgfpathlineto{\pgfqpoint{2.788640in}{0.731974in}}%
\pgfpathlineto{\pgfqpoint{2.797324in}{0.720038in}}%
\pgfpathlineto{\pgfqpoint{2.805043in}{0.713055in}}%
\pgfpathlineto{\pgfqpoint{2.811797in}{0.709984in}}%
\pgfpathlineto{\pgfqpoint{2.817586in}{0.709690in}}%
\pgfpathlineto{\pgfqpoint{2.823375in}{0.711563in}}%
\pgfpathlineto{\pgfqpoint{2.830129in}{0.716429in}}%
\pgfpathlineto{\pgfqpoint{2.837848in}{0.725331in}}%
\pgfpathlineto{\pgfqpoint{2.846532in}{0.739168in}}%
\pgfpathlineto{\pgfqpoint{2.857145in}{0.760558in}}%
\pgfpathlineto{\pgfqpoint{2.873548in}{0.799600in}}%
\pgfpathlineto{\pgfqpoint{2.897669in}{0.856698in}}%
\pgfpathlineto{\pgfqpoint{2.909248in}{0.878219in}}%
\pgfpathlineto{\pgfqpoint{2.917932in}{0.890098in}}%
\pgfpathlineto{\pgfqpoint{2.925650in}{0.897025in}}%
\pgfpathlineto{\pgfqpoint{2.932404in}{0.900044in}}%
\pgfpathlineto{\pgfqpoint{2.938194in}{0.900293in}}%
\pgfpathlineto{\pgfqpoint{2.943983in}{0.898375in}}%
\pgfpathlineto{\pgfqpoint{2.950737in}{0.893459in}}%
\pgfpathlineto{\pgfqpoint{2.958456in}{0.884503in}}%
\pgfpathlineto{\pgfqpoint{2.967140in}{0.870614in}}%
\pgfpathlineto{\pgfqpoint{2.977753in}{0.849176in}}%
\pgfpathlineto{\pgfqpoint{2.994156in}{0.810101in}}%
\pgfpathlineto{\pgfqpoint{3.018277in}{0.753049in}}%
\pgfpathlineto{\pgfqpoint{3.029856in}{0.731589in}}%
\pgfpathlineto{\pgfqpoint{3.038539in}{0.719766in}}%
\pgfpathlineto{\pgfqpoint{3.046258in}{0.712896in}}%
\pgfpathlineto{\pgfqpoint{3.053012in}{0.709929in}}%
\pgfpathlineto{\pgfqpoint{3.058801in}{0.709725in}}%
\pgfpathlineto{\pgfqpoint{3.064591in}{0.711688in}}%
\pgfpathlineto{\pgfqpoint{3.071345in}{0.716654in}}%
\pgfpathlineto{\pgfqpoint{3.079064in}{0.725664in}}%
\pgfpathlineto{\pgfqpoint{3.087747in}{0.739604in}}%
\pgfpathlineto{\pgfqpoint{3.098361in}{0.761090in}}%
\pgfpathlineto{\pgfqpoint{3.114764in}{0.800199in}}%
\pgfpathlineto{\pgfqpoint{3.116693in}{0.805000in}}%
\pgfpathlineto{\pgfqpoint{3.116693in}{0.805000in}}%
\pgfusepath{stroke}%
\end{pgfscope}%
\begin{pgfscope}%
\pgfpathrectangle{\pgfqpoint{0.225000in}{0.175000in}}{\pgfqpoint{4.050000in}{1.260000in}} %
\pgfusepath{clip}%
\pgfsetrectcap%
\pgfsetroundjoin%
\pgfsetlinewidth{1.003750pt}%
\definecolor{currentstroke}{rgb}{1.000000,0.000000,1.000000}%
\pgfsetstrokecolor{currentstroke}%
\pgfsetdash{}{0pt}%
\pgfpathmoveto{\pgfqpoint{3.116693in}{0.805000in}}%
\pgfpathlineto{\pgfqpoint{3.137920in}{1.106052in}}%
\pgfpathlineto{\pgfqpoint{3.148534in}{1.227715in}}%
\pgfpathlineto{\pgfqpoint{3.157217in}{1.303676in}}%
\pgfpathlineto{\pgfqpoint{3.163972in}{1.345311in}}%
\pgfpathlineto{\pgfqpoint{3.169761in}{1.367731in}}%
\pgfpathlineto{\pgfqpoint{3.173620in}{1.375587in}}%
\pgfpathlineto{\pgfqpoint{3.176515in}{1.377693in}}%
\pgfpathlineto{\pgfqpoint{3.179409in}{1.376538in}}%
\pgfpathlineto{\pgfqpoint{3.182304in}{1.372128in}}%
\pgfpathlineto{\pgfqpoint{3.186163in}{1.361234in}}%
\pgfpathlineto{\pgfqpoint{3.190988in}{1.339719in}}%
\pgfpathlineto{\pgfqpoint{3.196777in}{1.302788in}}%
\pgfpathlineto{\pgfqpoint{3.204496in}{1.236118in}}%
\pgfpathlineto{\pgfqpoint{3.213179in}{1.140474in}}%
\pgfpathlineto{\pgfqpoint{3.224758in}{0.987284in}}%
\pgfpathlineto{\pgfqpoint{3.266247in}{0.411316in}}%
\pgfpathlineto{\pgfqpoint{3.274931in}{0.327983in}}%
\pgfpathlineto{\pgfqpoint{3.282650in}{0.274247in}}%
\pgfpathlineto{\pgfqpoint{3.288439in}{0.247919in}}%
\pgfpathlineto{\pgfqpoint{3.293263in}{0.235640in}}%
\pgfpathlineto{\pgfqpoint{3.296158in}{0.232585in}}%
\pgfpathlineto{\pgfqpoint{3.299052in}{0.232789in}}%
\pgfpathlineto{\pgfqpoint{3.301947in}{0.236251in}}%
\pgfpathlineto{\pgfqpoint{3.305806in}{0.245897in}}%
\pgfpathlineto{\pgfqpoint{3.310631in}{0.265894in}}%
\pgfpathlineto{\pgfqpoint{3.316420in}{0.301099in}}%
\pgfpathlineto{\pgfqpoint{3.323174in}{0.356583in}}%
\pgfpathlineto{\pgfqpoint{3.331858in}{0.448011in}}%
\pgfpathlineto{\pgfqpoint{3.343436in}{0.597317in}}%
\pgfpathlineto{\pgfqpoint{3.365628in}{0.923019in}}%
\pgfpathlineto{\pgfqpoint{3.381066in}{1.133134in}}%
\pgfpathlineto{\pgfqpoint{3.391679in}{1.248900in}}%
\pgfpathlineto{\pgfqpoint{3.399398in}{1.312286in}}%
\pgfpathlineto{\pgfqpoint{3.406152in}{1.351016in}}%
\pgfpathlineto{\pgfqpoint{3.410976in}{1.368390in}}%
\pgfpathlineto{\pgfqpoint{3.414836in}{1.375887in}}%
\pgfpathlineto{\pgfqpoint{3.417730in}{1.377721in}}%
\pgfpathlineto{\pgfqpoint{3.420625in}{1.376294in}}%
\pgfpathlineto{\pgfqpoint{3.423520in}{1.371615in}}%
\pgfpathlineto{\pgfqpoint{3.427379in}{1.360364in}}%
\pgfpathlineto{\pgfqpoint{3.432203in}{1.338418in}}%
\pgfpathlineto{\pgfqpoint{3.437992in}{1.300996in}}%
\pgfpathlineto{\pgfqpoint{3.445711in}{1.233738in}}%
\pgfpathlineto{\pgfqpoint{3.455360in}{1.125713in}}%
\pgfpathlineto{\pgfqpoint{3.467903in}{0.956275in}}%
\pgfpathlineto{\pgfqpoint{3.504568in}{0.441013in}}%
\pgfpathlineto{\pgfqpoint{3.514217in}{0.342396in}}%
\pgfpathlineto{\pgfqpoint{3.521936in}{0.284233in}}%
\pgfpathlineto{\pgfqpoint{3.527725in}{0.254311in}}%
\pgfpathlineto{\pgfqpoint{3.532549in}{0.238921in}}%
\pgfpathlineto{\pgfqpoint{3.536408in}{0.233044in}}%
\pgfpathlineto{\pgfqpoint{3.539303in}{0.232432in}}%
\pgfpathlineto{\pgfqpoint{3.542198in}{0.235080in}}%
\pgfpathlineto{\pgfqpoint{3.546057in}{0.243654in}}%
\pgfpathlineto{\pgfqpoint{3.550881in}{0.262342in}}%
\pgfpathlineto{\pgfqpoint{3.556671in}{0.296052in}}%
\pgfpathlineto{\pgfqpoint{3.563425in}{0.349941in}}%
\pgfpathlineto{\pgfqpoint{3.572108in}{0.439625in}}%
\pgfpathlineto{\pgfqpoint{3.582722in}{0.574027in}}%
\pgfpathlineto{\pgfqpoint{3.601054in}{0.840998in}}%
\pgfpathlineto{\pgfqpoint{3.620352in}{1.112156in}}%
\pgfpathlineto{\pgfqpoint{3.630965in}{1.232542in}}%
\pgfpathlineto{\pgfqpoint{3.639649in}{1.307179in}}%
\pgfpathlineto{\pgfqpoint{3.646403in}{1.347658in}}%
\pgfpathlineto{\pgfqpoint{3.651227in}{1.366346in}}%
\pgfpathlineto{\pgfqpoint{3.655087in}{1.374920in}}%
\pgfpathlineto{\pgfqpoint{3.657981in}{1.377568in}}%
\pgfpathlineto{\pgfqpoint{3.660876in}{1.376956in}}%
\pgfpathlineto{\pgfqpoint{3.663770in}{1.373088in}}%
\pgfpathlineto{\pgfqpoint{3.667630in}{1.362906in}}%
\pgfpathlineto{\pgfqpoint{3.672454in}{1.342258in}}%
\pgfpathlineto{\pgfqpoint{3.678243in}{1.306311in}}%
\pgfpathlineto{\pgfqpoint{3.684997in}{1.250036in}}%
\pgfpathlineto{\pgfqpoint{3.693681in}{1.157748in}}%
\pgfpathlineto{\pgfqpoint{3.705259in}{1.007638in}}%
\pgfpathlineto{\pgfqpoint{3.729381in}{0.653725in}}%
\pgfpathlineto{\pgfqpoint{3.743854in}{0.460828in}}%
\pgfpathlineto{\pgfqpoint{3.753503in}{0.357705in}}%
\pgfpathlineto{\pgfqpoint{3.761221in}{0.295229in}}%
\pgfpathlineto{\pgfqpoint{3.767975in}{0.257377in}}%
\pgfpathlineto{\pgfqpoint{3.772800in}{0.240664in}}%
\pgfpathlineto{\pgfqpoint{3.776659in}{0.233706in}}%
\pgfpathlineto{\pgfqpoint{3.779554in}{0.232279in}}%
\pgfpathlineto{\pgfqpoint{3.782448in}{0.234113in}}%
\pgfpathlineto{\pgfqpoint{3.785343in}{0.239198in}}%
\pgfpathlineto{\pgfqpoint{3.789202in}{0.250981in}}%
\pgfpathlineto{\pgfqpoint{3.794027in}{0.273573in}}%
\pgfpathlineto{\pgfqpoint{3.799816in}{0.311727in}}%
\pgfpathlineto{\pgfqpoint{3.807535in}{0.379863in}}%
\pgfpathlineto{\pgfqpoint{3.817183in}{0.488778in}}%
\pgfpathlineto{\pgfqpoint{3.829727in}{0.658943in}}%
\pgfpathlineto{\pgfqpoint{3.865427in}{1.161989in}}%
\pgfpathlineto{\pgfqpoint{3.875075in}{1.262238in}}%
\pgfpathlineto{\pgfqpoint{3.882794in}{1.321955in}}%
\pgfpathlineto{\pgfqpoint{3.888583in}{1.353147in}}%
\pgfpathlineto{\pgfqpoint{3.893408in}{1.369641in}}%
\pgfpathlineto{\pgfqpoint{3.897267in}{1.376419in}}%
\pgfpathlineto{\pgfqpoint{3.900162in}{1.377710in}}%
\pgfpathlineto{\pgfqpoint{3.903056in}{1.375740in}}%
\pgfpathlineto{\pgfqpoint{3.905951in}{1.370520in}}%
\pgfpathlineto{\pgfqpoint{3.909810in}{1.358560in}}%
\pgfpathlineto{\pgfqpoint{3.914635in}{1.335753in}}%
\pgfpathlineto{\pgfqpoint{3.920424in}{1.297355in}}%
\pgfpathlineto{\pgfqpoint{3.928143in}{1.228928in}}%
\pgfpathlineto{\pgfqpoint{3.937791in}{1.119719in}}%
\pgfpathlineto{\pgfqpoint{3.950335in}{0.949315in}}%
\pgfpathlineto{\pgfqpoint{3.986034in}{0.446604in}}%
\pgfpathlineto{\pgfqpoint{3.995683in}{0.346680in}}%
\pgfpathlineto{\pgfqpoint{4.003402in}{0.287273in}}%
\pgfpathlineto{\pgfqpoint{4.009191in}{0.256333in}}%
\pgfpathlineto{\pgfqpoint{4.014015in}{0.240061in}}%
\pgfpathlineto{\pgfqpoint{4.017875in}{0.233462in}}%
\pgfpathlineto{\pgfqpoint{4.020770in}{0.232307in}}%
\pgfpathlineto{\pgfqpoint{4.023664in}{0.234413in}}%
\pgfpathlineto{\pgfqpoint{4.026559in}{0.239767in}}%
\pgfpathlineto{\pgfqpoint{4.030418in}{0.251905in}}%
\pgfpathlineto{\pgfqpoint{4.035242in}{0.274926in}}%
\pgfpathlineto{\pgfqpoint{4.041032in}{0.313567in}}%
\pgfpathlineto{\pgfqpoint{4.048751in}{0.382285in}}%
\pgfpathlineto{\pgfqpoint{4.058399in}{0.491788in}}%
\pgfpathlineto{\pgfqpoint{4.070942in}{0.662429in}}%
\pgfpathlineto{\pgfqpoint{4.080591in}{0.805000in}}%
\pgfpathlineto{\pgfqpoint{4.080591in}{0.805000in}}%
\pgfusepath{stroke}%
\end{pgfscope}%
\begin{pgfscope}%
\pgfsetrectcap%
\pgfsetmiterjoin%
\pgfsetlinewidth{1.003750pt}%
\definecolor{currentstroke}{rgb}{0.000000,0.000000,0.000000}%
\pgfsetstrokecolor{currentstroke}%
\pgfsetdash{}{0pt}%
\pgfpathmoveto{\pgfqpoint{0.225000in}{0.175000in}}%
\pgfpathlineto{\pgfqpoint{0.225000in}{1.435000in}}%
\pgfusepath{stroke}%
\end{pgfscope}%
\begin{pgfscope}%
\pgfsetrectcap%
\pgfsetmiterjoin%
\pgfsetlinewidth{1.003750pt}%
\definecolor{currentstroke}{rgb}{0.000000,0.000000,0.000000}%
\pgfsetstrokecolor{currentstroke}%
\pgfsetdash{}{0pt}%
\pgfpathmoveto{\pgfqpoint{0.225000in}{1.435000in}}%
\pgfpathlineto{\pgfqpoint{4.275000in}{1.435000in}}%
\pgfusepath{stroke}%
\end{pgfscope}%
\begin{pgfscope}%
\pgfsetrectcap%
\pgfsetmiterjoin%
\pgfsetlinewidth{1.003750pt}%
\definecolor{currentstroke}{rgb}{0.000000,0.000000,0.000000}%
\pgfsetstrokecolor{currentstroke}%
\pgfsetdash{}{0pt}%
\pgfpathmoveto{\pgfqpoint{4.275000in}{0.175000in}}%
\pgfpathlineto{\pgfqpoint{4.275000in}{1.435000in}}%
\pgfusepath{stroke}%
\end{pgfscope}%
\begin{pgfscope}%
\pgfsetrectcap%
\pgfsetmiterjoin%
\pgfsetlinewidth{1.003750pt}%
\definecolor{currentstroke}{rgb}{0.000000,0.000000,0.000000}%
\pgfsetstrokecolor{currentstroke}%
\pgfsetdash{}{0pt}%
\pgfpathmoveto{\pgfqpoint{0.225000in}{0.175000in}}%
\pgfpathlineto{\pgfqpoint{4.275000in}{0.175000in}}%
\pgfusepath{stroke}%
\end{pgfscope}%
\begin{pgfscope}%
\pgfsetbuttcap%
\pgfsetroundjoin%
\definecolor{currentfill}{rgb}{0.000000,0.000000,0.000000}%
\pgfsetfillcolor{currentfill}%
\pgfsetlinewidth{0.501875pt}%
\definecolor{currentstroke}{rgb}{0.000000,0.000000,0.000000}%
\pgfsetstrokecolor{currentstroke}%
\pgfsetdash{}{0pt}%
\pgfsys@defobject{currentmarker}{\pgfqpoint{0.000000in}{0.000000in}}{\pgfqpoint{0.000000in}{0.055556in}}{%
\pgfpathmoveto{\pgfqpoint{0.000000in}{0.000000in}}%
\pgfpathlineto{\pgfqpoint{0.000000in}{0.055556in}}%
\pgfusepath{stroke,fill}%
}%
\begin{pgfscope}%
\pgfsys@transformshift{0.225000in}{0.175000in}%
\pgfsys@useobject{currentmarker}{}%
\end{pgfscope}%
\end{pgfscope}%
\begin{pgfscope}%
\pgfsetbuttcap%
\pgfsetroundjoin%
\definecolor{currentfill}{rgb}{0.000000,0.000000,0.000000}%
\pgfsetfillcolor{currentfill}%
\pgfsetlinewidth{0.501875pt}%
\definecolor{currentstroke}{rgb}{0.000000,0.000000,0.000000}%
\pgfsetstrokecolor{currentstroke}%
\pgfsetdash{}{0pt}%
\pgfsys@defobject{currentmarker}{\pgfqpoint{0.000000in}{-0.055556in}}{\pgfqpoint{0.000000in}{0.000000in}}{%
\pgfpathmoveto{\pgfqpoint{0.000000in}{0.000000in}}%
\pgfpathlineto{\pgfqpoint{0.000000in}{-0.055556in}}%
\pgfusepath{stroke,fill}%
}%
\begin{pgfscope}%
\pgfsys@transformshift{0.225000in}{1.435000in}%
\pgfsys@useobject{currentmarker}{}%
\end{pgfscope}%
\end{pgfscope}%
\begin{pgfscope}%
\pgftext[x=0.225000in,y=0.119444in,,top]{\rmfamily\fontsize{9.000000}{10.800000}\selectfont \(\displaystyle 0.0000\)}%
\end{pgfscope}%
\begin{pgfscope}%
\pgfsetbuttcap%
\pgfsetroundjoin%
\definecolor{currentfill}{rgb}{0.000000,0.000000,0.000000}%
\pgfsetfillcolor{currentfill}%
\pgfsetlinewidth{0.501875pt}%
\definecolor{currentstroke}{rgb}{0.000000,0.000000,0.000000}%
\pgfsetstrokecolor{currentstroke}%
\pgfsetdash{}{0pt}%
\pgfsys@defobject{currentmarker}{\pgfqpoint{0.000000in}{0.000000in}}{\pgfqpoint{0.000000in}{0.055556in}}{%
\pgfpathmoveto{\pgfqpoint{0.000000in}{0.000000in}}%
\pgfpathlineto{\pgfqpoint{0.000000in}{0.055556in}}%
\pgfusepath{stroke,fill}%
}%
\begin{pgfscope}%
\pgfsys@transformshift{0.731250in}{0.175000in}%
\pgfsys@useobject{currentmarker}{}%
\end{pgfscope}%
\end{pgfscope}%
\begin{pgfscope}%
\pgfsetbuttcap%
\pgfsetroundjoin%
\definecolor{currentfill}{rgb}{0.000000,0.000000,0.000000}%
\pgfsetfillcolor{currentfill}%
\pgfsetlinewidth{0.501875pt}%
\definecolor{currentstroke}{rgb}{0.000000,0.000000,0.000000}%
\pgfsetstrokecolor{currentstroke}%
\pgfsetdash{}{0pt}%
\pgfsys@defobject{currentmarker}{\pgfqpoint{0.000000in}{-0.055556in}}{\pgfqpoint{0.000000in}{0.000000in}}{%
\pgfpathmoveto{\pgfqpoint{0.000000in}{0.000000in}}%
\pgfpathlineto{\pgfqpoint{0.000000in}{-0.055556in}}%
\pgfusepath{stroke,fill}%
}%
\begin{pgfscope}%
\pgfsys@transformshift{0.731250in}{1.435000in}%
\pgfsys@useobject{currentmarker}{}%
\end{pgfscope}%
\end{pgfscope}%
\begin{pgfscope}%
\pgftext[x=0.731250in,y=0.119444in,,top]{\rmfamily\fontsize{9.000000}{10.800000}\selectfont \(\displaystyle 0.0002\)}%
\end{pgfscope}%
\begin{pgfscope}%
\pgfsetbuttcap%
\pgfsetroundjoin%
\definecolor{currentfill}{rgb}{0.000000,0.000000,0.000000}%
\pgfsetfillcolor{currentfill}%
\pgfsetlinewidth{0.501875pt}%
\definecolor{currentstroke}{rgb}{0.000000,0.000000,0.000000}%
\pgfsetstrokecolor{currentstroke}%
\pgfsetdash{}{0pt}%
\pgfsys@defobject{currentmarker}{\pgfqpoint{0.000000in}{0.000000in}}{\pgfqpoint{0.000000in}{0.055556in}}{%
\pgfpathmoveto{\pgfqpoint{0.000000in}{0.000000in}}%
\pgfpathlineto{\pgfqpoint{0.000000in}{0.055556in}}%
\pgfusepath{stroke,fill}%
}%
\begin{pgfscope}%
\pgfsys@transformshift{1.237500in}{0.175000in}%
\pgfsys@useobject{currentmarker}{}%
\end{pgfscope}%
\end{pgfscope}%
\begin{pgfscope}%
\pgfsetbuttcap%
\pgfsetroundjoin%
\definecolor{currentfill}{rgb}{0.000000,0.000000,0.000000}%
\pgfsetfillcolor{currentfill}%
\pgfsetlinewidth{0.501875pt}%
\definecolor{currentstroke}{rgb}{0.000000,0.000000,0.000000}%
\pgfsetstrokecolor{currentstroke}%
\pgfsetdash{}{0pt}%
\pgfsys@defobject{currentmarker}{\pgfqpoint{0.000000in}{-0.055556in}}{\pgfqpoint{0.000000in}{0.000000in}}{%
\pgfpathmoveto{\pgfqpoint{0.000000in}{0.000000in}}%
\pgfpathlineto{\pgfqpoint{0.000000in}{-0.055556in}}%
\pgfusepath{stroke,fill}%
}%
\begin{pgfscope}%
\pgfsys@transformshift{1.237500in}{1.435000in}%
\pgfsys@useobject{currentmarker}{}%
\end{pgfscope}%
\end{pgfscope}%
\begin{pgfscope}%
\pgftext[x=1.237500in,y=0.119444in,,top]{\rmfamily\fontsize{9.000000}{10.800000}\selectfont \(\displaystyle 0.0004\)}%
\end{pgfscope}%
\begin{pgfscope}%
\pgfsetbuttcap%
\pgfsetroundjoin%
\definecolor{currentfill}{rgb}{0.000000,0.000000,0.000000}%
\pgfsetfillcolor{currentfill}%
\pgfsetlinewidth{0.501875pt}%
\definecolor{currentstroke}{rgb}{0.000000,0.000000,0.000000}%
\pgfsetstrokecolor{currentstroke}%
\pgfsetdash{}{0pt}%
\pgfsys@defobject{currentmarker}{\pgfqpoint{0.000000in}{0.000000in}}{\pgfqpoint{0.000000in}{0.055556in}}{%
\pgfpathmoveto{\pgfqpoint{0.000000in}{0.000000in}}%
\pgfpathlineto{\pgfqpoint{0.000000in}{0.055556in}}%
\pgfusepath{stroke,fill}%
}%
\begin{pgfscope}%
\pgfsys@transformshift{1.743750in}{0.175000in}%
\pgfsys@useobject{currentmarker}{}%
\end{pgfscope}%
\end{pgfscope}%
\begin{pgfscope}%
\pgfsetbuttcap%
\pgfsetroundjoin%
\definecolor{currentfill}{rgb}{0.000000,0.000000,0.000000}%
\pgfsetfillcolor{currentfill}%
\pgfsetlinewidth{0.501875pt}%
\definecolor{currentstroke}{rgb}{0.000000,0.000000,0.000000}%
\pgfsetstrokecolor{currentstroke}%
\pgfsetdash{}{0pt}%
\pgfsys@defobject{currentmarker}{\pgfqpoint{0.000000in}{-0.055556in}}{\pgfqpoint{0.000000in}{0.000000in}}{%
\pgfpathmoveto{\pgfqpoint{0.000000in}{0.000000in}}%
\pgfpathlineto{\pgfqpoint{0.000000in}{-0.055556in}}%
\pgfusepath{stroke,fill}%
}%
\begin{pgfscope}%
\pgfsys@transformshift{1.743750in}{1.435000in}%
\pgfsys@useobject{currentmarker}{}%
\end{pgfscope}%
\end{pgfscope}%
\begin{pgfscope}%
\pgftext[x=1.743750in,y=0.119444in,,top]{\rmfamily\fontsize{9.000000}{10.800000}\selectfont \(\displaystyle 0.0006\)}%
\end{pgfscope}%
\begin{pgfscope}%
\pgfsetbuttcap%
\pgfsetroundjoin%
\definecolor{currentfill}{rgb}{0.000000,0.000000,0.000000}%
\pgfsetfillcolor{currentfill}%
\pgfsetlinewidth{0.501875pt}%
\definecolor{currentstroke}{rgb}{0.000000,0.000000,0.000000}%
\pgfsetstrokecolor{currentstroke}%
\pgfsetdash{}{0pt}%
\pgfsys@defobject{currentmarker}{\pgfqpoint{0.000000in}{0.000000in}}{\pgfqpoint{0.000000in}{0.055556in}}{%
\pgfpathmoveto{\pgfqpoint{0.000000in}{0.000000in}}%
\pgfpathlineto{\pgfqpoint{0.000000in}{0.055556in}}%
\pgfusepath{stroke,fill}%
}%
\begin{pgfscope}%
\pgfsys@transformshift{2.250000in}{0.175000in}%
\pgfsys@useobject{currentmarker}{}%
\end{pgfscope}%
\end{pgfscope}%
\begin{pgfscope}%
\pgfsetbuttcap%
\pgfsetroundjoin%
\definecolor{currentfill}{rgb}{0.000000,0.000000,0.000000}%
\pgfsetfillcolor{currentfill}%
\pgfsetlinewidth{0.501875pt}%
\definecolor{currentstroke}{rgb}{0.000000,0.000000,0.000000}%
\pgfsetstrokecolor{currentstroke}%
\pgfsetdash{}{0pt}%
\pgfsys@defobject{currentmarker}{\pgfqpoint{0.000000in}{-0.055556in}}{\pgfqpoint{0.000000in}{0.000000in}}{%
\pgfpathmoveto{\pgfqpoint{0.000000in}{0.000000in}}%
\pgfpathlineto{\pgfqpoint{0.000000in}{-0.055556in}}%
\pgfusepath{stroke,fill}%
}%
\begin{pgfscope}%
\pgfsys@transformshift{2.250000in}{1.435000in}%
\pgfsys@useobject{currentmarker}{}%
\end{pgfscope}%
\end{pgfscope}%
\begin{pgfscope}%
\pgftext[x=2.250000in,y=0.119444in,,top]{\rmfamily\fontsize{9.000000}{10.800000}\selectfont \(\displaystyle 0.0008\)}%
\end{pgfscope}%
\begin{pgfscope}%
\pgfsetbuttcap%
\pgfsetroundjoin%
\definecolor{currentfill}{rgb}{0.000000,0.000000,0.000000}%
\pgfsetfillcolor{currentfill}%
\pgfsetlinewidth{0.501875pt}%
\definecolor{currentstroke}{rgb}{0.000000,0.000000,0.000000}%
\pgfsetstrokecolor{currentstroke}%
\pgfsetdash{}{0pt}%
\pgfsys@defobject{currentmarker}{\pgfqpoint{0.000000in}{0.000000in}}{\pgfqpoint{0.000000in}{0.055556in}}{%
\pgfpathmoveto{\pgfqpoint{0.000000in}{0.000000in}}%
\pgfpathlineto{\pgfqpoint{0.000000in}{0.055556in}}%
\pgfusepath{stroke,fill}%
}%
\begin{pgfscope}%
\pgfsys@transformshift{2.756250in}{0.175000in}%
\pgfsys@useobject{currentmarker}{}%
\end{pgfscope}%
\end{pgfscope}%
\begin{pgfscope}%
\pgfsetbuttcap%
\pgfsetroundjoin%
\definecolor{currentfill}{rgb}{0.000000,0.000000,0.000000}%
\pgfsetfillcolor{currentfill}%
\pgfsetlinewidth{0.501875pt}%
\definecolor{currentstroke}{rgb}{0.000000,0.000000,0.000000}%
\pgfsetstrokecolor{currentstroke}%
\pgfsetdash{}{0pt}%
\pgfsys@defobject{currentmarker}{\pgfqpoint{0.000000in}{-0.055556in}}{\pgfqpoint{0.000000in}{0.000000in}}{%
\pgfpathmoveto{\pgfqpoint{0.000000in}{0.000000in}}%
\pgfpathlineto{\pgfqpoint{0.000000in}{-0.055556in}}%
\pgfusepath{stroke,fill}%
}%
\begin{pgfscope}%
\pgfsys@transformshift{2.756250in}{1.435000in}%
\pgfsys@useobject{currentmarker}{}%
\end{pgfscope}%
\end{pgfscope}%
\begin{pgfscope}%
\pgftext[x=2.756250in,y=0.119444in,,top]{\rmfamily\fontsize{9.000000}{10.800000}\selectfont \(\displaystyle 0.0010\)}%
\end{pgfscope}%
\begin{pgfscope}%
\pgfsetbuttcap%
\pgfsetroundjoin%
\definecolor{currentfill}{rgb}{0.000000,0.000000,0.000000}%
\pgfsetfillcolor{currentfill}%
\pgfsetlinewidth{0.501875pt}%
\definecolor{currentstroke}{rgb}{0.000000,0.000000,0.000000}%
\pgfsetstrokecolor{currentstroke}%
\pgfsetdash{}{0pt}%
\pgfsys@defobject{currentmarker}{\pgfqpoint{0.000000in}{0.000000in}}{\pgfqpoint{0.000000in}{0.055556in}}{%
\pgfpathmoveto{\pgfqpoint{0.000000in}{0.000000in}}%
\pgfpathlineto{\pgfqpoint{0.000000in}{0.055556in}}%
\pgfusepath{stroke,fill}%
}%
\begin{pgfscope}%
\pgfsys@transformshift{3.262500in}{0.175000in}%
\pgfsys@useobject{currentmarker}{}%
\end{pgfscope}%
\end{pgfscope}%
\begin{pgfscope}%
\pgfsetbuttcap%
\pgfsetroundjoin%
\definecolor{currentfill}{rgb}{0.000000,0.000000,0.000000}%
\pgfsetfillcolor{currentfill}%
\pgfsetlinewidth{0.501875pt}%
\definecolor{currentstroke}{rgb}{0.000000,0.000000,0.000000}%
\pgfsetstrokecolor{currentstroke}%
\pgfsetdash{}{0pt}%
\pgfsys@defobject{currentmarker}{\pgfqpoint{0.000000in}{-0.055556in}}{\pgfqpoint{0.000000in}{0.000000in}}{%
\pgfpathmoveto{\pgfqpoint{0.000000in}{0.000000in}}%
\pgfpathlineto{\pgfqpoint{0.000000in}{-0.055556in}}%
\pgfusepath{stroke,fill}%
}%
\begin{pgfscope}%
\pgfsys@transformshift{3.262500in}{1.435000in}%
\pgfsys@useobject{currentmarker}{}%
\end{pgfscope}%
\end{pgfscope}%
\begin{pgfscope}%
\pgftext[x=3.262500in,y=0.119444in,,top]{\rmfamily\fontsize{9.000000}{10.800000}\selectfont \(\displaystyle 0.0012\)}%
\end{pgfscope}%
\begin{pgfscope}%
\pgfsetbuttcap%
\pgfsetroundjoin%
\definecolor{currentfill}{rgb}{0.000000,0.000000,0.000000}%
\pgfsetfillcolor{currentfill}%
\pgfsetlinewidth{0.501875pt}%
\definecolor{currentstroke}{rgb}{0.000000,0.000000,0.000000}%
\pgfsetstrokecolor{currentstroke}%
\pgfsetdash{}{0pt}%
\pgfsys@defobject{currentmarker}{\pgfqpoint{0.000000in}{0.000000in}}{\pgfqpoint{0.000000in}{0.055556in}}{%
\pgfpathmoveto{\pgfqpoint{0.000000in}{0.000000in}}%
\pgfpathlineto{\pgfqpoint{0.000000in}{0.055556in}}%
\pgfusepath{stroke,fill}%
}%
\begin{pgfscope}%
\pgfsys@transformshift{3.768750in}{0.175000in}%
\pgfsys@useobject{currentmarker}{}%
\end{pgfscope}%
\end{pgfscope}%
\begin{pgfscope}%
\pgfsetbuttcap%
\pgfsetroundjoin%
\definecolor{currentfill}{rgb}{0.000000,0.000000,0.000000}%
\pgfsetfillcolor{currentfill}%
\pgfsetlinewidth{0.501875pt}%
\definecolor{currentstroke}{rgb}{0.000000,0.000000,0.000000}%
\pgfsetstrokecolor{currentstroke}%
\pgfsetdash{}{0pt}%
\pgfsys@defobject{currentmarker}{\pgfqpoint{0.000000in}{-0.055556in}}{\pgfqpoint{0.000000in}{0.000000in}}{%
\pgfpathmoveto{\pgfqpoint{0.000000in}{0.000000in}}%
\pgfpathlineto{\pgfqpoint{0.000000in}{-0.055556in}}%
\pgfusepath{stroke,fill}%
}%
\begin{pgfscope}%
\pgfsys@transformshift{3.768750in}{1.435000in}%
\pgfsys@useobject{currentmarker}{}%
\end{pgfscope}%
\end{pgfscope}%
\begin{pgfscope}%
\pgftext[x=3.768750in,y=0.119444in,,top]{\rmfamily\fontsize{9.000000}{10.800000}\selectfont \(\displaystyle 0.0014\)}%
\end{pgfscope}%
\begin{pgfscope}%
\pgfsetbuttcap%
\pgfsetroundjoin%
\definecolor{currentfill}{rgb}{0.000000,0.000000,0.000000}%
\pgfsetfillcolor{currentfill}%
\pgfsetlinewidth{0.501875pt}%
\definecolor{currentstroke}{rgb}{0.000000,0.000000,0.000000}%
\pgfsetstrokecolor{currentstroke}%
\pgfsetdash{}{0pt}%
\pgfsys@defobject{currentmarker}{\pgfqpoint{0.000000in}{0.000000in}}{\pgfqpoint{0.000000in}{0.055556in}}{%
\pgfpathmoveto{\pgfqpoint{0.000000in}{0.000000in}}%
\pgfpathlineto{\pgfqpoint{0.000000in}{0.055556in}}%
\pgfusepath{stroke,fill}%
}%
\begin{pgfscope}%
\pgfsys@transformshift{4.275000in}{0.175000in}%
\pgfsys@useobject{currentmarker}{}%
\end{pgfscope}%
\end{pgfscope}%
\begin{pgfscope}%
\pgfsetbuttcap%
\pgfsetroundjoin%
\definecolor{currentfill}{rgb}{0.000000,0.000000,0.000000}%
\pgfsetfillcolor{currentfill}%
\pgfsetlinewidth{0.501875pt}%
\definecolor{currentstroke}{rgb}{0.000000,0.000000,0.000000}%
\pgfsetstrokecolor{currentstroke}%
\pgfsetdash{}{0pt}%
\pgfsys@defobject{currentmarker}{\pgfqpoint{0.000000in}{-0.055556in}}{\pgfqpoint{0.000000in}{0.000000in}}{%
\pgfpathmoveto{\pgfqpoint{0.000000in}{0.000000in}}%
\pgfpathlineto{\pgfqpoint{0.000000in}{-0.055556in}}%
\pgfusepath{stroke,fill}%
}%
\begin{pgfscope}%
\pgfsys@transformshift{4.275000in}{1.435000in}%
\pgfsys@useobject{currentmarker}{}%
\end{pgfscope}%
\end{pgfscope}%
\begin{pgfscope}%
\pgftext[x=4.275000in,y=0.119444in,,top]{\rmfamily\fontsize{9.000000}{10.800000}\selectfont \(\displaystyle 0.0016\)}%
\end{pgfscope}%
\begin{pgfscope}%
\pgfsetbuttcap%
\pgfsetroundjoin%
\definecolor{currentfill}{rgb}{0.000000,0.000000,0.000000}%
\pgfsetfillcolor{currentfill}%
\pgfsetlinewidth{0.501875pt}%
\definecolor{currentstroke}{rgb}{0.000000,0.000000,0.000000}%
\pgfsetstrokecolor{currentstroke}%
\pgfsetdash{}{0pt}%
\pgfsys@defobject{currentmarker}{\pgfqpoint{0.000000in}{0.000000in}}{\pgfqpoint{0.055556in}{0.000000in}}{%
\pgfpathmoveto{\pgfqpoint{0.000000in}{0.000000in}}%
\pgfpathlineto{\pgfqpoint{0.055556in}{0.000000in}}%
\pgfusepath{stroke,fill}%
}%
\begin{pgfscope}%
\pgfsys@transformshift{0.225000in}{0.232273in}%
\pgfsys@useobject{currentmarker}{}%
\end{pgfscope}%
\end{pgfscope}%
\begin{pgfscope}%
\pgfsetbuttcap%
\pgfsetroundjoin%
\definecolor{currentfill}{rgb}{0.000000,0.000000,0.000000}%
\pgfsetfillcolor{currentfill}%
\pgfsetlinewidth{0.501875pt}%
\definecolor{currentstroke}{rgb}{0.000000,0.000000,0.000000}%
\pgfsetstrokecolor{currentstroke}%
\pgfsetdash{}{0pt}%
\pgfsys@defobject{currentmarker}{\pgfqpoint{-0.055556in}{0.000000in}}{\pgfqpoint{0.000000in}{0.000000in}}{%
\pgfpathmoveto{\pgfqpoint{0.000000in}{0.000000in}}%
\pgfpathlineto{\pgfqpoint{-0.055556in}{0.000000in}}%
\pgfusepath{stroke,fill}%
}%
\begin{pgfscope}%
\pgfsys@transformshift{4.275000in}{0.232273in}%
\pgfsys@useobject{currentmarker}{}%
\end{pgfscope}%
\end{pgfscope}%
\begin{pgfscope}%
\pgftext[x=0.169444in,y=0.232273in,right,]{\rmfamily\fontsize{9.000000}{10.800000}\selectfont \(\displaystyle -30\)}%
\end{pgfscope}%
\begin{pgfscope}%
\pgfsetbuttcap%
\pgfsetroundjoin%
\definecolor{currentfill}{rgb}{0.000000,0.000000,0.000000}%
\pgfsetfillcolor{currentfill}%
\pgfsetlinewidth{0.501875pt}%
\definecolor{currentstroke}{rgb}{0.000000,0.000000,0.000000}%
\pgfsetstrokecolor{currentstroke}%
\pgfsetdash{}{0pt}%
\pgfsys@defobject{currentmarker}{\pgfqpoint{0.000000in}{0.000000in}}{\pgfqpoint{0.055556in}{0.000000in}}{%
\pgfpathmoveto{\pgfqpoint{0.000000in}{0.000000in}}%
\pgfpathlineto{\pgfqpoint{0.055556in}{0.000000in}}%
\pgfusepath{stroke,fill}%
}%
\begin{pgfscope}%
\pgfsys@transformshift{0.225000in}{0.423182in}%
\pgfsys@useobject{currentmarker}{}%
\end{pgfscope}%
\end{pgfscope}%
\begin{pgfscope}%
\pgfsetbuttcap%
\pgfsetroundjoin%
\definecolor{currentfill}{rgb}{0.000000,0.000000,0.000000}%
\pgfsetfillcolor{currentfill}%
\pgfsetlinewidth{0.501875pt}%
\definecolor{currentstroke}{rgb}{0.000000,0.000000,0.000000}%
\pgfsetstrokecolor{currentstroke}%
\pgfsetdash{}{0pt}%
\pgfsys@defobject{currentmarker}{\pgfqpoint{-0.055556in}{0.000000in}}{\pgfqpoint{0.000000in}{0.000000in}}{%
\pgfpathmoveto{\pgfqpoint{0.000000in}{0.000000in}}%
\pgfpathlineto{\pgfqpoint{-0.055556in}{0.000000in}}%
\pgfusepath{stroke,fill}%
}%
\begin{pgfscope}%
\pgfsys@transformshift{4.275000in}{0.423182in}%
\pgfsys@useobject{currentmarker}{}%
\end{pgfscope}%
\end{pgfscope}%
\begin{pgfscope}%
\pgftext[x=0.169444in,y=0.423182in,right,]{\rmfamily\fontsize{9.000000}{10.800000}\selectfont \(\displaystyle -20\)}%
\end{pgfscope}%
\begin{pgfscope}%
\pgfsetbuttcap%
\pgfsetroundjoin%
\definecolor{currentfill}{rgb}{0.000000,0.000000,0.000000}%
\pgfsetfillcolor{currentfill}%
\pgfsetlinewidth{0.501875pt}%
\definecolor{currentstroke}{rgb}{0.000000,0.000000,0.000000}%
\pgfsetstrokecolor{currentstroke}%
\pgfsetdash{}{0pt}%
\pgfsys@defobject{currentmarker}{\pgfqpoint{0.000000in}{0.000000in}}{\pgfqpoint{0.055556in}{0.000000in}}{%
\pgfpathmoveto{\pgfqpoint{0.000000in}{0.000000in}}%
\pgfpathlineto{\pgfqpoint{0.055556in}{0.000000in}}%
\pgfusepath{stroke,fill}%
}%
\begin{pgfscope}%
\pgfsys@transformshift{0.225000in}{0.614091in}%
\pgfsys@useobject{currentmarker}{}%
\end{pgfscope}%
\end{pgfscope}%
\begin{pgfscope}%
\pgfsetbuttcap%
\pgfsetroundjoin%
\definecolor{currentfill}{rgb}{0.000000,0.000000,0.000000}%
\pgfsetfillcolor{currentfill}%
\pgfsetlinewidth{0.501875pt}%
\definecolor{currentstroke}{rgb}{0.000000,0.000000,0.000000}%
\pgfsetstrokecolor{currentstroke}%
\pgfsetdash{}{0pt}%
\pgfsys@defobject{currentmarker}{\pgfqpoint{-0.055556in}{0.000000in}}{\pgfqpoint{0.000000in}{0.000000in}}{%
\pgfpathmoveto{\pgfqpoint{0.000000in}{0.000000in}}%
\pgfpathlineto{\pgfqpoint{-0.055556in}{0.000000in}}%
\pgfusepath{stroke,fill}%
}%
\begin{pgfscope}%
\pgfsys@transformshift{4.275000in}{0.614091in}%
\pgfsys@useobject{currentmarker}{}%
\end{pgfscope}%
\end{pgfscope}%
\begin{pgfscope}%
\pgftext[x=0.169444in,y=0.614091in,right,]{\rmfamily\fontsize{9.000000}{10.800000}\selectfont \(\displaystyle -10\)}%
\end{pgfscope}%
\begin{pgfscope}%
\pgfsetbuttcap%
\pgfsetroundjoin%
\definecolor{currentfill}{rgb}{0.000000,0.000000,0.000000}%
\pgfsetfillcolor{currentfill}%
\pgfsetlinewidth{0.501875pt}%
\definecolor{currentstroke}{rgb}{0.000000,0.000000,0.000000}%
\pgfsetstrokecolor{currentstroke}%
\pgfsetdash{}{0pt}%
\pgfsys@defobject{currentmarker}{\pgfqpoint{0.000000in}{0.000000in}}{\pgfqpoint{0.055556in}{0.000000in}}{%
\pgfpathmoveto{\pgfqpoint{0.000000in}{0.000000in}}%
\pgfpathlineto{\pgfqpoint{0.055556in}{0.000000in}}%
\pgfusepath{stroke,fill}%
}%
\begin{pgfscope}%
\pgfsys@transformshift{0.225000in}{0.805000in}%
\pgfsys@useobject{currentmarker}{}%
\end{pgfscope}%
\end{pgfscope}%
\begin{pgfscope}%
\pgfsetbuttcap%
\pgfsetroundjoin%
\definecolor{currentfill}{rgb}{0.000000,0.000000,0.000000}%
\pgfsetfillcolor{currentfill}%
\pgfsetlinewidth{0.501875pt}%
\definecolor{currentstroke}{rgb}{0.000000,0.000000,0.000000}%
\pgfsetstrokecolor{currentstroke}%
\pgfsetdash{}{0pt}%
\pgfsys@defobject{currentmarker}{\pgfqpoint{-0.055556in}{0.000000in}}{\pgfqpoint{0.000000in}{0.000000in}}{%
\pgfpathmoveto{\pgfqpoint{0.000000in}{0.000000in}}%
\pgfpathlineto{\pgfqpoint{-0.055556in}{0.000000in}}%
\pgfusepath{stroke,fill}%
}%
\begin{pgfscope}%
\pgfsys@transformshift{4.275000in}{0.805000in}%
\pgfsys@useobject{currentmarker}{}%
\end{pgfscope}%
\end{pgfscope}%
\begin{pgfscope}%
\pgftext[x=0.169444in,y=0.805000in,right,]{\rmfamily\fontsize{9.000000}{10.800000}\selectfont \(\displaystyle 0\)}%
\end{pgfscope}%
\begin{pgfscope}%
\pgfsetbuttcap%
\pgfsetroundjoin%
\definecolor{currentfill}{rgb}{0.000000,0.000000,0.000000}%
\pgfsetfillcolor{currentfill}%
\pgfsetlinewidth{0.501875pt}%
\definecolor{currentstroke}{rgb}{0.000000,0.000000,0.000000}%
\pgfsetstrokecolor{currentstroke}%
\pgfsetdash{}{0pt}%
\pgfsys@defobject{currentmarker}{\pgfqpoint{0.000000in}{0.000000in}}{\pgfqpoint{0.055556in}{0.000000in}}{%
\pgfpathmoveto{\pgfqpoint{0.000000in}{0.000000in}}%
\pgfpathlineto{\pgfqpoint{0.055556in}{0.000000in}}%
\pgfusepath{stroke,fill}%
}%
\begin{pgfscope}%
\pgfsys@transformshift{0.225000in}{0.995909in}%
\pgfsys@useobject{currentmarker}{}%
\end{pgfscope}%
\end{pgfscope}%
\begin{pgfscope}%
\pgfsetbuttcap%
\pgfsetroundjoin%
\definecolor{currentfill}{rgb}{0.000000,0.000000,0.000000}%
\pgfsetfillcolor{currentfill}%
\pgfsetlinewidth{0.501875pt}%
\definecolor{currentstroke}{rgb}{0.000000,0.000000,0.000000}%
\pgfsetstrokecolor{currentstroke}%
\pgfsetdash{}{0pt}%
\pgfsys@defobject{currentmarker}{\pgfqpoint{-0.055556in}{0.000000in}}{\pgfqpoint{0.000000in}{0.000000in}}{%
\pgfpathmoveto{\pgfqpoint{0.000000in}{0.000000in}}%
\pgfpathlineto{\pgfqpoint{-0.055556in}{0.000000in}}%
\pgfusepath{stroke,fill}%
}%
\begin{pgfscope}%
\pgfsys@transformshift{4.275000in}{0.995909in}%
\pgfsys@useobject{currentmarker}{}%
\end{pgfscope}%
\end{pgfscope}%
\begin{pgfscope}%
\pgftext[x=0.169444in,y=0.995909in,right,]{\rmfamily\fontsize{9.000000}{10.800000}\selectfont \(\displaystyle 10\)}%
\end{pgfscope}%
\begin{pgfscope}%
\pgfsetbuttcap%
\pgfsetroundjoin%
\definecolor{currentfill}{rgb}{0.000000,0.000000,0.000000}%
\pgfsetfillcolor{currentfill}%
\pgfsetlinewidth{0.501875pt}%
\definecolor{currentstroke}{rgb}{0.000000,0.000000,0.000000}%
\pgfsetstrokecolor{currentstroke}%
\pgfsetdash{}{0pt}%
\pgfsys@defobject{currentmarker}{\pgfqpoint{0.000000in}{0.000000in}}{\pgfqpoint{0.055556in}{0.000000in}}{%
\pgfpathmoveto{\pgfqpoint{0.000000in}{0.000000in}}%
\pgfpathlineto{\pgfqpoint{0.055556in}{0.000000in}}%
\pgfusepath{stroke,fill}%
}%
\begin{pgfscope}%
\pgfsys@transformshift{0.225000in}{1.186818in}%
\pgfsys@useobject{currentmarker}{}%
\end{pgfscope}%
\end{pgfscope}%
\begin{pgfscope}%
\pgfsetbuttcap%
\pgfsetroundjoin%
\definecolor{currentfill}{rgb}{0.000000,0.000000,0.000000}%
\pgfsetfillcolor{currentfill}%
\pgfsetlinewidth{0.501875pt}%
\definecolor{currentstroke}{rgb}{0.000000,0.000000,0.000000}%
\pgfsetstrokecolor{currentstroke}%
\pgfsetdash{}{0pt}%
\pgfsys@defobject{currentmarker}{\pgfqpoint{-0.055556in}{0.000000in}}{\pgfqpoint{0.000000in}{0.000000in}}{%
\pgfpathmoveto{\pgfqpoint{0.000000in}{0.000000in}}%
\pgfpathlineto{\pgfqpoint{-0.055556in}{0.000000in}}%
\pgfusepath{stroke,fill}%
}%
\begin{pgfscope}%
\pgfsys@transformshift{4.275000in}{1.186818in}%
\pgfsys@useobject{currentmarker}{}%
\end{pgfscope}%
\end{pgfscope}%
\begin{pgfscope}%
\pgftext[x=0.169444in,y=1.186818in,right,]{\rmfamily\fontsize{9.000000}{10.800000}\selectfont \(\displaystyle 20\)}%
\end{pgfscope}%
\begin{pgfscope}%
\pgfsetbuttcap%
\pgfsetroundjoin%
\definecolor{currentfill}{rgb}{0.000000,0.000000,0.000000}%
\pgfsetfillcolor{currentfill}%
\pgfsetlinewidth{0.501875pt}%
\definecolor{currentstroke}{rgb}{0.000000,0.000000,0.000000}%
\pgfsetstrokecolor{currentstroke}%
\pgfsetdash{}{0pt}%
\pgfsys@defobject{currentmarker}{\pgfqpoint{0.000000in}{0.000000in}}{\pgfqpoint{0.055556in}{0.000000in}}{%
\pgfpathmoveto{\pgfqpoint{0.000000in}{0.000000in}}%
\pgfpathlineto{\pgfqpoint{0.055556in}{0.000000in}}%
\pgfusepath{stroke,fill}%
}%
\begin{pgfscope}%
\pgfsys@transformshift{0.225000in}{1.377727in}%
\pgfsys@useobject{currentmarker}{}%
\end{pgfscope}%
\end{pgfscope}%
\begin{pgfscope}%
\pgfsetbuttcap%
\pgfsetroundjoin%
\definecolor{currentfill}{rgb}{0.000000,0.000000,0.000000}%
\pgfsetfillcolor{currentfill}%
\pgfsetlinewidth{0.501875pt}%
\definecolor{currentstroke}{rgb}{0.000000,0.000000,0.000000}%
\pgfsetstrokecolor{currentstroke}%
\pgfsetdash{}{0pt}%
\pgfsys@defobject{currentmarker}{\pgfqpoint{-0.055556in}{0.000000in}}{\pgfqpoint{0.000000in}{0.000000in}}{%
\pgfpathmoveto{\pgfqpoint{0.000000in}{0.000000in}}%
\pgfpathlineto{\pgfqpoint{-0.055556in}{0.000000in}}%
\pgfusepath{stroke,fill}%
}%
\begin{pgfscope}%
\pgfsys@transformshift{4.275000in}{1.377727in}%
\pgfsys@useobject{currentmarker}{}%
\end{pgfscope}%
\end{pgfscope}%
\begin{pgfscope}%
\pgftext[x=0.169444in,y=1.377727in,right,]{\rmfamily\fontsize{9.000000}{10.800000}\selectfont \(\displaystyle 30\)}%
\end{pgfscope}%
\begin{pgfscope}%
\pgftext[x=2.250000in,y=1.504444in,,base]{\rmfamily\fontsize{11.000000}{13.200000}\selectfont Moduliertes Signal}%
\end{pgfscope}%
\end{pgfpicture}%
\makeatother%
\endgroup%

    \caption{%
        \emph{Amplitude-shift  keying}: Oben   sind  die   zu  \"ubertragenden
        digitalen  Daten als  \code{1} und  \code{0} abgebildet. Die  mittlere
        Abbildung stellt eine typische Umsetzung des Konzepts mit harmonischer
        Tr\"agerschwingung und zwei verschiedenen Amplituden dar.\protect\\
        Das  unterste   Signal  ist  eine  vereinfachte   Darstellung  unserer
        Variante.  Die  Gleichspannung an  der Leitung  ist etwas  weniger als
        \SI{1}{\kilo\volt}; beim Kurzschluss eines Moduls erfolgt eine Abfolge
        von Spannungseinbr\"uchen auf der Leitung, welche die Daten kodiert.%
    }
    \label{fig:ask:concept}
\end{figure}


Sowohl  FSK wie  auch  unsere  Variante der  ASK  haben  ihre jeweiligen  Vor-
und  Nachteile,  auf  welche im  Kapitel  \emph{\titleref{chap:simu}}  genauer
eingegangen wird.
