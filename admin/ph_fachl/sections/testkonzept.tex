Um  reibungslose  Funktionalit\"at  zu gew\"ahrleisten  sind  intensive  Tests
notwendig. Sie  werden   aufgegliedert  gem\"ass  den  Modulen   des  Systems:
Hardware, bestehend aus Sensor und  Master, und Software, ebenfalls aufgeteilt
zwischen Sensor und Master.


% ---------------------------------------------------------------------------- %
\subsection{Gesamtsystem}
% ---------------------------------------------------------------------------- %

Um   das   System   automatisiert  zu   testen,   werden   Labor-Netzger\"ate,
Funktionsgeneratoren und  Oszilloskope verwendet,  welche \"uber  das Netzwerk
Daten  liefern k\"onnen  und  somit zur  \"Uberwachung  des 24-Stunden-  Tests
geeignet  sind. Ein  Computer  mit einer  UART-Schnittstelle  \"uberwacht  den
Herzschlag des Mikrokontrollers.

% ---------------------------------------------------------------------------- %
\subsection{Hardware}
% ---------------------------------------------------------------------------- %

F\"ur die  Hardware werden  das Master-Ger\"at  und die  Sensorplatine jeweils
einzeln  getestet.   Zus\"atzlich   zur  experimentellen  \"Uberpr\"ufung  des
Systems werden  zur \"Uberpr\"ufung von m\"oglichen  Worst-Case-Szenarios noch
Computersimulationen benutzt (z.B. LTSpice).


% ---------------------------------------------------------------------------- %
\subsubsection{Sensorplatine}
% ---------------------------------------------------------------------------- %

Zu Debugzwecken hat die Sensorplatine 2 LEDs, 2 Buttons und eine zus\"atzliche
UART-  sowie auch  eine  SWD-Schnittstelle. Mit  diesen Schnittstellen  sollte
es   auch  f\"ur   die  Softwareentwickler   m\"oglich  sein,  akkurate  Tests
durchzuf\"uhren.

\textbf{Spannungsversorgung}: Eine     stabile    Spannungsversorgung     wird
sichergestellt. Hier  ist  nicht der  genaue  Wert  entscheidend, sondern  die
Stabilit\"at  dessen. Falls  die Methode  mit  dem  Kurzschliessen der  Panels
gew\"ahlt wird,  muss sichergestellt sein, dass  die Spannungsversorgung nicht
einknickt!

\textbf{Mikrokontroller}: Um  nicht Zeit  beim  Suchen eines  Softwarefehlers,
welchen  es gar  nicht gibt,  zu verschwenden,  wird die  Funktionalit\"at des
Mikrokontrollers  durch  ein  Programm,  welches die  zwei  LEDs  alternierend
blinken l\"asst, geschrieben und geflasht. Ist diese H\"urde einmal geschafft,
kann die Software ohne Weiteres entwickelt und getestet werden.

\textbf{Kommunikation}: Erst wird die  Funktionalit\"at der UART-Schnittstelle
getestet,  ohne das  Modem zu  verwenden. Ist ihre  Funktionalt\"at gesichert,
wird das Modem gepr\"uft. Im Falle der PLL wird schrittweise gepr\"uft, ob das
Signal an den einzelnen Stationen korrekt anliegt.

\textbf{Energieverbrauch}: Es   wird   gemessen,  wieviel   Energie   effektiv
verbraucht wird, um das Einhalten der Vorgaben zu pr\"ufen.

\textbf{Sicherheit}: Wird der Ansatz  mit der Leistungsmodullierung verwendet,
muss sichergestellt werden, dass der  FET, welcher die Spannung herunterzieht,
nicht zu heiss wird.

\textbf{Dauertest}: Um  sicherzustellen, dass  der Betrieb  auch \"uber  lange
Zeit gew\"ahrleistet  ist, wird ein  Testsetup errichtet, welches  den Betrieb
w\"ahrend 24 Stunden \"uberwacht.


% ---------------------------------------------------------------------------- %
\subsubsection{Master}
% ---------------------------------------------------------------------------- %

Beim  Master wird  analog zur  Sensorplatine vorgegangen. Hier  muss noch  die
Funktionalit\"at des  GSM Modems (falls implementiert)  sowie der Strommessung
sichergestellt werden.


% ---------------------------------------------------------------------------- %
\subsection{Software}
% ---------------------------------------------------------------------------- %

Auch auf  der Software-Seite gilt  es, sowohl  die Sensorplatine wie  auch das
Masterger\"at separat und dann im Verbund zu testen.


% ---------------------------------------------------------------------------- %
\subsubsection{Sensorplatine}
% ---------------------------------------------------------------------------- %

Wie   erw\"ahnt   hat   die   Sensorplatine   zu   Testzwecken   zus\"atzliche
Schnittstellen. Zuerst   wird  dann   ein  Programm   getestet,  welches   die
Funktionalit\"at des  Mikrokontrollers garantiert. Dies geht Hand  in Hand mit
dem Testen der Hardware.

\textbf{Messung}: Es   wird  durch   Vergleiche  mit   externen  Messger\"aten
sichergestellt, dass die Sensorik korrekt misst und die Mittelwerte \"uber die
letzte Zeitperiode korrekt berechnet.

\textbf{Kommunikation}: Nach   dem   Versand   von   Daten   via   UART   wird
sichergestellt,  dass Kollisionen  von Daten  korrekt erkannt  und abgehandelt
werden.

\textbf{Deep-Sleep}: Um den  Stromverbrauch niedrig  zu halten, wird  der Chip
f\"ur  die Zeitperioden  in  welchen  er inaktiv  ist,  in den  Low-Power-Mode
versetzt. Es muss  getestet werden, dass er  dies korrekt tut und  diesen auch
korrekt wieder verl\"asst.

\textbf{Verl\"asslichkeit}: In einem  24 St\"undigen Dauertest  wird getestet,
dass sich die Software nicht irgendwann aufh\"angt.

% ---------------------------------------------------------------------------- %
\subsubsection{Master}
% ---------------------------------------------------------------------------- %

Beim Master  wird analog  zur Sensorplatine  vorgegangen. Hier muss  noch die
Funktionalit\"at des GSM Modems  sowie der Strommessung sichergestellt werden.
Der Master braucht eine bedeutend kompliziertere Logik.

\textbf{Blinky  LED}: Ein  einfaches  Programm   wird  verfasst,  welches  die
korrekte Ansteuerung von GPIO Pins durch alternierendes Blinken meldet.

\textbf{UART}: Auch    hier    ist    die    korrekte    Funktionsweise    der
UART-Schnittstellt zentral. Diese muss fr\"uhzeitig gepr\"uft werden.

\textbf{GSM (falls implementiert)}: Das GSM-Modul  wird gepr\"uft, in dem eine
einfache erste Botschaft verfasst wird und an ein Telefon geschickt wird.

\textbf{Sensorik}: Nat\"urlich  muss  auch  hier  die  Sensorik  mit  externen
Messger\"aten verifiziert werden.

\textbf{Alarmlogik}: F\"ur  die  Alarmlogik werden  fr\"uhzeitig  verschiedene
Testszenarios verfasst. Die  Alarmlogik muss diese dann  korrekt abhandeln, um
zu gew\"ahrleisten, dass keine falschen Alarme losgehen.
