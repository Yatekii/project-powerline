% **************************************************************************** %
\chapter{Daten von Solarmodulen}
\label{app:commercial:modules}
% **************************************************************************** %

Dieser Abschnitt  enth\"alt in Tabelle  \ref{tab:moduleData:Dimensions} einige
Eckdaten  von   kommerziell  erh\"altlichen  Modulen  mit   den  zugeh\"origen
Quellen. Diese   Informationen   sollen    prim\"ar   als   Anhaltspunkt   und
Vergleich  des   in  unseren  Simulationen  benutzten   Moduls  aus  Abschnitt
\ref{sec:simu:model:module} (ab Seite \pageref{sec:simu:model:module}) mit der
Praxis.

\begin{table}
    \centering
    \small
    \caption{%
        Daten   f\"ur   Solarmodule.  \textbf{pk}:   polykristallines   Panel,
        \textbf{mk}:     monokristallines     Panel.     \emph{Anmerkung}: Die
        Konfiguration  der Module  (wieviele  Zellen in  Serie  und wie  viele
        Str\"ange  parallel)  ist  mit   Ausnahme  des  Solarex  MSX-60  nicht
        angegeben. Es  ist  aber  bekannt,  in  welcher  Gr\"ossenordnung  die
        Spannung  pro  Zelle  ungef\"ahr  liegen sollte,  womit  man  aus  den
        angegebenen  Leerlaufspannungen  und  der Gesamtzahl  Zellen  auf  die
        Konfiguration eines Modules schliessen kann.%
    }
    \label{tab:moduleData:IU}
    \begin{tabular}{lp{20mm}lllll}
        \toprule
          \rotatebox{70}{\pbox{25mm}{Quelle}}
        & \rotatebox{70}{\pbox{25mm}{Modell}}
        & \rotatebox{70}{\pbox{25mm}{Kurzschluss-\\strom $I_{\mathrm{SC}}$}}
        & \rotatebox{70}{\pbox{25mm}{Leerlauf-\\spannung $V_{\mathrm{OC}}$}}
        & \rotatebox{70}{\pbox{25mm}{Anzahl Zellen \\(total)}}
        & \rotatebox{70}{\pbox{25mm}{Anzahl Zellen \\(Strang)}}
        & \rotatebox{70}{\pbox{25mm}{Leerlaufspan-\\nung pro Zelle}} \\
        \midrule

          \cite{ref:solar:bonkoungou}
        & Solarex MSX-60
        & \SI{3.8}{\ampere}
        & \SI{21.1}{\volt}
        & \num{36}
        & \num{36}
        & \SI{586}{\milli\volt}
        \\

          \cite{ref:solar:px85}
        & Sunset PX85 (\textbf{pk})
        & \SI{5.5}{\ampere}
        & \SI{21.5}{\volt}
        & \num{76}
        & \num{38}
        & \SI{566}{\milli\volt}
        \\

          \cite{ref:solar:as150}
        & Sunset Solargenerator AS150 (\textbf{mk})
        & \SI{8.7}{\ampere}
        & \SI{22.3}{\volt}
        & \num{36}
        & \num{36}
        & \SI{620}{\milli\volt}
        \\

          \cite{ref:solar:sunmodulePro}
        & Sunmodule Pro-Series XL SW320 (\textbf{mk})
        & \SI{9.41}{\ampere}
        & \SI{45.9}{\volt}
        & \num{72}
        & \num{72}
        & \SI{638}{\milli\volt}
        \\

        \bottomrule
    \end{tabular}
\end{table}


%\begin{table}[h!tb]
%    \centering
%    \caption{Abmessungen der Solarmodule aus Tabelle \ref{tab:moduleData:IU}}
%    \label{tab:moduleData:Dimensions}
%    \begin{tabular}{lp{20mm}ll}
%        \toprule
%          \rotatebox{70}{\pbox{25mm}{Quelle}}
%        & \rotatebox{70}{\pbox{25mm}{Modell}}
%        & \rotatebox{70}{\pbox{25mm}{Abmessungen Modul ($\si{\milli\meter} \si{\milli\meter}$)}}
%        & \rotatebox{70}{\pbox{25mm}{Abmessungen Zelle ($\si{\milli\meter} \si{\milli\meter}$)}} \\
%        \midrule
%
%          \cite{ref:solar:px85}
%        & Sunset PX85 (\textbf{pk})
%        & $1477 \times 660$
%        & $\approx 165 \times 75$
%        \\
%
%          \cite{ref:solar:as150}
%        & Sunset Solargenerator AS150 (\textbf{mk})
%        & $1480 \times 660$
%        & $\approx 155 \times 164$
%        \\
%
%          \cite{ref:solar:sunmodulePro}
%        & Sunmodule Pro-Series XL SW320 (\textbf{mk})
%        & $1985 \times 990$
%        & $156 \times 156$
%        \\
%        \bottomrule
%    \end{tabular}
%\end{table}
