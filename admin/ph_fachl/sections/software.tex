Da die Hardware  aus zwei einzelnen Boards besteht, wird  nat\"urlich auch pro
Board eine separate Software ben\"otigt.

% ---------------------------------------------------------------------------- %
\subsection{Sensorplatine}
% ---------------------------------------------------------------------------- %

Der  in der  Sensorplatine  verbaute  Atmel-Chip wird  in  C programmiert,  da
sich  diese Programmiersprache  durch die  Hardwaren\"ahe optimal  f\"ur diese
Aufgabenstellung  eignet. Die Aufgabe  der  Software besteht  hier darin,  die
gemessenen Daten zu verarbeiten, falls n\"otig Pr\"ufsummen zu berechnen sowie
die zu sendenden  Daten als Pakete bereitzustellen, um sie  auf die DC-Leitung
modulieren zu k\"onnen. Zus\"atzlich muss  zur Identifikation beim Master eine
sensorspezifische  Nummer  eingelesen  und ebenfalls  \"uber  die  Netzleitung
kommuniziert werden k\"onnen.

% ---------------------------------------------------------------------------- %
\subsection{Master}
% ---------------------------------------------------------------------------- %

Beim  Master gibt  es  zwei  Hauptbereiche, die  zu  betrachten sind.   Zuerst
muss  eine passende  Plattform  in Form  eines  Betriebssystems gew\"ahlt  und
konfiguriert werden.   Anschliessend muss  die passende Technologie  f\"ur die
Implementation der eigentlichen Applikationen gew\"ahlt werden.


% ---------------------------------------------------------------------------- %
\subsubsection{Betriebssystem}
% ---------------------------------------------------------------------------- %

F\"ur  das Raspberry  Pi  sind angepasste  Versionen diverser  Betriebssysteme
verf\"ugbar. Das  am  besten  unterst\"utzte  ist eine  Variante  von  Debian,
genannt Raspbian. Diese bietet neben  grundlegenden Systemfunktionen auch eine
Vielzahl von  Programmbibliotheken, welche zur Entwicklung  von Auswertung und
Benutzerinterface von grossem Nutzen sind.

Zur Erh\"ohgung der  Lebensdauer der SD-Karte, auf der  das Betriebssystem des
RasPi  gespeichert  ist,  werden  die Systemlog-Daten  kontinuierlich  in  ein
RAM-Dateisystem  (\code{/var/}  als  \code{tempfs}  eingebunden)  gespeichert.
Nachteil  dieser  L\"osung  ist,  dass bei  Verlust  der  Stromversorgung  die
Log-Daten verloren  gehen. Da diese aber  f\"ur uns nicht von  Interesse sind,
ist dies akzeptabel.

Die vom Betriebssystem verursachten Schreibvorg\"ange werden weiter reduziert,
indem das  Aktualisieren der Zugriffszeit  beim Lesen einer  Datei unterbunden
wird (Mount-Option \code{noatime}).

Diese  beiden  Massnahmen  reduzieren  die Anzahl  Schreibvorg\"ange  auf  der
SD-Karte bedeutend und erh\"ohen somit ihre Lebensdauer.

Die \"Uberwachungsdaten  der Solarmodule  werden hingegen fortlaufend  auf der
SD-Karte  gespeichert, da  diese Daten  einerseits wichtig  sind (Datenverlust
nicht w\"unschenswert)  und nicht viele Schreibvorg\"ange  verursachen (kleine
Datenmenge).


% ---------------------------------------------------------------------------- %
\clearpage
\subsubsection{Applikationen}
% ---------------------------------------------------------------------------- %

Die Programmierung des Masters erfolgt in  Python, da hier die zur Verf\"ugung
stehenden Libraries besser mit dem  verwendeten Raspberry Pi kombiniert werden
k\"onnen als mit C. Dabei ist die Software einerseits daf\"ur zust\"andig, die
von  den Sensoren  \"ubermittelten,  demodulierten Daten  zu  erfassen und  zu
verarbeiten. Die  erhaltenen  Daten  m\"ussen  verglichen,  eventuell  defekte
Module  evaluiert, Fehler  ber\"ucksichtigt sowie  bei einem  erkannten Defekt
dieser gemeldet werden.

Andererseits    m\"ussen    die    Sensoren   anhand    der    \"ubermittelten
Identifikationsnummern den einzelnen Solarmodulen zugewiesen werden, damit die
Messwerte korrekt miteinander verglichen werden k\"onnen.
