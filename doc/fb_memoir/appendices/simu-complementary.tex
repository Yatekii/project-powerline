% **************************************************************************** %
\chapter{Erg\"anzende Simulationsergebnisse}
\label{app:simu:complementary}
% **************************************************************************** %


Dieses   Kapitel  beinhaltet   erg\"anzende  Simulationsergebnisse,   die  aus
Gr\"unden der \"Ubersichtlichkeit nicht im Hauptteil zu finden sind.


% ---------------------------------------------------------------------------- %
\section{Modul mit zwei parallelen Str\"angen}
\label{app:sec:simu:complementary:36x2}
% ---------------------------------------------------------------------------- %

\begin{figure}[h!tb]
    \centering
    \begin{tikzpicture}
       \begin{scope}[x={(0mm,0mm)},y={(120mm,0.95\textwidth)}]
           \begin{axis}[%
                   height=40mm,
                   width=\textwidth,
                   at={(0,70mm)},
                   grid=both,
                   xlabel=Zeit (\si{\micro\second}),
                   ylabel=Strom (\si{\ampere}),
                   %x unit=u,
                   change x base=true,
                   x SI prefix=micro,
               ]
               \addplot[-,color=blue] table {data/module-72cells--I-MOSFET--1000u.dat};
           \end{axis}
           \begin{axis}[%
                   height=40mm,
                   width=\textwidth,
                   at={(0,35mm)},
                   grid=both,
                   xlabel=Zeit (\si{\micro\second}),
                   ylabel=Spannung (\si{\volt}),
                   %x unit=u,
                   change x base=true,
                   x SI prefix=micro,
               ]
               \addplot[-,color=magenta] table {data/module-72cells--U-MOSFET--1000u.dat};
           \end{axis}
           \begin{axis}[%
                   height=40mm,
                   width=\textwidth,
                   at={(0,0)},
                   grid=both,
                   xlabel=Zeit (\si{\micro\second}),
                   ylabel=Leistung (\si{\watt}),
                   %x unit=u,
                   change x base=true,
                   x SI prefix=micro,
               ]
               \addplot[-,color=teal] table {data/module-72cells--P-MOSFET--1000u.dat};
           \end{axis}
       \end{scope}
   \end{tikzpicture}
   \caption{%
       Transientensimulation des Kurzschlussverfahrens f\"ur ASK mit einem $36 \times 2$-Modul%
   }
\end{figure}
\todo{``Abbilddung sowieso auf Seite diesunddas ist ein vergr\"osserter Ausschnitt eines Peaks aus dieser Simulation''}


\begin{figure}[h!tb]
    \centering
    \begin{tikzpicture}
       \begin{scope}[x={(0mm,0mm)},y={(120mm,0.95\textwidth)}]
           \begin{axis}[%
                   height=40mm,
                   width=\textwidth,
                   at={(0,70mm)},
                   grid=both,
                   xlabel=Zeit (\si{\micro\second}),
                   ylabel=Strom (\si{\ampere}),
                   %x unit=u,
                   change x base=true,
                   x SI prefix=micro,
               ]
               \addplot[-,color=blue] table {data/module-72cells-series--I-MOSFET--1000u.dat};
           \end{axis}
           \begin{axis}[%
                   height=40mm,
                   width=\textwidth,
                   at={(0,35mm)},
                   grid=both,
                   xlabel=Zeit (\si{\micro\second}),
                   ylabel=Spannung (\si{\volt}),
                   %x unit=u,
                   change x base=true,
                   x SI prefix=micro,
               ]
               \addplot[-,color=magenta] table {data/module-72cells-series--U-MOSFET--1000u.dat};
           \end{axis}
           \begin{axis}[%
                   height=40mm,
                   width=\textwidth,
                   at={(0,0)},
                   grid=both,
                   xlabel=Zeit (\si{\micro\second}),
                   ylabel=Leistung (\si{\watt}),
                   %x unit=u,
                   change x base=true,
                   x SI prefix=micro,
               ]
               \addplot[-,color=teal] table {data/module-72cells-series--P-MOSFET--1000u.dat};
           \end{axis}
       \end{scope}
   \end{tikzpicture}
   \caption{%
       Transientensimulation des Kurzschlussverfahrens f\"ur ASK mit einem $72 \times 1$-Modul%
   }
\end{figure}

\todo{``Abbilddung sowieso ist ein vergr\"osserter Ausschnitt eines Peaks aus dieser Simualtin}
