% **************************************************************************** %
\chapter*{Management Summary}
\label{chap:mgmtsum}
% **************************************************************************** %

\section*{Main Ideas and facts}
For  Project 4  in the  spring  semester 2015  the students  were assigned  to
develop  an electro  mechanical system  to  keep a  laser pointer  steady. The
device should be able to emit a laser beam and keep it steady even if the user
is nervous and  trembling. The goal of this project is  to develop a prototype
that  is small  and  light enough  to  carry around  and  hold a  presentation
with. Another focus was on the budget: The  production costs of a series of 20
pieces should not exceed 120 CHF per unit.

\section*{Background}
In the autumn semester  2014, a group of students were  assigned with the task
to  find out  if  itwould be  possible  to develop  such  an anti-shake  laser
pointer, the results were inconclusive. Theparticipants agreed that the device
should be able to perform at a frequency range of 3 to 20 Hz.

\section*{Results}
The final prototype  did not meet the expectations. It is  able to correct low
frequency movements,but it struggles when  it is shaken at higher frequencies.
The prototype is small, it blends in with the usual laser pointers although it
is slightly  larger in  diameter. The estimated  price per  unit to  produce a
series  of 20  is around  70 SFr. However,  this figure  does not  include the
casing,  which was  produced for  free  by the  FHNW mechanical  workshop. The
prototype is powered by  a battery that is rechargeable by USB  and is able to
last eight hours.

\section*{Conclusions}
The approach of deflecting the laser beam with stepper motors seems promising,
the power usage is manageable and it allows a very small size, while the price
for the  parts is low. While the  overall concept and the  peripheral hardware
are solid and meet the expectations, there is the problem that the device does
not work at  higher frequencies.  The stepper motors used  in this project are
probably a  bit too cheap and  tend to break. A simulation  in Simulink showed
that the problem lies within  the laser deflection mechanism. We estimate that
the moment  of inertia  of the components  is too much  for the  small stepper
motors.

\section*{Recommendation}
To get rid of the flaws  in higher frequency performance, we suggest improving
the laser deflection mechanism. Most likely  reducing weight would not cut it,
as laser diodes can’t  get much smaller than the one  we use. This means the
whole  deflection  mechanism  should be  redesigned. Also,  different  stepper
motors should be considered.
