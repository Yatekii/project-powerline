% **************************************************************************** %
\chapter{L\"osungsans\"atze und Simulationen}
\label{chap:simu}
% **************************************************************************** %

Bevor  simuliert  werden  kann,  m\"ussen  die  daf\"ur  ben\"otigten  Modelle
vorhanden sein. Es  wird zuerst ein  Modell f\"ur eine  Solarzelle entwickelt.
Anschliessend wird dieses Zellenmodell benutzt, um ein Modul aufzubauen.

In  diesem Abschnitt  werden Schaltungen  f\"ur drei  L\"osungsans\"atze (zwei
zur  FSK, eine  zur  ASK)  vorgestellt, mit  \code{LTspice  IV} simuliert  und
beurteilt.\todo{LTspice: source}

Es werden  jeweils die  Schaltung f\"ur  den Sender,  den Empf\"anger  und das
Gesamtsystem untersucht.


% ---------------------------------------------------------------------------- %
\section{Modellierung einer PV-Zelle}
\label{sec:simu:model:cell}
% ---------------------------------------------------------------------------- %


Ziel ist, eine Zelle zu modellieren, welche folgende Eigenschaften erf\"ullt:

\begin{align}
    U_{\mathrm{Leerlauf}}                    &= \SI{600}{\milli\volt} \\
    I_{\mathrm{Kurzschluss, polykristallin}} &=  \SI{5}{\ampere}      \\
    I_{\mathrm{Kurzschluss, monokristallin}} &= \SI{10}{\ampere}
\end{align}

Die Herleitung  dieser Werte ist  in Anhang \ref{app:subsec:cell:UI}  ab Seite
\pageref{app:subsec:cell:UI} zu  finden und basiert auf  den PV-Modulen, deren
Daten in Tabelle \ref{tab:moduleData:IU} auf Seite \pageref{tab:moduleData:IU}
aufgelistet  sind.   Es  werden  zwei  Zielwerte  f\"ur  den  Kurzschlussstrom
angegeben;  einer f\"ur  monokristalline Zellen,  einer f\"ur  polykristalline
Zellen.

Als  Grundlage f\"ur  das  Modell dient  das  Eindiodenmodell einer  PV-Zelle,
erweitert  um  eine parallele  Kapazit\"at  $C$  (basierend auf  Informationen
aus \cite{ref:solar:scofield} und \cite{ref:solar:friesen}). Die resultierende
Schaltung ist dargestellt in \fref{fig:circuit:solarCell}.

\clearpage
\begin{figure}[h!tb]
    \centering
    % Parametrized version, can be put at coordinates x and y
%
%\newcommand{\CTSolarCell}[2]{%
%    % Source to top terminal
%    (0+#1,0+#2) to[current source,i=$I_{\mathrm{Zelle}}$] (0+#1,4+#2) -- (5+#1,4+#2) to[R,l^=$R_{\mathrm{S}}$] (10+#1,4+#2) node[circ] {}
%
%    % Open terminal at bottom
%    (10+#1,0+#2) node[circ] {} -- (0+#1,0+#2)
%
%    % Voltage arrow over open terminals
%    %(10+#1,3.7+#2) -- (10+#1,0.3+#2) node[currarrow,rotate=-90] {}
%    %(10+#1,2+#2) node[anchor=west] {$V_{\mathrm{offen}}$}
%
%    % Parallel elements
%    (1.5+#1,4+#2) to[empty diode,*-*,l_=$D$] (1.5+#1,0+#2)
%    (3+#1,4+#2) to[C,*-*,l_=$C$] (3+#1,0+#2)
%    (4.5+#1,4+#2) to[R,*-*,l_=$R_{\mathrm{P}}$] (4.5+#1,0+#2)%
%}


% Single Diode
%\begin{circuitikz}
%    \draw
%    % Source to top terminal
%    (0,0) to[current source,i=$I_{\mathrm{Zelle}}$] (0,3) -- (5,3) to[R,l^=$R_{\mathrm{S}}$] (8,3) node[ocirc] {}
%
%    % Open terminal at bottom
%    (8,0) node[ocirc] {} -- (0,0)
%
%    % Voltage arrow over open terminals
%    (8,2.7) -- (8,0.3) node[currarrow,rotate=-90] {}
%    (8,1.5) node[anchor=west] {$V_{\mathrm{offen}}$}
%
%    % Parallel elements
%    (1.5,3) to[empty diode,*-*,l_=$D$] (1.5,0)
%    (3,3) to[C,*-*,l_=$C$] (3,0)
%    (4.5,3) to[R,*-*,l_=$R_{\mathrm{P}}$] (4.5,0)
%    ;
%\end{circuitikz}

% Two Diodes
\begin{circuitikz}
    \draw
    % Source to top terminal
    (0,0) to[current source,i=$I_{\mathrm{Ph}}$] (0,3) -- (7.5,3) to[R,l^=$R_{\mathrm{S}}$] (11,3) node[ocirc] {}

    % Open terminal at bottom
    (11,0) node[ocirc] {} -- (0,0)

    % Voltage arrow over open terminals
    (11,2.7) -- (11,0.3) node[currarrow,rotate=-90] {}
    (11,1.5) node[anchor=west] {$V_{\mathrm{offen}}$}

    % Parallel elements
    (2,3) to[empty diode,*-*,l_=$D_{\mathrm{Diff}}$] (2,0)
    (4.5,3) to[empty diode,*-*,l_=$D_{\mathrm{Rekomb}}$] (4.5,0)
    (6,3) to[C,*-*,l_=$C$] (6,0)
    (7.5,3) to[R,*-*,l_=$R_{\mathrm{Sh}}$] (7.5,0)
    ;
\end{circuitikz}

    \caption{%
        Schaltschema    zur    Modellierung    einer    Solarzelle    gem\"ass
        Eindiodenmodell mit zus\"atzlicher Kapazit\"at%
    }
    \label{fig:circuit:solarCell}
\end{figure}

Bei diesem Modell sind insgesamt f\"unf Parameter zu bestimmen:

\begin{itemize}
    \firmlist
    \item
        Seriewiderstand $R_{\mathrm{S}}$
    \item
        Shunt-Widerstand $R_{\mathrm{P}}$
    \item
        Parasit\"are Kapazit\"at $C$
    \item
        Diode   $D$: Reverse    Saturation   Current    $I_{\mathrm{S}}$   und
        Idealit\"atsfaktor $n$
\end{itemize}

Es werden zuerst die Kapazit\"at und die Widerst\"ande bestimmt.

In \cite{ref:solar:scofield} wurden $C$, $R_{\mathrm{S}}$ und $R_{\mathrm{P}}$
einer  Solarzelle der  Gr\"osse  \SI{0.43}{\centi\meter\squared} \"uber  einen
Frequenzbereicht von \SI{1}{\kilo\hertz} bis \SI{1}{\mega\hertz} gemessen. Die
Resultate waren:

\begin{alignat}{3}
    \label{eq:scofield:C}
    C_{\mathrm{Messung}}    &= \SI{8}{\nano\farad} \; & \text{bis} \; & \SI{20}{\nano\farad} \;  & = & \; \SI{14 \pm 6}{\nano\farad} \\
    \label{eq:scofield:Rs}
    R_{\mathrm{S, Messung}} &= \SI{0.2}{\ohm}      \; & \text{bis} \; & \SI{20}{\ohm}        \;  & = & \; \SI{10.1 \pm 9.9}{\ohm}     \\
    \label{eq:scofield:Rp}
    R_{\mathrm{P, Messung}} &= \SI{0.5}{\kilo\ohm} \; & \text{bis} \; & \SI{500}{\kilo\ohm}  \;  & = & \; \SI{250.25 \pm 249.75}{\kilo\ohm}
\end{alignat}

Um die  obigen Werte f\"ur unsere  Zwecke verwenden zu k\"onnen,  m\"ussen sie
auf eine Zelle der Fl\"ache \SI{240}{\centi\meter\squared} um ungef\"ahr einen
Faktor \num{600} hochskaliert werden (zur Herleitung dieses Werts siehe Anhang
\ref{app:subsec:cell:surface} ab Seite \pageref{app:subsec:cell:surface})

$R_{\mathrm{P}}$   und  $R_{\mathrm{S}}$   skalieren  umgekehrt   proportional
zur   Zellfl\"ache,    wogegen   $C$   bei   gr\"osser    werdender   Fl\"ache
ansteigt~\cite{ref:solar:scofield}. Bei    der   ASK-Implementation    mittels
gesteuertem   Kurzschluss    (siehe   Abbildung    \ref{fig:ask:concept}   auf
Seite    \pageref{fig:ask:concept}     und    Simulationen     in    Abschnitt
\ref{subsec:simu:ask:sensor})  nehmen   die  Schwierigkeiten   mit  steigendem
Ausgangsstrom   der  Zelle   zu. Das   Modell  wird   deshalb  auf   maximalen
Ausgangsstrom getrimmt:

\begin{symbols}
    \firmlist
    \item[$C$]
        Beim Kurzschliessen der  Zelle treten Stromspitzen auf,  wenn sich der
        Kondensator  $C$  entl\"adt. Je  gr\"osser  dessen  Kapazit\"at,  umso
        h\"oher diese Stromspitzen.  Wir  nehmen also \SI{20}{\nano\farad} aus
        Gleichung \ref{eq:scofield:C} als Ausgangswert.
    \item[$R_{\mathrm{S}}$]
        Der aus der Solarzelle fliessende Strom wird gr\"osser, je kleiner der
        Seriewiderstand vor dem Ausgang ist. Es wird daher der niedrigere Wert
        von \SI{0.2}{\ohm} aus Gleichung \ref{eq:scofield:Rs} gew\"ahlt.
    \item[$R_{\mathrm{P}}$]
        Je    gr\"osser   der    Parallelwiderstand   im    Verh\"altnis   zum
        Seriewiderstand  ist, um  so mehr  Strom fliesst  aus dem  Ausgang der
        Zelle.  Es wird deshalb der h\"ochste Wert von \SI{500}{\kilo\ohm} aus
        Gleichung \ref{eq:scofield:Rp} als Ausgangswert verwendet.
\end{symbols}

Die skalierten Werte sind somit:
\begin{alignat}{2}
    C               &= C_{\mathrm{Messung}}    &\cdot 600 &= \SI{12}{\micro\farad} \\
    R_{\mathrm{S,}} &= R_{\mathrm{S, Messung}} &\div  600 &= \SI{333}{\micro\ohm}    \\
    R_{\mathrm{P,}} &= R_{\mathrm{P, Messung}} &\div  600 &= \SI{833}{\ohm}
\end{alignat}



Es bleiben noch die Parameter der Diode zu bestimmen. Ausganslage f\"ur das
Diodenmodell ist die Shockley-Diodengleichung:\todo{source?}

\begin{equation}
    \label{eq:diode}
    I_{\mathrm{D}} = I_{\mathrm{S}} \cdot \left( \exp\left(\frac{q \cdot V}{n \cdot k \cdot T}\right) - 1 \right)
\end{equation}

\begin{conditions}
    I_{\mathrm{D}} & Diodenstrom \\
    I_{\mathrm{S}} & Reverse saturation current \\
    q              & Elementarladung eines Elektrons (\SI{1.602e-19}{\coulomb}) \\
    V              & Diodenspannung \\
    n              & Idealit\"atsfaktor \\
    k              & Boltzmannkonstante (\SI{1.38e-23}{\joule\per\kelvin}) \\
    T              & Diodentemperatur \\
\end{conditions}

Der   Reverse   Saturation  Current   ist   der   Strom,  der   beim   Anlegen
einer  negativen   Spannung  \"uber  die   Diode  fliesst,  bevor   die  Diode
durchbricht~\cite{ref:solar:diodeCharacteristics}.    Er  liegt   bei  kleinen
Dioden \"ublicherweise Bereich von Nano-Amp\`ere bis Femto-Amp\`ere\todo{F\"ur
k\"aufliche Dioden, nicht  Solarzellen}, bei einer Solarzelle  wird er h\"oher
liegen, da diese gr\"osser ist.

Wie  man an  Gleichung \ref{eq:diode}  erkennen kann,  steigt der  Diodenstrom
f\"ur eine gegebene Spannung, wenn der Reverse Saturation Current ansteigt.

Der Idealit\"atsfaktor ist ein Indikator  f\"ur den Spannungsabfall \"uber der
Diode in  Abh\"angigkeit des durchfliessenden Stromes  und liegt normalerweise
zwischen  1  (ideale  Diode)   und  2. Je  gr\"osser  der  Idealit\"atsfaktor,
umso  h\"oher  der Spannungsabfall  \"uber  der  Diode f\"ur  einen  gegebenen
Strom  (bzw.  umso kleiner  der  Strom  bei einer  fixen  Spannung). Abbildung
\ref{fig:diodeVI:IS} zeigt das Strom-Spannungsverhalten  einer Diode sowie den
Einfluss von $I_{\mathrm{S}}$ und $n$ schematisch.

\begin{figure}[h!tb]
    %% Creator: Matplotlib, PGF backend
%%
%% To include the figure in your LaTeX document, write
%%   \input{<filename>.pgf}
%%
%% Make sure the required packages are loaded in your preamble
%%   \usepackage{pgf}
%%
%% Figures using additional raster images can only be included by \input if
%% they are in the same directory as the main LaTeX file. For loading figures
%% from other directories you can use the `import` package
%%   \usepackage{import}
%% and then include the figures with
%%   \import{<path to file>}{<filename>.pgf}
%%
%% Matplotlib used the following preamble
%%   \usepackage{fontspec}
%%   \setmainfont{Bitstream Vera Serif}
%%   \setsansfont{Bitstream Vera Sans}
%%   \setmonofont{Bitstream Vera Sans Mono}
%%
\begingroup%
\makeatletter%
\begin{pgfpicture}%
\pgfpathrectangle{\pgfpointorigin}{\pgfqpoint{4.500000in}{3.000000in}}%
\pgfusepath{use as bounding box, clip}%
\begin{pgfscope}%
\pgfsetbuttcap%
\pgfsetmiterjoin%
\pgfsetlinewidth{0.000000pt}%
\definecolor{currentstroke}{rgb}{0.000000,0.000000,0.000000}%
\pgfsetstrokecolor{currentstroke}%
\pgfsetstrokeopacity{0.000000}%
\pgfsetdash{}{0pt}%
\pgfpathmoveto{\pgfqpoint{0.000000in}{0.000000in}}%
\pgfpathlineto{\pgfqpoint{4.500000in}{0.000000in}}%
\pgfpathlineto{\pgfqpoint{4.500000in}{3.000000in}}%
\pgfpathlineto{\pgfqpoint{0.000000in}{3.000000in}}%
\pgfpathclose%
\pgfusepath{}%
\end{pgfscope}%
\begin{pgfscope}%
\pgfsetbuttcap%
\pgfsetmiterjoin%
\pgfsetlinewidth{0.000000pt}%
\definecolor{currentstroke}{rgb}{0.000000,0.000000,0.000000}%
\pgfsetstrokecolor{currentstroke}%
\pgfsetstrokeopacity{0.000000}%
\pgfsetdash{}{0pt}%
\pgfpathmoveto{\pgfqpoint{0.900000in}{0.600000in}}%
\pgfpathlineto{\pgfqpoint{4.275000in}{0.600000in}}%
\pgfpathlineto{\pgfqpoint{4.275000in}{2.550000in}}%
\pgfpathlineto{\pgfqpoint{0.900000in}{2.550000in}}%
\pgfpathclose%
\pgfusepath{}%
\end{pgfscope}%
\begin{pgfscope}%
\pgfpathrectangle{\pgfqpoint{0.900000in}{0.600000in}}{\pgfqpoint{3.375000in}{1.950000in}} %
\pgfusepath{clip}%
\pgfsetbuttcap%
\pgfsetroundjoin%
\definecolor{currentfill}{rgb}{1.000000,0.000000,1.000000}%
\pgfsetfillcolor{currentfill}%
\pgfsetfillopacity{0.500000}%
\pgfsetlinewidth{0.000000pt}%
\definecolor{currentstroke}{rgb}{1.000000,0.000000,1.000000}%
\pgfsetstrokecolor{currentstroke}%
\pgfsetstrokeopacity{0.500000}%
\pgfsetdash{}{0pt}%
\pgfpathmoveto{\pgfqpoint{1.650000in}{1.575000in}}%
\pgfpathlineto{\pgfqpoint{1.650000in}{1.330031in}}%
\pgfpathlineto{\pgfqpoint{1.651879in}{1.330031in}}%
\pgfpathlineto{\pgfqpoint{1.653758in}{1.330031in}}%
\pgfpathlineto{\pgfqpoint{1.655636in}{1.330031in}}%
\pgfpathlineto{\pgfqpoint{1.657515in}{1.330031in}}%
\pgfpathlineto{\pgfqpoint{1.659394in}{1.330031in}}%
\pgfpathlineto{\pgfqpoint{1.661273in}{1.330031in}}%
\pgfpathlineto{\pgfqpoint{1.663151in}{1.330031in}}%
\pgfpathlineto{\pgfqpoint{1.665030in}{1.330031in}}%
\pgfpathlineto{\pgfqpoint{1.666909in}{1.330031in}}%
\pgfpathlineto{\pgfqpoint{1.668788in}{1.330031in}}%
\pgfpathlineto{\pgfqpoint{1.670666in}{1.330031in}}%
\pgfpathlineto{\pgfqpoint{1.672545in}{1.330031in}}%
\pgfpathlineto{\pgfqpoint{1.674424in}{1.330031in}}%
\pgfpathlineto{\pgfqpoint{1.676303in}{1.330031in}}%
\pgfpathlineto{\pgfqpoint{1.678181in}{1.330031in}}%
\pgfpathlineto{\pgfqpoint{1.680060in}{1.330031in}}%
\pgfpathlineto{\pgfqpoint{1.681939in}{1.330031in}}%
\pgfpathlineto{\pgfqpoint{1.683818in}{1.330031in}}%
\pgfpathlineto{\pgfqpoint{1.685696in}{1.330031in}}%
\pgfpathlineto{\pgfqpoint{1.687575in}{1.330031in}}%
\pgfpathlineto{\pgfqpoint{1.689454in}{1.330032in}}%
\pgfpathlineto{\pgfqpoint{1.691333in}{1.330032in}}%
\pgfpathlineto{\pgfqpoint{1.693211in}{1.330032in}}%
\pgfpathlineto{\pgfqpoint{1.695090in}{1.330032in}}%
\pgfpathlineto{\pgfqpoint{1.696969in}{1.330032in}}%
\pgfpathlineto{\pgfqpoint{1.698848in}{1.330032in}}%
\pgfpathlineto{\pgfqpoint{1.700726in}{1.330032in}}%
\pgfpathlineto{\pgfqpoint{1.702605in}{1.330032in}}%
\pgfpathlineto{\pgfqpoint{1.704484in}{1.330032in}}%
\pgfpathlineto{\pgfqpoint{1.706363in}{1.330032in}}%
\pgfpathlineto{\pgfqpoint{1.708241in}{1.330032in}}%
\pgfpathlineto{\pgfqpoint{1.710120in}{1.330032in}}%
\pgfpathlineto{\pgfqpoint{1.711999in}{1.330032in}}%
\pgfpathlineto{\pgfqpoint{1.713878in}{1.330032in}}%
\pgfpathlineto{\pgfqpoint{1.715757in}{1.330032in}}%
\pgfpathlineto{\pgfqpoint{1.717635in}{1.330032in}}%
\pgfpathlineto{\pgfqpoint{1.719514in}{1.330032in}}%
\pgfpathlineto{\pgfqpoint{1.721393in}{1.330032in}}%
\pgfpathlineto{\pgfqpoint{1.723272in}{1.330032in}}%
\pgfpathlineto{\pgfqpoint{1.725150in}{1.330032in}}%
\pgfpathlineto{\pgfqpoint{1.727029in}{1.330032in}}%
\pgfpathlineto{\pgfqpoint{1.728908in}{1.330032in}}%
\pgfpathlineto{\pgfqpoint{1.730787in}{1.330032in}}%
\pgfpathlineto{\pgfqpoint{1.732665in}{1.330032in}}%
\pgfpathlineto{\pgfqpoint{1.734544in}{1.330032in}}%
\pgfpathlineto{\pgfqpoint{1.736423in}{1.330032in}}%
\pgfpathlineto{\pgfqpoint{1.738302in}{1.330032in}}%
\pgfpathlineto{\pgfqpoint{1.740180in}{1.330032in}}%
\pgfpathlineto{\pgfqpoint{1.742059in}{1.330032in}}%
\pgfpathlineto{\pgfqpoint{1.743938in}{1.330032in}}%
\pgfpathlineto{\pgfqpoint{1.745817in}{1.330032in}}%
\pgfpathlineto{\pgfqpoint{1.747695in}{1.330032in}}%
\pgfpathlineto{\pgfqpoint{1.749574in}{1.330032in}}%
\pgfpathlineto{\pgfqpoint{1.751453in}{1.330032in}}%
\pgfpathlineto{\pgfqpoint{1.753332in}{1.330032in}}%
\pgfpathlineto{\pgfqpoint{1.755210in}{1.330032in}}%
\pgfpathlineto{\pgfqpoint{1.757089in}{1.330032in}}%
\pgfpathlineto{\pgfqpoint{1.758968in}{1.330032in}}%
\pgfpathlineto{\pgfqpoint{1.760847in}{1.330032in}}%
\pgfpathlineto{\pgfqpoint{1.762725in}{1.330032in}}%
\pgfpathlineto{\pgfqpoint{1.764604in}{1.330032in}}%
\pgfpathlineto{\pgfqpoint{1.766483in}{1.330032in}}%
\pgfpathlineto{\pgfqpoint{1.768362in}{1.330032in}}%
\pgfpathlineto{\pgfqpoint{1.770240in}{1.330032in}}%
\pgfpathlineto{\pgfqpoint{1.772119in}{1.330032in}}%
\pgfpathlineto{\pgfqpoint{1.773998in}{1.330032in}}%
\pgfpathlineto{\pgfqpoint{1.775877in}{1.330032in}}%
\pgfpathlineto{\pgfqpoint{1.777756in}{1.330032in}}%
\pgfpathlineto{\pgfqpoint{1.779634in}{1.330032in}}%
\pgfpathlineto{\pgfqpoint{1.781513in}{1.330032in}}%
\pgfpathlineto{\pgfqpoint{1.783392in}{1.330032in}}%
\pgfpathlineto{\pgfqpoint{1.785271in}{1.330032in}}%
\pgfpathlineto{\pgfqpoint{1.787149in}{1.330032in}}%
\pgfpathlineto{\pgfqpoint{1.789028in}{1.330032in}}%
\pgfpathlineto{\pgfqpoint{1.790907in}{1.330033in}}%
\pgfpathlineto{\pgfqpoint{1.792786in}{1.330033in}}%
\pgfpathlineto{\pgfqpoint{1.794664in}{1.330033in}}%
\pgfpathlineto{\pgfqpoint{1.796543in}{1.330033in}}%
\pgfpathlineto{\pgfqpoint{1.798422in}{1.330033in}}%
\pgfpathlineto{\pgfqpoint{1.800301in}{1.330033in}}%
\pgfpathlineto{\pgfqpoint{1.802179in}{1.330033in}}%
\pgfpathlineto{\pgfqpoint{1.804058in}{1.330033in}}%
\pgfpathlineto{\pgfqpoint{1.805937in}{1.330033in}}%
\pgfpathlineto{\pgfqpoint{1.807816in}{1.330033in}}%
\pgfpathlineto{\pgfqpoint{1.809694in}{1.330033in}}%
\pgfpathlineto{\pgfqpoint{1.811573in}{1.330033in}}%
\pgfpathlineto{\pgfqpoint{1.813452in}{1.330033in}}%
\pgfpathlineto{\pgfqpoint{1.815331in}{1.330033in}}%
\pgfpathlineto{\pgfqpoint{1.817209in}{1.330033in}}%
\pgfpathlineto{\pgfqpoint{1.819088in}{1.330033in}}%
\pgfpathlineto{\pgfqpoint{1.820967in}{1.330033in}}%
\pgfpathlineto{\pgfqpoint{1.822846in}{1.330033in}}%
\pgfpathlineto{\pgfqpoint{1.824724in}{1.330033in}}%
\pgfpathlineto{\pgfqpoint{1.826603in}{1.330034in}}%
\pgfpathlineto{\pgfqpoint{1.828482in}{1.330034in}}%
\pgfpathlineto{\pgfqpoint{1.830361in}{1.330034in}}%
\pgfpathlineto{\pgfqpoint{1.832239in}{1.330034in}}%
\pgfpathlineto{\pgfqpoint{1.834118in}{1.330034in}}%
\pgfpathlineto{\pgfqpoint{1.835997in}{1.330034in}}%
\pgfpathlineto{\pgfqpoint{1.837876in}{1.330034in}}%
\pgfpathlineto{\pgfqpoint{1.839755in}{1.330034in}}%
\pgfpathlineto{\pgfqpoint{1.841633in}{1.330034in}}%
\pgfpathlineto{\pgfqpoint{1.843512in}{1.330034in}}%
\pgfpathlineto{\pgfqpoint{1.845391in}{1.330034in}}%
\pgfpathlineto{\pgfqpoint{1.847270in}{1.330034in}}%
\pgfpathlineto{\pgfqpoint{1.849148in}{1.330035in}}%
\pgfpathlineto{\pgfqpoint{1.851027in}{1.330035in}}%
\pgfpathlineto{\pgfqpoint{1.852906in}{1.330035in}}%
\pgfpathlineto{\pgfqpoint{1.854785in}{1.330035in}}%
\pgfpathlineto{\pgfqpoint{1.856663in}{1.330035in}}%
\pgfpathlineto{\pgfqpoint{1.858542in}{1.330035in}}%
\pgfpathlineto{\pgfqpoint{1.860421in}{1.330035in}}%
\pgfpathlineto{\pgfqpoint{1.862300in}{1.330035in}}%
\pgfpathlineto{\pgfqpoint{1.864178in}{1.330035in}}%
\pgfpathlineto{\pgfqpoint{1.866057in}{1.330036in}}%
\pgfpathlineto{\pgfqpoint{1.867936in}{1.330036in}}%
\pgfpathlineto{\pgfqpoint{1.869815in}{1.330036in}}%
\pgfpathlineto{\pgfqpoint{1.871693in}{1.330036in}}%
\pgfpathlineto{\pgfqpoint{1.873572in}{1.330036in}}%
\pgfpathlineto{\pgfqpoint{1.875451in}{1.330036in}}%
\pgfpathlineto{\pgfqpoint{1.877330in}{1.330036in}}%
\pgfpathlineto{\pgfqpoint{1.879208in}{1.330037in}}%
\pgfpathlineto{\pgfqpoint{1.881087in}{1.330037in}}%
\pgfpathlineto{\pgfqpoint{1.882966in}{1.330037in}}%
\pgfpathlineto{\pgfqpoint{1.884845in}{1.330037in}}%
\pgfpathlineto{\pgfqpoint{1.886723in}{1.330037in}}%
\pgfpathlineto{\pgfqpoint{1.888602in}{1.330037in}}%
\pgfpathlineto{\pgfqpoint{1.890481in}{1.330038in}}%
\pgfpathlineto{\pgfqpoint{1.892360in}{1.330038in}}%
\pgfpathlineto{\pgfqpoint{1.894238in}{1.330038in}}%
\pgfpathlineto{\pgfqpoint{1.896117in}{1.330038in}}%
\pgfpathlineto{\pgfqpoint{1.897996in}{1.330038in}}%
\pgfpathlineto{\pgfqpoint{1.899875in}{1.330039in}}%
\pgfpathlineto{\pgfqpoint{1.901754in}{1.330039in}}%
\pgfpathlineto{\pgfqpoint{1.903632in}{1.330039in}}%
\pgfpathlineto{\pgfqpoint{1.905511in}{1.330039in}}%
\pgfpathlineto{\pgfqpoint{1.907390in}{1.330040in}}%
\pgfpathlineto{\pgfqpoint{1.909269in}{1.330040in}}%
\pgfpathlineto{\pgfqpoint{1.911147in}{1.330040in}}%
\pgfpathlineto{\pgfqpoint{1.913026in}{1.330040in}}%
\pgfpathlineto{\pgfqpoint{1.914905in}{1.330041in}}%
\pgfpathlineto{\pgfqpoint{1.916784in}{1.330041in}}%
\pgfpathlineto{\pgfqpoint{1.918662in}{1.330041in}}%
\pgfpathlineto{\pgfqpoint{1.920541in}{1.330041in}}%
\pgfpathlineto{\pgfqpoint{1.922420in}{1.330042in}}%
\pgfpathlineto{\pgfqpoint{1.924299in}{1.330042in}}%
\pgfpathlineto{\pgfqpoint{1.926177in}{1.330042in}}%
\pgfpathlineto{\pgfqpoint{1.928056in}{1.330043in}}%
\pgfpathlineto{\pgfqpoint{1.929935in}{1.330043in}}%
\pgfpathlineto{\pgfqpoint{1.931814in}{1.330044in}}%
\pgfpathlineto{\pgfqpoint{1.933692in}{1.330044in}}%
\pgfpathlineto{\pgfqpoint{1.935571in}{1.330044in}}%
\pgfpathlineto{\pgfqpoint{1.937450in}{1.330045in}}%
\pgfpathlineto{\pgfqpoint{1.939329in}{1.330045in}}%
\pgfpathlineto{\pgfqpoint{1.941207in}{1.330045in}}%
\pgfpathlineto{\pgfqpoint{1.943086in}{1.330046in}}%
\pgfpathlineto{\pgfqpoint{1.944965in}{1.330046in}}%
\pgfpathlineto{\pgfqpoint{1.946844in}{1.330047in}}%
\pgfpathlineto{\pgfqpoint{1.948722in}{1.330047in}}%
\pgfpathlineto{\pgfqpoint{1.950601in}{1.330048in}}%
\pgfpathlineto{\pgfqpoint{1.952480in}{1.330048in}}%
\pgfpathlineto{\pgfqpoint{1.954359in}{1.330049in}}%
\pgfpathlineto{\pgfqpoint{1.956237in}{1.330049in}}%
\pgfpathlineto{\pgfqpoint{1.958116in}{1.330050in}}%
\pgfpathlineto{\pgfqpoint{1.959995in}{1.330050in}}%
\pgfpathlineto{\pgfqpoint{1.961874in}{1.330051in}}%
\pgfpathlineto{\pgfqpoint{1.963753in}{1.330052in}}%
\pgfpathlineto{\pgfqpoint{1.965631in}{1.330052in}}%
\pgfpathlineto{\pgfqpoint{1.967510in}{1.330053in}}%
\pgfpathlineto{\pgfqpoint{1.969389in}{1.330054in}}%
\pgfpathlineto{\pgfqpoint{1.971268in}{1.330054in}}%
\pgfpathlineto{\pgfqpoint{1.973146in}{1.330055in}}%
\pgfpathlineto{\pgfqpoint{1.975025in}{1.330056in}}%
\pgfpathlineto{\pgfqpoint{1.976904in}{1.330056in}}%
\pgfpathlineto{\pgfqpoint{1.978783in}{1.330057in}}%
\pgfpathlineto{\pgfqpoint{1.980661in}{1.330058in}}%
\pgfpathlineto{\pgfqpoint{1.982540in}{1.330059in}}%
\pgfpathlineto{\pgfqpoint{1.984419in}{1.330060in}}%
\pgfpathlineto{\pgfqpoint{1.986298in}{1.330061in}}%
\pgfpathlineto{\pgfqpoint{1.988176in}{1.330061in}}%
\pgfpathlineto{\pgfqpoint{1.990055in}{1.330062in}}%
\pgfpathlineto{\pgfqpoint{1.991934in}{1.330063in}}%
\pgfpathlineto{\pgfqpoint{1.993813in}{1.330064in}}%
\pgfpathlineto{\pgfqpoint{1.995691in}{1.330065in}}%
\pgfpathlineto{\pgfqpoint{1.997570in}{1.330066in}}%
\pgfpathlineto{\pgfqpoint{1.999449in}{1.330067in}}%
\pgfpathlineto{\pgfqpoint{2.001328in}{1.330069in}}%
\pgfpathlineto{\pgfqpoint{2.003206in}{1.330070in}}%
\pgfpathlineto{\pgfqpoint{2.005085in}{1.330071in}}%
\pgfpathlineto{\pgfqpoint{2.006964in}{1.330072in}}%
\pgfpathlineto{\pgfqpoint{2.008843in}{1.330073in}}%
\pgfpathlineto{\pgfqpoint{2.010721in}{1.330075in}}%
\pgfpathlineto{\pgfqpoint{2.012600in}{1.330076in}}%
\pgfpathlineto{\pgfqpoint{2.014479in}{1.330077in}}%
\pgfpathlineto{\pgfqpoint{2.016358in}{1.330079in}}%
\pgfpathlineto{\pgfqpoint{2.018236in}{1.330080in}}%
\pgfpathlineto{\pgfqpoint{2.020115in}{1.330082in}}%
\pgfpathlineto{\pgfqpoint{2.021994in}{1.330083in}}%
\pgfpathlineto{\pgfqpoint{2.023873in}{1.330085in}}%
\pgfpathlineto{\pgfqpoint{2.025752in}{1.330086in}}%
\pgfpathlineto{\pgfqpoint{2.027630in}{1.330088in}}%
\pgfpathlineto{\pgfqpoint{2.029509in}{1.330090in}}%
\pgfpathlineto{\pgfqpoint{2.031388in}{1.330092in}}%
\pgfpathlineto{\pgfqpoint{2.033267in}{1.330093in}}%
\pgfpathlineto{\pgfqpoint{2.035145in}{1.330095in}}%
\pgfpathlineto{\pgfqpoint{2.037024in}{1.330097in}}%
\pgfpathlineto{\pgfqpoint{2.038903in}{1.330099in}}%
\pgfpathlineto{\pgfqpoint{2.040782in}{1.330101in}}%
\pgfpathlineto{\pgfqpoint{2.042660in}{1.330103in}}%
\pgfpathlineto{\pgfqpoint{2.044539in}{1.330106in}}%
\pgfpathlineto{\pgfqpoint{2.046418in}{1.330108in}}%
\pgfpathlineto{\pgfqpoint{2.048297in}{1.330110in}}%
\pgfpathlineto{\pgfqpoint{2.050175in}{1.330113in}}%
\pgfpathlineto{\pgfqpoint{2.052054in}{1.330115in}}%
\pgfpathlineto{\pgfqpoint{2.053933in}{1.330118in}}%
\pgfpathlineto{\pgfqpoint{2.055812in}{1.330120in}}%
\pgfpathlineto{\pgfqpoint{2.057690in}{1.330123in}}%
\pgfpathlineto{\pgfqpoint{2.059569in}{1.330126in}}%
\pgfpathlineto{\pgfqpoint{2.061448in}{1.330129in}}%
\pgfpathlineto{\pgfqpoint{2.063327in}{1.330132in}}%
\pgfpathlineto{\pgfqpoint{2.065205in}{1.330135in}}%
\pgfpathlineto{\pgfqpoint{2.067084in}{1.330138in}}%
\pgfpathlineto{\pgfqpoint{2.068963in}{1.330141in}}%
\pgfpathlineto{\pgfqpoint{2.070842in}{1.330145in}}%
\pgfpathlineto{\pgfqpoint{2.072720in}{1.330148in}}%
\pgfpathlineto{\pgfqpoint{2.074599in}{1.330152in}}%
\pgfpathlineto{\pgfqpoint{2.076478in}{1.330155in}}%
\pgfpathlineto{\pgfqpoint{2.078357in}{1.330159in}}%
\pgfpathlineto{\pgfqpoint{2.080235in}{1.330163in}}%
\pgfpathlineto{\pgfqpoint{2.082114in}{1.330167in}}%
\pgfpathlineto{\pgfqpoint{2.083993in}{1.330171in}}%
\pgfpathlineto{\pgfqpoint{2.085872in}{1.330175in}}%
\pgfpathlineto{\pgfqpoint{2.087751in}{1.330180in}}%
\pgfpathlineto{\pgfqpoint{2.089629in}{1.330184in}}%
\pgfpathlineto{\pgfqpoint{2.091508in}{1.330189in}}%
\pgfpathlineto{\pgfqpoint{2.093387in}{1.330194in}}%
\pgfpathlineto{\pgfqpoint{2.095266in}{1.330199in}}%
\pgfpathlineto{\pgfqpoint{2.097144in}{1.330204in}}%
\pgfpathlineto{\pgfqpoint{2.099023in}{1.330209in}}%
\pgfpathlineto{\pgfqpoint{2.100902in}{1.330215in}}%
\pgfpathlineto{\pgfqpoint{2.102781in}{1.330220in}}%
\pgfpathlineto{\pgfqpoint{2.104659in}{1.330226in}}%
\pgfpathlineto{\pgfqpoint{2.106538in}{1.330232in}}%
\pgfpathlineto{\pgfqpoint{2.108417in}{1.330238in}}%
\pgfpathlineto{\pgfqpoint{2.110296in}{1.330244in}}%
\pgfpathlineto{\pgfqpoint{2.112174in}{1.330251in}}%
\pgfpathlineto{\pgfqpoint{2.114053in}{1.330257in}}%
\pgfpathlineto{\pgfqpoint{2.115932in}{1.330264in}}%
\pgfpathlineto{\pgfqpoint{2.117811in}{1.330271in}}%
\pgfpathlineto{\pgfqpoint{2.119689in}{1.330279in}}%
\pgfpathlineto{\pgfqpoint{2.121568in}{1.330286in}}%
\pgfpathlineto{\pgfqpoint{2.123447in}{1.330294in}}%
\pgfpathlineto{\pgfqpoint{2.125326in}{1.330302in}}%
\pgfpathlineto{\pgfqpoint{2.127204in}{1.330310in}}%
\pgfpathlineto{\pgfqpoint{2.129083in}{1.330319in}}%
\pgfpathlineto{\pgfqpoint{2.130962in}{1.330328in}}%
\pgfpathlineto{\pgfqpoint{2.132841in}{1.330337in}}%
\pgfpathlineto{\pgfqpoint{2.134719in}{1.330346in}}%
\pgfpathlineto{\pgfqpoint{2.136598in}{1.330356in}}%
\pgfpathlineto{\pgfqpoint{2.138477in}{1.330366in}}%
\pgfpathlineto{\pgfqpoint{2.140356in}{1.330376in}}%
\pgfpathlineto{\pgfqpoint{2.142234in}{1.330386in}}%
\pgfpathlineto{\pgfqpoint{2.144113in}{1.330397in}}%
\pgfpathlineto{\pgfqpoint{2.145992in}{1.330408in}}%
\pgfpathlineto{\pgfqpoint{2.147871in}{1.330420in}}%
\pgfpathlineto{\pgfqpoint{2.149749in}{1.330432in}}%
\pgfpathlineto{\pgfqpoint{2.151628in}{1.330444in}}%
\pgfpathlineto{\pgfqpoint{2.153507in}{1.330456in}}%
\pgfpathlineto{\pgfqpoint{2.155386in}{1.330469in}}%
\pgfpathlineto{\pgfqpoint{2.157265in}{1.330483in}}%
\pgfpathlineto{\pgfqpoint{2.159143in}{1.330496in}}%
\pgfpathlineto{\pgfqpoint{2.161022in}{1.330511in}}%
\pgfpathlineto{\pgfqpoint{2.162901in}{1.330525in}}%
\pgfpathlineto{\pgfqpoint{2.164780in}{1.330540in}}%
\pgfpathlineto{\pgfqpoint{2.166658in}{1.330556in}}%
\pgfpathlineto{\pgfqpoint{2.168537in}{1.330572in}}%
\pgfpathlineto{\pgfqpoint{2.170416in}{1.330588in}}%
\pgfpathlineto{\pgfqpoint{2.172295in}{1.330605in}}%
\pgfpathlineto{\pgfqpoint{2.174173in}{1.330623in}}%
\pgfpathlineto{\pgfqpoint{2.176052in}{1.330641in}}%
\pgfpathlineto{\pgfqpoint{2.177931in}{1.330659in}}%
\pgfpathlineto{\pgfqpoint{2.179810in}{1.330678in}}%
\pgfpathlineto{\pgfqpoint{2.181688in}{1.330698in}}%
\pgfpathlineto{\pgfqpoint{2.183567in}{1.330719in}}%
\pgfpathlineto{\pgfqpoint{2.185446in}{1.330739in}}%
\pgfpathlineto{\pgfqpoint{2.187325in}{1.330761in}}%
\pgfpathlineto{\pgfqpoint{2.189203in}{1.330783in}}%
\pgfpathlineto{\pgfqpoint{2.191082in}{1.330806in}}%
\pgfpathlineto{\pgfqpoint{2.192961in}{1.330830in}}%
\pgfpathlineto{\pgfqpoint{2.194840in}{1.330854in}}%
\pgfpathlineto{\pgfqpoint{2.196718in}{1.330879in}}%
\pgfpathlineto{\pgfqpoint{2.198597in}{1.330905in}}%
\pgfpathlineto{\pgfqpoint{2.200476in}{1.330932in}}%
\pgfpathlineto{\pgfqpoint{2.202355in}{1.330959in}}%
\pgfpathlineto{\pgfqpoint{2.204233in}{1.330987in}}%
\pgfpathlineto{\pgfqpoint{2.206112in}{1.331016in}}%
\pgfpathlineto{\pgfqpoint{2.207991in}{1.331046in}}%
\pgfpathlineto{\pgfqpoint{2.209870in}{1.331077in}}%
\pgfpathlineto{\pgfqpoint{2.211748in}{1.331109in}}%
\pgfpathlineto{\pgfqpoint{2.213627in}{1.331142in}}%
\pgfpathlineto{\pgfqpoint{2.215506in}{1.331176in}}%
\pgfpathlineto{\pgfqpoint{2.217385in}{1.331211in}}%
\pgfpathlineto{\pgfqpoint{2.219264in}{1.331246in}}%
\pgfpathlineto{\pgfqpoint{2.221142in}{1.331283in}}%
\pgfpathlineto{\pgfqpoint{2.223021in}{1.331322in}}%
\pgfpathlineto{\pgfqpoint{2.224900in}{1.331361in}}%
\pgfpathlineto{\pgfqpoint{2.226779in}{1.331401in}}%
\pgfpathlineto{\pgfqpoint{2.228657in}{1.331443in}}%
\pgfpathlineto{\pgfqpoint{2.230536in}{1.331486in}}%
\pgfpathlineto{\pgfqpoint{2.232415in}{1.331530in}}%
\pgfpathlineto{\pgfqpoint{2.234294in}{1.331576in}}%
\pgfpathlineto{\pgfqpoint{2.236172in}{1.331623in}}%
\pgfpathlineto{\pgfqpoint{2.238051in}{1.331671in}}%
\pgfpathlineto{\pgfqpoint{2.239930in}{1.331721in}}%
\pgfpathlineto{\pgfqpoint{2.241809in}{1.331772in}}%
\pgfpathlineto{\pgfqpoint{2.243687in}{1.331825in}}%
\pgfpathlineto{\pgfqpoint{2.245566in}{1.331880in}}%
\pgfpathlineto{\pgfqpoint{2.247445in}{1.331936in}}%
\pgfpathlineto{\pgfqpoint{2.249324in}{1.331994in}}%
\pgfpathlineto{\pgfqpoint{2.251202in}{1.332053in}}%
\pgfpathlineto{\pgfqpoint{2.253081in}{1.332115in}}%
\pgfpathlineto{\pgfqpoint{2.254960in}{1.332178in}}%
\pgfpathlineto{\pgfqpoint{2.256839in}{1.332243in}}%
\pgfpathlineto{\pgfqpoint{2.258717in}{1.332310in}}%
\pgfpathlineto{\pgfqpoint{2.260596in}{1.332380in}}%
\pgfpathlineto{\pgfqpoint{2.262475in}{1.332451in}}%
\pgfpathlineto{\pgfqpoint{2.264354in}{1.332524in}}%
\pgfpathlineto{\pgfqpoint{2.266232in}{1.332600in}}%
\pgfpathlineto{\pgfqpoint{2.268111in}{1.332678in}}%
\pgfpathlineto{\pgfqpoint{2.269990in}{1.332758in}}%
\pgfpathlineto{\pgfqpoint{2.271869in}{1.332841in}}%
\pgfpathlineto{\pgfqpoint{2.273747in}{1.332926in}}%
\pgfpathlineto{\pgfqpoint{2.275626in}{1.333014in}}%
\pgfpathlineto{\pgfqpoint{2.277505in}{1.333104in}}%
\pgfpathlineto{\pgfqpoint{2.279384in}{1.333197in}}%
\pgfpathlineto{\pgfqpoint{2.281263in}{1.333293in}}%
\pgfpathlineto{\pgfqpoint{2.283141in}{1.333392in}}%
\pgfpathlineto{\pgfqpoint{2.285020in}{1.333493in}}%
\pgfpathlineto{\pgfqpoint{2.286899in}{1.333598in}}%
\pgfpathlineto{\pgfqpoint{2.288778in}{1.333706in}}%
\pgfpathlineto{\pgfqpoint{2.290656in}{1.333817in}}%
\pgfpathlineto{\pgfqpoint{2.292535in}{1.333932in}}%
\pgfpathlineto{\pgfqpoint{2.294414in}{1.334050in}}%
\pgfpathlineto{\pgfqpoint{2.296293in}{1.334171in}}%
\pgfpathlineto{\pgfqpoint{2.298171in}{1.334296in}}%
\pgfpathlineto{\pgfqpoint{2.300050in}{1.334425in}}%
\pgfpathlineto{\pgfqpoint{2.301929in}{1.334558in}}%
\pgfpathlineto{\pgfqpoint{2.303808in}{1.334695in}}%
\pgfpathlineto{\pgfqpoint{2.305686in}{1.334836in}}%
\pgfpathlineto{\pgfqpoint{2.307565in}{1.334980in}}%
\pgfpathlineto{\pgfqpoint{2.309444in}{1.335130in}}%
\pgfpathlineto{\pgfqpoint{2.311323in}{1.335283in}}%
\pgfpathlineto{\pgfqpoint{2.313201in}{1.335442in}}%
\pgfpathlineto{\pgfqpoint{2.315080in}{1.335605in}}%
\pgfpathlineto{\pgfqpoint{2.316959in}{1.335773in}}%
\pgfpathlineto{\pgfqpoint{2.318838in}{1.335946in}}%
\pgfpathlineto{\pgfqpoint{2.320716in}{1.336124in}}%
\pgfpathlineto{\pgfqpoint{2.322595in}{1.336307in}}%
\pgfpathlineto{\pgfqpoint{2.324474in}{1.336496in}}%
\pgfpathlineto{\pgfqpoint{2.326353in}{1.336690in}}%
\pgfpathlineto{\pgfqpoint{2.328231in}{1.336890in}}%
\pgfpathlineto{\pgfqpoint{2.330110in}{1.337096in}}%
\pgfpathlineto{\pgfqpoint{2.331989in}{1.337308in}}%
\pgfpathlineto{\pgfqpoint{2.333868in}{1.337526in}}%
\pgfpathlineto{\pgfqpoint{2.335746in}{1.337751in}}%
\pgfpathlineto{\pgfqpoint{2.337625in}{1.337982in}}%
\pgfpathlineto{\pgfqpoint{2.339504in}{1.338220in}}%
\pgfpathlineto{\pgfqpoint{2.341383in}{1.338465in}}%
\pgfpathlineto{\pgfqpoint{2.343262in}{1.338718in}}%
\pgfpathlineto{\pgfqpoint{2.345140in}{1.338977in}}%
\pgfpathlineto{\pgfqpoint{2.347019in}{1.339245in}}%
\pgfpathlineto{\pgfqpoint{2.348898in}{1.339520in}}%
\pgfpathlineto{\pgfqpoint{2.350777in}{1.339803in}}%
\pgfpathlineto{\pgfqpoint{2.352655in}{1.340094in}}%
\pgfpathlineto{\pgfqpoint{2.354534in}{1.340394in}}%
\pgfpathlineto{\pgfqpoint{2.356413in}{1.340703in}}%
\pgfpathlineto{\pgfqpoint{2.358292in}{1.341020in}}%
\pgfpathlineto{\pgfqpoint{2.360170in}{1.341347in}}%
\pgfpathlineto{\pgfqpoint{2.362049in}{1.341683in}}%
\pgfpathlineto{\pgfqpoint{2.363928in}{1.342029in}}%
\pgfpathlineto{\pgfqpoint{2.365807in}{1.342385in}}%
\pgfpathlineto{\pgfqpoint{2.367685in}{1.342751in}}%
\pgfpathlineto{\pgfqpoint{2.369564in}{1.343128in}}%
\pgfpathlineto{\pgfqpoint{2.371443in}{1.343515in}}%
\pgfpathlineto{\pgfqpoint{2.373322in}{1.343914in}}%
\pgfpathlineto{\pgfqpoint{2.375200in}{1.344324in}}%
\pgfpathlineto{\pgfqpoint{2.377079in}{1.344745in}}%
\pgfpathlineto{\pgfqpoint{2.378958in}{1.345179in}}%
\pgfpathlineto{\pgfqpoint{2.380837in}{1.345625in}}%
\pgfpathlineto{\pgfqpoint{2.382715in}{1.346084in}}%
\pgfpathlineto{\pgfqpoint{2.384594in}{1.346555in}}%
\pgfpathlineto{\pgfqpoint{2.386473in}{1.347040in}}%
\pgfpathlineto{\pgfqpoint{2.388352in}{1.347538in}}%
\pgfpathlineto{\pgfqpoint{2.390230in}{1.348051in}}%
\pgfpathlineto{\pgfqpoint{2.392109in}{1.348578in}}%
\pgfpathlineto{\pgfqpoint{2.393988in}{1.349119in}}%
\pgfpathlineto{\pgfqpoint{2.395867in}{1.349676in}}%
\pgfpathlineto{\pgfqpoint{2.397745in}{1.350248in}}%
\pgfpathlineto{\pgfqpoint{2.399624in}{1.350835in}}%
\pgfpathlineto{\pgfqpoint{2.401503in}{1.351440in}}%
\pgfpathlineto{\pgfqpoint{2.403382in}{1.352060in}}%
\pgfpathlineto{\pgfqpoint{2.405261in}{1.352698in}}%
\pgfpathlineto{\pgfqpoint{2.407139in}{1.353353in}}%
\pgfpathlineto{\pgfqpoint{2.409018in}{1.354026in}}%
\pgfpathlineto{\pgfqpoint{2.410897in}{1.354717in}}%
\pgfpathlineto{\pgfqpoint{2.412776in}{1.355427in}}%
\pgfpathlineto{\pgfqpoint{2.414654in}{1.356156in}}%
\pgfpathlineto{\pgfqpoint{2.416533in}{1.356905in}}%
\pgfpathlineto{\pgfqpoint{2.418412in}{1.357674in}}%
\pgfpathlineto{\pgfqpoint{2.420291in}{1.358463in}}%
\pgfpathlineto{\pgfqpoint{2.422169in}{1.359273in}}%
\pgfpathlineto{\pgfqpoint{2.424048in}{1.360105in}}%
\pgfpathlineto{\pgfqpoint{2.425927in}{1.360958in}}%
\pgfpathlineto{\pgfqpoint{2.427806in}{1.361834in}}%
\pgfpathlineto{\pgfqpoint{2.429684in}{1.362732in}}%
\pgfpathlineto{\pgfqpoint{2.431563in}{1.363654in}}%
\pgfpathlineto{\pgfqpoint{2.433442in}{1.364600in}}%
\pgfpathlineto{\pgfqpoint{2.435321in}{1.365570in}}%
\pgfpathlineto{\pgfqpoint{2.437199in}{1.366565in}}%
\pgfpathlineto{\pgfqpoint{2.439078in}{1.367584in}}%
\pgfpathlineto{\pgfqpoint{2.440957in}{1.368630in}}%
\pgfpathlineto{\pgfqpoint{2.442836in}{1.369702in}}%
\pgfpathlineto{\pgfqpoint{2.444714in}{1.370801in}}%
\pgfpathlineto{\pgfqpoint{2.446593in}{1.371927in}}%
\pgfpathlineto{\pgfqpoint{2.448472in}{1.373081in}}%
\pgfpathlineto{\pgfqpoint{2.450351in}{1.374263in}}%
\pgfpathlineto{\pgfqpoint{2.452229in}{1.375473in}}%
\pgfpathlineto{\pgfqpoint{2.454108in}{1.376713in}}%
\pgfpathlineto{\pgfqpoint{2.455987in}{1.377983in}}%
\pgfpathlineto{\pgfqpoint{2.457866in}{1.379283in}}%
\pgfpathlineto{\pgfqpoint{2.459744in}{1.380614in}}%
\pgfpathlineto{\pgfqpoint{2.461623in}{1.381976in}}%
\pgfpathlineto{\pgfqpoint{2.463502in}{1.383369in}}%
\pgfpathlineto{\pgfqpoint{2.465381in}{1.384795in}}%
\pgfpathlineto{\pgfqpoint{2.467260in}{1.386253in}}%
\pgfpathlineto{\pgfqpoint{2.469138in}{1.387744in}}%
\pgfpathlineto{\pgfqpoint{2.471017in}{1.389269in}}%
\pgfpathlineto{\pgfqpoint{2.472896in}{1.390828in}}%
\pgfpathlineto{\pgfqpoint{2.474775in}{1.392421in}}%
\pgfpathlineto{\pgfqpoint{2.476653in}{1.394048in}}%
\pgfpathlineto{\pgfqpoint{2.478532in}{1.395711in}}%
\pgfpathlineto{\pgfqpoint{2.480411in}{1.397409in}}%
\pgfpathlineto{\pgfqpoint{2.482290in}{1.399143in}}%
\pgfpathlineto{\pgfqpoint{2.484168in}{1.400914in}}%
\pgfpathlineto{\pgfqpoint{2.486047in}{1.402720in}}%
\pgfpathlineto{\pgfqpoint{2.487926in}{1.404564in}}%
\pgfpathlineto{\pgfqpoint{2.489805in}{1.406444in}}%
\pgfpathlineto{\pgfqpoint{2.491683in}{1.408362in}}%
\pgfpathlineto{\pgfqpoint{2.493562in}{1.410317in}}%
\pgfpathlineto{\pgfqpoint{2.495441in}{1.412311in}}%
\pgfpathlineto{\pgfqpoint{2.497320in}{1.414342in}}%
\pgfpathlineto{\pgfqpoint{2.499198in}{1.416411in}}%
\pgfpathlineto{\pgfqpoint{2.501077in}{1.418518in}}%
\pgfpathlineto{\pgfqpoint{2.502956in}{1.420664in}}%
\pgfpathlineto{\pgfqpoint{2.504835in}{1.422848in}}%
\pgfpathlineto{\pgfqpoint{2.506713in}{1.425070in}}%
\pgfpathlineto{\pgfqpoint{2.508592in}{1.427331in}}%
\pgfpathlineto{\pgfqpoint{2.510471in}{1.429629in}}%
\pgfpathlineto{\pgfqpoint{2.512350in}{1.431967in}}%
\pgfpathlineto{\pgfqpoint{2.514228in}{1.434342in}}%
\pgfpathlineto{\pgfqpoint{2.516107in}{1.436755in}}%
\pgfpathlineto{\pgfqpoint{2.517986in}{1.439206in}}%
\pgfpathlineto{\pgfqpoint{2.519865in}{1.441694in}}%
\pgfpathlineto{\pgfqpoint{2.521743in}{1.444220in}}%
\pgfpathlineto{\pgfqpoint{2.523622in}{1.446782in}}%
\pgfpathlineto{\pgfqpoint{2.525501in}{1.449381in}}%
\pgfpathlineto{\pgfqpoint{2.527380in}{1.452017in}}%
\pgfpathlineto{\pgfqpoint{2.529259in}{1.454687in}}%
\pgfpathlineto{\pgfqpoint{2.531137in}{1.457393in}}%
\pgfpathlineto{\pgfqpoint{2.533016in}{1.460134in}}%
\pgfpathlineto{\pgfqpoint{2.534895in}{1.462908in}}%
\pgfpathlineto{\pgfqpoint{2.536774in}{1.465716in}}%
\pgfpathlineto{\pgfqpoint{2.538652in}{1.468557in}}%
\pgfpathlineto{\pgfqpoint{2.540531in}{1.471429in}}%
\pgfpathlineto{\pgfqpoint{2.542410in}{1.474333in}}%
\pgfpathlineto{\pgfqpoint{2.544289in}{1.477267in}}%
\pgfpathlineto{\pgfqpoint{2.546167in}{1.480230in}}%
\pgfpathlineto{\pgfqpoint{2.548046in}{1.483222in}}%
\pgfpathlineto{\pgfqpoint{2.549925in}{1.486242in}}%
\pgfpathlineto{\pgfqpoint{2.551804in}{1.489288in}}%
\pgfpathlineto{\pgfqpoint{2.553682in}{1.492360in}}%
\pgfpathlineto{\pgfqpoint{2.555561in}{1.495456in}}%
\pgfpathlineto{\pgfqpoint{2.557440in}{1.498576in}}%
\pgfpathlineto{\pgfqpoint{2.559319in}{1.501718in}}%
\pgfpathlineto{\pgfqpoint{2.561197in}{1.504881in}}%
\pgfpathlineto{\pgfqpoint{2.563076in}{1.508063in}}%
\pgfpathlineto{\pgfqpoint{2.564955in}{1.511264in}}%
\pgfpathlineto{\pgfqpoint{2.566834in}{1.514483in}}%
\pgfpathlineto{\pgfqpoint{2.568712in}{1.517717in}}%
\pgfpathlineto{\pgfqpoint{2.570591in}{1.520966in}}%
\pgfpathlineto{\pgfqpoint{2.572470in}{1.524228in}}%
\pgfpathlineto{\pgfqpoint{2.574349in}{1.527501in}}%
\pgfpathlineto{\pgfqpoint{2.576227in}{1.530786in}}%
\pgfpathlineto{\pgfqpoint{2.578106in}{1.534078in}}%
\pgfpathlineto{\pgfqpoint{2.579985in}{1.537379in}}%
\pgfpathlineto{\pgfqpoint{2.581864in}{1.540685in}}%
\pgfpathlineto{\pgfqpoint{2.583742in}{1.543996in}}%
\pgfpathlineto{\pgfqpoint{2.585621in}{1.547310in}}%
\pgfpathlineto{\pgfqpoint{2.587500in}{1.550625in}}%
\pgfpathlineto{\pgfqpoint{2.587500in}{1.575000in}}%
\pgfpathlineto{\pgfqpoint{2.587500in}{1.575000in}}%
\pgfpathlineto{\pgfqpoint{2.585621in}{1.575000in}}%
\pgfpathlineto{\pgfqpoint{2.583742in}{1.575000in}}%
\pgfpathlineto{\pgfqpoint{2.581864in}{1.575000in}}%
\pgfpathlineto{\pgfqpoint{2.579985in}{1.575000in}}%
\pgfpathlineto{\pgfqpoint{2.578106in}{1.575000in}}%
\pgfpathlineto{\pgfqpoint{2.576227in}{1.575000in}}%
\pgfpathlineto{\pgfqpoint{2.574349in}{1.575000in}}%
\pgfpathlineto{\pgfqpoint{2.572470in}{1.575000in}}%
\pgfpathlineto{\pgfqpoint{2.570591in}{1.575000in}}%
\pgfpathlineto{\pgfqpoint{2.568712in}{1.575000in}}%
\pgfpathlineto{\pgfqpoint{2.566834in}{1.575000in}}%
\pgfpathlineto{\pgfqpoint{2.564955in}{1.575000in}}%
\pgfpathlineto{\pgfqpoint{2.563076in}{1.575000in}}%
\pgfpathlineto{\pgfqpoint{2.561197in}{1.575000in}}%
\pgfpathlineto{\pgfqpoint{2.559319in}{1.575000in}}%
\pgfpathlineto{\pgfqpoint{2.557440in}{1.575000in}}%
\pgfpathlineto{\pgfqpoint{2.555561in}{1.575000in}}%
\pgfpathlineto{\pgfqpoint{2.553682in}{1.575000in}}%
\pgfpathlineto{\pgfqpoint{2.551804in}{1.575000in}}%
\pgfpathlineto{\pgfqpoint{2.549925in}{1.575000in}}%
\pgfpathlineto{\pgfqpoint{2.548046in}{1.575000in}}%
\pgfpathlineto{\pgfqpoint{2.546167in}{1.575000in}}%
\pgfpathlineto{\pgfqpoint{2.544289in}{1.575000in}}%
\pgfpathlineto{\pgfqpoint{2.542410in}{1.575000in}}%
\pgfpathlineto{\pgfqpoint{2.540531in}{1.575000in}}%
\pgfpathlineto{\pgfqpoint{2.538652in}{1.575000in}}%
\pgfpathlineto{\pgfqpoint{2.536774in}{1.575000in}}%
\pgfpathlineto{\pgfqpoint{2.534895in}{1.575000in}}%
\pgfpathlineto{\pgfqpoint{2.533016in}{1.575000in}}%
\pgfpathlineto{\pgfqpoint{2.531137in}{1.575000in}}%
\pgfpathlineto{\pgfqpoint{2.529259in}{1.575000in}}%
\pgfpathlineto{\pgfqpoint{2.527380in}{1.575000in}}%
\pgfpathlineto{\pgfqpoint{2.525501in}{1.575000in}}%
\pgfpathlineto{\pgfqpoint{2.523622in}{1.575000in}}%
\pgfpathlineto{\pgfqpoint{2.521743in}{1.575000in}}%
\pgfpathlineto{\pgfqpoint{2.519865in}{1.575000in}}%
\pgfpathlineto{\pgfqpoint{2.517986in}{1.575000in}}%
\pgfpathlineto{\pgfqpoint{2.516107in}{1.575000in}}%
\pgfpathlineto{\pgfqpoint{2.514228in}{1.575000in}}%
\pgfpathlineto{\pgfqpoint{2.512350in}{1.575000in}}%
\pgfpathlineto{\pgfqpoint{2.510471in}{1.575000in}}%
\pgfpathlineto{\pgfqpoint{2.508592in}{1.575000in}}%
\pgfpathlineto{\pgfqpoint{2.506713in}{1.575000in}}%
\pgfpathlineto{\pgfqpoint{2.504835in}{1.575000in}}%
\pgfpathlineto{\pgfqpoint{2.502956in}{1.575000in}}%
\pgfpathlineto{\pgfqpoint{2.501077in}{1.575000in}}%
\pgfpathlineto{\pgfqpoint{2.499198in}{1.575000in}}%
\pgfpathlineto{\pgfqpoint{2.497320in}{1.575000in}}%
\pgfpathlineto{\pgfqpoint{2.495441in}{1.575000in}}%
\pgfpathlineto{\pgfqpoint{2.493562in}{1.575000in}}%
\pgfpathlineto{\pgfqpoint{2.491683in}{1.575000in}}%
\pgfpathlineto{\pgfqpoint{2.489805in}{1.575000in}}%
\pgfpathlineto{\pgfqpoint{2.487926in}{1.575000in}}%
\pgfpathlineto{\pgfqpoint{2.486047in}{1.575000in}}%
\pgfpathlineto{\pgfqpoint{2.484168in}{1.575000in}}%
\pgfpathlineto{\pgfqpoint{2.482290in}{1.575000in}}%
\pgfpathlineto{\pgfqpoint{2.480411in}{1.575000in}}%
\pgfpathlineto{\pgfqpoint{2.478532in}{1.575000in}}%
\pgfpathlineto{\pgfqpoint{2.476653in}{1.575000in}}%
\pgfpathlineto{\pgfqpoint{2.474775in}{1.575000in}}%
\pgfpathlineto{\pgfqpoint{2.472896in}{1.575000in}}%
\pgfpathlineto{\pgfqpoint{2.471017in}{1.575000in}}%
\pgfpathlineto{\pgfqpoint{2.469138in}{1.575000in}}%
\pgfpathlineto{\pgfqpoint{2.467260in}{1.575000in}}%
\pgfpathlineto{\pgfqpoint{2.465381in}{1.575000in}}%
\pgfpathlineto{\pgfqpoint{2.463502in}{1.575000in}}%
\pgfpathlineto{\pgfqpoint{2.461623in}{1.575000in}}%
\pgfpathlineto{\pgfqpoint{2.459744in}{1.575000in}}%
\pgfpathlineto{\pgfqpoint{2.457866in}{1.575000in}}%
\pgfpathlineto{\pgfqpoint{2.455987in}{1.575000in}}%
\pgfpathlineto{\pgfqpoint{2.454108in}{1.575000in}}%
\pgfpathlineto{\pgfqpoint{2.452229in}{1.575000in}}%
\pgfpathlineto{\pgfqpoint{2.450351in}{1.575000in}}%
\pgfpathlineto{\pgfqpoint{2.448472in}{1.575000in}}%
\pgfpathlineto{\pgfqpoint{2.446593in}{1.575000in}}%
\pgfpathlineto{\pgfqpoint{2.444714in}{1.575000in}}%
\pgfpathlineto{\pgfqpoint{2.442836in}{1.575000in}}%
\pgfpathlineto{\pgfqpoint{2.440957in}{1.575000in}}%
\pgfpathlineto{\pgfqpoint{2.439078in}{1.575000in}}%
\pgfpathlineto{\pgfqpoint{2.437199in}{1.575000in}}%
\pgfpathlineto{\pgfqpoint{2.435321in}{1.575000in}}%
\pgfpathlineto{\pgfqpoint{2.433442in}{1.575000in}}%
\pgfpathlineto{\pgfqpoint{2.431563in}{1.575000in}}%
\pgfpathlineto{\pgfqpoint{2.429684in}{1.575000in}}%
\pgfpathlineto{\pgfqpoint{2.427806in}{1.575000in}}%
\pgfpathlineto{\pgfqpoint{2.425927in}{1.575000in}}%
\pgfpathlineto{\pgfqpoint{2.424048in}{1.575000in}}%
\pgfpathlineto{\pgfqpoint{2.422169in}{1.575000in}}%
\pgfpathlineto{\pgfqpoint{2.420291in}{1.575000in}}%
\pgfpathlineto{\pgfqpoint{2.418412in}{1.575000in}}%
\pgfpathlineto{\pgfqpoint{2.416533in}{1.575000in}}%
\pgfpathlineto{\pgfqpoint{2.414654in}{1.575000in}}%
\pgfpathlineto{\pgfqpoint{2.412776in}{1.575000in}}%
\pgfpathlineto{\pgfqpoint{2.410897in}{1.575000in}}%
\pgfpathlineto{\pgfqpoint{2.409018in}{1.575000in}}%
\pgfpathlineto{\pgfqpoint{2.407139in}{1.575000in}}%
\pgfpathlineto{\pgfqpoint{2.405261in}{1.575000in}}%
\pgfpathlineto{\pgfqpoint{2.403382in}{1.575000in}}%
\pgfpathlineto{\pgfqpoint{2.401503in}{1.575000in}}%
\pgfpathlineto{\pgfqpoint{2.399624in}{1.575000in}}%
\pgfpathlineto{\pgfqpoint{2.397745in}{1.575000in}}%
\pgfpathlineto{\pgfqpoint{2.395867in}{1.575000in}}%
\pgfpathlineto{\pgfqpoint{2.393988in}{1.575000in}}%
\pgfpathlineto{\pgfqpoint{2.392109in}{1.575000in}}%
\pgfpathlineto{\pgfqpoint{2.390230in}{1.575000in}}%
\pgfpathlineto{\pgfqpoint{2.388352in}{1.575000in}}%
\pgfpathlineto{\pgfqpoint{2.386473in}{1.575000in}}%
\pgfpathlineto{\pgfqpoint{2.384594in}{1.575000in}}%
\pgfpathlineto{\pgfqpoint{2.382715in}{1.575000in}}%
\pgfpathlineto{\pgfqpoint{2.380837in}{1.575000in}}%
\pgfpathlineto{\pgfqpoint{2.378958in}{1.575000in}}%
\pgfpathlineto{\pgfqpoint{2.377079in}{1.575000in}}%
\pgfpathlineto{\pgfqpoint{2.375200in}{1.575000in}}%
\pgfpathlineto{\pgfqpoint{2.373322in}{1.575000in}}%
\pgfpathlineto{\pgfqpoint{2.371443in}{1.575000in}}%
\pgfpathlineto{\pgfqpoint{2.369564in}{1.575000in}}%
\pgfpathlineto{\pgfqpoint{2.367685in}{1.575000in}}%
\pgfpathlineto{\pgfqpoint{2.365807in}{1.575000in}}%
\pgfpathlineto{\pgfqpoint{2.363928in}{1.575000in}}%
\pgfpathlineto{\pgfqpoint{2.362049in}{1.575000in}}%
\pgfpathlineto{\pgfqpoint{2.360170in}{1.575000in}}%
\pgfpathlineto{\pgfqpoint{2.358292in}{1.575000in}}%
\pgfpathlineto{\pgfqpoint{2.356413in}{1.575000in}}%
\pgfpathlineto{\pgfqpoint{2.354534in}{1.575000in}}%
\pgfpathlineto{\pgfqpoint{2.352655in}{1.575000in}}%
\pgfpathlineto{\pgfqpoint{2.350777in}{1.575000in}}%
\pgfpathlineto{\pgfqpoint{2.348898in}{1.575000in}}%
\pgfpathlineto{\pgfqpoint{2.347019in}{1.575000in}}%
\pgfpathlineto{\pgfqpoint{2.345140in}{1.575000in}}%
\pgfpathlineto{\pgfqpoint{2.343262in}{1.575000in}}%
\pgfpathlineto{\pgfqpoint{2.341383in}{1.575000in}}%
\pgfpathlineto{\pgfqpoint{2.339504in}{1.575000in}}%
\pgfpathlineto{\pgfqpoint{2.337625in}{1.575000in}}%
\pgfpathlineto{\pgfqpoint{2.335746in}{1.575000in}}%
\pgfpathlineto{\pgfqpoint{2.333868in}{1.575000in}}%
\pgfpathlineto{\pgfqpoint{2.331989in}{1.575000in}}%
\pgfpathlineto{\pgfqpoint{2.330110in}{1.575000in}}%
\pgfpathlineto{\pgfqpoint{2.328231in}{1.575000in}}%
\pgfpathlineto{\pgfqpoint{2.326353in}{1.575000in}}%
\pgfpathlineto{\pgfqpoint{2.324474in}{1.575000in}}%
\pgfpathlineto{\pgfqpoint{2.322595in}{1.575000in}}%
\pgfpathlineto{\pgfqpoint{2.320716in}{1.575000in}}%
\pgfpathlineto{\pgfqpoint{2.318838in}{1.575000in}}%
\pgfpathlineto{\pgfqpoint{2.316959in}{1.575000in}}%
\pgfpathlineto{\pgfqpoint{2.315080in}{1.575000in}}%
\pgfpathlineto{\pgfqpoint{2.313201in}{1.575000in}}%
\pgfpathlineto{\pgfqpoint{2.311323in}{1.575000in}}%
\pgfpathlineto{\pgfqpoint{2.309444in}{1.575000in}}%
\pgfpathlineto{\pgfqpoint{2.307565in}{1.575000in}}%
\pgfpathlineto{\pgfqpoint{2.305686in}{1.575000in}}%
\pgfpathlineto{\pgfqpoint{2.303808in}{1.575000in}}%
\pgfpathlineto{\pgfqpoint{2.301929in}{1.575000in}}%
\pgfpathlineto{\pgfqpoint{2.300050in}{1.575000in}}%
\pgfpathlineto{\pgfqpoint{2.298171in}{1.575000in}}%
\pgfpathlineto{\pgfqpoint{2.296293in}{1.575000in}}%
\pgfpathlineto{\pgfqpoint{2.294414in}{1.575000in}}%
\pgfpathlineto{\pgfqpoint{2.292535in}{1.575000in}}%
\pgfpathlineto{\pgfqpoint{2.290656in}{1.575000in}}%
\pgfpathlineto{\pgfqpoint{2.288778in}{1.575000in}}%
\pgfpathlineto{\pgfqpoint{2.286899in}{1.575000in}}%
\pgfpathlineto{\pgfqpoint{2.285020in}{1.575000in}}%
\pgfpathlineto{\pgfqpoint{2.283141in}{1.575000in}}%
\pgfpathlineto{\pgfqpoint{2.281263in}{1.575000in}}%
\pgfpathlineto{\pgfqpoint{2.279384in}{1.575000in}}%
\pgfpathlineto{\pgfqpoint{2.277505in}{1.575000in}}%
\pgfpathlineto{\pgfqpoint{2.275626in}{1.575000in}}%
\pgfpathlineto{\pgfqpoint{2.273747in}{1.575000in}}%
\pgfpathlineto{\pgfqpoint{2.271869in}{1.575000in}}%
\pgfpathlineto{\pgfqpoint{2.269990in}{1.575000in}}%
\pgfpathlineto{\pgfqpoint{2.268111in}{1.575000in}}%
\pgfpathlineto{\pgfqpoint{2.266232in}{1.575000in}}%
\pgfpathlineto{\pgfqpoint{2.264354in}{1.575000in}}%
\pgfpathlineto{\pgfqpoint{2.262475in}{1.575000in}}%
\pgfpathlineto{\pgfqpoint{2.260596in}{1.575000in}}%
\pgfpathlineto{\pgfqpoint{2.258717in}{1.575000in}}%
\pgfpathlineto{\pgfqpoint{2.256839in}{1.575000in}}%
\pgfpathlineto{\pgfqpoint{2.254960in}{1.575000in}}%
\pgfpathlineto{\pgfqpoint{2.253081in}{1.575000in}}%
\pgfpathlineto{\pgfqpoint{2.251202in}{1.575000in}}%
\pgfpathlineto{\pgfqpoint{2.249324in}{1.575000in}}%
\pgfpathlineto{\pgfqpoint{2.247445in}{1.575000in}}%
\pgfpathlineto{\pgfqpoint{2.245566in}{1.575000in}}%
\pgfpathlineto{\pgfqpoint{2.243687in}{1.575000in}}%
\pgfpathlineto{\pgfqpoint{2.241809in}{1.575000in}}%
\pgfpathlineto{\pgfqpoint{2.239930in}{1.575000in}}%
\pgfpathlineto{\pgfqpoint{2.238051in}{1.575000in}}%
\pgfpathlineto{\pgfqpoint{2.236172in}{1.575000in}}%
\pgfpathlineto{\pgfqpoint{2.234294in}{1.575000in}}%
\pgfpathlineto{\pgfqpoint{2.232415in}{1.575000in}}%
\pgfpathlineto{\pgfqpoint{2.230536in}{1.575000in}}%
\pgfpathlineto{\pgfqpoint{2.228657in}{1.575000in}}%
\pgfpathlineto{\pgfqpoint{2.226779in}{1.575000in}}%
\pgfpathlineto{\pgfqpoint{2.224900in}{1.575000in}}%
\pgfpathlineto{\pgfqpoint{2.223021in}{1.575000in}}%
\pgfpathlineto{\pgfqpoint{2.221142in}{1.575000in}}%
\pgfpathlineto{\pgfqpoint{2.219264in}{1.575000in}}%
\pgfpathlineto{\pgfqpoint{2.217385in}{1.575000in}}%
\pgfpathlineto{\pgfqpoint{2.215506in}{1.575000in}}%
\pgfpathlineto{\pgfqpoint{2.213627in}{1.575000in}}%
\pgfpathlineto{\pgfqpoint{2.211748in}{1.575000in}}%
\pgfpathlineto{\pgfqpoint{2.209870in}{1.575000in}}%
\pgfpathlineto{\pgfqpoint{2.207991in}{1.575000in}}%
\pgfpathlineto{\pgfqpoint{2.206112in}{1.575000in}}%
\pgfpathlineto{\pgfqpoint{2.204233in}{1.575000in}}%
\pgfpathlineto{\pgfqpoint{2.202355in}{1.575000in}}%
\pgfpathlineto{\pgfqpoint{2.200476in}{1.575000in}}%
\pgfpathlineto{\pgfqpoint{2.198597in}{1.575000in}}%
\pgfpathlineto{\pgfqpoint{2.196718in}{1.575000in}}%
\pgfpathlineto{\pgfqpoint{2.194840in}{1.575000in}}%
\pgfpathlineto{\pgfqpoint{2.192961in}{1.575000in}}%
\pgfpathlineto{\pgfqpoint{2.191082in}{1.575000in}}%
\pgfpathlineto{\pgfqpoint{2.189203in}{1.575000in}}%
\pgfpathlineto{\pgfqpoint{2.187325in}{1.575000in}}%
\pgfpathlineto{\pgfqpoint{2.185446in}{1.575000in}}%
\pgfpathlineto{\pgfqpoint{2.183567in}{1.575000in}}%
\pgfpathlineto{\pgfqpoint{2.181688in}{1.575000in}}%
\pgfpathlineto{\pgfqpoint{2.179810in}{1.575000in}}%
\pgfpathlineto{\pgfqpoint{2.177931in}{1.575000in}}%
\pgfpathlineto{\pgfqpoint{2.176052in}{1.575000in}}%
\pgfpathlineto{\pgfqpoint{2.174173in}{1.575000in}}%
\pgfpathlineto{\pgfqpoint{2.172295in}{1.575000in}}%
\pgfpathlineto{\pgfqpoint{2.170416in}{1.575000in}}%
\pgfpathlineto{\pgfqpoint{2.168537in}{1.575000in}}%
\pgfpathlineto{\pgfqpoint{2.166658in}{1.575000in}}%
\pgfpathlineto{\pgfqpoint{2.164780in}{1.575000in}}%
\pgfpathlineto{\pgfqpoint{2.162901in}{1.575000in}}%
\pgfpathlineto{\pgfqpoint{2.161022in}{1.575000in}}%
\pgfpathlineto{\pgfqpoint{2.159143in}{1.575000in}}%
\pgfpathlineto{\pgfqpoint{2.157265in}{1.575000in}}%
\pgfpathlineto{\pgfqpoint{2.155386in}{1.575000in}}%
\pgfpathlineto{\pgfqpoint{2.153507in}{1.575000in}}%
\pgfpathlineto{\pgfqpoint{2.151628in}{1.575000in}}%
\pgfpathlineto{\pgfqpoint{2.149749in}{1.575000in}}%
\pgfpathlineto{\pgfqpoint{2.147871in}{1.575000in}}%
\pgfpathlineto{\pgfqpoint{2.145992in}{1.575000in}}%
\pgfpathlineto{\pgfqpoint{2.144113in}{1.575000in}}%
\pgfpathlineto{\pgfqpoint{2.142234in}{1.575000in}}%
\pgfpathlineto{\pgfqpoint{2.140356in}{1.575000in}}%
\pgfpathlineto{\pgfqpoint{2.138477in}{1.575000in}}%
\pgfpathlineto{\pgfqpoint{2.136598in}{1.575000in}}%
\pgfpathlineto{\pgfqpoint{2.134719in}{1.575000in}}%
\pgfpathlineto{\pgfqpoint{2.132841in}{1.575000in}}%
\pgfpathlineto{\pgfqpoint{2.130962in}{1.575000in}}%
\pgfpathlineto{\pgfqpoint{2.129083in}{1.575000in}}%
\pgfpathlineto{\pgfqpoint{2.127204in}{1.575000in}}%
\pgfpathlineto{\pgfqpoint{2.125326in}{1.575000in}}%
\pgfpathlineto{\pgfqpoint{2.123447in}{1.575000in}}%
\pgfpathlineto{\pgfqpoint{2.121568in}{1.575000in}}%
\pgfpathlineto{\pgfqpoint{2.119689in}{1.575000in}}%
\pgfpathlineto{\pgfqpoint{2.117811in}{1.575000in}}%
\pgfpathlineto{\pgfqpoint{2.115932in}{1.575000in}}%
\pgfpathlineto{\pgfqpoint{2.114053in}{1.575000in}}%
\pgfpathlineto{\pgfqpoint{2.112174in}{1.575000in}}%
\pgfpathlineto{\pgfqpoint{2.110296in}{1.575000in}}%
\pgfpathlineto{\pgfqpoint{2.108417in}{1.575000in}}%
\pgfpathlineto{\pgfqpoint{2.106538in}{1.575000in}}%
\pgfpathlineto{\pgfqpoint{2.104659in}{1.575000in}}%
\pgfpathlineto{\pgfqpoint{2.102781in}{1.575000in}}%
\pgfpathlineto{\pgfqpoint{2.100902in}{1.575000in}}%
\pgfpathlineto{\pgfqpoint{2.099023in}{1.575000in}}%
\pgfpathlineto{\pgfqpoint{2.097144in}{1.575000in}}%
\pgfpathlineto{\pgfqpoint{2.095266in}{1.575000in}}%
\pgfpathlineto{\pgfqpoint{2.093387in}{1.575000in}}%
\pgfpathlineto{\pgfqpoint{2.091508in}{1.575000in}}%
\pgfpathlineto{\pgfqpoint{2.089629in}{1.575000in}}%
\pgfpathlineto{\pgfqpoint{2.087751in}{1.575000in}}%
\pgfpathlineto{\pgfqpoint{2.085872in}{1.575000in}}%
\pgfpathlineto{\pgfqpoint{2.083993in}{1.575000in}}%
\pgfpathlineto{\pgfqpoint{2.082114in}{1.575000in}}%
\pgfpathlineto{\pgfqpoint{2.080235in}{1.575000in}}%
\pgfpathlineto{\pgfqpoint{2.078357in}{1.575000in}}%
\pgfpathlineto{\pgfqpoint{2.076478in}{1.575000in}}%
\pgfpathlineto{\pgfqpoint{2.074599in}{1.575000in}}%
\pgfpathlineto{\pgfqpoint{2.072720in}{1.575000in}}%
\pgfpathlineto{\pgfqpoint{2.070842in}{1.575000in}}%
\pgfpathlineto{\pgfqpoint{2.068963in}{1.575000in}}%
\pgfpathlineto{\pgfqpoint{2.067084in}{1.575000in}}%
\pgfpathlineto{\pgfqpoint{2.065205in}{1.575000in}}%
\pgfpathlineto{\pgfqpoint{2.063327in}{1.575000in}}%
\pgfpathlineto{\pgfqpoint{2.061448in}{1.575000in}}%
\pgfpathlineto{\pgfqpoint{2.059569in}{1.575000in}}%
\pgfpathlineto{\pgfqpoint{2.057690in}{1.575000in}}%
\pgfpathlineto{\pgfqpoint{2.055812in}{1.575000in}}%
\pgfpathlineto{\pgfqpoint{2.053933in}{1.575000in}}%
\pgfpathlineto{\pgfqpoint{2.052054in}{1.575000in}}%
\pgfpathlineto{\pgfqpoint{2.050175in}{1.575000in}}%
\pgfpathlineto{\pgfqpoint{2.048297in}{1.575000in}}%
\pgfpathlineto{\pgfqpoint{2.046418in}{1.575000in}}%
\pgfpathlineto{\pgfqpoint{2.044539in}{1.575000in}}%
\pgfpathlineto{\pgfqpoint{2.042660in}{1.575000in}}%
\pgfpathlineto{\pgfqpoint{2.040782in}{1.575000in}}%
\pgfpathlineto{\pgfqpoint{2.038903in}{1.575000in}}%
\pgfpathlineto{\pgfqpoint{2.037024in}{1.575000in}}%
\pgfpathlineto{\pgfqpoint{2.035145in}{1.575000in}}%
\pgfpathlineto{\pgfqpoint{2.033267in}{1.575000in}}%
\pgfpathlineto{\pgfqpoint{2.031388in}{1.575000in}}%
\pgfpathlineto{\pgfqpoint{2.029509in}{1.575000in}}%
\pgfpathlineto{\pgfqpoint{2.027630in}{1.575000in}}%
\pgfpathlineto{\pgfqpoint{2.025752in}{1.575000in}}%
\pgfpathlineto{\pgfqpoint{2.023873in}{1.575000in}}%
\pgfpathlineto{\pgfqpoint{2.021994in}{1.575000in}}%
\pgfpathlineto{\pgfqpoint{2.020115in}{1.575000in}}%
\pgfpathlineto{\pgfqpoint{2.018236in}{1.575000in}}%
\pgfpathlineto{\pgfqpoint{2.016358in}{1.575000in}}%
\pgfpathlineto{\pgfqpoint{2.014479in}{1.575000in}}%
\pgfpathlineto{\pgfqpoint{2.012600in}{1.575000in}}%
\pgfpathlineto{\pgfqpoint{2.010721in}{1.575000in}}%
\pgfpathlineto{\pgfqpoint{2.008843in}{1.575000in}}%
\pgfpathlineto{\pgfqpoint{2.006964in}{1.575000in}}%
\pgfpathlineto{\pgfqpoint{2.005085in}{1.575000in}}%
\pgfpathlineto{\pgfqpoint{2.003206in}{1.575000in}}%
\pgfpathlineto{\pgfqpoint{2.001328in}{1.575000in}}%
\pgfpathlineto{\pgfqpoint{1.999449in}{1.575000in}}%
\pgfpathlineto{\pgfqpoint{1.997570in}{1.575000in}}%
\pgfpathlineto{\pgfqpoint{1.995691in}{1.575000in}}%
\pgfpathlineto{\pgfqpoint{1.993813in}{1.575000in}}%
\pgfpathlineto{\pgfqpoint{1.991934in}{1.575000in}}%
\pgfpathlineto{\pgfqpoint{1.990055in}{1.575000in}}%
\pgfpathlineto{\pgfqpoint{1.988176in}{1.575000in}}%
\pgfpathlineto{\pgfqpoint{1.986298in}{1.575000in}}%
\pgfpathlineto{\pgfqpoint{1.984419in}{1.575000in}}%
\pgfpathlineto{\pgfqpoint{1.982540in}{1.575000in}}%
\pgfpathlineto{\pgfqpoint{1.980661in}{1.575000in}}%
\pgfpathlineto{\pgfqpoint{1.978783in}{1.575000in}}%
\pgfpathlineto{\pgfqpoint{1.976904in}{1.575000in}}%
\pgfpathlineto{\pgfqpoint{1.975025in}{1.575000in}}%
\pgfpathlineto{\pgfqpoint{1.973146in}{1.575000in}}%
\pgfpathlineto{\pgfqpoint{1.971268in}{1.575000in}}%
\pgfpathlineto{\pgfqpoint{1.969389in}{1.575000in}}%
\pgfpathlineto{\pgfqpoint{1.967510in}{1.575000in}}%
\pgfpathlineto{\pgfqpoint{1.965631in}{1.575000in}}%
\pgfpathlineto{\pgfqpoint{1.963753in}{1.575000in}}%
\pgfpathlineto{\pgfqpoint{1.961874in}{1.575000in}}%
\pgfpathlineto{\pgfqpoint{1.959995in}{1.575000in}}%
\pgfpathlineto{\pgfqpoint{1.958116in}{1.575000in}}%
\pgfpathlineto{\pgfqpoint{1.956237in}{1.575000in}}%
\pgfpathlineto{\pgfqpoint{1.954359in}{1.575000in}}%
\pgfpathlineto{\pgfqpoint{1.952480in}{1.575000in}}%
\pgfpathlineto{\pgfqpoint{1.950601in}{1.575000in}}%
\pgfpathlineto{\pgfqpoint{1.948722in}{1.575000in}}%
\pgfpathlineto{\pgfqpoint{1.946844in}{1.575000in}}%
\pgfpathlineto{\pgfqpoint{1.944965in}{1.575000in}}%
\pgfpathlineto{\pgfqpoint{1.943086in}{1.575000in}}%
\pgfpathlineto{\pgfqpoint{1.941207in}{1.575000in}}%
\pgfpathlineto{\pgfqpoint{1.939329in}{1.575000in}}%
\pgfpathlineto{\pgfqpoint{1.937450in}{1.575000in}}%
\pgfpathlineto{\pgfqpoint{1.935571in}{1.575000in}}%
\pgfpathlineto{\pgfqpoint{1.933692in}{1.575000in}}%
\pgfpathlineto{\pgfqpoint{1.931814in}{1.575000in}}%
\pgfpathlineto{\pgfqpoint{1.929935in}{1.575000in}}%
\pgfpathlineto{\pgfqpoint{1.928056in}{1.575000in}}%
\pgfpathlineto{\pgfqpoint{1.926177in}{1.575000in}}%
\pgfpathlineto{\pgfqpoint{1.924299in}{1.575000in}}%
\pgfpathlineto{\pgfqpoint{1.922420in}{1.575000in}}%
\pgfpathlineto{\pgfqpoint{1.920541in}{1.575000in}}%
\pgfpathlineto{\pgfqpoint{1.918662in}{1.575000in}}%
\pgfpathlineto{\pgfqpoint{1.916784in}{1.575000in}}%
\pgfpathlineto{\pgfqpoint{1.914905in}{1.575000in}}%
\pgfpathlineto{\pgfqpoint{1.913026in}{1.575000in}}%
\pgfpathlineto{\pgfqpoint{1.911147in}{1.575000in}}%
\pgfpathlineto{\pgfqpoint{1.909269in}{1.575000in}}%
\pgfpathlineto{\pgfqpoint{1.907390in}{1.575000in}}%
\pgfpathlineto{\pgfqpoint{1.905511in}{1.575000in}}%
\pgfpathlineto{\pgfqpoint{1.903632in}{1.575000in}}%
\pgfpathlineto{\pgfqpoint{1.901754in}{1.575000in}}%
\pgfpathlineto{\pgfqpoint{1.899875in}{1.575000in}}%
\pgfpathlineto{\pgfqpoint{1.897996in}{1.575000in}}%
\pgfpathlineto{\pgfqpoint{1.896117in}{1.575000in}}%
\pgfpathlineto{\pgfqpoint{1.894238in}{1.575000in}}%
\pgfpathlineto{\pgfqpoint{1.892360in}{1.575000in}}%
\pgfpathlineto{\pgfqpoint{1.890481in}{1.575000in}}%
\pgfpathlineto{\pgfqpoint{1.888602in}{1.575000in}}%
\pgfpathlineto{\pgfqpoint{1.886723in}{1.575000in}}%
\pgfpathlineto{\pgfqpoint{1.884845in}{1.575000in}}%
\pgfpathlineto{\pgfqpoint{1.882966in}{1.575000in}}%
\pgfpathlineto{\pgfqpoint{1.881087in}{1.575000in}}%
\pgfpathlineto{\pgfqpoint{1.879208in}{1.575000in}}%
\pgfpathlineto{\pgfqpoint{1.877330in}{1.575000in}}%
\pgfpathlineto{\pgfqpoint{1.875451in}{1.575000in}}%
\pgfpathlineto{\pgfqpoint{1.873572in}{1.575000in}}%
\pgfpathlineto{\pgfqpoint{1.871693in}{1.575000in}}%
\pgfpathlineto{\pgfqpoint{1.869815in}{1.575000in}}%
\pgfpathlineto{\pgfqpoint{1.867936in}{1.575000in}}%
\pgfpathlineto{\pgfqpoint{1.866057in}{1.575000in}}%
\pgfpathlineto{\pgfqpoint{1.864178in}{1.575000in}}%
\pgfpathlineto{\pgfqpoint{1.862300in}{1.575000in}}%
\pgfpathlineto{\pgfqpoint{1.860421in}{1.575000in}}%
\pgfpathlineto{\pgfqpoint{1.858542in}{1.575000in}}%
\pgfpathlineto{\pgfqpoint{1.856663in}{1.575000in}}%
\pgfpathlineto{\pgfqpoint{1.854785in}{1.575000in}}%
\pgfpathlineto{\pgfqpoint{1.852906in}{1.575000in}}%
\pgfpathlineto{\pgfqpoint{1.851027in}{1.575000in}}%
\pgfpathlineto{\pgfqpoint{1.849148in}{1.575000in}}%
\pgfpathlineto{\pgfqpoint{1.847270in}{1.575000in}}%
\pgfpathlineto{\pgfqpoint{1.845391in}{1.575000in}}%
\pgfpathlineto{\pgfqpoint{1.843512in}{1.575000in}}%
\pgfpathlineto{\pgfqpoint{1.841633in}{1.575000in}}%
\pgfpathlineto{\pgfqpoint{1.839755in}{1.575000in}}%
\pgfpathlineto{\pgfqpoint{1.837876in}{1.575000in}}%
\pgfpathlineto{\pgfqpoint{1.835997in}{1.575000in}}%
\pgfpathlineto{\pgfqpoint{1.834118in}{1.575000in}}%
\pgfpathlineto{\pgfqpoint{1.832239in}{1.575000in}}%
\pgfpathlineto{\pgfqpoint{1.830361in}{1.575000in}}%
\pgfpathlineto{\pgfqpoint{1.828482in}{1.575000in}}%
\pgfpathlineto{\pgfqpoint{1.826603in}{1.575000in}}%
\pgfpathlineto{\pgfqpoint{1.824724in}{1.575000in}}%
\pgfpathlineto{\pgfqpoint{1.822846in}{1.575000in}}%
\pgfpathlineto{\pgfqpoint{1.820967in}{1.575000in}}%
\pgfpathlineto{\pgfqpoint{1.819088in}{1.575000in}}%
\pgfpathlineto{\pgfqpoint{1.817209in}{1.575000in}}%
\pgfpathlineto{\pgfqpoint{1.815331in}{1.575000in}}%
\pgfpathlineto{\pgfqpoint{1.813452in}{1.575000in}}%
\pgfpathlineto{\pgfqpoint{1.811573in}{1.575000in}}%
\pgfpathlineto{\pgfqpoint{1.809694in}{1.575000in}}%
\pgfpathlineto{\pgfqpoint{1.807816in}{1.575000in}}%
\pgfpathlineto{\pgfqpoint{1.805937in}{1.575000in}}%
\pgfpathlineto{\pgfqpoint{1.804058in}{1.575000in}}%
\pgfpathlineto{\pgfqpoint{1.802179in}{1.575000in}}%
\pgfpathlineto{\pgfqpoint{1.800301in}{1.575000in}}%
\pgfpathlineto{\pgfqpoint{1.798422in}{1.575000in}}%
\pgfpathlineto{\pgfqpoint{1.796543in}{1.575000in}}%
\pgfpathlineto{\pgfqpoint{1.794664in}{1.575000in}}%
\pgfpathlineto{\pgfqpoint{1.792786in}{1.575000in}}%
\pgfpathlineto{\pgfqpoint{1.790907in}{1.575000in}}%
\pgfpathlineto{\pgfqpoint{1.789028in}{1.575000in}}%
\pgfpathlineto{\pgfqpoint{1.787149in}{1.575000in}}%
\pgfpathlineto{\pgfqpoint{1.785271in}{1.575000in}}%
\pgfpathlineto{\pgfqpoint{1.783392in}{1.575000in}}%
\pgfpathlineto{\pgfqpoint{1.781513in}{1.575000in}}%
\pgfpathlineto{\pgfqpoint{1.779634in}{1.575000in}}%
\pgfpathlineto{\pgfqpoint{1.777756in}{1.575000in}}%
\pgfpathlineto{\pgfqpoint{1.775877in}{1.575000in}}%
\pgfpathlineto{\pgfqpoint{1.773998in}{1.575000in}}%
\pgfpathlineto{\pgfqpoint{1.772119in}{1.575000in}}%
\pgfpathlineto{\pgfqpoint{1.770240in}{1.575000in}}%
\pgfpathlineto{\pgfqpoint{1.768362in}{1.575000in}}%
\pgfpathlineto{\pgfqpoint{1.766483in}{1.575000in}}%
\pgfpathlineto{\pgfqpoint{1.764604in}{1.575000in}}%
\pgfpathlineto{\pgfqpoint{1.762725in}{1.575000in}}%
\pgfpathlineto{\pgfqpoint{1.760847in}{1.575000in}}%
\pgfpathlineto{\pgfqpoint{1.758968in}{1.575000in}}%
\pgfpathlineto{\pgfqpoint{1.757089in}{1.575000in}}%
\pgfpathlineto{\pgfqpoint{1.755210in}{1.575000in}}%
\pgfpathlineto{\pgfqpoint{1.753332in}{1.575000in}}%
\pgfpathlineto{\pgfqpoint{1.751453in}{1.575000in}}%
\pgfpathlineto{\pgfqpoint{1.749574in}{1.575000in}}%
\pgfpathlineto{\pgfqpoint{1.747695in}{1.575000in}}%
\pgfpathlineto{\pgfqpoint{1.745817in}{1.575000in}}%
\pgfpathlineto{\pgfqpoint{1.743938in}{1.575000in}}%
\pgfpathlineto{\pgfqpoint{1.742059in}{1.575000in}}%
\pgfpathlineto{\pgfqpoint{1.740180in}{1.575000in}}%
\pgfpathlineto{\pgfqpoint{1.738302in}{1.575000in}}%
\pgfpathlineto{\pgfqpoint{1.736423in}{1.575000in}}%
\pgfpathlineto{\pgfqpoint{1.734544in}{1.575000in}}%
\pgfpathlineto{\pgfqpoint{1.732665in}{1.575000in}}%
\pgfpathlineto{\pgfqpoint{1.730787in}{1.575000in}}%
\pgfpathlineto{\pgfqpoint{1.728908in}{1.575000in}}%
\pgfpathlineto{\pgfqpoint{1.727029in}{1.575000in}}%
\pgfpathlineto{\pgfqpoint{1.725150in}{1.575000in}}%
\pgfpathlineto{\pgfqpoint{1.723272in}{1.575000in}}%
\pgfpathlineto{\pgfqpoint{1.721393in}{1.575000in}}%
\pgfpathlineto{\pgfqpoint{1.719514in}{1.575000in}}%
\pgfpathlineto{\pgfqpoint{1.717635in}{1.575000in}}%
\pgfpathlineto{\pgfqpoint{1.715757in}{1.575000in}}%
\pgfpathlineto{\pgfqpoint{1.713878in}{1.575000in}}%
\pgfpathlineto{\pgfqpoint{1.711999in}{1.575000in}}%
\pgfpathlineto{\pgfqpoint{1.710120in}{1.575000in}}%
\pgfpathlineto{\pgfqpoint{1.708241in}{1.575000in}}%
\pgfpathlineto{\pgfqpoint{1.706363in}{1.575000in}}%
\pgfpathlineto{\pgfqpoint{1.704484in}{1.575000in}}%
\pgfpathlineto{\pgfqpoint{1.702605in}{1.575000in}}%
\pgfpathlineto{\pgfqpoint{1.700726in}{1.575000in}}%
\pgfpathlineto{\pgfqpoint{1.698848in}{1.575000in}}%
\pgfpathlineto{\pgfqpoint{1.696969in}{1.575000in}}%
\pgfpathlineto{\pgfqpoint{1.695090in}{1.575000in}}%
\pgfpathlineto{\pgfqpoint{1.693211in}{1.575000in}}%
\pgfpathlineto{\pgfqpoint{1.691333in}{1.575000in}}%
\pgfpathlineto{\pgfqpoint{1.689454in}{1.575000in}}%
\pgfpathlineto{\pgfqpoint{1.687575in}{1.575000in}}%
\pgfpathlineto{\pgfqpoint{1.685696in}{1.575000in}}%
\pgfpathlineto{\pgfqpoint{1.683818in}{1.575000in}}%
\pgfpathlineto{\pgfqpoint{1.681939in}{1.575000in}}%
\pgfpathlineto{\pgfqpoint{1.680060in}{1.575000in}}%
\pgfpathlineto{\pgfqpoint{1.678181in}{1.575000in}}%
\pgfpathlineto{\pgfqpoint{1.676303in}{1.575000in}}%
\pgfpathlineto{\pgfqpoint{1.674424in}{1.575000in}}%
\pgfpathlineto{\pgfqpoint{1.672545in}{1.575000in}}%
\pgfpathlineto{\pgfqpoint{1.670666in}{1.575000in}}%
\pgfpathlineto{\pgfqpoint{1.668788in}{1.575000in}}%
\pgfpathlineto{\pgfqpoint{1.666909in}{1.575000in}}%
\pgfpathlineto{\pgfqpoint{1.665030in}{1.575000in}}%
\pgfpathlineto{\pgfqpoint{1.663151in}{1.575000in}}%
\pgfpathlineto{\pgfqpoint{1.661273in}{1.575000in}}%
\pgfpathlineto{\pgfqpoint{1.659394in}{1.575000in}}%
\pgfpathlineto{\pgfqpoint{1.657515in}{1.575000in}}%
\pgfpathlineto{\pgfqpoint{1.655636in}{1.575000in}}%
\pgfpathlineto{\pgfqpoint{1.653758in}{1.575000in}}%
\pgfpathlineto{\pgfqpoint{1.651879in}{1.575000in}}%
\pgfpathlineto{\pgfqpoint{1.650000in}{1.575000in}}%
\pgfpathclose%
\pgfusepath{fill}%
\end{pgfscope}%
\begin{pgfscope}%
\pgfpathrectangle{\pgfqpoint{0.900000in}{0.600000in}}{\pgfqpoint{3.375000in}{1.950000in}} %
\pgfusepath{clip}%
\pgfsetbuttcap%
\pgfsetroundjoin%
\definecolor{currentfill}{rgb}{1.000000,0.000000,1.000000}%
\pgfsetfillcolor{currentfill}%
\pgfsetfillopacity{0.500000}%
\pgfsetlinewidth{0.000000pt}%
\definecolor{currentstroke}{rgb}{1.000000,0.000000,1.000000}%
\pgfsetstrokecolor{currentstroke}%
\pgfsetstrokeopacity{0.500000}%
\pgfsetdash{}{0pt}%
\pgfpathmoveto{\pgfqpoint{1.518750in}{1.575000in}}%
\pgfpathlineto{\pgfqpoint{1.518750in}{1.250871in}}%
\pgfpathlineto{\pgfqpoint{1.519015in}{1.251146in}}%
\pgfpathlineto{\pgfqpoint{1.519280in}{1.251420in}}%
\pgfpathlineto{\pgfqpoint{1.519545in}{1.251694in}}%
\pgfpathlineto{\pgfqpoint{1.519810in}{1.251966in}}%
\pgfpathlineto{\pgfqpoint{1.520075in}{1.252238in}}%
\pgfpathlineto{\pgfqpoint{1.520339in}{1.252509in}}%
\pgfpathlineto{\pgfqpoint{1.520604in}{1.252778in}}%
\pgfpathlineto{\pgfqpoint{1.520869in}{1.253047in}}%
\pgfpathlineto{\pgfqpoint{1.521134in}{1.253315in}}%
\pgfpathlineto{\pgfqpoint{1.521399in}{1.253582in}}%
\pgfpathlineto{\pgfqpoint{1.521664in}{1.253848in}}%
\pgfpathlineto{\pgfqpoint{1.521929in}{1.254113in}}%
\pgfpathlineto{\pgfqpoint{1.522194in}{1.254378in}}%
\pgfpathlineto{\pgfqpoint{1.522459in}{1.254641in}}%
\pgfpathlineto{\pgfqpoint{1.522724in}{1.254904in}}%
\pgfpathlineto{\pgfqpoint{1.522988in}{1.255165in}}%
\pgfpathlineto{\pgfqpoint{1.523253in}{1.255426in}}%
\pgfpathlineto{\pgfqpoint{1.523518in}{1.255686in}}%
\pgfpathlineto{\pgfqpoint{1.523783in}{1.255944in}}%
\pgfpathlineto{\pgfqpoint{1.524048in}{1.256202in}}%
\pgfpathlineto{\pgfqpoint{1.524313in}{1.256460in}}%
\pgfpathlineto{\pgfqpoint{1.524578in}{1.256716in}}%
\pgfpathlineto{\pgfqpoint{1.524843in}{1.256971in}}%
\pgfpathlineto{\pgfqpoint{1.525108in}{1.257226in}}%
\pgfpathlineto{\pgfqpoint{1.525373in}{1.257479in}}%
\pgfpathlineto{\pgfqpoint{1.525638in}{1.257732in}}%
\pgfpathlineto{\pgfqpoint{1.525902in}{1.257984in}}%
\pgfpathlineto{\pgfqpoint{1.526167in}{1.258235in}}%
\pgfpathlineto{\pgfqpoint{1.526432in}{1.258485in}}%
\pgfpathlineto{\pgfqpoint{1.526697in}{1.258734in}}%
\pgfpathlineto{\pgfqpoint{1.526962in}{1.258983in}}%
\pgfpathlineto{\pgfqpoint{1.527227in}{1.259230in}}%
\pgfpathlineto{\pgfqpoint{1.527492in}{1.259477in}}%
\pgfpathlineto{\pgfqpoint{1.527757in}{1.259723in}}%
\pgfpathlineto{\pgfqpoint{1.528022in}{1.259968in}}%
\pgfpathlineto{\pgfqpoint{1.528287in}{1.260212in}}%
\pgfpathlineto{\pgfqpoint{1.528551in}{1.260455in}}%
\pgfpathlineto{\pgfqpoint{1.528816in}{1.260698in}}%
\pgfpathlineto{\pgfqpoint{1.529081in}{1.260940in}}%
\pgfpathlineto{\pgfqpoint{1.529346in}{1.261181in}}%
\pgfpathlineto{\pgfqpoint{1.529611in}{1.261421in}}%
\pgfpathlineto{\pgfqpoint{1.529876in}{1.261660in}}%
\pgfpathlineto{\pgfqpoint{1.530141in}{1.261898in}}%
\pgfpathlineto{\pgfqpoint{1.530406in}{1.262136in}}%
\pgfpathlineto{\pgfqpoint{1.530671in}{1.262373in}}%
\pgfpathlineto{\pgfqpoint{1.530936in}{1.262609in}}%
\pgfpathlineto{\pgfqpoint{1.531201in}{1.262844in}}%
\pgfpathlineto{\pgfqpoint{1.531465in}{1.263078in}}%
\pgfpathlineto{\pgfqpoint{1.531730in}{1.263312in}}%
\pgfpathlineto{\pgfqpoint{1.531995in}{1.263544in}}%
\pgfpathlineto{\pgfqpoint{1.532260in}{1.263776in}}%
\pgfpathlineto{\pgfqpoint{1.532525in}{1.264008in}}%
\pgfpathlineto{\pgfqpoint{1.532790in}{1.264238in}}%
\pgfpathlineto{\pgfqpoint{1.533055in}{1.264468in}}%
\pgfpathlineto{\pgfqpoint{1.533320in}{1.264696in}}%
\pgfpathlineto{\pgfqpoint{1.533585in}{1.264924in}}%
\pgfpathlineto{\pgfqpoint{1.533850in}{1.265152in}}%
\pgfpathlineto{\pgfqpoint{1.534114in}{1.265378in}}%
\pgfpathlineto{\pgfqpoint{1.534379in}{1.265604in}}%
\pgfpathlineto{\pgfqpoint{1.534644in}{1.265829in}}%
\pgfpathlineto{\pgfqpoint{1.534909in}{1.266053in}}%
\pgfpathlineto{\pgfqpoint{1.535174in}{1.266276in}}%
\pgfpathlineto{\pgfqpoint{1.535439in}{1.266499in}}%
\pgfpathlineto{\pgfqpoint{1.535704in}{1.266721in}}%
\pgfpathlineto{\pgfqpoint{1.535969in}{1.266942in}}%
\pgfpathlineto{\pgfqpoint{1.536234in}{1.267162in}}%
\pgfpathlineto{\pgfqpoint{1.536499in}{1.267381in}}%
\pgfpathlineto{\pgfqpoint{1.536764in}{1.267600in}}%
\pgfpathlineto{\pgfqpoint{1.537028in}{1.267818in}}%
\pgfpathlineto{\pgfqpoint{1.537293in}{1.268036in}}%
\pgfpathlineto{\pgfqpoint{1.537558in}{1.268252in}}%
\pgfpathlineto{\pgfqpoint{1.537823in}{1.268468in}}%
\pgfpathlineto{\pgfqpoint{1.538088in}{1.268683in}}%
\pgfpathlineto{\pgfqpoint{1.538353in}{1.268897in}}%
\pgfpathlineto{\pgfqpoint{1.538618in}{1.269111in}}%
\pgfpathlineto{\pgfqpoint{1.538883in}{1.269324in}}%
\pgfpathlineto{\pgfqpoint{1.539148in}{1.269536in}}%
\pgfpathlineto{\pgfqpoint{1.539413in}{1.269748in}}%
\pgfpathlineto{\pgfqpoint{1.539677in}{1.269958in}}%
\pgfpathlineto{\pgfqpoint{1.539942in}{1.270168in}}%
\pgfpathlineto{\pgfqpoint{1.540207in}{1.270378in}}%
\pgfpathlineto{\pgfqpoint{1.540472in}{1.270586in}}%
\pgfpathlineto{\pgfqpoint{1.540737in}{1.270794in}}%
\pgfpathlineto{\pgfqpoint{1.541002in}{1.271001in}}%
\pgfpathlineto{\pgfqpoint{1.541267in}{1.271207in}}%
\pgfpathlineto{\pgfqpoint{1.541532in}{1.271413in}}%
\pgfpathlineto{\pgfqpoint{1.541797in}{1.271618in}}%
\pgfpathlineto{\pgfqpoint{1.542062in}{1.271822in}}%
\pgfpathlineto{\pgfqpoint{1.542327in}{1.272026in}}%
\pgfpathlineto{\pgfqpoint{1.542591in}{1.272229in}}%
\pgfpathlineto{\pgfqpoint{1.542856in}{1.272431in}}%
\pgfpathlineto{\pgfqpoint{1.543121in}{1.272633in}}%
\pgfpathlineto{\pgfqpoint{1.543386in}{1.272833in}}%
\pgfpathlineto{\pgfqpoint{1.543651in}{1.273034in}}%
\pgfpathlineto{\pgfqpoint{1.543916in}{1.273233in}}%
\pgfpathlineto{\pgfqpoint{1.544181in}{1.273432in}}%
\pgfpathlineto{\pgfqpoint{1.544446in}{1.273630in}}%
\pgfpathlineto{\pgfqpoint{1.544711in}{1.273827in}}%
\pgfpathlineto{\pgfqpoint{1.544976in}{1.274024in}}%
\pgfpathlineto{\pgfqpoint{1.545240in}{1.274220in}}%
\pgfpathlineto{\pgfqpoint{1.545505in}{1.274415in}}%
\pgfpathlineto{\pgfqpoint{1.545770in}{1.274610in}}%
\pgfpathlineto{\pgfqpoint{1.546035in}{1.274804in}}%
\pgfpathlineto{\pgfqpoint{1.546300in}{1.274998in}}%
\pgfpathlineto{\pgfqpoint{1.546565in}{1.275190in}}%
\pgfpathlineto{\pgfqpoint{1.546830in}{1.275382in}}%
\pgfpathlineto{\pgfqpoint{1.547095in}{1.275574in}}%
\pgfpathlineto{\pgfqpoint{1.547360in}{1.275764in}}%
\pgfpathlineto{\pgfqpoint{1.547625in}{1.275955in}}%
\pgfpathlineto{\pgfqpoint{1.547890in}{1.276144in}}%
\pgfpathlineto{\pgfqpoint{1.548154in}{1.276333in}}%
\pgfpathlineto{\pgfqpoint{1.548419in}{1.276521in}}%
\pgfpathlineto{\pgfqpoint{1.548684in}{1.276708in}}%
\pgfpathlineto{\pgfqpoint{1.548949in}{1.276895in}}%
\pgfpathlineto{\pgfqpoint{1.549214in}{1.277082in}}%
\pgfpathlineto{\pgfqpoint{1.549479in}{1.277267in}}%
\pgfpathlineto{\pgfqpoint{1.549744in}{1.277452in}}%
\pgfpathlineto{\pgfqpoint{1.550009in}{1.277636in}}%
\pgfpathlineto{\pgfqpoint{1.550274in}{1.277820in}}%
\pgfpathlineto{\pgfqpoint{1.550539in}{1.278003in}}%
\pgfpathlineto{\pgfqpoint{1.550803in}{1.278185in}}%
\pgfpathlineto{\pgfqpoint{1.551068in}{1.278367in}}%
\pgfpathlineto{\pgfqpoint{1.551333in}{1.278548in}}%
\pgfpathlineto{\pgfqpoint{1.551598in}{1.278729in}}%
\pgfpathlineto{\pgfqpoint{1.551863in}{1.278909in}}%
\pgfpathlineto{\pgfqpoint{1.552128in}{1.279088in}}%
\pgfpathlineto{\pgfqpoint{1.552393in}{1.279267in}}%
\pgfpathlineto{\pgfqpoint{1.552658in}{1.279445in}}%
\pgfpathlineto{\pgfqpoint{1.552923in}{1.279623in}}%
\pgfpathlineto{\pgfqpoint{1.553188in}{1.279799in}}%
\pgfpathlineto{\pgfqpoint{1.553453in}{1.279976in}}%
\pgfpathlineto{\pgfqpoint{1.553717in}{1.280151in}}%
\pgfpathlineto{\pgfqpoint{1.553982in}{1.280326in}}%
\pgfpathlineto{\pgfqpoint{1.554247in}{1.280501in}}%
\pgfpathlineto{\pgfqpoint{1.554512in}{1.280675in}}%
\pgfpathlineto{\pgfqpoint{1.554777in}{1.280848in}}%
\pgfpathlineto{\pgfqpoint{1.555042in}{1.281021in}}%
\pgfpathlineto{\pgfqpoint{1.555307in}{1.281193in}}%
\pgfpathlineto{\pgfqpoint{1.555572in}{1.281364in}}%
\pgfpathlineto{\pgfqpoint{1.555837in}{1.281535in}}%
\pgfpathlineto{\pgfqpoint{1.556102in}{1.281705in}}%
\pgfpathlineto{\pgfqpoint{1.556366in}{1.281875in}}%
\pgfpathlineto{\pgfqpoint{1.556631in}{1.282044in}}%
\pgfpathlineto{\pgfqpoint{1.556896in}{1.282213in}}%
\pgfpathlineto{\pgfqpoint{1.557161in}{1.282381in}}%
\pgfpathlineto{\pgfqpoint{1.557426in}{1.282548in}}%
\pgfpathlineto{\pgfqpoint{1.557691in}{1.282715in}}%
\pgfpathlineto{\pgfqpoint{1.557956in}{1.282881in}}%
\pgfpathlineto{\pgfqpoint{1.558221in}{1.283047in}}%
\pgfpathlineto{\pgfqpoint{1.558486in}{1.283212in}}%
\pgfpathlineto{\pgfqpoint{1.558751in}{1.283377in}}%
\pgfpathlineto{\pgfqpoint{1.559016in}{1.283541in}}%
\pgfpathlineto{\pgfqpoint{1.559280in}{1.283704in}}%
\pgfpathlineto{\pgfqpoint{1.559545in}{1.283867in}}%
\pgfpathlineto{\pgfqpoint{1.559810in}{1.284030in}}%
\pgfpathlineto{\pgfqpoint{1.560075in}{1.284191in}}%
\pgfpathlineto{\pgfqpoint{1.560340in}{1.284353in}}%
\pgfpathlineto{\pgfqpoint{1.560605in}{1.284513in}}%
\pgfpathlineto{\pgfqpoint{1.560870in}{1.284673in}}%
\pgfpathlineto{\pgfqpoint{1.561135in}{1.284833in}}%
\pgfpathlineto{\pgfqpoint{1.561400in}{1.284992in}}%
\pgfpathlineto{\pgfqpoint{1.561665in}{1.285150in}}%
\pgfpathlineto{\pgfqpoint{1.561929in}{1.285308in}}%
\pgfpathlineto{\pgfqpoint{1.562194in}{1.285466in}}%
\pgfpathlineto{\pgfqpoint{1.562459in}{1.285623in}}%
\pgfpathlineto{\pgfqpoint{1.562724in}{1.285779in}}%
\pgfpathlineto{\pgfqpoint{1.562989in}{1.285935in}}%
\pgfpathlineto{\pgfqpoint{1.563254in}{1.286090in}}%
\pgfpathlineto{\pgfqpoint{1.563519in}{1.286245in}}%
\pgfpathlineto{\pgfqpoint{1.563784in}{1.286399in}}%
\pgfpathlineto{\pgfqpoint{1.564049in}{1.286553in}}%
\pgfpathlineto{\pgfqpoint{1.564314in}{1.286706in}}%
\pgfpathlineto{\pgfqpoint{1.564579in}{1.286858in}}%
\pgfpathlineto{\pgfqpoint{1.564843in}{1.287010in}}%
\pgfpathlineto{\pgfqpoint{1.565108in}{1.287162in}}%
\pgfpathlineto{\pgfqpoint{1.565373in}{1.287313in}}%
\pgfpathlineto{\pgfqpoint{1.565638in}{1.287463in}}%
\pgfpathlineto{\pgfqpoint{1.565903in}{1.287613in}}%
\pgfpathlineto{\pgfqpoint{1.566168in}{1.287763in}}%
\pgfpathlineto{\pgfqpoint{1.566433in}{1.287912in}}%
\pgfpathlineto{\pgfqpoint{1.566698in}{1.288060in}}%
\pgfpathlineto{\pgfqpoint{1.566963in}{1.288208in}}%
\pgfpathlineto{\pgfqpoint{1.567228in}{1.288356in}}%
\pgfpathlineto{\pgfqpoint{1.567492in}{1.288503in}}%
\pgfpathlineto{\pgfqpoint{1.567757in}{1.288649in}}%
\pgfpathlineto{\pgfqpoint{1.568022in}{1.288795in}}%
\pgfpathlineto{\pgfqpoint{1.568287in}{1.288941in}}%
\pgfpathlineto{\pgfqpoint{1.568552in}{1.289086in}}%
\pgfpathlineto{\pgfqpoint{1.568817in}{1.289230in}}%
\pgfpathlineto{\pgfqpoint{1.569082in}{1.289374in}}%
\pgfpathlineto{\pgfqpoint{1.569347in}{1.289517in}}%
\pgfpathlineto{\pgfqpoint{1.569612in}{1.289660in}}%
\pgfpathlineto{\pgfqpoint{1.569877in}{1.289803in}}%
\pgfpathlineto{\pgfqpoint{1.570142in}{1.289945in}}%
\pgfpathlineto{\pgfqpoint{1.570406in}{1.290086in}}%
\pgfpathlineto{\pgfqpoint{1.570671in}{1.290227in}}%
\pgfpathlineto{\pgfqpoint{1.570936in}{1.290368in}}%
\pgfpathlineto{\pgfqpoint{1.571201in}{1.290508in}}%
\pgfpathlineto{\pgfqpoint{1.571466in}{1.290648in}}%
\pgfpathlineto{\pgfqpoint{1.571731in}{1.290787in}}%
\pgfpathlineto{\pgfqpoint{1.571996in}{1.290925in}}%
\pgfpathlineto{\pgfqpoint{1.572261in}{1.291064in}}%
\pgfpathlineto{\pgfqpoint{1.572526in}{1.291201in}}%
\pgfpathlineto{\pgfqpoint{1.572791in}{1.291338in}}%
\pgfpathlineto{\pgfqpoint{1.573055in}{1.291475in}}%
\pgfpathlineto{\pgfqpoint{1.573320in}{1.291611in}}%
\pgfpathlineto{\pgfqpoint{1.573585in}{1.291747in}}%
\pgfpathlineto{\pgfqpoint{1.573850in}{1.291883in}}%
\pgfpathlineto{\pgfqpoint{1.574115in}{1.292017in}}%
\pgfpathlineto{\pgfqpoint{1.574380in}{1.292152in}}%
\pgfpathlineto{\pgfqpoint{1.574645in}{1.292286in}}%
\pgfpathlineto{\pgfqpoint{1.574910in}{1.292419in}}%
\pgfpathlineto{\pgfqpoint{1.575175in}{1.292552in}}%
\pgfpathlineto{\pgfqpoint{1.575440in}{1.292685in}}%
\pgfpathlineto{\pgfqpoint{1.575705in}{1.292817in}}%
\pgfpathlineto{\pgfqpoint{1.575969in}{1.292949in}}%
\pgfpathlineto{\pgfqpoint{1.576234in}{1.293080in}}%
\pgfpathlineto{\pgfqpoint{1.576499in}{1.293211in}}%
\pgfpathlineto{\pgfqpoint{1.576764in}{1.293341in}}%
\pgfpathlineto{\pgfqpoint{1.577029in}{1.293471in}}%
\pgfpathlineto{\pgfqpoint{1.577294in}{1.293600in}}%
\pgfpathlineto{\pgfqpoint{1.577559in}{1.293729in}}%
\pgfpathlineto{\pgfqpoint{1.577824in}{1.293858in}}%
\pgfpathlineto{\pgfqpoint{1.578089in}{1.293986in}}%
\pgfpathlineto{\pgfqpoint{1.578354in}{1.294114in}}%
\pgfpathlineto{\pgfqpoint{1.578618in}{1.294241in}}%
\pgfpathlineto{\pgfqpoint{1.578883in}{1.294368in}}%
\pgfpathlineto{\pgfqpoint{1.579148in}{1.294494in}}%
\pgfpathlineto{\pgfqpoint{1.579413in}{1.294620in}}%
\pgfpathlineto{\pgfqpoint{1.579678in}{1.294745in}}%
\pgfpathlineto{\pgfqpoint{1.579943in}{1.294870in}}%
\pgfpathlineto{\pgfqpoint{1.580208in}{1.294995in}}%
\pgfpathlineto{\pgfqpoint{1.580473in}{1.295119in}}%
\pgfpathlineto{\pgfqpoint{1.580738in}{1.295243in}}%
\pgfpathlineto{\pgfqpoint{1.581003in}{1.295366in}}%
\pgfpathlineto{\pgfqpoint{1.581268in}{1.295489in}}%
\pgfpathlineto{\pgfqpoint{1.581532in}{1.295612in}}%
\pgfpathlineto{\pgfqpoint{1.581797in}{1.295734in}}%
\pgfpathlineto{\pgfqpoint{1.582062in}{1.295856in}}%
\pgfpathlineto{\pgfqpoint{1.582327in}{1.295977in}}%
\pgfpathlineto{\pgfqpoint{1.582592in}{1.296098in}}%
\pgfpathlineto{\pgfqpoint{1.582857in}{1.296218in}}%
\pgfpathlineto{\pgfqpoint{1.583122in}{1.296338in}}%
\pgfpathlineto{\pgfqpoint{1.583387in}{1.296458in}}%
\pgfpathlineto{\pgfqpoint{1.583652in}{1.296577in}}%
\pgfpathlineto{\pgfqpoint{1.583917in}{1.296696in}}%
\pgfpathlineto{\pgfqpoint{1.584181in}{1.296814in}}%
\pgfpathlineto{\pgfqpoint{1.584446in}{1.296932in}}%
\pgfpathlineto{\pgfqpoint{1.584711in}{1.297050in}}%
\pgfpathlineto{\pgfqpoint{1.584976in}{1.297167in}}%
\pgfpathlineto{\pgfqpoint{1.585241in}{1.297284in}}%
\pgfpathlineto{\pgfqpoint{1.585506in}{1.297400in}}%
\pgfpathlineto{\pgfqpoint{1.585771in}{1.297516in}}%
\pgfpathlineto{\pgfqpoint{1.586036in}{1.297632in}}%
\pgfpathlineto{\pgfqpoint{1.586301in}{1.297747in}}%
\pgfpathlineto{\pgfqpoint{1.586566in}{1.297861in}}%
\pgfpathlineto{\pgfqpoint{1.586831in}{1.297976in}}%
\pgfpathlineto{\pgfqpoint{1.587095in}{1.298090in}}%
\pgfpathlineto{\pgfqpoint{1.587360in}{1.298203in}}%
\pgfpathlineto{\pgfqpoint{1.587625in}{1.298317in}}%
\pgfpathlineto{\pgfqpoint{1.587890in}{1.298430in}}%
\pgfpathlineto{\pgfqpoint{1.588155in}{1.298542in}}%
\pgfpathlineto{\pgfqpoint{1.588420in}{1.298654in}}%
\pgfpathlineto{\pgfqpoint{1.588685in}{1.298766in}}%
\pgfpathlineto{\pgfqpoint{1.588950in}{1.298877in}}%
\pgfpathlineto{\pgfqpoint{1.589215in}{1.298988in}}%
\pgfpathlineto{\pgfqpoint{1.589480in}{1.299098in}}%
\pgfpathlineto{\pgfqpoint{1.589744in}{1.299209in}}%
\pgfpathlineto{\pgfqpoint{1.590009in}{1.299318in}}%
\pgfpathlineto{\pgfqpoint{1.590274in}{1.299428in}}%
\pgfpathlineto{\pgfqpoint{1.590539in}{1.299537in}}%
\pgfpathlineto{\pgfqpoint{1.590804in}{1.299645in}}%
\pgfpathlineto{\pgfqpoint{1.591069in}{1.299754in}}%
\pgfpathlineto{\pgfqpoint{1.591334in}{1.299862in}}%
\pgfpathlineto{\pgfqpoint{1.591599in}{1.299969in}}%
\pgfpathlineto{\pgfqpoint{1.591864in}{1.300076in}}%
\pgfpathlineto{\pgfqpoint{1.592129in}{1.300183in}}%
\pgfpathlineto{\pgfqpoint{1.592394in}{1.300290in}}%
\pgfpathlineto{\pgfqpoint{1.592658in}{1.300396in}}%
\pgfpathlineto{\pgfqpoint{1.592923in}{1.300501in}}%
\pgfpathlineto{\pgfqpoint{1.593188in}{1.300607in}}%
\pgfpathlineto{\pgfqpoint{1.593453in}{1.300712in}}%
\pgfpathlineto{\pgfqpoint{1.593718in}{1.300816in}}%
\pgfpathlineto{\pgfqpoint{1.593983in}{1.300920in}}%
\pgfpathlineto{\pgfqpoint{1.594248in}{1.301024in}}%
\pgfpathlineto{\pgfqpoint{1.594513in}{1.301128in}}%
\pgfpathlineto{\pgfqpoint{1.594778in}{1.301231in}}%
\pgfpathlineto{\pgfqpoint{1.595043in}{1.301334in}}%
\pgfpathlineto{\pgfqpoint{1.595307in}{1.301436in}}%
\pgfpathlineto{\pgfqpoint{1.595572in}{1.301539in}}%
\pgfpathlineto{\pgfqpoint{1.595837in}{1.301640in}}%
\pgfpathlineto{\pgfqpoint{1.596102in}{1.301742in}}%
\pgfpathlineto{\pgfqpoint{1.596367in}{1.301843in}}%
\pgfpathlineto{\pgfqpoint{1.596632in}{1.301944in}}%
\pgfpathlineto{\pgfqpoint{1.596897in}{1.302044in}}%
\pgfpathlineto{\pgfqpoint{1.597162in}{1.302144in}}%
\pgfpathlineto{\pgfqpoint{1.597427in}{1.302244in}}%
\pgfpathlineto{\pgfqpoint{1.597692in}{1.302343in}}%
\pgfpathlineto{\pgfqpoint{1.597957in}{1.302442in}}%
\pgfpathlineto{\pgfqpoint{1.598221in}{1.302541in}}%
\pgfpathlineto{\pgfqpoint{1.598486in}{1.302639in}}%
\pgfpathlineto{\pgfqpoint{1.598751in}{1.302737in}}%
\pgfpathlineto{\pgfqpoint{1.599016in}{1.302835in}}%
\pgfpathlineto{\pgfqpoint{1.599281in}{1.302932in}}%
\pgfpathlineto{\pgfqpoint{1.599546in}{1.303029in}}%
\pgfpathlineto{\pgfqpoint{1.599811in}{1.303126in}}%
\pgfpathlineto{\pgfqpoint{1.600076in}{1.303222in}}%
\pgfpathlineto{\pgfqpoint{1.600341in}{1.303318in}}%
\pgfpathlineto{\pgfqpoint{1.600606in}{1.303414in}}%
\pgfpathlineto{\pgfqpoint{1.600870in}{1.303510in}}%
\pgfpathlineto{\pgfqpoint{1.601135in}{1.303605in}}%
\pgfpathlineto{\pgfqpoint{1.601400in}{1.303699in}}%
\pgfpathlineto{\pgfqpoint{1.601665in}{1.303794in}}%
\pgfpathlineto{\pgfqpoint{1.601930in}{1.303888in}}%
\pgfpathlineto{\pgfqpoint{1.602195in}{1.303981in}}%
\pgfpathlineto{\pgfqpoint{1.602460in}{1.304075in}}%
\pgfpathlineto{\pgfqpoint{1.602725in}{1.304168in}}%
\pgfpathlineto{\pgfqpoint{1.602990in}{1.304261in}}%
\pgfpathlineto{\pgfqpoint{1.603255in}{1.304353in}}%
\pgfpathlineto{\pgfqpoint{1.603520in}{1.304445in}}%
\pgfpathlineto{\pgfqpoint{1.603784in}{1.304537in}}%
\pgfpathlineto{\pgfqpoint{1.604049in}{1.304629in}}%
\pgfpathlineto{\pgfqpoint{1.604314in}{1.304720in}}%
\pgfpathlineto{\pgfqpoint{1.604579in}{1.304811in}}%
\pgfpathlineto{\pgfqpoint{1.604844in}{1.304901in}}%
\pgfpathlineto{\pgfqpoint{1.605109in}{1.304992in}}%
\pgfpathlineto{\pgfqpoint{1.605374in}{1.305082in}}%
\pgfpathlineto{\pgfqpoint{1.605639in}{1.305171in}}%
\pgfpathlineto{\pgfqpoint{1.605904in}{1.305261in}}%
\pgfpathlineto{\pgfqpoint{1.606169in}{1.305350in}}%
\pgfpathlineto{\pgfqpoint{1.606433in}{1.305438in}}%
\pgfpathlineto{\pgfqpoint{1.606698in}{1.305527in}}%
\pgfpathlineto{\pgfqpoint{1.606963in}{1.305615in}}%
\pgfpathlineto{\pgfqpoint{1.607228in}{1.305703in}}%
\pgfpathlineto{\pgfqpoint{1.607493in}{1.305790in}}%
\pgfpathlineto{\pgfqpoint{1.607758in}{1.305878in}}%
\pgfpathlineto{\pgfqpoint{1.608023in}{1.305964in}}%
\pgfpathlineto{\pgfqpoint{1.608288in}{1.306051in}}%
\pgfpathlineto{\pgfqpoint{1.608553in}{1.306137in}}%
\pgfpathlineto{\pgfqpoint{1.608818in}{1.306223in}}%
\pgfpathlineto{\pgfqpoint{1.609083in}{1.306309in}}%
\pgfpathlineto{\pgfqpoint{1.609347in}{1.306395in}}%
\pgfpathlineto{\pgfqpoint{1.609612in}{1.306480in}}%
\pgfpathlineto{\pgfqpoint{1.609877in}{1.306565in}}%
\pgfpathlineto{\pgfqpoint{1.610142in}{1.306649in}}%
\pgfpathlineto{\pgfqpoint{1.610407in}{1.306733in}}%
\pgfpathlineto{\pgfqpoint{1.610672in}{1.306817in}}%
\pgfpathlineto{\pgfqpoint{1.610937in}{1.306901in}}%
\pgfpathlineto{\pgfqpoint{1.611202in}{1.306985in}}%
\pgfpathlineto{\pgfqpoint{1.611467in}{1.307068in}}%
\pgfpathlineto{\pgfqpoint{1.611732in}{1.307151in}}%
\pgfpathlineto{\pgfqpoint{1.611996in}{1.307233in}}%
\pgfpathlineto{\pgfqpoint{1.612261in}{1.307315in}}%
\pgfpathlineto{\pgfqpoint{1.612526in}{1.307397in}}%
\pgfpathlineto{\pgfqpoint{1.612791in}{1.307479in}}%
\pgfpathlineto{\pgfqpoint{1.613056in}{1.307561in}}%
\pgfpathlineto{\pgfqpoint{1.613321in}{1.307642in}}%
\pgfpathlineto{\pgfqpoint{1.613586in}{1.307723in}}%
\pgfpathlineto{\pgfqpoint{1.613851in}{1.307803in}}%
\pgfpathlineto{\pgfqpoint{1.614116in}{1.307884in}}%
\pgfpathlineto{\pgfqpoint{1.614381in}{1.307964in}}%
\pgfpathlineto{\pgfqpoint{1.614646in}{1.308043in}}%
\pgfpathlineto{\pgfqpoint{1.614910in}{1.308123in}}%
\pgfpathlineto{\pgfqpoint{1.615175in}{1.308202in}}%
\pgfpathlineto{\pgfqpoint{1.615440in}{1.308281in}}%
\pgfpathlineto{\pgfqpoint{1.615705in}{1.308360in}}%
\pgfpathlineto{\pgfqpoint{1.615970in}{1.308438in}}%
\pgfpathlineto{\pgfqpoint{1.616235in}{1.308516in}}%
\pgfpathlineto{\pgfqpoint{1.616500in}{1.308594in}}%
\pgfpathlineto{\pgfqpoint{1.616765in}{1.308672in}}%
\pgfpathlineto{\pgfqpoint{1.617030in}{1.308749in}}%
\pgfpathlineto{\pgfqpoint{1.617295in}{1.308826in}}%
\pgfpathlineto{\pgfqpoint{1.617559in}{1.308903in}}%
\pgfpathlineto{\pgfqpoint{1.617824in}{1.308980in}}%
\pgfpathlineto{\pgfqpoint{1.618089in}{1.309056in}}%
\pgfpathlineto{\pgfqpoint{1.618354in}{1.309132in}}%
\pgfpathlineto{\pgfqpoint{1.618619in}{1.309208in}}%
\pgfpathlineto{\pgfqpoint{1.618884in}{1.309283in}}%
\pgfpathlineto{\pgfqpoint{1.619149in}{1.309359in}}%
\pgfpathlineto{\pgfqpoint{1.619414in}{1.309434in}}%
\pgfpathlineto{\pgfqpoint{1.619679in}{1.309508in}}%
\pgfpathlineto{\pgfqpoint{1.619944in}{1.309583in}}%
\pgfpathlineto{\pgfqpoint{1.620209in}{1.309657in}}%
\pgfpathlineto{\pgfqpoint{1.620473in}{1.309731in}}%
\pgfpathlineto{\pgfqpoint{1.620738in}{1.309805in}}%
\pgfpathlineto{\pgfqpoint{1.621003in}{1.309878in}}%
\pgfpathlineto{\pgfqpoint{1.621268in}{1.309951in}}%
\pgfpathlineto{\pgfqpoint{1.621533in}{1.310024in}}%
\pgfpathlineto{\pgfqpoint{1.621798in}{1.310097in}}%
\pgfpathlineto{\pgfqpoint{1.622063in}{1.310170in}}%
\pgfpathlineto{\pgfqpoint{1.622328in}{1.310242in}}%
\pgfpathlineto{\pgfqpoint{1.622593in}{1.310314in}}%
\pgfpathlineto{\pgfqpoint{1.622858in}{1.310385in}}%
\pgfpathlineto{\pgfqpoint{1.623122in}{1.310457in}}%
\pgfpathlineto{\pgfqpoint{1.623387in}{1.310528in}}%
\pgfpathlineto{\pgfqpoint{1.623652in}{1.310599in}}%
\pgfpathlineto{\pgfqpoint{1.623917in}{1.310670in}}%
\pgfpathlineto{\pgfqpoint{1.624182in}{1.310740in}}%
\pgfpathlineto{\pgfqpoint{1.624447in}{1.310811in}}%
\pgfpathlineto{\pgfqpoint{1.624712in}{1.310881in}}%
\pgfpathlineto{\pgfqpoint{1.624977in}{1.310950in}}%
\pgfpathlineto{\pgfqpoint{1.625242in}{1.311020in}}%
\pgfpathlineto{\pgfqpoint{1.625507in}{1.311089in}}%
\pgfpathlineto{\pgfqpoint{1.625772in}{1.311158in}}%
\pgfpathlineto{\pgfqpoint{1.626036in}{1.311227in}}%
\pgfpathlineto{\pgfqpoint{1.626301in}{1.311296in}}%
\pgfpathlineto{\pgfqpoint{1.626566in}{1.311364in}}%
\pgfpathlineto{\pgfqpoint{1.626831in}{1.311432in}}%
\pgfpathlineto{\pgfqpoint{1.627096in}{1.311500in}}%
\pgfpathlineto{\pgfqpoint{1.627361in}{1.311568in}}%
\pgfpathlineto{\pgfqpoint{1.627626in}{1.311635in}}%
\pgfpathlineto{\pgfqpoint{1.627891in}{1.311703in}}%
\pgfpathlineto{\pgfqpoint{1.628156in}{1.311769in}}%
\pgfpathlineto{\pgfqpoint{1.628421in}{1.311836in}}%
\pgfpathlineto{\pgfqpoint{1.628685in}{1.311903in}}%
\pgfpathlineto{\pgfqpoint{1.628950in}{1.311969in}}%
\pgfpathlineto{\pgfqpoint{1.629215in}{1.312035in}}%
\pgfpathlineto{\pgfqpoint{1.629480in}{1.312101in}}%
\pgfpathlineto{\pgfqpoint{1.629745in}{1.312167in}}%
\pgfpathlineto{\pgfqpoint{1.630010in}{1.312232in}}%
\pgfpathlineto{\pgfqpoint{1.630275in}{1.312297in}}%
\pgfpathlineto{\pgfqpoint{1.630540in}{1.312362in}}%
\pgfpathlineto{\pgfqpoint{1.630805in}{1.312427in}}%
\pgfpathlineto{\pgfqpoint{1.631070in}{1.312491in}}%
\pgfpathlineto{\pgfqpoint{1.631335in}{1.312555in}}%
\pgfpathlineto{\pgfqpoint{1.631599in}{1.312619in}}%
\pgfpathlineto{\pgfqpoint{1.631864in}{1.312683in}}%
\pgfpathlineto{\pgfqpoint{1.632129in}{1.312747in}}%
\pgfpathlineto{\pgfqpoint{1.632394in}{1.312810in}}%
\pgfpathlineto{\pgfqpoint{1.632659in}{1.312873in}}%
\pgfpathlineto{\pgfqpoint{1.632924in}{1.312936in}}%
\pgfpathlineto{\pgfqpoint{1.633189in}{1.312999in}}%
\pgfpathlineto{\pgfqpoint{1.633454in}{1.313062in}}%
\pgfpathlineto{\pgfqpoint{1.633719in}{1.313124in}}%
\pgfpathlineto{\pgfqpoint{1.633984in}{1.313186in}}%
\pgfpathlineto{\pgfqpoint{1.634248in}{1.313248in}}%
\pgfpathlineto{\pgfqpoint{1.634513in}{1.313310in}}%
\pgfpathlineto{\pgfqpoint{1.634778in}{1.313371in}}%
\pgfpathlineto{\pgfqpoint{1.635043in}{1.313432in}}%
\pgfpathlineto{\pgfqpoint{1.635308in}{1.313493in}}%
\pgfpathlineto{\pgfqpoint{1.635573in}{1.313554in}}%
\pgfpathlineto{\pgfqpoint{1.635838in}{1.313615in}}%
\pgfpathlineto{\pgfqpoint{1.636103in}{1.313675in}}%
\pgfpathlineto{\pgfqpoint{1.636368in}{1.313736in}}%
\pgfpathlineto{\pgfqpoint{1.636633in}{1.313796in}}%
\pgfpathlineto{\pgfqpoint{1.636898in}{1.313855in}}%
\pgfpathlineto{\pgfqpoint{1.637162in}{1.313915in}}%
\pgfpathlineto{\pgfqpoint{1.637427in}{1.313974in}}%
\pgfpathlineto{\pgfqpoint{1.637692in}{1.314033in}}%
\pgfpathlineto{\pgfqpoint{1.637957in}{1.314092in}}%
\pgfpathlineto{\pgfqpoint{1.638222in}{1.314151in}}%
\pgfpathlineto{\pgfqpoint{1.638487in}{1.314210in}}%
\pgfpathlineto{\pgfqpoint{1.638752in}{1.314268in}}%
\pgfpathlineto{\pgfqpoint{1.639017in}{1.314326in}}%
\pgfpathlineto{\pgfqpoint{1.639282in}{1.314384in}}%
\pgfpathlineto{\pgfqpoint{1.639547in}{1.314442in}}%
\pgfpathlineto{\pgfqpoint{1.639811in}{1.314500in}}%
\pgfpathlineto{\pgfqpoint{1.640076in}{1.314557in}}%
\pgfpathlineto{\pgfqpoint{1.640341in}{1.314614in}}%
\pgfpathlineto{\pgfqpoint{1.640606in}{1.314671in}}%
\pgfpathlineto{\pgfqpoint{1.640871in}{1.314728in}}%
\pgfpathlineto{\pgfqpoint{1.641136in}{1.314785in}}%
\pgfpathlineto{\pgfqpoint{1.641401in}{1.314841in}}%
\pgfpathlineto{\pgfqpoint{1.641666in}{1.314897in}}%
\pgfpathlineto{\pgfqpoint{1.641931in}{1.314953in}}%
\pgfpathlineto{\pgfqpoint{1.642196in}{1.315009in}}%
\pgfpathlineto{\pgfqpoint{1.642461in}{1.315065in}}%
\pgfpathlineto{\pgfqpoint{1.642725in}{1.315120in}}%
\pgfpathlineto{\pgfqpoint{1.642990in}{1.315176in}}%
\pgfpathlineto{\pgfqpoint{1.643255in}{1.315231in}}%
\pgfpathlineto{\pgfqpoint{1.643520in}{1.315285in}}%
\pgfpathlineto{\pgfqpoint{1.643785in}{1.315340in}}%
\pgfpathlineto{\pgfqpoint{1.644050in}{1.315395in}}%
\pgfpathlineto{\pgfqpoint{1.644315in}{1.315449in}}%
\pgfpathlineto{\pgfqpoint{1.644580in}{1.315503in}}%
\pgfpathlineto{\pgfqpoint{1.644845in}{1.315557in}}%
\pgfpathlineto{\pgfqpoint{1.645110in}{1.315611in}}%
\pgfpathlineto{\pgfqpoint{1.645374in}{1.315664in}}%
\pgfpathlineto{\pgfqpoint{1.645639in}{1.315718in}}%
\pgfpathlineto{\pgfqpoint{1.645904in}{1.315771in}}%
\pgfpathlineto{\pgfqpoint{1.646169in}{1.315824in}}%
\pgfpathlineto{\pgfqpoint{1.646434in}{1.315877in}}%
\pgfpathlineto{\pgfqpoint{1.646699in}{1.315930in}}%
\pgfpathlineto{\pgfqpoint{1.646964in}{1.315982in}}%
\pgfpathlineto{\pgfqpoint{1.647229in}{1.316034in}}%
\pgfpathlineto{\pgfqpoint{1.647494in}{1.316087in}}%
\pgfpathlineto{\pgfqpoint{1.647759in}{1.316138in}}%
\pgfpathlineto{\pgfqpoint{1.648024in}{1.316190in}}%
\pgfpathlineto{\pgfqpoint{1.648288in}{1.316242in}}%
\pgfpathlineto{\pgfqpoint{1.648553in}{1.316293in}}%
\pgfpathlineto{\pgfqpoint{1.648818in}{1.316344in}}%
\pgfpathlineto{\pgfqpoint{1.649083in}{1.316396in}}%
\pgfpathlineto{\pgfqpoint{1.649348in}{1.316446in}}%
\pgfpathlineto{\pgfqpoint{1.649613in}{1.316497in}}%
\pgfpathlineto{\pgfqpoint{1.649878in}{1.316548in}}%
\pgfpathlineto{\pgfqpoint{1.650143in}{1.316598in}}%
\pgfpathlineto{\pgfqpoint{1.650408in}{1.316648in}}%
\pgfpathlineto{\pgfqpoint{1.650673in}{1.316698in}}%
\pgfpathlineto{\pgfqpoint{1.650937in}{1.316748in}}%
\pgfpathlineto{\pgfqpoint{1.650937in}{1.575000in}}%
\pgfpathlineto{\pgfqpoint{1.650937in}{1.575000in}}%
\pgfpathlineto{\pgfqpoint{1.650673in}{1.575000in}}%
\pgfpathlineto{\pgfqpoint{1.650408in}{1.575000in}}%
\pgfpathlineto{\pgfqpoint{1.650143in}{1.575000in}}%
\pgfpathlineto{\pgfqpoint{1.649878in}{1.575000in}}%
\pgfpathlineto{\pgfqpoint{1.649613in}{1.575000in}}%
\pgfpathlineto{\pgfqpoint{1.649348in}{1.575000in}}%
\pgfpathlineto{\pgfqpoint{1.649083in}{1.575000in}}%
\pgfpathlineto{\pgfqpoint{1.648818in}{1.575000in}}%
\pgfpathlineto{\pgfqpoint{1.648553in}{1.575000in}}%
\pgfpathlineto{\pgfqpoint{1.648288in}{1.575000in}}%
\pgfpathlineto{\pgfqpoint{1.648024in}{1.575000in}}%
\pgfpathlineto{\pgfqpoint{1.647759in}{1.575000in}}%
\pgfpathlineto{\pgfqpoint{1.647494in}{1.575000in}}%
\pgfpathlineto{\pgfqpoint{1.647229in}{1.575000in}}%
\pgfpathlineto{\pgfqpoint{1.646964in}{1.575000in}}%
\pgfpathlineto{\pgfqpoint{1.646699in}{1.575000in}}%
\pgfpathlineto{\pgfqpoint{1.646434in}{1.575000in}}%
\pgfpathlineto{\pgfqpoint{1.646169in}{1.575000in}}%
\pgfpathlineto{\pgfqpoint{1.645904in}{1.575000in}}%
\pgfpathlineto{\pgfqpoint{1.645639in}{1.575000in}}%
\pgfpathlineto{\pgfqpoint{1.645374in}{1.575000in}}%
\pgfpathlineto{\pgfqpoint{1.645110in}{1.575000in}}%
\pgfpathlineto{\pgfqpoint{1.644845in}{1.575000in}}%
\pgfpathlineto{\pgfqpoint{1.644580in}{1.575000in}}%
\pgfpathlineto{\pgfqpoint{1.644315in}{1.575000in}}%
\pgfpathlineto{\pgfqpoint{1.644050in}{1.575000in}}%
\pgfpathlineto{\pgfqpoint{1.643785in}{1.575000in}}%
\pgfpathlineto{\pgfqpoint{1.643520in}{1.575000in}}%
\pgfpathlineto{\pgfqpoint{1.643255in}{1.575000in}}%
\pgfpathlineto{\pgfqpoint{1.642990in}{1.575000in}}%
\pgfpathlineto{\pgfqpoint{1.642725in}{1.575000in}}%
\pgfpathlineto{\pgfqpoint{1.642461in}{1.575000in}}%
\pgfpathlineto{\pgfqpoint{1.642196in}{1.575000in}}%
\pgfpathlineto{\pgfqpoint{1.641931in}{1.575000in}}%
\pgfpathlineto{\pgfqpoint{1.641666in}{1.575000in}}%
\pgfpathlineto{\pgfqpoint{1.641401in}{1.575000in}}%
\pgfpathlineto{\pgfqpoint{1.641136in}{1.575000in}}%
\pgfpathlineto{\pgfqpoint{1.640871in}{1.575000in}}%
\pgfpathlineto{\pgfqpoint{1.640606in}{1.575000in}}%
\pgfpathlineto{\pgfqpoint{1.640341in}{1.575000in}}%
\pgfpathlineto{\pgfqpoint{1.640076in}{1.575000in}}%
\pgfpathlineto{\pgfqpoint{1.639811in}{1.575000in}}%
\pgfpathlineto{\pgfqpoint{1.639547in}{1.575000in}}%
\pgfpathlineto{\pgfqpoint{1.639282in}{1.575000in}}%
\pgfpathlineto{\pgfqpoint{1.639017in}{1.575000in}}%
\pgfpathlineto{\pgfqpoint{1.638752in}{1.575000in}}%
\pgfpathlineto{\pgfqpoint{1.638487in}{1.575000in}}%
\pgfpathlineto{\pgfqpoint{1.638222in}{1.575000in}}%
\pgfpathlineto{\pgfqpoint{1.637957in}{1.575000in}}%
\pgfpathlineto{\pgfqpoint{1.637692in}{1.575000in}}%
\pgfpathlineto{\pgfqpoint{1.637427in}{1.575000in}}%
\pgfpathlineto{\pgfqpoint{1.637162in}{1.575000in}}%
\pgfpathlineto{\pgfqpoint{1.636898in}{1.575000in}}%
\pgfpathlineto{\pgfqpoint{1.636633in}{1.575000in}}%
\pgfpathlineto{\pgfqpoint{1.636368in}{1.575000in}}%
\pgfpathlineto{\pgfqpoint{1.636103in}{1.575000in}}%
\pgfpathlineto{\pgfqpoint{1.635838in}{1.575000in}}%
\pgfpathlineto{\pgfqpoint{1.635573in}{1.575000in}}%
\pgfpathlineto{\pgfqpoint{1.635308in}{1.575000in}}%
\pgfpathlineto{\pgfqpoint{1.635043in}{1.575000in}}%
\pgfpathlineto{\pgfqpoint{1.634778in}{1.575000in}}%
\pgfpathlineto{\pgfqpoint{1.634513in}{1.575000in}}%
\pgfpathlineto{\pgfqpoint{1.634248in}{1.575000in}}%
\pgfpathlineto{\pgfqpoint{1.633984in}{1.575000in}}%
\pgfpathlineto{\pgfqpoint{1.633719in}{1.575000in}}%
\pgfpathlineto{\pgfqpoint{1.633454in}{1.575000in}}%
\pgfpathlineto{\pgfqpoint{1.633189in}{1.575000in}}%
\pgfpathlineto{\pgfqpoint{1.632924in}{1.575000in}}%
\pgfpathlineto{\pgfqpoint{1.632659in}{1.575000in}}%
\pgfpathlineto{\pgfqpoint{1.632394in}{1.575000in}}%
\pgfpathlineto{\pgfqpoint{1.632129in}{1.575000in}}%
\pgfpathlineto{\pgfqpoint{1.631864in}{1.575000in}}%
\pgfpathlineto{\pgfqpoint{1.631599in}{1.575000in}}%
\pgfpathlineto{\pgfqpoint{1.631335in}{1.575000in}}%
\pgfpathlineto{\pgfqpoint{1.631070in}{1.575000in}}%
\pgfpathlineto{\pgfqpoint{1.630805in}{1.575000in}}%
\pgfpathlineto{\pgfqpoint{1.630540in}{1.575000in}}%
\pgfpathlineto{\pgfqpoint{1.630275in}{1.575000in}}%
\pgfpathlineto{\pgfqpoint{1.630010in}{1.575000in}}%
\pgfpathlineto{\pgfqpoint{1.629745in}{1.575000in}}%
\pgfpathlineto{\pgfqpoint{1.629480in}{1.575000in}}%
\pgfpathlineto{\pgfqpoint{1.629215in}{1.575000in}}%
\pgfpathlineto{\pgfqpoint{1.628950in}{1.575000in}}%
\pgfpathlineto{\pgfqpoint{1.628685in}{1.575000in}}%
\pgfpathlineto{\pgfqpoint{1.628421in}{1.575000in}}%
\pgfpathlineto{\pgfqpoint{1.628156in}{1.575000in}}%
\pgfpathlineto{\pgfqpoint{1.627891in}{1.575000in}}%
\pgfpathlineto{\pgfqpoint{1.627626in}{1.575000in}}%
\pgfpathlineto{\pgfqpoint{1.627361in}{1.575000in}}%
\pgfpathlineto{\pgfqpoint{1.627096in}{1.575000in}}%
\pgfpathlineto{\pgfqpoint{1.626831in}{1.575000in}}%
\pgfpathlineto{\pgfqpoint{1.626566in}{1.575000in}}%
\pgfpathlineto{\pgfqpoint{1.626301in}{1.575000in}}%
\pgfpathlineto{\pgfqpoint{1.626036in}{1.575000in}}%
\pgfpathlineto{\pgfqpoint{1.625772in}{1.575000in}}%
\pgfpathlineto{\pgfqpoint{1.625507in}{1.575000in}}%
\pgfpathlineto{\pgfqpoint{1.625242in}{1.575000in}}%
\pgfpathlineto{\pgfqpoint{1.624977in}{1.575000in}}%
\pgfpathlineto{\pgfqpoint{1.624712in}{1.575000in}}%
\pgfpathlineto{\pgfqpoint{1.624447in}{1.575000in}}%
\pgfpathlineto{\pgfqpoint{1.624182in}{1.575000in}}%
\pgfpathlineto{\pgfqpoint{1.623917in}{1.575000in}}%
\pgfpathlineto{\pgfqpoint{1.623652in}{1.575000in}}%
\pgfpathlineto{\pgfqpoint{1.623387in}{1.575000in}}%
\pgfpathlineto{\pgfqpoint{1.623122in}{1.575000in}}%
\pgfpathlineto{\pgfqpoint{1.622858in}{1.575000in}}%
\pgfpathlineto{\pgfqpoint{1.622593in}{1.575000in}}%
\pgfpathlineto{\pgfqpoint{1.622328in}{1.575000in}}%
\pgfpathlineto{\pgfqpoint{1.622063in}{1.575000in}}%
\pgfpathlineto{\pgfqpoint{1.621798in}{1.575000in}}%
\pgfpathlineto{\pgfqpoint{1.621533in}{1.575000in}}%
\pgfpathlineto{\pgfqpoint{1.621268in}{1.575000in}}%
\pgfpathlineto{\pgfqpoint{1.621003in}{1.575000in}}%
\pgfpathlineto{\pgfqpoint{1.620738in}{1.575000in}}%
\pgfpathlineto{\pgfqpoint{1.620473in}{1.575000in}}%
\pgfpathlineto{\pgfqpoint{1.620209in}{1.575000in}}%
\pgfpathlineto{\pgfqpoint{1.619944in}{1.575000in}}%
\pgfpathlineto{\pgfqpoint{1.619679in}{1.575000in}}%
\pgfpathlineto{\pgfqpoint{1.619414in}{1.575000in}}%
\pgfpathlineto{\pgfqpoint{1.619149in}{1.575000in}}%
\pgfpathlineto{\pgfqpoint{1.618884in}{1.575000in}}%
\pgfpathlineto{\pgfqpoint{1.618619in}{1.575000in}}%
\pgfpathlineto{\pgfqpoint{1.618354in}{1.575000in}}%
\pgfpathlineto{\pgfqpoint{1.618089in}{1.575000in}}%
\pgfpathlineto{\pgfqpoint{1.617824in}{1.575000in}}%
\pgfpathlineto{\pgfqpoint{1.617559in}{1.575000in}}%
\pgfpathlineto{\pgfqpoint{1.617295in}{1.575000in}}%
\pgfpathlineto{\pgfqpoint{1.617030in}{1.575000in}}%
\pgfpathlineto{\pgfqpoint{1.616765in}{1.575000in}}%
\pgfpathlineto{\pgfqpoint{1.616500in}{1.575000in}}%
\pgfpathlineto{\pgfqpoint{1.616235in}{1.575000in}}%
\pgfpathlineto{\pgfqpoint{1.615970in}{1.575000in}}%
\pgfpathlineto{\pgfqpoint{1.615705in}{1.575000in}}%
\pgfpathlineto{\pgfqpoint{1.615440in}{1.575000in}}%
\pgfpathlineto{\pgfqpoint{1.615175in}{1.575000in}}%
\pgfpathlineto{\pgfqpoint{1.614910in}{1.575000in}}%
\pgfpathlineto{\pgfqpoint{1.614646in}{1.575000in}}%
\pgfpathlineto{\pgfqpoint{1.614381in}{1.575000in}}%
\pgfpathlineto{\pgfqpoint{1.614116in}{1.575000in}}%
\pgfpathlineto{\pgfqpoint{1.613851in}{1.575000in}}%
\pgfpathlineto{\pgfqpoint{1.613586in}{1.575000in}}%
\pgfpathlineto{\pgfqpoint{1.613321in}{1.575000in}}%
\pgfpathlineto{\pgfqpoint{1.613056in}{1.575000in}}%
\pgfpathlineto{\pgfqpoint{1.612791in}{1.575000in}}%
\pgfpathlineto{\pgfqpoint{1.612526in}{1.575000in}}%
\pgfpathlineto{\pgfqpoint{1.612261in}{1.575000in}}%
\pgfpathlineto{\pgfqpoint{1.611996in}{1.575000in}}%
\pgfpathlineto{\pgfqpoint{1.611732in}{1.575000in}}%
\pgfpathlineto{\pgfqpoint{1.611467in}{1.575000in}}%
\pgfpathlineto{\pgfqpoint{1.611202in}{1.575000in}}%
\pgfpathlineto{\pgfqpoint{1.610937in}{1.575000in}}%
\pgfpathlineto{\pgfqpoint{1.610672in}{1.575000in}}%
\pgfpathlineto{\pgfqpoint{1.610407in}{1.575000in}}%
\pgfpathlineto{\pgfqpoint{1.610142in}{1.575000in}}%
\pgfpathlineto{\pgfqpoint{1.609877in}{1.575000in}}%
\pgfpathlineto{\pgfqpoint{1.609612in}{1.575000in}}%
\pgfpathlineto{\pgfqpoint{1.609347in}{1.575000in}}%
\pgfpathlineto{\pgfqpoint{1.609083in}{1.575000in}}%
\pgfpathlineto{\pgfqpoint{1.608818in}{1.575000in}}%
\pgfpathlineto{\pgfqpoint{1.608553in}{1.575000in}}%
\pgfpathlineto{\pgfqpoint{1.608288in}{1.575000in}}%
\pgfpathlineto{\pgfqpoint{1.608023in}{1.575000in}}%
\pgfpathlineto{\pgfqpoint{1.607758in}{1.575000in}}%
\pgfpathlineto{\pgfqpoint{1.607493in}{1.575000in}}%
\pgfpathlineto{\pgfqpoint{1.607228in}{1.575000in}}%
\pgfpathlineto{\pgfqpoint{1.606963in}{1.575000in}}%
\pgfpathlineto{\pgfqpoint{1.606698in}{1.575000in}}%
\pgfpathlineto{\pgfqpoint{1.606433in}{1.575000in}}%
\pgfpathlineto{\pgfqpoint{1.606169in}{1.575000in}}%
\pgfpathlineto{\pgfqpoint{1.605904in}{1.575000in}}%
\pgfpathlineto{\pgfqpoint{1.605639in}{1.575000in}}%
\pgfpathlineto{\pgfqpoint{1.605374in}{1.575000in}}%
\pgfpathlineto{\pgfqpoint{1.605109in}{1.575000in}}%
\pgfpathlineto{\pgfqpoint{1.604844in}{1.575000in}}%
\pgfpathlineto{\pgfqpoint{1.604579in}{1.575000in}}%
\pgfpathlineto{\pgfqpoint{1.604314in}{1.575000in}}%
\pgfpathlineto{\pgfqpoint{1.604049in}{1.575000in}}%
\pgfpathlineto{\pgfqpoint{1.603784in}{1.575000in}}%
\pgfpathlineto{\pgfqpoint{1.603520in}{1.575000in}}%
\pgfpathlineto{\pgfqpoint{1.603255in}{1.575000in}}%
\pgfpathlineto{\pgfqpoint{1.602990in}{1.575000in}}%
\pgfpathlineto{\pgfqpoint{1.602725in}{1.575000in}}%
\pgfpathlineto{\pgfqpoint{1.602460in}{1.575000in}}%
\pgfpathlineto{\pgfqpoint{1.602195in}{1.575000in}}%
\pgfpathlineto{\pgfqpoint{1.601930in}{1.575000in}}%
\pgfpathlineto{\pgfqpoint{1.601665in}{1.575000in}}%
\pgfpathlineto{\pgfqpoint{1.601400in}{1.575000in}}%
\pgfpathlineto{\pgfqpoint{1.601135in}{1.575000in}}%
\pgfpathlineto{\pgfqpoint{1.600870in}{1.575000in}}%
\pgfpathlineto{\pgfqpoint{1.600606in}{1.575000in}}%
\pgfpathlineto{\pgfqpoint{1.600341in}{1.575000in}}%
\pgfpathlineto{\pgfqpoint{1.600076in}{1.575000in}}%
\pgfpathlineto{\pgfqpoint{1.599811in}{1.575000in}}%
\pgfpathlineto{\pgfqpoint{1.599546in}{1.575000in}}%
\pgfpathlineto{\pgfqpoint{1.599281in}{1.575000in}}%
\pgfpathlineto{\pgfqpoint{1.599016in}{1.575000in}}%
\pgfpathlineto{\pgfqpoint{1.598751in}{1.575000in}}%
\pgfpathlineto{\pgfqpoint{1.598486in}{1.575000in}}%
\pgfpathlineto{\pgfqpoint{1.598221in}{1.575000in}}%
\pgfpathlineto{\pgfqpoint{1.597957in}{1.575000in}}%
\pgfpathlineto{\pgfqpoint{1.597692in}{1.575000in}}%
\pgfpathlineto{\pgfqpoint{1.597427in}{1.575000in}}%
\pgfpathlineto{\pgfqpoint{1.597162in}{1.575000in}}%
\pgfpathlineto{\pgfqpoint{1.596897in}{1.575000in}}%
\pgfpathlineto{\pgfqpoint{1.596632in}{1.575000in}}%
\pgfpathlineto{\pgfqpoint{1.596367in}{1.575000in}}%
\pgfpathlineto{\pgfqpoint{1.596102in}{1.575000in}}%
\pgfpathlineto{\pgfqpoint{1.595837in}{1.575000in}}%
\pgfpathlineto{\pgfqpoint{1.595572in}{1.575000in}}%
\pgfpathlineto{\pgfqpoint{1.595307in}{1.575000in}}%
\pgfpathlineto{\pgfqpoint{1.595043in}{1.575000in}}%
\pgfpathlineto{\pgfqpoint{1.594778in}{1.575000in}}%
\pgfpathlineto{\pgfqpoint{1.594513in}{1.575000in}}%
\pgfpathlineto{\pgfqpoint{1.594248in}{1.575000in}}%
\pgfpathlineto{\pgfqpoint{1.593983in}{1.575000in}}%
\pgfpathlineto{\pgfqpoint{1.593718in}{1.575000in}}%
\pgfpathlineto{\pgfqpoint{1.593453in}{1.575000in}}%
\pgfpathlineto{\pgfqpoint{1.593188in}{1.575000in}}%
\pgfpathlineto{\pgfqpoint{1.592923in}{1.575000in}}%
\pgfpathlineto{\pgfqpoint{1.592658in}{1.575000in}}%
\pgfpathlineto{\pgfqpoint{1.592394in}{1.575000in}}%
\pgfpathlineto{\pgfqpoint{1.592129in}{1.575000in}}%
\pgfpathlineto{\pgfqpoint{1.591864in}{1.575000in}}%
\pgfpathlineto{\pgfqpoint{1.591599in}{1.575000in}}%
\pgfpathlineto{\pgfqpoint{1.591334in}{1.575000in}}%
\pgfpathlineto{\pgfqpoint{1.591069in}{1.575000in}}%
\pgfpathlineto{\pgfqpoint{1.590804in}{1.575000in}}%
\pgfpathlineto{\pgfqpoint{1.590539in}{1.575000in}}%
\pgfpathlineto{\pgfqpoint{1.590274in}{1.575000in}}%
\pgfpathlineto{\pgfqpoint{1.590009in}{1.575000in}}%
\pgfpathlineto{\pgfqpoint{1.589744in}{1.575000in}}%
\pgfpathlineto{\pgfqpoint{1.589480in}{1.575000in}}%
\pgfpathlineto{\pgfqpoint{1.589215in}{1.575000in}}%
\pgfpathlineto{\pgfqpoint{1.588950in}{1.575000in}}%
\pgfpathlineto{\pgfqpoint{1.588685in}{1.575000in}}%
\pgfpathlineto{\pgfqpoint{1.588420in}{1.575000in}}%
\pgfpathlineto{\pgfqpoint{1.588155in}{1.575000in}}%
\pgfpathlineto{\pgfqpoint{1.587890in}{1.575000in}}%
\pgfpathlineto{\pgfqpoint{1.587625in}{1.575000in}}%
\pgfpathlineto{\pgfqpoint{1.587360in}{1.575000in}}%
\pgfpathlineto{\pgfqpoint{1.587095in}{1.575000in}}%
\pgfpathlineto{\pgfqpoint{1.586831in}{1.575000in}}%
\pgfpathlineto{\pgfqpoint{1.586566in}{1.575000in}}%
\pgfpathlineto{\pgfqpoint{1.586301in}{1.575000in}}%
\pgfpathlineto{\pgfqpoint{1.586036in}{1.575000in}}%
\pgfpathlineto{\pgfqpoint{1.585771in}{1.575000in}}%
\pgfpathlineto{\pgfqpoint{1.585506in}{1.575000in}}%
\pgfpathlineto{\pgfqpoint{1.585241in}{1.575000in}}%
\pgfpathlineto{\pgfqpoint{1.584976in}{1.575000in}}%
\pgfpathlineto{\pgfqpoint{1.584711in}{1.575000in}}%
\pgfpathlineto{\pgfqpoint{1.584446in}{1.575000in}}%
\pgfpathlineto{\pgfqpoint{1.584181in}{1.575000in}}%
\pgfpathlineto{\pgfqpoint{1.583917in}{1.575000in}}%
\pgfpathlineto{\pgfqpoint{1.583652in}{1.575000in}}%
\pgfpathlineto{\pgfqpoint{1.583387in}{1.575000in}}%
\pgfpathlineto{\pgfqpoint{1.583122in}{1.575000in}}%
\pgfpathlineto{\pgfqpoint{1.582857in}{1.575000in}}%
\pgfpathlineto{\pgfqpoint{1.582592in}{1.575000in}}%
\pgfpathlineto{\pgfqpoint{1.582327in}{1.575000in}}%
\pgfpathlineto{\pgfqpoint{1.582062in}{1.575000in}}%
\pgfpathlineto{\pgfqpoint{1.581797in}{1.575000in}}%
\pgfpathlineto{\pgfqpoint{1.581532in}{1.575000in}}%
\pgfpathlineto{\pgfqpoint{1.581268in}{1.575000in}}%
\pgfpathlineto{\pgfqpoint{1.581003in}{1.575000in}}%
\pgfpathlineto{\pgfqpoint{1.580738in}{1.575000in}}%
\pgfpathlineto{\pgfqpoint{1.580473in}{1.575000in}}%
\pgfpathlineto{\pgfqpoint{1.580208in}{1.575000in}}%
\pgfpathlineto{\pgfqpoint{1.579943in}{1.575000in}}%
\pgfpathlineto{\pgfqpoint{1.579678in}{1.575000in}}%
\pgfpathlineto{\pgfqpoint{1.579413in}{1.575000in}}%
\pgfpathlineto{\pgfqpoint{1.579148in}{1.575000in}}%
\pgfpathlineto{\pgfqpoint{1.578883in}{1.575000in}}%
\pgfpathlineto{\pgfqpoint{1.578618in}{1.575000in}}%
\pgfpathlineto{\pgfqpoint{1.578354in}{1.575000in}}%
\pgfpathlineto{\pgfqpoint{1.578089in}{1.575000in}}%
\pgfpathlineto{\pgfqpoint{1.577824in}{1.575000in}}%
\pgfpathlineto{\pgfqpoint{1.577559in}{1.575000in}}%
\pgfpathlineto{\pgfqpoint{1.577294in}{1.575000in}}%
\pgfpathlineto{\pgfqpoint{1.577029in}{1.575000in}}%
\pgfpathlineto{\pgfqpoint{1.576764in}{1.575000in}}%
\pgfpathlineto{\pgfqpoint{1.576499in}{1.575000in}}%
\pgfpathlineto{\pgfqpoint{1.576234in}{1.575000in}}%
\pgfpathlineto{\pgfqpoint{1.575969in}{1.575000in}}%
\pgfpathlineto{\pgfqpoint{1.575705in}{1.575000in}}%
\pgfpathlineto{\pgfqpoint{1.575440in}{1.575000in}}%
\pgfpathlineto{\pgfqpoint{1.575175in}{1.575000in}}%
\pgfpathlineto{\pgfqpoint{1.574910in}{1.575000in}}%
\pgfpathlineto{\pgfqpoint{1.574645in}{1.575000in}}%
\pgfpathlineto{\pgfqpoint{1.574380in}{1.575000in}}%
\pgfpathlineto{\pgfqpoint{1.574115in}{1.575000in}}%
\pgfpathlineto{\pgfqpoint{1.573850in}{1.575000in}}%
\pgfpathlineto{\pgfqpoint{1.573585in}{1.575000in}}%
\pgfpathlineto{\pgfqpoint{1.573320in}{1.575000in}}%
\pgfpathlineto{\pgfqpoint{1.573055in}{1.575000in}}%
\pgfpathlineto{\pgfqpoint{1.572791in}{1.575000in}}%
\pgfpathlineto{\pgfqpoint{1.572526in}{1.575000in}}%
\pgfpathlineto{\pgfqpoint{1.572261in}{1.575000in}}%
\pgfpathlineto{\pgfqpoint{1.571996in}{1.575000in}}%
\pgfpathlineto{\pgfqpoint{1.571731in}{1.575000in}}%
\pgfpathlineto{\pgfqpoint{1.571466in}{1.575000in}}%
\pgfpathlineto{\pgfqpoint{1.571201in}{1.575000in}}%
\pgfpathlineto{\pgfqpoint{1.570936in}{1.575000in}}%
\pgfpathlineto{\pgfqpoint{1.570671in}{1.575000in}}%
\pgfpathlineto{\pgfqpoint{1.570406in}{1.575000in}}%
\pgfpathlineto{\pgfqpoint{1.570142in}{1.575000in}}%
\pgfpathlineto{\pgfqpoint{1.569877in}{1.575000in}}%
\pgfpathlineto{\pgfqpoint{1.569612in}{1.575000in}}%
\pgfpathlineto{\pgfqpoint{1.569347in}{1.575000in}}%
\pgfpathlineto{\pgfqpoint{1.569082in}{1.575000in}}%
\pgfpathlineto{\pgfqpoint{1.568817in}{1.575000in}}%
\pgfpathlineto{\pgfqpoint{1.568552in}{1.575000in}}%
\pgfpathlineto{\pgfqpoint{1.568287in}{1.575000in}}%
\pgfpathlineto{\pgfqpoint{1.568022in}{1.575000in}}%
\pgfpathlineto{\pgfqpoint{1.567757in}{1.575000in}}%
\pgfpathlineto{\pgfqpoint{1.567492in}{1.575000in}}%
\pgfpathlineto{\pgfqpoint{1.567228in}{1.575000in}}%
\pgfpathlineto{\pgfqpoint{1.566963in}{1.575000in}}%
\pgfpathlineto{\pgfqpoint{1.566698in}{1.575000in}}%
\pgfpathlineto{\pgfqpoint{1.566433in}{1.575000in}}%
\pgfpathlineto{\pgfqpoint{1.566168in}{1.575000in}}%
\pgfpathlineto{\pgfqpoint{1.565903in}{1.575000in}}%
\pgfpathlineto{\pgfqpoint{1.565638in}{1.575000in}}%
\pgfpathlineto{\pgfqpoint{1.565373in}{1.575000in}}%
\pgfpathlineto{\pgfqpoint{1.565108in}{1.575000in}}%
\pgfpathlineto{\pgfqpoint{1.564843in}{1.575000in}}%
\pgfpathlineto{\pgfqpoint{1.564579in}{1.575000in}}%
\pgfpathlineto{\pgfqpoint{1.564314in}{1.575000in}}%
\pgfpathlineto{\pgfqpoint{1.564049in}{1.575000in}}%
\pgfpathlineto{\pgfqpoint{1.563784in}{1.575000in}}%
\pgfpathlineto{\pgfqpoint{1.563519in}{1.575000in}}%
\pgfpathlineto{\pgfqpoint{1.563254in}{1.575000in}}%
\pgfpathlineto{\pgfqpoint{1.562989in}{1.575000in}}%
\pgfpathlineto{\pgfqpoint{1.562724in}{1.575000in}}%
\pgfpathlineto{\pgfqpoint{1.562459in}{1.575000in}}%
\pgfpathlineto{\pgfqpoint{1.562194in}{1.575000in}}%
\pgfpathlineto{\pgfqpoint{1.561929in}{1.575000in}}%
\pgfpathlineto{\pgfqpoint{1.561665in}{1.575000in}}%
\pgfpathlineto{\pgfqpoint{1.561400in}{1.575000in}}%
\pgfpathlineto{\pgfqpoint{1.561135in}{1.575000in}}%
\pgfpathlineto{\pgfqpoint{1.560870in}{1.575000in}}%
\pgfpathlineto{\pgfqpoint{1.560605in}{1.575000in}}%
\pgfpathlineto{\pgfqpoint{1.560340in}{1.575000in}}%
\pgfpathlineto{\pgfqpoint{1.560075in}{1.575000in}}%
\pgfpathlineto{\pgfqpoint{1.559810in}{1.575000in}}%
\pgfpathlineto{\pgfqpoint{1.559545in}{1.575000in}}%
\pgfpathlineto{\pgfqpoint{1.559280in}{1.575000in}}%
\pgfpathlineto{\pgfqpoint{1.559016in}{1.575000in}}%
\pgfpathlineto{\pgfqpoint{1.558751in}{1.575000in}}%
\pgfpathlineto{\pgfqpoint{1.558486in}{1.575000in}}%
\pgfpathlineto{\pgfqpoint{1.558221in}{1.575000in}}%
\pgfpathlineto{\pgfqpoint{1.557956in}{1.575000in}}%
\pgfpathlineto{\pgfqpoint{1.557691in}{1.575000in}}%
\pgfpathlineto{\pgfqpoint{1.557426in}{1.575000in}}%
\pgfpathlineto{\pgfqpoint{1.557161in}{1.575000in}}%
\pgfpathlineto{\pgfqpoint{1.556896in}{1.575000in}}%
\pgfpathlineto{\pgfqpoint{1.556631in}{1.575000in}}%
\pgfpathlineto{\pgfqpoint{1.556366in}{1.575000in}}%
\pgfpathlineto{\pgfqpoint{1.556102in}{1.575000in}}%
\pgfpathlineto{\pgfqpoint{1.555837in}{1.575000in}}%
\pgfpathlineto{\pgfqpoint{1.555572in}{1.575000in}}%
\pgfpathlineto{\pgfqpoint{1.555307in}{1.575000in}}%
\pgfpathlineto{\pgfqpoint{1.555042in}{1.575000in}}%
\pgfpathlineto{\pgfqpoint{1.554777in}{1.575000in}}%
\pgfpathlineto{\pgfqpoint{1.554512in}{1.575000in}}%
\pgfpathlineto{\pgfqpoint{1.554247in}{1.575000in}}%
\pgfpathlineto{\pgfqpoint{1.553982in}{1.575000in}}%
\pgfpathlineto{\pgfqpoint{1.553717in}{1.575000in}}%
\pgfpathlineto{\pgfqpoint{1.553453in}{1.575000in}}%
\pgfpathlineto{\pgfqpoint{1.553188in}{1.575000in}}%
\pgfpathlineto{\pgfqpoint{1.552923in}{1.575000in}}%
\pgfpathlineto{\pgfqpoint{1.552658in}{1.575000in}}%
\pgfpathlineto{\pgfqpoint{1.552393in}{1.575000in}}%
\pgfpathlineto{\pgfqpoint{1.552128in}{1.575000in}}%
\pgfpathlineto{\pgfqpoint{1.551863in}{1.575000in}}%
\pgfpathlineto{\pgfqpoint{1.551598in}{1.575000in}}%
\pgfpathlineto{\pgfqpoint{1.551333in}{1.575000in}}%
\pgfpathlineto{\pgfqpoint{1.551068in}{1.575000in}}%
\pgfpathlineto{\pgfqpoint{1.550803in}{1.575000in}}%
\pgfpathlineto{\pgfqpoint{1.550539in}{1.575000in}}%
\pgfpathlineto{\pgfqpoint{1.550274in}{1.575000in}}%
\pgfpathlineto{\pgfqpoint{1.550009in}{1.575000in}}%
\pgfpathlineto{\pgfqpoint{1.549744in}{1.575000in}}%
\pgfpathlineto{\pgfqpoint{1.549479in}{1.575000in}}%
\pgfpathlineto{\pgfqpoint{1.549214in}{1.575000in}}%
\pgfpathlineto{\pgfqpoint{1.548949in}{1.575000in}}%
\pgfpathlineto{\pgfqpoint{1.548684in}{1.575000in}}%
\pgfpathlineto{\pgfqpoint{1.548419in}{1.575000in}}%
\pgfpathlineto{\pgfqpoint{1.548154in}{1.575000in}}%
\pgfpathlineto{\pgfqpoint{1.547890in}{1.575000in}}%
\pgfpathlineto{\pgfqpoint{1.547625in}{1.575000in}}%
\pgfpathlineto{\pgfqpoint{1.547360in}{1.575000in}}%
\pgfpathlineto{\pgfqpoint{1.547095in}{1.575000in}}%
\pgfpathlineto{\pgfqpoint{1.546830in}{1.575000in}}%
\pgfpathlineto{\pgfqpoint{1.546565in}{1.575000in}}%
\pgfpathlineto{\pgfqpoint{1.546300in}{1.575000in}}%
\pgfpathlineto{\pgfqpoint{1.546035in}{1.575000in}}%
\pgfpathlineto{\pgfqpoint{1.545770in}{1.575000in}}%
\pgfpathlineto{\pgfqpoint{1.545505in}{1.575000in}}%
\pgfpathlineto{\pgfqpoint{1.545240in}{1.575000in}}%
\pgfpathlineto{\pgfqpoint{1.544976in}{1.575000in}}%
\pgfpathlineto{\pgfqpoint{1.544711in}{1.575000in}}%
\pgfpathlineto{\pgfqpoint{1.544446in}{1.575000in}}%
\pgfpathlineto{\pgfqpoint{1.544181in}{1.575000in}}%
\pgfpathlineto{\pgfqpoint{1.543916in}{1.575000in}}%
\pgfpathlineto{\pgfqpoint{1.543651in}{1.575000in}}%
\pgfpathlineto{\pgfqpoint{1.543386in}{1.575000in}}%
\pgfpathlineto{\pgfqpoint{1.543121in}{1.575000in}}%
\pgfpathlineto{\pgfqpoint{1.542856in}{1.575000in}}%
\pgfpathlineto{\pgfqpoint{1.542591in}{1.575000in}}%
\pgfpathlineto{\pgfqpoint{1.542327in}{1.575000in}}%
\pgfpathlineto{\pgfqpoint{1.542062in}{1.575000in}}%
\pgfpathlineto{\pgfqpoint{1.541797in}{1.575000in}}%
\pgfpathlineto{\pgfqpoint{1.541532in}{1.575000in}}%
\pgfpathlineto{\pgfqpoint{1.541267in}{1.575000in}}%
\pgfpathlineto{\pgfqpoint{1.541002in}{1.575000in}}%
\pgfpathlineto{\pgfqpoint{1.540737in}{1.575000in}}%
\pgfpathlineto{\pgfqpoint{1.540472in}{1.575000in}}%
\pgfpathlineto{\pgfqpoint{1.540207in}{1.575000in}}%
\pgfpathlineto{\pgfqpoint{1.539942in}{1.575000in}}%
\pgfpathlineto{\pgfqpoint{1.539677in}{1.575000in}}%
\pgfpathlineto{\pgfqpoint{1.539413in}{1.575000in}}%
\pgfpathlineto{\pgfqpoint{1.539148in}{1.575000in}}%
\pgfpathlineto{\pgfqpoint{1.538883in}{1.575000in}}%
\pgfpathlineto{\pgfqpoint{1.538618in}{1.575000in}}%
\pgfpathlineto{\pgfqpoint{1.538353in}{1.575000in}}%
\pgfpathlineto{\pgfqpoint{1.538088in}{1.575000in}}%
\pgfpathlineto{\pgfqpoint{1.537823in}{1.575000in}}%
\pgfpathlineto{\pgfqpoint{1.537558in}{1.575000in}}%
\pgfpathlineto{\pgfqpoint{1.537293in}{1.575000in}}%
\pgfpathlineto{\pgfqpoint{1.537028in}{1.575000in}}%
\pgfpathlineto{\pgfqpoint{1.536764in}{1.575000in}}%
\pgfpathlineto{\pgfqpoint{1.536499in}{1.575000in}}%
\pgfpathlineto{\pgfqpoint{1.536234in}{1.575000in}}%
\pgfpathlineto{\pgfqpoint{1.535969in}{1.575000in}}%
\pgfpathlineto{\pgfqpoint{1.535704in}{1.575000in}}%
\pgfpathlineto{\pgfqpoint{1.535439in}{1.575000in}}%
\pgfpathlineto{\pgfqpoint{1.535174in}{1.575000in}}%
\pgfpathlineto{\pgfqpoint{1.534909in}{1.575000in}}%
\pgfpathlineto{\pgfqpoint{1.534644in}{1.575000in}}%
\pgfpathlineto{\pgfqpoint{1.534379in}{1.575000in}}%
\pgfpathlineto{\pgfqpoint{1.534114in}{1.575000in}}%
\pgfpathlineto{\pgfqpoint{1.533850in}{1.575000in}}%
\pgfpathlineto{\pgfqpoint{1.533585in}{1.575000in}}%
\pgfpathlineto{\pgfqpoint{1.533320in}{1.575000in}}%
\pgfpathlineto{\pgfqpoint{1.533055in}{1.575000in}}%
\pgfpathlineto{\pgfqpoint{1.532790in}{1.575000in}}%
\pgfpathlineto{\pgfqpoint{1.532525in}{1.575000in}}%
\pgfpathlineto{\pgfqpoint{1.532260in}{1.575000in}}%
\pgfpathlineto{\pgfqpoint{1.531995in}{1.575000in}}%
\pgfpathlineto{\pgfqpoint{1.531730in}{1.575000in}}%
\pgfpathlineto{\pgfqpoint{1.531465in}{1.575000in}}%
\pgfpathlineto{\pgfqpoint{1.531201in}{1.575000in}}%
\pgfpathlineto{\pgfqpoint{1.530936in}{1.575000in}}%
\pgfpathlineto{\pgfqpoint{1.530671in}{1.575000in}}%
\pgfpathlineto{\pgfqpoint{1.530406in}{1.575000in}}%
\pgfpathlineto{\pgfqpoint{1.530141in}{1.575000in}}%
\pgfpathlineto{\pgfqpoint{1.529876in}{1.575000in}}%
\pgfpathlineto{\pgfqpoint{1.529611in}{1.575000in}}%
\pgfpathlineto{\pgfqpoint{1.529346in}{1.575000in}}%
\pgfpathlineto{\pgfqpoint{1.529081in}{1.575000in}}%
\pgfpathlineto{\pgfqpoint{1.528816in}{1.575000in}}%
\pgfpathlineto{\pgfqpoint{1.528551in}{1.575000in}}%
\pgfpathlineto{\pgfqpoint{1.528287in}{1.575000in}}%
\pgfpathlineto{\pgfqpoint{1.528022in}{1.575000in}}%
\pgfpathlineto{\pgfqpoint{1.527757in}{1.575000in}}%
\pgfpathlineto{\pgfqpoint{1.527492in}{1.575000in}}%
\pgfpathlineto{\pgfqpoint{1.527227in}{1.575000in}}%
\pgfpathlineto{\pgfqpoint{1.526962in}{1.575000in}}%
\pgfpathlineto{\pgfqpoint{1.526697in}{1.575000in}}%
\pgfpathlineto{\pgfqpoint{1.526432in}{1.575000in}}%
\pgfpathlineto{\pgfqpoint{1.526167in}{1.575000in}}%
\pgfpathlineto{\pgfqpoint{1.525902in}{1.575000in}}%
\pgfpathlineto{\pgfqpoint{1.525638in}{1.575000in}}%
\pgfpathlineto{\pgfqpoint{1.525373in}{1.575000in}}%
\pgfpathlineto{\pgfqpoint{1.525108in}{1.575000in}}%
\pgfpathlineto{\pgfqpoint{1.524843in}{1.575000in}}%
\pgfpathlineto{\pgfqpoint{1.524578in}{1.575000in}}%
\pgfpathlineto{\pgfqpoint{1.524313in}{1.575000in}}%
\pgfpathlineto{\pgfqpoint{1.524048in}{1.575000in}}%
\pgfpathlineto{\pgfqpoint{1.523783in}{1.575000in}}%
\pgfpathlineto{\pgfqpoint{1.523518in}{1.575000in}}%
\pgfpathlineto{\pgfqpoint{1.523253in}{1.575000in}}%
\pgfpathlineto{\pgfqpoint{1.522988in}{1.575000in}}%
\pgfpathlineto{\pgfqpoint{1.522724in}{1.575000in}}%
\pgfpathlineto{\pgfqpoint{1.522459in}{1.575000in}}%
\pgfpathlineto{\pgfqpoint{1.522194in}{1.575000in}}%
\pgfpathlineto{\pgfqpoint{1.521929in}{1.575000in}}%
\pgfpathlineto{\pgfqpoint{1.521664in}{1.575000in}}%
\pgfpathlineto{\pgfqpoint{1.521399in}{1.575000in}}%
\pgfpathlineto{\pgfqpoint{1.521134in}{1.575000in}}%
\pgfpathlineto{\pgfqpoint{1.520869in}{1.575000in}}%
\pgfpathlineto{\pgfqpoint{1.520604in}{1.575000in}}%
\pgfpathlineto{\pgfqpoint{1.520339in}{1.575000in}}%
\pgfpathlineto{\pgfqpoint{1.520075in}{1.575000in}}%
\pgfpathlineto{\pgfqpoint{1.519810in}{1.575000in}}%
\pgfpathlineto{\pgfqpoint{1.519545in}{1.575000in}}%
\pgfpathlineto{\pgfqpoint{1.519280in}{1.575000in}}%
\pgfpathlineto{\pgfqpoint{1.519015in}{1.575000in}}%
\pgfpathlineto{\pgfqpoint{1.518750in}{1.575000in}}%
\pgfpathclose%
\pgfusepath{fill}%
\end{pgfscope}%
\begin{pgfscope}%
\pgfpathrectangle{\pgfqpoint{0.900000in}{0.600000in}}{\pgfqpoint{3.375000in}{1.950000in}} %
\pgfusepath{clip}%
\pgfsetbuttcap%
\pgfsetmiterjoin%
\definecolor{currentfill}{rgb}{0.000000,0.000000,0.000000}%
\pgfsetfillcolor{currentfill}%
\pgfsetlinewidth{1.003750pt}%
\definecolor{currentstroke}{rgb}{0.000000,0.000000,0.000000}%
\pgfsetstrokecolor{currentstroke}%
\pgfsetdash{}{0pt}%
\pgfpathmoveto{\pgfqpoint{4.181250in}{1.575000in}}%
\pgfpathlineto{\pgfqpoint{4.087500in}{1.526250in}}%
\pgfpathlineto{\pgfqpoint{4.087500in}{1.574976in}}%
\pgfpathlineto{\pgfqpoint{3.900000in}{1.574976in}}%
\pgfpathlineto{\pgfqpoint{3.900000in}{1.575024in}}%
\pgfpathlineto{\pgfqpoint{4.087500in}{1.575024in}}%
\pgfpathlineto{\pgfqpoint{4.087500in}{1.623750in}}%
\pgfpathclose%
\pgfusepath{stroke,fill}%
\end{pgfscope}%
\begin{pgfscope}%
\pgfpathrectangle{\pgfqpoint{0.900000in}{0.600000in}}{\pgfqpoint{3.375000in}{1.950000in}} %
\pgfusepath{clip}%
\pgfsetbuttcap%
\pgfsetmiterjoin%
\definecolor{currentfill}{rgb}{0.000000,0.000000,0.000000}%
\pgfsetfillcolor{currentfill}%
\pgfsetlinewidth{1.003750pt}%
\definecolor{currentstroke}{rgb}{0.000000,0.000000,0.000000}%
\pgfsetstrokecolor{currentstroke}%
\pgfsetdash{}{0pt}%
\pgfpathmoveto{\pgfqpoint{2.596875in}{2.501250in}}%
\pgfpathlineto{\pgfqpoint{2.643750in}{2.403750in}}%
\pgfpathlineto{\pgfqpoint{2.597812in}{2.403750in}}%
\pgfpathlineto{\pgfqpoint{2.597812in}{2.306250in}}%
\pgfpathlineto{\pgfqpoint{2.595938in}{2.306250in}}%
\pgfpathlineto{\pgfqpoint{2.595938in}{2.403750in}}%
\pgfpathlineto{\pgfqpoint{2.550000in}{2.403750in}}%
\pgfpathclose%
\pgfusepath{stroke,fill}%
\end{pgfscope}%
\begin{pgfscope}%
\pgfpathrectangle{\pgfqpoint{0.900000in}{0.600000in}}{\pgfqpoint{3.375000in}{1.950000in}} %
\pgfusepath{clip}%
\pgfsetrectcap%
\pgfsetroundjoin%
\pgfsetlinewidth{1.003750pt}%
\definecolor{currentstroke}{rgb}{1.000000,0.000000,0.000000}%
\pgfsetstrokecolor{currentstroke}%
\pgfsetdash{}{0pt}%
\pgfpathmoveto{\pgfqpoint{3.337500in}{1.770185in}}%
\pgfpathlineto{\pgfqpoint{3.550711in}{1.771688in}}%
\pgfpathlineto{\pgfqpoint{3.641639in}{1.774333in}}%
\pgfpathlineto{\pgfqpoint{3.704348in}{1.778299in}}%
\pgfpathlineto{\pgfqpoint{3.751380in}{1.783512in}}%
\pgfpathlineto{\pgfqpoint{3.789005in}{1.789956in}}%
\pgfpathlineto{\pgfqpoint{3.820360in}{1.797618in}}%
\pgfpathlineto{\pgfqpoint{3.845443in}{1.805817in}}%
\pgfpathlineto{\pgfqpoint{3.870527in}{1.816451in}}%
\pgfpathlineto{\pgfqpoint{3.892475in}{1.828315in}}%
\pgfpathlineto{\pgfqpoint{3.911288in}{1.840869in}}%
\pgfpathlineto{\pgfqpoint{3.930100in}{1.856126in}}%
\pgfpathlineto{\pgfqpoint{3.948913in}{1.874667in}}%
\pgfpathlineto{\pgfqpoint{3.964590in}{1.893133in}}%
\pgfpathlineto{\pgfqpoint{3.980268in}{1.914856in}}%
\pgfpathlineto{\pgfqpoint{3.995945in}{1.940412in}}%
\pgfpathlineto{\pgfqpoint{4.011622in}{1.970477in}}%
\pgfpathlineto{\pgfqpoint{4.027299in}{2.005845in}}%
\pgfpathlineto{\pgfqpoint{4.042977in}{2.047454in}}%
\pgfpathlineto{\pgfqpoint{4.058654in}{2.096403in}}%
\pgfpathlineto{\pgfqpoint{4.074331in}{2.153988in}}%
\pgfpathlineto{\pgfqpoint{4.090008in}{2.221732in}}%
\pgfpathlineto{\pgfqpoint{4.105686in}{2.301427in}}%
\pgfpathlineto{\pgfqpoint{4.121363in}{2.395183in}}%
\pgfpathlineto{\pgfqpoint{4.137040in}{2.505479in}}%
\pgfpathlineto{\pgfqpoint{4.143932in}{2.560000in}}%
\pgfpathlineto{\pgfqpoint{4.143932in}{2.560000in}}%
\pgfusepath{stroke}%
\end{pgfscope}%
\begin{pgfscope}%
\pgfpathrectangle{\pgfqpoint{0.900000in}{0.600000in}}{\pgfqpoint{3.375000in}{1.950000in}} %
\pgfusepath{clip}%
\pgfsetrectcap%
\pgfsetroundjoin%
\pgfsetlinewidth{1.003750pt}%
\definecolor{currentstroke}{rgb}{0.000000,0.000000,1.000000}%
\pgfsetstrokecolor{currentstroke}%
\pgfsetdash{}{0pt}%
\pgfpathmoveto{\pgfqpoint{3.337500in}{1.771293in}}%
\pgfpathlineto{\pgfqpoint{3.419022in}{1.773719in}}%
\pgfpathlineto{\pgfqpoint{3.472324in}{1.777419in}}%
\pgfpathlineto{\pgfqpoint{3.509950in}{1.782079in}}%
\pgfpathlineto{\pgfqpoint{3.541304in}{1.788131in}}%
\pgfpathlineto{\pgfqpoint{3.566388in}{1.795093in}}%
\pgfpathlineto{\pgfqpoint{3.588336in}{1.803346in}}%
\pgfpathlineto{\pgfqpoint{3.607149in}{1.812549in}}%
\pgfpathlineto{\pgfqpoint{3.625962in}{1.824292in}}%
\pgfpathlineto{\pgfqpoint{3.641639in}{1.836518in}}%
\pgfpathlineto{\pgfqpoint{3.657316in}{1.851497in}}%
\pgfpathlineto{\pgfqpoint{3.672993in}{1.869850in}}%
\pgfpathlineto{\pgfqpoint{3.688671in}{1.892335in}}%
\pgfpathlineto{\pgfqpoint{3.701212in}{1.913918in}}%
\pgfpathlineto{\pgfqpoint{3.713754in}{1.939308in}}%
\pgfpathlineto{\pgfqpoint{3.726296in}{1.969178in}}%
\pgfpathlineto{\pgfqpoint{3.738838in}{2.004318in}}%
\pgfpathlineto{\pgfqpoint{3.751380in}{2.045656in}}%
\pgfpathlineto{\pgfqpoint{3.763921in}{2.094288in}}%
\pgfpathlineto{\pgfqpoint{3.776463in}{2.151500in}}%
\pgfpathlineto{\pgfqpoint{3.789005in}{2.218805in}}%
\pgfpathlineto{\pgfqpoint{3.801547in}{2.297985in}}%
\pgfpathlineto{\pgfqpoint{3.814089in}{2.391133in}}%
\pgfpathlineto{\pgfqpoint{3.826630in}{2.500715in}}%
\pgfpathlineto{\pgfqpoint{3.832647in}{2.560000in}}%
\pgfpathlineto{\pgfqpoint{3.832647in}{2.560000in}}%
\pgfusepath{stroke}%
\end{pgfscope}%
\begin{pgfscope}%
\pgfpathrectangle{\pgfqpoint{0.900000in}{0.600000in}}{\pgfqpoint{3.375000in}{1.950000in}} %
\pgfusepath{clip}%
\pgfsetrectcap%
\pgfsetroundjoin%
\pgfsetlinewidth{1.003750pt}%
\definecolor{currentstroke}{rgb}{0.000000,0.501961,0.000000}%
\pgfsetstrokecolor{currentstroke}%
\pgfsetdash{}{0pt}%
\pgfpathmoveto{\pgfqpoint{3.337500in}{1.782935in}}%
\pgfpathlineto{\pgfqpoint{3.368855in}{1.789416in}}%
\pgfpathlineto{\pgfqpoint{3.393938in}{1.796872in}}%
\pgfpathlineto{\pgfqpoint{3.415886in}{1.805710in}}%
\pgfpathlineto{\pgfqpoint{3.434699in}{1.815566in}}%
\pgfpathlineto{\pgfqpoint{3.453512in}{1.828141in}}%
\pgfpathlineto{\pgfqpoint{3.469189in}{1.841234in}}%
\pgfpathlineto{\pgfqpoint{3.484866in}{1.857276in}}%
\pgfpathlineto{\pgfqpoint{3.500543in}{1.876930in}}%
\pgfpathlineto{\pgfqpoint{3.516221in}{1.901010in}}%
\pgfpathlineto{\pgfqpoint{3.528763in}{1.924123in}}%
\pgfpathlineto{\pgfqpoint{3.541304in}{1.951314in}}%
\pgfpathlineto{\pgfqpoint{3.553846in}{1.983302in}}%
\pgfpathlineto{\pgfqpoint{3.566388in}{2.020933in}}%
\pgfpathlineto{\pgfqpoint{3.578930in}{2.065204in}}%
\pgfpathlineto{\pgfqpoint{3.591472in}{2.117284in}}%
\pgfpathlineto{\pgfqpoint{3.604013in}{2.178553in}}%
\pgfpathlineto{\pgfqpoint{3.616555in}{2.250631in}}%
\pgfpathlineto{\pgfqpoint{3.629097in}{2.335426in}}%
\pgfpathlineto{\pgfqpoint{3.641639in}{2.435180in}}%
\pgfpathlineto{\pgfqpoint{3.654902in}{2.560000in}}%
\pgfpathlineto{\pgfqpoint{3.654902in}{2.560000in}}%
\pgfusepath{stroke}%
\end{pgfscope}%
\begin{pgfscope}%
\pgfpathrectangle{\pgfqpoint{0.900000in}{0.600000in}}{\pgfqpoint{3.375000in}{1.950000in}} %
\pgfusepath{clip}%
\pgfsetrectcap%
\pgfsetroundjoin%
\pgfsetlinewidth{1.003750pt}%
\definecolor{currentstroke}{rgb}{0.000000,0.000000,1.000000}%
\pgfsetstrokecolor{currentstroke}%
\pgfsetdash{}{0pt}%
\pgfpathmoveto{\pgfqpoint{1.347279in}{0.590000in}}%
\pgfpathlineto{\pgfqpoint{1.357776in}{0.684367in}}%
\pgfpathlineto{\pgfqpoint{1.370318in}{0.781377in}}%
\pgfpathlineto{\pgfqpoint{1.382860in}{0.863839in}}%
\pgfpathlineto{\pgfqpoint{1.395401in}{0.933935in}}%
\pgfpathlineto{\pgfqpoint{1.407943in}{0.993518in}}%
\pgfpathlineto{\pgfqpoint{1.420485in}{1.044166in}}%
\pgfpathlineto{\pgfqpoint{1.433027in}{1.087219in}}%
\pgfpathlineto{\pgfqpoint{1.445569in}{1.123815in}}%
\pgfpathlineto{\pgfqpoint{1.458110in}{1.154923in}}%
\pgfpathlineto{\pgfqpoint{1.473788in}{1.187332in}}%
\pgfpathlineto{\pgfqpoint{1.489465in}{1.213784in}}%
\pgfpathlineto{\pgfqpoint{1.505142in}{1.235375in}}%
\pgfpathlineto{\pgfqpoint{1.520819in}{1.252997in}}%
\pgfpathlineto{\pgfqpoint{1.536497in}{1.267380in}}%
\pgfpathlineto{\pgfqpoint{1.552174in}{1.279119in}}%
\pgfpathlineto{\pgfqpoint{1.570987in}{1.290395in}}%
\pgfpathlineto{\pgfqpoint{1.589799in}{1.299231in}}%
\pgfpathlineto{\pgfqpoint{1.611747in}{1.307156in}}%
\pgfpathlineto{\pgfqpoint{1.636831in}{1.313840in}}%
\pgfpathlineto{\pgfqpoint{1.668186in}{1.319652in}}%
\pgfpathlineto{\pgfqpoint{1.705811in}{1.324127in}}%
\pgfpathlineto{\pgfqpoint{1.752843in}{1.327377in}}%
\pgfpathlineto{\pgfqpoint{1.821823in}{1.329665in}}%
\pgfpathlineto{\pgfqpoint{1.837500in}{1.329957in}}%
\pgfpathlineto{\pgfqpoint{1.837500in}{1.329957in}}%
\pgfusepath{stroke}%
\end{pgfscope}%
\begin{pgfscope}%
\pgfpathrectangle{\pgfqpoint{0.900000in}{0.600000in}}{\pgfqpoint{3.375000in}{1.950000in}} %
\pgfusepath{clip}%
\pgfsetrectcap%
\pgfsetroundjoin%
\pgfsetlinewidth{1.003750pt}%
\definecolor{currentstroke}{rgb}{0.000000,0.000000,1.000000}%
\pgfsetstrokecolor{currentstroke}%
\pgfsetdash{}{0pt}%
\pgfpathmoveto{\pgfqpoint{1.837500in}{1.330034in}}%
\pgfpathlineto{\pgfqpoint{2.213754in}{1.331144in}}%
\pgfpathlineto{\pgfqpoint{2.283988in}{1.333437in}}%
\pgfpathlineto{\pgfqpoint{2.329139in}{1.336989in}}%
\pgfpathlineto{\pgfqpoint{2.364256in}{1.342091in}}%
\pgfpathlineto{\pgfqpoint{2.389339in}{1.347806in}}%
\pgfpathlineto{\pgfqpoint{2.409406in}{1.354167in}}%
\pgfpathlineto{\pgfqpoint{2.429473in}{1.362630in}}%
\pgfpathlineto{\pgfqpoint{2.449540in}{1.373749in}}%
\pgfpathlineto{\pgfqpoint{2.469607in}{1.388122in}}%
\pgfpathlineto{\pgfqpoint{2.484657in}{1.401380in}}%
\pgfpathlineto{\pgfqpoint{2.499707in}{1.416978in}}%
\pgfpathlineto{\pgfqpoint{2.519774in}{1.441574in}}%
\pgfpathlineto{\pgfqpoint{2.539841in}{1.470371in}}%
\pgfpathlineto{\pgfqpoint{2.564925in}{1.511213in}}%
\pgfpathlineto{\pgfqpoint{2.640176in}{1.638446in}}%
\pgfpathlineto{\pgfqpoint{2.660242in}{1.666232in}}%
\pgfpathlineto{\pgfqpoint{2.680309in}{1.689740in}}%
\pgfpathlineto{\pgfqpoint{2.700376in}{1.708958in}}%
\pgfpathlineto{\pgfqpoint{2.720443in}{1.724236in}}%
\pgfpathlineto{\pgfqpoint{2.740510in}{1.736112in}}%
\pgfpathlineto{\pgfqpoint{2.760577in}{1.745184in}}%
\pgfpathlineto{\pgfqpoint{2.780644in}{1.752023in}}%
\pgfpathlineto{\pgfqpoint{2.805727in}{1.758181in}}%
\pgfpathlineto{\pgfqpoint{2.835828in}{1.763072in}}%
\pgfpathlineto{\pgfqpoint{2.875962in}{1.766895in}}%
\pgfpathlineto{\pgfqpoint{2.931145in}{1.769420in}}%
\pgfpathlineto{\pgfqpoint{3.016430in}{1.770758in}}%
\pgfpathlineto{\pgfqpoint{3.232149in}{1.771204in}}%
\pgfpathlineto{\pgfqpoint{3.337500in}{1.771216in}}%
\pgfpathlineto{\pgfqpoint{3.337500in}{1.771216in}}%
\pgfusepath{stroke}%
\end{pgfscope}%
\begin{pgfscope}%
\pgfpathrectangle{\pgfqpoint{0.900000in}{0.600000in}}{\pgfqpoint{3.375000in}{1.950000in}} %
\pgfusepath{clip}%
\pgfsetrectcap%
\pgfsetroundjoin%
\pgfsetlinewidth{1.003750pt}%
\definecolor{currentstroke}{rgb}{0.000000,0.000000,0.000000}%
\pgfsetstrokecolor{currentstroke}%
\pgfsetdash{}{0pt}%
\pgfpathmoveto{\pgfqpoint{0.900000in}{1.575000in}}%
\pgfpathlineto{\pgfqpoint{4.275000in}{1.575000in}}%
\pgfusepath{stroke}%
\end{pgfscope}%
\begin{pgfscope}%
\pgfpathrectangle{\pgfqpoint{0.900000in}{0.600000in}}{\pgfqpoint{3.375000in}{1.950000in}} %
\pgfusepath{clip}%
\pgfsetrectcap%
\pgfsetroundjoin%
\pgfsetlinewidth{1.003750pt}%
\definecolor{currentstroke}{rgb}{0.000000,0.000000,0.000000}%
\pgfsetstrokecolor{currentstroke}%
\pgfsetdash{}{0pt}%
\pgfpathmoveto{\pgfqpoint{2.596875in}{0.600000in}}%
\pgfpathlineto{\pgfqpoint{2.596875in}{2.550000in}}%
\pgfusepath{stroke}%
\end{pgfscope}%
\begin{pgfscope}%
\pgfsetrectcap%
\pgfsetmiterjoin%
\pgfsetlinewidth{1.003750pt}%
\definecolor{currentstroke}{rgb}{0.000000,0.000000,0.000000}%
\pgfsetstrokecolor{currentstroke}%
\pgfsetdash{}{0pt}%
\pgfpathmoveto{\pgfqpoint{0.900000in}{0.600000in}}%
\pgfpathlineto{\pgfqpoint{4.275000in}{0.600000in}}%
\pgfusepath{stroke}%
\end{pgfscope}%
\begin{pgfscope}%
\pgfsetrectcap%
\pgfsetmiterjoin%
\pgfsetlinewidth{1.003750pt}%
\definecolor{currentstroke}{rgb}{0.000000,0.000000,0.000000}%
\pgfsetstrokecolor{currentstroke}%
\pgfsetdash{}{0pt}%
\pgfpathmoveto{\pgfqpoint{0.900000in}{2.550000in}}%
\pgfpathlineto{\pgfqpoint{4.275000in}{2.550000in}}%
\pgfusepath{stroke}%
\end{pgfscope}%
\begin{pgfscope}%
\pgfsetrectcap%
\pgfsetmiterjoin%
\pgfsetlinewidth{1.003750pt}%
\definecolor{currentstroke}{rgb}{0.000000,0.000000,0.000000}%
\pgfsetstrokecolor{currentstroke}%
\pgfsetdash{}{0pt}%
\pgfpathmoveto{\pgfqpoint{4.275000in}{0.600000in}}%
\pgfpathlineto{\pgfqpoint{4.275000in}{2.550000in}}%
\pgfusepath{stroke}%
\end{pgfscope}%
\begin{pgfscope}%
\pgfsetrectcap%
\pgfsetmiterjoin%
\pgfsetlinewidth{1.003750pt}%
\definecolor{currentstroke}{rgb}{0.000000,0.000000,0.000000}%
\pgfsetstrokecolor{currentstroke}%
\pgfsetdash{}{0pt}%
\pgfpathmoveto{\pgfqpoint{0.900000in}{0.600000in}}%
\pgfpathlineto{\pgfqpoint{0.900000in}{2.550000in}}%
\pgfusepath{stroke}%
\end{pgfscope}%
\begin{pgfscope}%
\pgftext[x=2.587500in,y=0.530556in,,top]{\rmfamily\fontsize{9.000000}{10.800000}\selectfont Spannung (V)}%
\end{pgfscope}%
\begin{pgfscope}%
\pgftext[x=0.830556in,y=1.575000in,,bottom,rotate=90.000000]{\rmfamily\fontsize{9.000000}{10.800000}\selectfont Strom (A)}%
\end{pgfscope}%
\begin{pgfscope}%
\pgftext[x=2.025000in,y=1.623750in,left,base]{\rmfamily\fontsize{9.000000}{10.800000}\selectfont \(\displaystyle I_S\)}%
\end{pgfscope}%
\begin{pgfscope}%
\pgftext[x=2.587500in,y=2.619444in,,base]{\rmfamily\fontsize{11.000000}{13.200000}\selectfont IV-Kurve einer Diode}%
\end{pgfscope}%
\end{pgfpicture}%
\makeatother%
\endgroup%

    \caption{%
        Vereinfachte Strom-Spannungs-Kurve einer Diode. Der Reverse Saturation
        Current   $I_{\mathrm{S}}$   tritt   im   \textcolor{magenta}{magenta}
        eingef\"arbten Bereich auf. Der angegebene Zahlenwert bezieht sich auf
        den  Bereich  der  Kurve,  in der  $I_{\mathrm{S}}$  relativ  konstant
        ist.\protect\\
        Die    \textcolor{blue}{blaue   Kurve}    dient   als    Referenz. Die
        \textcolor{red}{rote  Kurve}  zeigt  den Einfluss  eines  ansteigenden
        Idealit\"atsfaktors  (also  st\"arkeres  Abweichen von  einer  idealen
        Diode) auf die IV-Kennlinie.\protect\\
        Die \textcolor{green!50!black}{gr\"une Kurve} zeigt den Einfluss eines
        ansteigenden  Reverse Saturation  Current: Der Diodenstrom  steigt bei
        einer gegebenen Diodenspannung.\protect\\
        Zur  Verbesserung  der  \"Ubersichtlichkeit  wurden  die  abweichenden
        Kurven nur  im positiven Bereich geplottet,  nat\"urlich \"andert sich
        auch das Verhalten im Reverse-Betrieb analog.%
    }
    \label{fig:diodeVI:IS}
\end{figure}

\todo{Reverse-Betrieb-Kurven: analog?}

Zur  Simulation  in  \code{LTspice}  wird das  Diodenmodell  auf  den  Reverse
Saturation  Current   und  den  Idealit\"atsfaktor  reduziert,   und  folgende
gesch\"atzte Parameter als Ausgangslage f\"ur den Iterationsprozess benutzt:

\begin{center}
    \code{.model diode1 D(IS=1e-6 N=2)}
\end{center}

Es wird die Schaltung gem\"ass Abbildung \ref{fig:circuit:solarCell} von Seite
\pageref{fig:circuit:solarCell} aufgebaut und die  oben bestimmten Werte f\"ur
$R_{\mathrm{S}}$, $R_{\mathrm{P}}$ und $C$ eingesetzt.

Mit einer  Transientensimulation (\code{.tran  1m}) wird die  Zelle simuliert,
ihre  Leerlaufspannung  $V_{\mathrm{offen}}$  gemessen und  mit  dem  Zielwert
aus   Gleichung~\ref{eq:voffen}  verglichen\todo{``wird''   nur  einmal   oder
wiederholen?}.  Anschliessend  werden die  Werte f\"ur \code{IS}  und \code{N}
angepasst, bis die gew\"unschte Leerlaufspannung erreicht ist.

Nach einigen Iterationen liefert dieser Prozess\footnotemark:

\footnotetext{%
    Dies ist eine m\"ogliche L\"osung. Es gibt nat\"urlich noch beliebig viele
    weitere  Kombinationen von  \code{IS} und  \code{N}, welche  die gegebenen
    Bedingungen  erf\"ullen. Wir   sind  hier  lediglich  an   einer  L\"osung
    interessiert, von  der wir zuversichtlich  sind, dass sie  ein hinreichend
    gutes Modell liefert.%
}

\begin{align}
    \label{eq:cell:diode:IS:N:result}
    I_s &= \SI{4}{\micro\ampere} \\
    N   &= 1.65
\end{align}

Somit ist  das Modell  einer einzelnen  Zelle bestimmt  und kann  dazu benutzt
werden,  ein   Modul  aufzubauen.


% ---------------------------------------------------------------------------- %
\section{Modellierung eines PV-Moduls}
\label{sec:simu:model:module}
% ---------------------------------------------------------------------------- %

\todo{referenz auf vereinfachtes Modell}

Wie  in   Anhang  \ref{app:simu:module}  ab   Seite  \pageref{app:simu:module}
erw\"ahnt,     werden     zwei      verschiedene     Module     simuliert: Ein
Modul    mit    zwei    parallelen     Str\"angen    zu    je    36    seriell
geschalteten    Zellen     (\fref{fig:ltspice:module:cellBased:36x2},    Seite
\pageref{fig:ltspice:module:cellBased:36x2})    und   ein    Modul   mit    72
in   Serie   geschalteten  Zellen   (\fref{fig:ltspice:module:cellBased:72x1},
Seite     \pageref{fig:ltspice:module:cellBased:72x1}). Die     $36     \times
2$-Konfiguration\todo{Bindestrich?}   hat   eine  h\"ohere   Kapazit\"at   als
die   $72  \times   1$-Anordnung,   liefert  daf\"ur   aber   nur  die   halbe
Ausgangsspannung. Der Ausgangsstrom  beider Module wurde  auf \SI{10}{\ampere}
festgelegt,  die Stromquellen  im  $36 \times  2$-Modul  geben also  lediglich
\SI{5}{\ampere}  ab. Basierend auf  Tabelle \ref{tab:moduleData:IU}  von Seite
\pageref{tab:moduleData:IU}  mit Informationen  zu kommerziell  erh\"altlichen
Solarmodulen scheint uns dieser Ansatz gerechtfertigt.

\fref{fig:ltspice:solarCell}   zeigt   die  Implementation   unseres   Modells
einer     Solarzelle      gem\"ass     \fref{fig:circuit:solarCell}     (Seite
\pageref{fig:circuit:solarCell} in Abschnitt \ref{subsubsec:hw:ask:modell}) in
\code{LTspice}.

Ein MOSFET  wird benutzt, um  die Zelle  gesteuert (in unserer  Simulation bei
einer Tr\"agerfrequenz von \SI{10}{\kilo\hertz}) kurzschliessen zu k\"onnen.

Es    soll   das    f\"ur   den    MOSFET   schlimmere    Szenario   evaluiert
werden   (mehr    Strom/Leistung   durch    den   MOSFET\todo{Korrekt?}). Dazu
werden   aus  72   Zellen  zwei   verschiedene  Module   aufgebaut: Ein  Modul
\"ahnlich    zum    Sunset   Solargenerator    AS150    \cite{ref:solar:as150}
mit    ca. \SI{10}{\ampere}   Kurzschlussstrom    und   etwa    \SI{22}{\volt}
Leerlaufspannung  (\fref{fig:ltspice:module:cellBased:36x2}),  und  ein  Modul
\"ahnlich  zum Sunmodule  Pro-Series  XL SWB320  \cite{ref:solar:sunmodulePro}
mit   \SI{10}{\ampere}   Kurzschlussstrom    und   ungef\"ahr   \SI{45}{\volt}
Leerlaufspannung (\fref{fig:ltspice:module:cellBased:72x1}).

Weil die gesamte Kapazit\"at bei in Serie geschalteten Kondensatoren sinkt und
bei  paralleler  Anordnung steigt,  hat  die  $2 \times  36$-Parallelschaltung
\todo{Bindestrich und Spaces?} etwa die doppelte Kapazit\"at des Einzelstrangs
aus 72 Zellen. Der  durch den MOSFET fliessende Stroms, die  \"uber den MOSFET
abfallende Spannung und die im  MOSFET dissipierte Leistung beim Durchschalten
des Transistors sollen zwischen den beiden Szenarien verglichen werden.

Die zu  diesen Schaltungen geh\"orenden \code{.asc}-Dateien  sind elektronisch
verf\"ugbar  (Datentr\"ager  siehe Anhang  \ref{app:electronicStorage},  Seite
\pageref{app:electronicStorage}).

\todo{kleines Bild der gesamten Schaltung?}


% ---------------------------------------------------------------------------- %
\section{Frequenzumtastung: Kapazitive Einkopplung}
\label{sec:simu:fsk:capacitive}
% ---------------------------------------------------------------------------- %

% ---------------------------------------------------------------------------- %
\subsection{Sender}
\label{sec:simu:fsk:capacitive:transmitter}
% ---------------------------------------------------------------------------- %

% ---------------------------------------------------------------------------- %
\subsection{Empf\"anger}
\label{sec:simu:fsk:capacitive:receiver}
% ---------------------------------------------------------------------------- %

% ---------------------------------------------------------------------------- %
\subsection{Gesamtsystem}
\label{sec:simu:fsk:capacitive:system}
% ---------------------------------------------------------------------------- %

% ---------------------------------------------------------------------------- %
\section{Frequenzumtastung: Induktive Einkopplung}
\label{sec:simu:fsk:inductive}
% ---------------------------------------------------------------------------- %

Eine induktive Einkopplung legt eine  Spule um die DC-Leitung. Auf diese Spule
wird von der FSK-Schaltung das zu  \"ubertragende Signal gegeben und die Spule
induziert in  der DC-Leitung  entsprechende Spannungs-Rippel, die  vom \Master
ausgewertet werden k\"onnen. Der entsprechende  Schaltkreis ist schematisch in
Abbildung \ref{fig:circ:coupling:inductive} dargestellt.

Verglichen mit  Kondensatoren sind Spulen relativ  gross und teuer. Allerdings
ist das Prinzip  der induktiven Einkopplung gut dokumentiert  und die Aussicht
auf Erfolg (bei vern\"unftigem Aufwand) somit gut.

\begin{figure}[h!tb]
    \centering
    \begin{circuitikz}
    \draw
    (-1,0) to[empty photodiode,o-,l_=PV-Modul] (1,0) to[short] (6,0)
    %(2,-2) to[short,o-] (2,-0.05) to[L=L] (4,-0.05) to[short,-o] (4,-2)
    (2,-2) -- (2,-0.05) to[L,l^=Kopplung] (4,-0.05) -- (4,-2) to[sinusoidal voltage source,l^=$U_{\mathrm{Signal}}$] (2,-2)
    ;
\end{circuitikz}

    \caption{Induktive Einkopplung}
    \label{fig:circ:coupling:inductive}
\end{figure}

% ---------------------------------------------------------------------------- %
\subsection{Sender}
\label{sec:simu:fsk:inductive:transmitter}
% ---------------------------------------------------------------------------- %

% ---------------------------------------------------------------------------- %
\subsection{Empf\"anger}
\label{sec:simu:fsk:inductive:receiver}
% ---------------------------------------------------------------------------- %

% ---------------------------------------------------------------------------- %
\subsection{Gesamtsystem}
\label{sec:simu:fsk:inductive:inductive}
% ---------------------------------------------------------------------------- %


% ---------------------------------------------------------------------------- %
\section{Amplitudenumtastung}
\label{sec:simu:ask}
% ---------------------------------------------------------------------------- %

Wie in Abschnitt \todo{reference} erw\"ahnt, wird bei dieser L\"osungsvariante
jeweils   ein   Modul   gesteuert   kurzgeschlossen. Dies   verursacht   kurze
Spannungseinbr\"uche auf  der DC-Leitung,  welche vom  Empf\"anger ausgewertet
werden k\"onnen, wie in Abbildung \todo{reference}vereinfact dargestellt.

Potentielle Probleme sind bei  dieser L\"osungsvariante in folgenden Bereichen
zu suchen:

\begin{symbols}
    \firmlist
    \item[\textbf{Induktivit\"at der Leitung:}]
        Der  Spannungsabfall   auf  der  DC-Leitung  wird   bei  geschlossenem
        Stromkreis zu Strom\"anderungen auf der DC-Leitung f\"uhren. Dies wird
        eine Spannungs\"anderung auf der DC-Leitung bewirken\footnotemark.
        \footnotetext{%
            Spannung in Abh\"angigkeit der Strom\"anderung:
            $v = L \cdot \frac{\mathrm{d}i}{\mathrm{d}t}$%
        }

        Gem\"ass    Lenz'scher     Regel    \todo{reference}     wird    diese
        Spannungs\"anderung  so gerichtet  sein,  dass  sich der  zugeh\"orige
        Strom der aufgezwungenen \"Anderung widersetzt.

        Es  kann also  sein, dass  die  Induktivit\"at der  DC-Leitung das  zu
        \"ubertragende  Signal  so  stark  kompensiert,  dass  es  nicht  mehr
        detektierbar  ist. Je   h\"oher  die  Frequenz,  mit   der  das  Modul
        kurzgeschlossen  und  wieder  ge\"offnet   wird,  um  so  h\"oher  die
        zugeh\"origen Strom\"anderungen
        $\frac{\mathrm{d}i}{\mathrm{d}{t}}$
        und somit Impedanz der Induktivit\"at (gem\"ass
        $Z_L = j \omega L$).
    \item[\textbf{Kapazit\"at des Solarmoduls:}]
        Dem  Solarmodul   wird  bei   diesem  Vorgehen  das   Verhalten  einer
        Wechselstromquelle   aufgezwungen. Besitzt  es   interne  parasit\"are
        Kapazit\"aten,  k\"onnen  diese  bei  den  abrupten  \"Anderungen  der
        Spannung hohe Str\"ome im  kurzgeschlossenen Pfad und seinen Bauteilen
        verursachen.\footnotemark
        \footnotetext{%
            Strom\"anderung in Abh\"angigkeit der Spannungs\"anderung:
            $I(t) = C \cdot \frac{\mathrm{V(t)}}{\mathrm{d}t}$%
        }

        Besitzen    diese   Bauteile    Ohm'sche   Widerst\"ande,    entstehen
        entsprechende thermische  Verluste, welche die  Bauteile besch\"adigen
        k\"onnen.
\end{symbols}

Die    Tr\"agerfrequenz    in    den    folgenden    Simulationen    betr\"agt
\SI{10}{\kilo\hertz}.


% ---------------------------------------------------------------------------- %
\subsection{Sender}
\label{subsec:simu:ask:sensor}
% ---------------------------------------------------------------------------- %

Das  gesteuerte Kurzschliessen  des Moduls  wird mit  einem MOSFET  umgesetzt,
welcher zwischen  Eingang und  Ausgang des Moduls  durchschalten kann  und vom
Microcontroller auf dem Sensor gesteuert wird. \fref{fig:module:mosfet:simple}
zeigt diesen Aufbau schematisch.

Die  Ansteuerung des  Transistors  erfolgt mit  \SI{3.3}{\volt},  da dies  die
maximale Spannung  ist, welche der  auf dem Sensor  platzierte Microcontroller
ausgeben kann.

\begin{figure}[h!tb]
    \centering
    % Pro memoriam:
%
%            |  D
%      | |---+
%      |
%      | |<--+  B
%      |     |
% G ---+ |---+
%            |  S
%(mos.B) node[anchor=west] {B}
%(mos.G) node[anchor=east] {G}
%(mos.D) node[anchor=north] {D}
%(mos.S) node[anchor=south] {S}

\begin{circuitikz}
    \small
    \draw
    (4.5,1) node[nigfete] (mos) {MOSFET}

    (0,0) to[empty photodiode,l_=PV-Modul] (0,2) -- (4.5,2) -- (mos.D)
    %(0,0) to[dcisource,l_=PV-Modul] (0,4) -- (8,4) -- (mos.D)
    (mos.S) -- (4.5,0) -- (0,0)

    (2.25,0.725) to[sinusoidal voltage source,l^=Controller] (mos.G)
    (2.25,0.725) -- (2,0.725) node[sground] {}
    ;
\end{circuitikz}

    \caption{%
        Gesteuerter     Kurzschluss     eines    Solarmoduls     mit     einem
        microcontroller-gesteuerten       Transistor. Die      vollst\"andigen
        \code{LTspice}-Schaltungen  sind  in Anhang  \ref{app:simu:module}  ab
        Seite \pageref{app:simu:module} dokumentiert.%
    }
    \label{fig:module:mosfet:simple}
\end{figure}

%TODO: line width
\begin{figure}
    \begin{tikzpicture}
       \begin{scope}[x={(0mm,0mm)},y={(120mm,\textwidth)}]
           \begin{axis}[%
                   height=40mm,
                   width=\textwidth,
                   at={(0,70mm)},
                   grid=both,
                   xlabel=Zeit (\si{\micro\second}),
                   ylabel=Strom (\si{\ampere}),
                   %x unit=u,
                   change x base=true,
                   %line width = 1pt,
                   thick,
                   x SI prefix=micro,
               ]
               \addplot[-,color=blue] table {data/module-72cells--I-MOSFET--0005u.dat};
           \end{axis}
           \begin{axis}[%
                   height=40mm,
                   width=\textwidth,
                   at={(0,35mm)},
                   grid=both,
                   xlabel=Zeit (\si{\micro\second}),
                   ylabel=Spannung (\si{\volt}),
                   %x unit=u,
                   change x base=true,
                   x SI prefix=micro,
               ]
               \addplot[-,color=magenta] table {data/module-72cells--U-MOSFET--0005u.dat};
           \end{axis}
           \begin{axis}[%
                   height=40mm,
                   width=\textwidth,
                   at={(0,0)},
                   grid=both,
                   xlabel=Zeit (\si{\micro\second}),
                   ylabel=Leistung (\si{\watt}),
                   %x unit=u,
                   change x base=true,
                   x SI prefix=micro,
               ]
               \addplot[-,color=teal] table {data/module-72cells--P-MOSFET--0005u.dat};
           \end{axis}
       \end{scope}
   \end{tikzpicture}
   \caption{%
       Verlauf    von    Strom,    Spannung    und    Leistung    am    MOSFET
       bei   einer   Schaltfrequenz   von   \SI{10}{\kilo\hertz}   bei   einer
       Modulkonfiguration    von    $36     \times    2$    Zellen    gem\"ass
       Schema    in    \fref{fig:ltspice:module:cellBased:36x2}   auf    Seite
       \ref{fig:ltspice:module:cellBased:36x2}.\protect\\
       Anhang       \ref{app:sec:simu:complementary:36x2}      auf       Seite
       \pageref{app:sec:simu:complementary:36x2} beinhaltet zum Vergleich noch
       Simulationen f\"ur den Zeitraum von einer Millisekunde.%
   }
   \label{fig:simu:results:36x2:3u}
\end{figure}

\begin{figure}
    \begin{tikzpicture}
       \begin{scope}[x={(0mm,0mm)},y={(120mm,\textwidth)}]
           \begin{axis}[%
                   height=40mm,
                   width=\textwidth,
                   at={(0,70mm)},
                   grid=both,
                   xlabel=Zeit (\si{\micro\second}),
                   ylabel=Strom (\si{\ampere}),
                   %x unit=u,
                   change x base=true,
                   x SI prefix=micro,
               ]
               \addplot[-,color=blue] table {data/module-72cells-series--I-MOSFET--0003u.dat};
           \end{axis}
           \begin{axis}[%
                   height=40mm,
                   width=\textwidth,
                   at={(0,35mm)},
                   grid=both,
                   xlabel=Zeit (\si{\micro\second}),
                   ylabel=Spannung (\si{\volt}),
                   %x unit=u,
                   change x base=true,
                   x SI prefix=micro,
               ]
               \addplot[-,color=magenta] table {data/module-72cells-series--U-MOSFET--0003u.dat};
           \end{axis}
           \begin{axis}[%
                   height=40mm,
                   width=\textwidth,
                   at={(0,0)},
                   grid=both,
                   xlabel=Zeit (\si{\micro\second}),
                   ylabel=Leistung (\si{\watt}),
                   %x unit=u,
                   change x base=true,
                   x SI prefix=micro,
               ]
               \addplot[-,color=teal] table {data/module-72cells-series--P-MOSFET--0003u.dat};
           \end{axis}
       \end{scope}
   \end{tikzpicture}
   \caption{%
       Verlauf    von    Strom,    Spannung    und    Leistung    am    MOSFET
       bei   einer   Schaltfrequenz   von   \SI{10}{\kilo\hertz}   bei   einer
       Modulkonfiguration    von    $72     \times    1$    Zellen    gem\"ass
       Schema    in    \fref{fig:ltspice:module:cellBased:36x2}   auf    Seite
       \ref{fig:ltspice:module:cellBased:72x1}. \protect\\
       Anhang     \ref{app:sec:simu:complementary:72x1}     auf     Seite
       \pageref{app:sec:simu:complementary:72x1} beinhaltet zum Vergleich noch
       Simulationen f\"ur den Zeitraum von einer Millisekunde.%
   }
   \label{fig:simu:results:72x1:3u}
\end{figure}

Die    Resultate   der    Simulation   f\"ur    ein   $36    \times   2$-Modul
sind   in   \fref{fig:simu:results:36x2:3u}   f\"ur  einen   Zeitbereich   von
\SI{3}{\micro\second}  dargestellt,  die  Ergebnisse   f\"ur  das  $72  \times
1$-Modul in \fref{fig:simu:results:72x1:3u}.  Tabelle \ref{tab:36x2:72x1:heat}
fasst   die  wichtigsten   Eckdaten  der   Simulationen  zusammen,   inklusive
Durchschnittswerte f\"ur Strom und Leistung.

\begin{table}[h!tb]
    \centering
    \caption{%
        Eckdaten zur Simulation des Kurzschlussverfahrens f\"ur ein Solarpanel
        mit  $36   \times  2$  Zellen  und   ein  Panel  mit  $72   \times  1$
        Zellen.\protect\\
        Bei     den     Durchschnittswerten     sind     sowohl     Ergebnisse
        f\"ur      \SI{1}{\milli\second}     wie      auch     \SI{1}{\second}
        angegeben,    um   zu    zeigen,   dass    sich   die    Konfiguration
        bereits    bei    einer    Millisekunde   stabilisiert    hat    (auch
        zu     sehen    in     Anhang    \ref{app:sec:simu:complementary:36x2}
        auf      Seite      \pageref{app:sec:simu:complementary:36x2}      und
        Anhang      \ref{app:sec:simu:complementary:72x1}       auf      Seite
        \pageref{app:sec:simu:complementary:72x1}).\protect\\
        Die     Dauer      der     Spitze     ist     beim      $36     \times
        2$-Modul     \SI{2}{\micro\second}    und     \SI{1.78}{\micro\second}
        beim    $72    \times    1$-Modul    (siehe    unterste    Plots    in
        den     Abbildungen      \ref{fig:simu:results:36x2:3u}     respektive
        \ref{fig:simu:results:72x1:3u}).%
    }
    \label{tab:36x2:72x1:heat}
    \begin{tabular}{lrr}

    \toprule
    Kriterium                                      & $36 \times 2$-Modul & $72 \times 1$-Modul \\
    \midrule
     Spitzenwert Strom:                            & \SI{32}{\ampere}    & \SI{24}{\ampere}    \\
     Spitzenwert Leistung:                         & \SI{310}{\watt}     & \SI{535}{\watt}     \\
     $\overline{P}$ f\"ur Dauer der Spitze         & \SI{158.77}{\watt}  & \SI{267.46}{\watt}  \\
     $\overline{P}$ f\"ur \SI{1}{\milli\second}    & \SI{4.312}{\watt}   & \SI{6.0577}{\watt}  \\
     $\overline{P}$ f\"ur \SI{1}{\second}          & \SI{4.3047}{\watt}  & \SI{6.0433}{\watt}  \\
    \bottomrule
    \end{tabular}
\end{table}

Aus \fref{fig:simu:results:36x2:5u}  und \tref{tab:36x2:72x1:heat}  ziehen wir
folgende Schlussfolgerungen:

\begin{enumerate}
    \firmlist
    \item
        Der  Spitzenwert f\"ur  den  Strom ist  hoch,  aber Transistoren,  die
        solche Str\"ome  verkraften k\"onnen,  sind zu  vern\"unftigen Preisen
        erh\"altlich.
    \item
        Die    Spitzenwerte    der    Leistungen   zwar    nur    sehr    kurz
        (etwa   \SI{2}{\micro\second}),   aber   sehr  hoch. Auch   wenn   der
        Durchschnittliche Leistungswert  weit unter  der Grenze  von g\"unstig
        erh\"altlichen  MOSFETs   liegt,  k\"onnte  die   Leistungsspitze  den
        Transistor irreversibel sch\"adigen.
    \item
        Der  durchschnittliche  Leistungsverbrauch  ist  weit  \"uber  dem  im
        Pflichtenheft angestrebten Wert von \SI{100}{\milli\watt}.
\end{enumerate}
\todo{korrekte Schlussfolgerungen?}


% ---------------------------------------------------------------------------- %
\clearpage
\subsection{\Master (Empf\"anger)}
\label{subsec:simu:ask:recv}
% ---------------------------------------------------------------------------- %


% ---------------------------------------------------------------------------- %
\subsection{Gesamtsystem}
\label{subsec:simu:ask:total}
% ---------------------------------------------------------------------------- %
