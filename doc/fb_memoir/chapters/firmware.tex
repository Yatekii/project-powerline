% **************************************************************************** %
\chapter{Software}
\label{chap:software}
% **************************************************************************** %


An  dieser  Stelle wird  genauer  erl\"autert,  welche \"Uberlegungen  in  die
Firmware  unseres  Systems  geflossen  sind. Weil  Sensor  und  Master-Ger\"at
jeweils  eine   eigene  Firmware   ben\"otigen,  ist  auch   dieser  Abschnitt
entsprechend aufgeteilt.

\anweisung   Diagramme    zur   Erkl\"arung    des   Aufbaus    der   Firmware
verwenden. Trockener    Text    \"uber    Software   ist    nicht    besonders
leserfreundlich.

\anweisung Bei der Verwendung von  Libraries sollen diese kurz beschrieben und
der Grund ihrer Wahl erl\"autert werden.

\anweisung   Unterkapitel  entsprechend   Gliederung  der   Firmware  sinnvoll
erg\"anzen.


% ---------------------------------------------------------------------------- %
\section{Datenfluss}
\label{sec:fw:dataFlow}
% ---------------------------------------------------------------------------- %


% ---------------------------------------------------------------------------- %
\section{Firmware \Sensor}
\label{sec:firmware:sensor}
% ---------------------------------------------------------------------------- %


% ---------------------------------------------------------------------------- %
\section{Software \Master}
\label{sec:software:master}
% ---------------------------------------------------------------------------- %

Es wird  im Folgenden auf  die Software des \Master~  genauer eingegangen. Die
verwendeten  Komponenten  werden beschrieben  und  es  wird auf  die  Lizenzen
ebendieser  Komponenten eingegangen. Anschliessend  werden der  Aufbau unseres
Software-Stacks und die Funktionsprinzipien dokumentiert.


% ---------------------------------------------------------------------------- %
\subsection{Komponenten}
\label{subsec:software:master:components}
% ---------------------------------------------------------------------------- %

Als   Betriebssystem   kommt   Raspbian   zum   Einsatz. Raspbian   ist   eine
Linux-Distribution,  welche auf  Debian  \todo{link auf  website} aufbaut  und
Erweiterungen f\"ur  zus\"atzliche Display-Funktionalit\"at f\"ur  den \Raspi~
besitzt.

Unsere eigene Software basiert auf  Python3 und QT5. In Python3 stehen diverse
Bibliotheken  \todo{welche?} zur  Verf\"ugung,  welche  es uns  erm\"oglichen,
das  Rad  nicht   neu  erfinden  zu  m\"ussen. Sattdessen   k\"onnen  wir  uns
darauf  konzentrieren, die  eigentliche Funktionalit\"at  unseres Ger\"ats  zu
implementieren.

\todo{layering diagramm}

\todo{layering wird hier integriert, kein separater abschnitt}

 - Grundlage für die Software bildet das angepasste Betriebssystem.
 - Darauf aufgebaut sind Programmbibliotheken zur Abstraktion von Betriebsystemfunktionen und weiteren Hardwareaufrufen.
 - Diese werden in der Master-Software eingebunden und mittels Python-API angesprochen.


% ---------------------------------------------------------------------------- %
\subsection{Lizenzen}
\label{subsec:software:master:licenses}
% ---------------------------------------------------------------------------- %

Unterschiedliche konditionen zur Verwendung der Software
 - GPL für Linux kernel
 - Closed Source Firmware Blobs für rpi und display
 - DFSG für alle weitere Software, da von Debian kontrolliert

In Ordung für kommerzielle Zwecke, darf kein Geld verlangt werden
Eigener Quellcode nicht öffentlich und kann als Bestandteil des Produkts verkauft werden.


% ---------------------------------------------------------------------------- %
\subsection{Threads}
\label{subsec:software:master:threads}
% ---------------------------------------------------------------------------- %

 - Hauptthread initialisiert Datenbank und startet Threads, unter welchen die eigentlichen Funktionen aufgeteilt sind.
 - Unterthreads werden mittels Semaphoren vom Hauptthread koordiniert. Dies gewährleistet, dass Arbeitsschritte nur mit aktuellen Daten durchgeführt werden.
 - Geteilte Ressourcen wie die Datenbank und Logging-Funktionalität sind threadsafe.
    - Logging verwendet eine Queue um Anfragen zu speichern und hintereinander abzuarbeiten.
    - Die verwendete Datenbankbibliothek beinhaltet Funkionalität um aus mehreren Threads parallel aufgerufen zu werden. Dazu werden die Anfragen aus jeweils einem Thread gruppiert und atomar ausgeführt.
 - Alle anderen externen Ressourcen, dies sind I2C-bus, UART und GPIO, werden jeweils von nur einem Thread genutzt, wodurch keine Konflikte auftreten können.


% ---------------------------------------------------------------------------- %
\subsection{Aufgabenteilung}
\label{subsec:software:master:taskSeparation}
% ---------------------------------------------------------------------------- %

 - Datensammlung: Empfangen von Spannungs- und Strommesswerte von den Sensoren und abspeichern in der Datenbank.
 - Datenauswertung: Untersuchen der gespeicherten Messwerte auf Defekte Panels
 - Benutzerinterface
   - Graphisch (Display / Touchscreen): Konfigurieren der Einstellungen von Alarmierung und Telefonnummer.
   - Modem (SMS): Versand von SMS beim Eintritt von vordefinierten Ereignissen.
   - Elektrisch (Relais / Tasten): Schalten der eingebauten Relais
 - Datenbankverwaltung: Initialisierung der Datenbank und Vorbereiten der Datenbankverbindung
 - Prozesssteuerung: Verwaltung der einzelnen Threads und ihrem Zustand


% ---------------------------------------------------------------------------- %
\subsection{Benutzeroberfl\"ache}
\label{subsec:software:master:GUI}
% ---------------------------------------------------------------------------- %

% ---------------------------------------------------------------------------- %
\subsection{Datenbank}
\label{subsec:software:master:database}
% ---------------------------------------------------------------------------- %

% ---------------------------------------------------------------------------- %
\subsection{Funktionen}
\label{subsec:software:master:functions}
% ---------------------------------------------------------------------------- %
