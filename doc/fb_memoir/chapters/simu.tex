% **************************************************************************** %
\chapter{L\"osungsans\"atze und Simulationen}
\label{chap:simu}
% **************************************************************************** %

Bevor  simuliert  werden  kann,  m\"ussen  die  daf\"ur  ben\"otigten  Modelle
vorhanden sein. Es  wird zuerst ein  Modell f\"ur eine  Solarzelle entwickelt.
Anschliessend wird dieses Zellenmodell benutzt, um ein Modul aufzubauen.

In  diesem Abschnitt  werden Schaltungen  f\"ur drei  L\"osungsans\"atze (zwei
zur  FSK, eine  zur  ASK)  vorgestellt, mit  \code{LTspice  IV} simuliert  und
beurteilt.\todo{LTspice: source}

Es werden  jeweils die  Schaltung f\"ur  den Sender,  den Empf\"anger  und das
Gesamtsystem untersucht.


% ---------------------------------------------------------------------------- %
\section{Modellierung einer PV-Zelle}
\label{sec:simu:model:cell}
% ---------------------------------------------------------------------------- %


Ziel ist, eine Zelle zu modellieren, welche folgende Eigenschaften erf\"ullt:

\begin{align}
    U_{\mathrm{Leerlauf}}                    &= \SI{600}{\milli\volt} \\
    I_{\mathrm{Kurzschluss, polykristallin}} &=  \SI{5}{\ampere}      \\
    I_{\mathrm{Kurzschluss, monokristallin}} &= \SI{10}{\ampere}
\end{align}

Die Herleitung  dieser Werte ist  in Anhang \ref{app:subsec:cell:UI}  ab Seite
\pageref{app:subsec:cell:UI}  zu finden  und  basiert auf  den Modulen,  deren
Daten in Tabelle \ref{tab:moduleData:IU} auf Seite \pageref{tab:moduleData:IU}
aufgelistet  sind.   Es  werden  zwei  Zielwerte  f\"ur  den  Kurzschlussstrom
angegeben;  einer f\"ur  monokristalline Zellen,  einer f\"ur  polykristalline
Zellen.

Als  Grundlage f\"ur  das  Modell dient  das  Eindiodenmodell einer  PV-Zelle,
erweitert  um  eine parallele  Kapazit\"at  $C$  (basierend auf  Informationen
aus  \cite{ref:solar:scofield} und  \cite{ref:solar:friesen}), dargestellt  in
\fref{fig:circuit:solarCell}.

\clearpage
\begin{figure}[h!tb]
    \centering
    \begin{circuitikz}
    \draw
    % Source to top terminal
    (0,0) to[current source,i=$I_{\mathrm{Zelle}}$] (0,4) -- (5,4) to[R,l^=$R_{\mathrm{S}}$] (10,4) node[ocirc] {}

    % Open terminal at bottom
    (10,0) node[ocirc] {} -- (0,0)

    % Parallel elements
    (1.5,4) to[empty diode,*-*,l_=$D$] (1.5,0)
    (3,4) to[C,*-*,l_=$C$] (3,0)
    (4.5,4) to[R,*-*,l_=$R_{\mathrm{P}}$] (4.5,0)
    ;
\end{circuitikz}

    \caption{%
        Schaltschema    zur    Modellierung    einer    Solarzelle    gem\"ass
        Eindiodenmodell mit zus\"atzlicher Kapazit\"at%
    }
    \label{fig:circuit:solarCell}
\end{figure}

Bei diesem Modell sind insgesamt f\"unf Parameter zu bestimmen:

\begin{itemize}
    \firmlist
    \item
        Seriewiderstand $R_{\mathrm{S}}$
    \item
        Shunt-Widerstand $R_{\mathrm{P}}$
    \item
        Parasit\"are Kapazit\"at $C$
    \item
        Diode   $D$: Reverse    Saturation   Current    $I_{\mathrm{S}}$   und
        Idealit\"atsfaktor $n$
\end{itemize}

Es werden zuerst die Kapazit\"at und die Widerst\"ande bestimmt.

In \cite{ref:solar:scofield} wurden $C$, $R_{\mathrm{S}}$ und $R_{\mathrm{P}}$
einer  Solarzelle der  Gr\"osse  \SI{0.43}{\centi\meter\squared} \"uber  einen
Frequenzbereicht von \SI{1}{\kilo\hertz} bis \SI{1}{\mega\hertz} gemessen. Die
Resultate waren:

\begin{alignat}{3}
    \label{eq:scofield:C}
    C_{\mathrm{Messung}}    &= \SI{8}{\nano\farad} \; & \text{bis} \; & \SI{20}{\nano\farad} \;  & = & \; \SI{14 \pm 6}{\nano\farad} \\
    \label{eq:scofield:Rs}
    R_{\mathrm{S, Messung}} &= \SI{0.2}{\ohm}      \; & \text{bis} \; & \SI{20}{\ohm}        \;  & = & \; \SI{10.1 \pm 9.9}{\ohm}     \\
    \label{eq:scofield:Rp}
    R_{\mathrm{P, Messung}} &= \SI{0.5}{\kilo\ohm} \; & \text{bis} \; & \SI{500}{\kilo\ohm}  \;  & = & \; \SI{250.25 \pm 249.75}{\kilo\ohm}
\end{alignat}

Um   die   obigen  Werte   f\"ur   unsere   Zwecke  verwenden   zu   k\"onnen,
m\"ussen  sie  auf  eine  Zelle  der  Fl\"ache  \SI{240}{\centi\meter\squared}
(zur  Herleitung  dieses  Werts  siehe  Anhang  \ref{app:sec:cell:surface}  ab
Seite  \pageref{app:sec:cell:surface}) um  ungef\"ahr  einen Faktor  \num{600}
hochskaliert werden.

$R_{\mathrm{P}}$   und  $R_{\mathrm{S}}$   skalieren  umgekehrt   proportional
zur   Zellfl\"ache,    wogegen   $C$   bei   gr\"osser    werdender   Fl\"ache
ansteigt~\cite{ref:solar:scofield}. Das  schlimmstm\"ogliche   Szenario  f\"ur
unseren   Fall  tritt   ein,  wenn   der  Ausgangsstrom   der  Zelle   maximal
wird\todo{korrekt?}. Folgende Werte werden deshalb gew\"ahlt:

\begin{symbols}
    \firmlist
    \item[$C$]
        Beim Kurzschliessen der  Zelle treten Stromspitzen auf,  wenn sich der
        Kondensator  $C$  entl\"adt. Je  gr\"osser  dessen  Kapazit\"at,  umso
        h\"oher diese Stromspitzen.  Wir  nehmen also \SI{20}{\nano\farad} aus
        Gleichung \ref{eq:scofield:C} als Ausgangswert.
    \item[$R_{\mathrm{S}}$]
        Der aus der Solarzelle fliessende Strom wird gr\"osser, je kleiner der
        Seriewiderstand vor dem Ausgang ist. Es wird daher der niedrigere Wert
        von \SI{0.2}{\ohm} aus Gleichung \ref{eq:scofield:Rs} gew\"ahlt.
    \item[$R_{\mathrm{P}}$]
        Je    gr\"osser   der    Parallelwiderstand   im    Verh\"altnis   zum
        Seriewiderstand  ist, um  so mehr  Strom fliesst  aus dem  Ausgang der
        Zelle.  Es wird deshalb der h\"ochste Wert von \SI{500}{\kilo\ohm} aus
        Gleichung \ref{eq:scofield:Rp} als Ausgangswert verwendet.
\end{symbols}

Die skalierten Werte sind somit:
\begin{alignat}{2}
    C               &= C_{\mathrm{Messung}}    &\cdot 600 &= \SI{12}{\micro\farad} \\
    R_{\mathrm{S,}} &= R_{\mathrm{S, Messung}} &\div  600 &= \SI{333}{\micro\ohm}    \\
    R_{\mathrm{P,}} &= R_{\mathrm{P, Messung}} &\div  600 &= \SI{833}{\ohm}
\end{alignat}



Es bleiben noch die Parameter der Diode zu bestimmen. Ausganslage f\"ur das
Diodenmodell ist die Shockley-Diodengleichung:\todo{source?}

\begin{equation}
    \label{eq:diode}
    I_{\mathrm{D}} = I_{\mathrm{S}} \cdot \left( \exp\left(\frac{q \cdot V}{n \cdot k \cdot T}\right) - 1 \right)
\end{equation}

\begin{conditions}
    I_{\mathrm{D}} & Diodenstrom \\
    I_{\mathrm{S}} & Reverse saturation current \\
    q              & Elementarladung eines Elektrons (\SI{1.602e-19}{\coulomb}) \\
    V              & Diodenspannung \\
    n              & Idealit\"atsfaktor \\
    k              & Boltzmannkonstante (\SI{1.38e-23}{\joule\per\kelvin}) \\
    T              & Diodentemperatur \\
\end{conditions}

Der   Reverse   Saturation  Current   ist   der   Strom,  der   beim   Anlegen
einer  negativen   Spannung  \"uber  die   Diode  fliesst,  bevor   die  Diode
durchbricht~\cite{ref:solar:diodeCharacteristics}.    Er  liegt   bei  kleinen
Dioden \"ublicherweise Bereich von Nano-Amp\`ere bis Femto-Amp\`ere\todo{F\"ur
k\"aufliche Dioden, nicht  Solarzellen}, bei einer Solarzelle  wird er h\"oher
liegen, da diese gr\"osser ist.

Wie  man an  Gleichung \ref{eq:diode}  erkennen kann,  steigt der  Diodenstrom
f\"ur eine gegebene Spannung, wenn der Reverse Saturation Current ansteigt.

Der Idealit\"atsfaktor ist ein Indikator  f\"ur den Spannungsabfall \"uber der
Diode in  Abh\"angigkeit des durchfliessenden Stromes  und liegt normalerweise
zwischen  1  (ideale  Diode)   und  2. Je  gr\"osser  der  Idealit\"atsfaktor,
umso  h\"oher  der Spannungsabfall  \"uber  der  Diode f\"ur  einen  gegebenen
Strom  (bzw.  umso kleiner  der  Strom  bei einer  fixen  Spannung). Abbildung
\ref{fig:diodeVI:IS} zeigt das Strom-Spannungsverhalten  einer Diode sowie den
Einfluss von $I_{\mathrm{S}}$ und $n$ schematisch.

\begin{figure}[h!tb]
    %% Creator: Matplotlib, PGF backend
%%
%% To include the figure in your LaTeX document, write
%%   \input{<filename>.pgf}
%%
%% Make sure the required packages are loaded in your preamble
%%   \usepackage{pgf}
%%
%% Figures using additional raster images can only be included by \input if
%% they are in the same directory as the main LaTeX file. For loading figures
%% from other directories you can use the `import` package
%%   \usepackage{import}
%% and then include the figures with
%%   \import{<path to file>}{<filename>.pgf}
%%
%% Matplotlib used the following preamble
%%   \usepackage{fontspec}
%%   \setmainfont{Bitstream Vera Serif}
%%   \setsansfont{Bitstream Vera Sans}
%%   \setmonofont{Bitstream Vera Sans Mono}
%%
\begingroup%
\makeatletter%
\begin{pgfpicture}%
\pgfpathrectangle{\pgfpointorigin}{\pgfqpoint{4.500000in}{3.000000in}}%
\pgfusepath{use as bounding box, clip}%
\begin{pgfscope}%
\pgfsetbuttcap%
\pgfsetmiterjoin%
\pgfsetlinewidth{0.000000pt}%
\definecolor{currentstroke}{rgb}{0.000000,0.000000,0.000000}%
\pgfsetstrokecolor{currentstroke}%
\pgfsetstrokeopacity{0.000000}%
\pgfsetdash{}{0pt}%
\pgfpathmoveto{\pgfqpoint{0.000000in}{0.000000in}}%
\pgfpathlineto{\pgfqpoint{4.500000in}{0.000000in}}%
\pgfpathlineto{\pgfqpoint{4.500000in}{3.000000in}}%
\pgfpathlineto{\pgfqpoint{0.000000in}{3.000000in}}%
\pgfpathclose%
\pgfusepath{}%
\end{pgfscope}%
\begin{pgfscope}%
\pgfsetbuttcap%
\pgfsetmiterjoin%
\pgfsetlinewidth{0.000000pt}%
\definecolor{currentstroke}{rgb}{0.000000,0.000000,0.000000}%
\pgfsetstrokecolor{currentstroke}%
\pgfsetstrokeopacity{0.000000}%
\pgfsetdash{}{0pt}%
\pgfpathmoveto{\pgfqpoint{0.900000in}{0.600000in}}%
\pgfpathlineto{\pgfqpoint{4.275000in}{0.600000in}}%
\pgfpathlineto{\pgfqpoint{4.275000in}{2.550000in}}%
\pgfpathlineto{\pgfqpoint{0.900000in}{2.550000in}}%
\pgfpathclose%
\pgfusepath{}%
\end{pgfscope}%
\begin{pgfscope}%
\pgfpathrectangle{\pgfqpoint{0.900000in}{0.600000in}}{\pgfqpoint{3.375000in}{1.950000in}} %
\pgfusepath{clip}%
\pgfsetbuttcap%
\pgfsetroundjoin%
\definecolor{currentfill}{rgb}{1.000000,0.000000,1.000000}%
\pgfsetfillcolor{currentfill}%
\pgfsetfillopacity{0.500000}%
\pgfsetlinewidth{0.000000pt}%
\definecolor{currentstroke}{rgb}{1.000000,0.000000,1.000000}%
\pgfsetstrokecolor{currentstroke}%
\pgfsetstrokeopacity{0.500000}%
\pgfsetdash{}{0pt}%
\pgfpathmoveto{\pgfqpoint{2.212500in}{1.575000in}}%
\pgfpathlineto{\pgfqpoint{2.212500in}{1.331122in}}%
\pgfpathlineto{\pgfqpoint{2.213270in}{1.331136in}}%
\pgfpathlineto{\pgfqpoint{2.214041in}{1.331149in}}%
\pgfpathlineto{\pgfqpoint{2.214811in}{1.331163in}}%
\pgfpathlineto{\pgfqpoint{2.215581in}{1.331177in}}%
\pgfpathlineto{\pgfqpoint{2.216351in}{1.331191in}}%
\pgfpathlineto{\pgfqpoint{2.217122in}{1.331206in}}%
\pgfpathlineto{\pgfqpoint{2.217892in}{1.331220in}}%
\pgfpathlineto{\pgfqpoint{2.218662in}{1.331235in}}%
\pgfpathlineto{\pgfqpoint{2.219433in}{1.331250in}}%
\pgfpathlineto{\pgfqpoint{2.220203in}{1.331265in}}%
\pgfpathlineto{\pgfqpoint{2.220973in}{1.331280in}}%
\pgfpathlineto{\pgfqpoint{2.221743in}{1.331296in}}%
\pgfpathlineto{\pgfqpoint{2.222514in}{1.331311in}}%
\pgfpathlineto{\pgfqpoint{2.223284in}{1.331327in}}%
\pgfpathlineto{\pgfqpoint{2.224054in}{1.331343in}}%
\pgfpathlineto{\pgfqpoint{2.224825in}{1.331359in}}%
\pgfpathlineto{\pgfqpoint{2.225595in}{1.331376in}}%
\pgfpathlineto{\pgfqpoint{2.226365in}{1.331392in}}%
\pgfpathlineto{\pgfqpoint{2.227136in}{1.331409in}}%
\pgfpathlineto{\pgfqpoint{2.227906in}{1.331426in}}%
\pgfpathlineto{\pgfqpoint{2.228676in}{1.331443in}}%
\pgfpathlineto{\pgfqpoint{2.229446in}{1.331461in}}%
\pgfpathlineto{\pgfqpoint{2.230217in}{1.331478in}}%
\pgfpathlineto{\pgfqpoint{2.230987in}{1.331496in}}%
\pgfpathlineto{\pgfqpoint{2.231757in}{1.331514in}}%
\pgfpathlineto{\pgfqpoint{2.232528in}{1.331533in}}%
\pgfpathlineto{\pgfqpoint{2.233298in}{1.331551in}}%
\pgfpathlineto{\pgfqpoint{2.234068in}{1.331570in}}%
\pgfpathlineto{\pgfqpoint{2.234838in}{1.331589in}}%
\pgfpathlineto{\pgfqpoint{2.235609in}{1.331608in}}%
\pgfpathlineto{\pgfqpoint{2.236379in}{1.331628in}}%
\pgfpathlineto{\pgfqpoint{2.237149in}{1.331648in}}%
\pgfpathlineto{\pgfqpoint{2.237920in}{1.331668in}}%
\pgfpathlineto{\pgfqpoint{2.238690in}{1.331688in}}%
\pgfpathlineto{\pgfqpoint{2.239460in}{1.331708in}}%
\pgfpathlineto{\pgfqpoint{2.240230in}{1.331729in}}%
\pgfpathlineto{\pgfqpoint{2.241001in}{1.331750in}}%
\pgfpathlineto{\pgfqpoint{2.241771in}{1.331771in}}%
\pgfpathlineto{\pgfqpoint{2.242541in}{1.331793in}}%
\pgfpathlineto{\pgfqpoint{2.243312in}{1.331814in}}%
\pgfpathlineto{\pgfqpoint{2.244082in}{1.331836in}}%
\pgfpathlineto{\pgfqpoint{2.244852in}{1.331859in}}%
\pgfpathlineto{\pgfqpoint{2.245622in}{1.331881in}}%
\pgfpathlineto{\pgfqpoint{2.246393in}{1.331904in}}%
\pgfpathlineto{\pgfqpoint{2.247163in}{1.331927in}}%
\pgfpathlineto{\pgfqpoint{2.247933in}{1.331951in}}%
\pgfpathlineto{\pgfqpoint{2.248704in}{1.331974in}}%
\pgfpathlineto{\pgfqpoint{2.249474in}{1.331998in}}%
\pgfpathlineto{\pgfqpoint{2.250244in}{1.332023in}}%
\pgfpathlineto{\pgfqpoint{2.251015in}{1.332047in}}%
\pgfpathlineto{\pgfqpoint{2.251785in}{1.332072in}}%
\pgfpathlineto{\pgfqpoint{2.252555in}{1.332097in}}%
\pgfpathlineto{\pgfqpoint{2.253325in}{1.332123in}}%
\pgfpathlineto{\pgfqpoint{2.254096in}{1.332149in}}%
\pgfpathlineto{\pgfqpoint{2.254866in}{1.332175in}}%
\pgfpathlineto{\pgfqpoint{2.255636in}{1.332201in}}%
\pgfpathlineto{\pgfqpoint{2.256407in}{1.332228in}}%
\pgfpathlineto{\pgfqpoint{2.257177in}{1.332255in}}%
\pgfpathlineto{\pgfqpoint{2.257947in}{1.332283in}}%
\pgfpathlineto{\pgfqpoint{2.258717in}{1.332310in}}%
\pgfpathlineto{\pgfqpoint{2.259488in}{1.332338in}}%
\pgfpathlineto{\pgfqpoint{2.260258in}{1.332367in}}%
\pgfpathlineto{\pgfqpoint{2.261028in}{1.332396in}}%
\pgfpathlineto{\pgfqpoint{2.261799in}{1.332425in}}%
\pgfpathlineto{\pgfqpoint{2.262569in}{1.332454in}}%
\pgfpathlineto{\pgfqpoint{2.263339in}{1.332484in}}%
\pgfpathlineto{\pgfqpoint{2.264109in}{1.332515in}}%
\pgfpathlineto{\pgfqpoint{2.264880in}{1.332545in}}%
\pgfpathlineto{\pgfqpoint{2.265650in}{1.332576in}}%
\pgfpathlineto{\pgfqpoint{2.266420in}{1.332608in}}%
\pgfpathlineto{\pgfqpoint{2.267191in}{1.332639in}}%
\pgfpathlineto{\pgfqpoint{2.267961in}{1.332671in}}%
\pgfpathlineto{\pgfqpoint{2.268731in}{1.332704in}}%
\pgfpathlineto{\pgfqpoint{2.269502in}{1.332737in}}%
\pgfpathlineto{\pgfqpoint{2.270272in}{1.332770in}}%
\pgfpathlineto{\pgfqpoint{2.271042in}{1.332804in}}%
\pgfpathlineto{\pgfqpoint{2.271812in}{1.332838in}}%
\pgfpathlineto{\pgfqpoint{2.272583in}{1.332873in}}%
\pgfpathlineto{\pgfqpoint{2.273353in}{1.332908in}}%
\pgfpathlineto{\pgfqpoint{2.274123in}{1.332943in}}%
\pgfpathlineto{\pgfqpoint{2.274894in}{1.332979in}}%
\pgfpathlineto{\pgfqpoint{2.275664in}{1.333015in}}%
\pgfpathlineto{\pgfqpoint{2.276434in}{1.333052in}}%
\pgfpathlineto{\pgfqpoint{2.277204in}{1.333089in}}%
\pgfpathlineto{\pgfqpoint{2.277975in}{1.333127in}}%
\pgfpathlineto{\pgfqpoint{2.278745in}{1.333165in}}%
\pgfpathlineto{\pgfqpoint{2.279515in}{1.333204in}}%
\pgfpathlineto{\pgfqpoint{2.280286in}{1.333243in}}%
\pgfpathlineto{\pgfqpoint{2.281056in}{1.333282in}}%
\pgfpathlineto{\pgfqpoint{2.281826in}{1.333322in}}%
\pgfpathlineto{\pgfqpoint{2.282596in}{1.333363in}}%
\pgfpathlineto{\pgfqpoint{2.283367in}{1.333404in}}%
\pgfpathlineto{\pgfqpoint{2.284137in}{1.333445in}}%
\pgfpathlineto{\pgfqpoint{2.284907in}{1.333487in}}%
\pgfpathlineto{\pgfqpoint{2.285678in}{1.333530in}}%
\pgfpathlineto{\pgfqpoint{2.286448in}{1.333573in}}%
\pgfpathlineto{\pgfqpoint{2.287218in}{1.333616in}}%
\pgfpathlineto{\pgfqpoint{2.287988in}{1.333660in}}%
\pgfpathlineto{\pgfqpoint{2.288759in}{1.333705in}}%
\pgfpathlineto{\pgfqpoint{2.289529in}{1.333750in}}%
\pgfpathlineto{\pgfqpoint{2.290299in}{1.333796in}}%
\pgfpathlineto{\pgfqpoint{2.291070in}{1.333842in}}%
\pgfpathlineto{\pgfqpoint{2.291840in}{1.333889in}}%
\pgfpathlineto{\pgfqpoint{2.292610in}{1.333937in}}%
\pgfpathlineto{\pgfqpoint{2.293381in}{1.333985in}}%
\pgfpathlineto{\pgfqpoint{2.294151in}{1.334033in}}%
\pgfpathlineto{\pgfqpoint{2.294921in}{1.334082in}}%
\pgfpathlineto{\pgfqpoint{2.295691in}{1.334132in}}%
\pgfpathlineto{\pgfqpoint{2.296462in}{1.334182in}}%
\pgfpathlineto{\pgfqpoint{2.297232in}{1.334233in}}%
\pgfpathlineto{\pgfqpoint{2.298002in}{1.334285in}}%
\pgfpathlineto{\pgfqpoint{2.298773in}{1.334337in}}%
\pgfpathlineto{\pgfqpoint{2.299543in}{1.334390in}}%
\pgfpathlineto{\pgfqpoint{2.300313in}{1.334444in}}%
\pgfpathlineto{\pgfqpoint{2.301083in}{1.334498in}}%
\pgfpathlineto{\pgfqpoint{2.301854in}{1.334553in}}%
\pgfpathlineto{\pgfqpoint{2.302624in}{1.334608in}}%
\pgfpathlineto{\pgfqpoint{2.303394in}{1.334664in}}%
\pgfpathlineto{\pgfqpoint{2.304165in}{1.334721in}}%
\pgfpathlineto{\pgfqpoint{2.304935in}{1.334779in}}%
\pgfpathlineto{\pgfqpoint{2.305705in}{1.334837in}}%
\pgfpathlineto{\pgfqpoint{2.306475in}{1.334896in}}%
\pgfpathlineto{\pgfqpoint{2.307246in}{1.334956in}}%
\pgfpathlineto{\pgfqpoint{2.308016in}{1.335016in}}%
\pgfpathlineto{\pgfqpoint{2.308786in}{1.335077in}}%
\pgfpathlineto{\pgfqpoint{2.309557in}{1.335139in}}%
\pgfpathlineto{\pgfqpoint{2.310327in}{1.335201in}}%
\pgfpathlineto{\pgfqpoint{2.311097in}{1.335265in}}%
\pgfpathlineto{\pgfqpoint{2.311867in}{1.335329in}}%
\pgfpathlineto{\pgfqpoint{2.312638in}{1.335394in}}%
\pgfpathlineto{\pgfqpoint{2.313408in}{1.335460in}}%
\pgfpathlineto{\pgfqpoint{2.314178in}{1.335526in}}%
\pgfpathlineto{\pgfqpoint{2.314949in}{1.335593in}}%
\pgfpathlineto{\pgfqpoint{2.315719in}{1.335661in}}%
\pgfpathlineto{\pgfqpoint{2.316489in}{1.335730in}}%
\pgfpathlineto{\pgfqpoint{2.317260in}{1.335800in}}%
\pgfpathlineto{\pgfqpoint{2.318030in}{1.335871in}}%
\pgfpathlineto{\pgfqpoint{2.318800in}{1.335942in}}%
\pgfpathlineto{\pgfqpoint{2.319570in}{1.336014in}}%
\pgfpathlineto{\pgfqpoint{2.320341in}{1.336088in}}%
\pgfpathlineto{\pgfqpoint{2.321111in}{1.336162in}}%
\pgfpathlineto{\pgfqpoint{2.321881in}{1.336237in}}%
\pgfpathlineto{\pgfqpoint{2.322652in}{1.336312in}}%
\pgfpathlineto{\pgfqpoint{2.323422in}{1.336389in}}%
\pgfpathlineto{\pgfqpoint{2.324192in}{1.336467in}}%
\pgfpathlineto{\pgfqpoint{2.324962in}{1.336546in}}%
\pgfpathlineto{\pgfqpoint{2.325733in}{1.336625in}}%
\pgfpathlineto{\pgfqpoint{2.326503in}{1.336706in}}%
\pgfpathlineto{\pgfqpoint{2.327273in}{1.336787in}}%
\pgfpathlineto{\pgfqpoint{2.328044in}{1.336870in}}%
\pgfpathlineto{\pgfqpoint{2.328814in}{1.336953in}}%
\pgfpathlineto{\pgfqpoint{2.329584in}{1.337038in}}%
\pgfpathlineto{\pgfqpoint{2.330354in}{1.337123in}}%
\pgfpathlineto{\pgfqpoint{2.331125in}{1.337210in}}%
\pgfpathlineto{\pgfqpoint{2.331895in}{1.337297in}}%
\pgfpathlineto{\pgfqpoint{2.332665in}{1.337386in}}%
\pgfpathlineto{\pgfqpoint{2.333436in}{1.337475in}}%
\pgfpathlineto{\pgfqpoint{2.334206in}{1.337566in}}%
\pgfpathlineto{\pgfqpoint{2.334976in}{1.337658in}}%
\pgfpathlineto{\pgfqpoint{2.335746in}{1.337751in}}%
\pgfpathlineto{\pgfqpoint{2.336517in}{1.337845in}}%
\pgfpathlineto{\pgfqpoint{2.337287in}{1.337940in}}%
\pgfpathlineto{\pgfqpoint{2.338057in}{1.338036in}}%
\pgfpathlineto{\pgfqpoint{2.338828in}{1.338134in}}%
\pgfpathlineto{\pgfqpoint{2.339598in}{1.338232in}}%
\pgfpathlineto{\pgfqpoint{2.340368in}{1.338332in}}%
\pgfpathlineto{\pgfqpoint{2.341139in}{1.338433in}}%
\pgfpathlineto{\pgfqpoint{2.341909in}{1.338535in}}%
\pgfpathlineto{\pgfqpoint{2.342679in}{1.338639in}}%
\pgfpathlineto{\pgfqpoint{2.343449in}{1.338743in}}%
\pgfpathlineto{\pgfqpoint{2.344220in}{1.338849in}}%
\pgfpathlineto{\pgfqpoint{2.344990in}{1.338956in}}%
\pgfpathlineto{\pgfqpoint{2.345760in}{1.339065in}}%
\pgfpathlineto{\pgfqpoint{2.346531in}{1.339175in}}%
\pgfpathlineto{\pgfqpoint{2.347301in}{1.339286in}}%
\pgfpathlineto{\pgfqpoint{2.348071in}{1.339398in}}%
\pgfpathlineto{\pgfqpoint{2.348841in}{1.339511in}}%
\pgfpathlineto{\pgfqpoint{2.349612in}{1.339627in}}%
\pgfpathlineto{\pgfqpoint{2.350382in}{1.339743in}}%
\pgfpathlineto{\pgfqpoint{2.351152in}{1.339861in}}%
\pgfpathlineto{\pgfqpoint{2.351923in}{1.339980in}}%
\pgfpathlineto{\pgfqpoint{2.352693in}{1.340100in}}%
\pgfpathlineto{\pgfqpoint{2.353463in}{1.340222in}}%
\pgfpathlineto{\pgfqpoint{2.354233in}{1.340346in}}%
\pgfpathlineto{\pgfqpoint{2.355004in}{1.340471in}}%
\pgfpathlineto{\pgfqpoint{2.355774in}{1.340597in}}%
\pgfpathlineto{\pgfqpoint{2.356544in}{1.340725in}}%
\pgfpathlineto{\pgfqpoint{2.357315in}{1.340854in}}%
\pgfpathlineto{\pgfqpoint{2.358085in}{1.340985in}}%
\pgfpathlineto{\pgfqpoint{2.358855in}{1.341117in}}%
\pgfpathlineto{\pgfqpoint{2.359626in}{1.341251in}}%
\pgfpathlineto{\pgfqpoint{2.360396in}{1.341387in}}%
\pgfpathlineto{\pgfqpoint{2.361166in}{1.341524in}}%
\pgfpathlineto{\pgfqpoint{2.361936in}{1.341663in}}%
\pgfpathlineto{\pgfqpoint{2.362707in}{1.341803in}}%
\pgfpathlineto{\pgfqpoint{2.363477in}{1.341945in}}%
\pgfpathlineto{\pgfqpoint{2.364247in}{1.342089in}}%
\pgfpathlineto{\pgfqpoint{2.365018in}{1.342234in}}%
\pgfpathlineto{\pgfqpoint{2.365788in}{1.342382in}}%
\pgfpathlineto{\pgfqpoint{2.366558in}{1.342530in}}%
\pgfpathlineto{\pgfqpoint{2.367328in}{1.342681in}}%
\pgfpathlineto{\pgfqpoint{2.368099in}{1.342833in}}%
\pgfpathlineto{\pgfqpoint{2.368869in}{1.342987in}}%
\pgfpathlineto{\pgfqpoint{2.369639in}{1.343143in}}%
\pgfpathlineto{\pgfqpoint{2.370410in}{1.343301in}}%
\pgfpathlineto{\pgfqpoint{2.371180in}{1.343461in}}%
\pgfpathlineto{\pgfqpoint{2.371950in}{1.343622in}}%
\pgfpathlineto{\pgfqpoint{2.372720in}{1.343785in}}%
\pgfpathlineto{\pgfqpoint{2.373491in}{1.343950in}}%
\pgfpathlineto{\pgfqpoint{2.374261in}{1.344117in}}%
\pgfpathlineto{\pgfqpoint{2.375031in}{1.344287in}}%
\pgfpathlineto{\pgfqpoint{2.375802in}{1.344458in}}%
\pgfpathlineto{\pgfqpoint{2.376572in}{1.344630in}}%
\pgfpathlineto{\pgfqpoint{2.377342in}{1.344805in}}%
\pgfpathlineto{\pgfqpoint{2.378112in}{1.344982in}}%
\pgfpathlineto{\pgfqpoint{2.378883in}{1.345162in}}%
\pgfpathlineto{\pgfqpoint{2.379653in}{1.345343in}}%
\pgfpathlineto{\pgfqpoint{2.380423in}{1.345526in}}%
\pgfpathlineto{\pgfqpoint{2.381194in}{1.345711in}}%
\pgfpathlineto{\pgfqpoint{2.381964in}{1.345899in}}%
\pgfpathlineto{\pgfqpoint{2.382734in}{1.346088in}}%
\pgfpathlineto{\pgfqpoint{2.383505in}{1.346280in}}%
\pgfpathlineto{\pgfqpoint{2.384275in}{1.346474in}}%
\pgfpathlineto{\pgfqpoint{2.385045in}{1.346670in}}%
\pgfpathlineto{\pgfqpoint{2.385815in}{1.346869in}}%
\pgfpathlineto{\pgfqpoint{2.386586in}{1.347069in}}%
\pgfpathlineto{\pgfqpoint{2.387356in}{1.347272in}}%
\pgfpathlineto{\pgfqpoint{2.388126in}{1.347478in}}%
\pgfpathlineto{\pgfqpoint{2.388897in}{1.347686in}}%
\pgfpathlineto{\pgfqpoint{2.389667in}{1.347896in}}%
\pgfpathlineto{\pgfqpoint{2.390437in}{1.348108in}}%
\pgfpathlineto{\pgfqpoint{2.391207in}{1.348323in}}%
\pgfpathlineto{\pgfqpoint{2.391978in}{1.348540in}}%
\pgfpathlineto{\pgfqpoint{2.392748in}{1.348760in}}%
\pgfpathlineto{\pgfqpoint{2.393518in}{1.348982in}}%
\pgfpathlineto{\pgfqpoint{2.394289in}{1.349207in}}%
\pgfpathlineto{\pgfqpoint{2.395059in}{1.349434in}}%
\pgfpathlineto{\pgfqpoint{2.395829in}{1.349664in}}%
\pgfpathlineto{\pgfqpoint{2.396599in}{1.349897in}}%
\pgfpathlineto{\pgfqpoint{2.397370in}{1.350132in}}%
\pgfpathlineto{\pgfqpoint{2.398140in}{1.350370in}}%
\pgfpathlineto{\pgfqpoint{2.398910in}{1.350610in}}%
\pgfpathlineto{\pgfqpoint{2.399681in}{1.350853in}}%
\pgfpathlineto{\pgfqpoint{2.400451in}{1.351099in}}%
\pgfpathlineto{\pgfqpoint{2.401221in}{1.351348in}}%
\pgfpathlineto{\pgfqpoint{2.401991in}{1.351599in}}%
\pgfpathlineto{\pgfqpoint{2.402762in}{1.351854in}}%
\pgfpathlineto{\pgfqpoint{2.403532in}{1.352111in}}%
\pgfpathlineto{\pgfqpoint{2.404302in}{1.352371in}}%
\pgfpathlineto{\pgfqpoint{2.405073in}{1.352633in}}%
\pgfpathlineto{\pgfqpoint{2.405843in}{1.352899in}}%
\pgfpathlineto{\pgfqpoint{2.406613in}{1.353168in}}%
\pgfpathlineto{\pgfqpoint{2.407384in}{1.353440in}}%
\pgfpathlineto{\pgfqpoint{2.408154in}{1.353714in}}%
\pgfpathlineto{\pgfqpoint{2.408924in}{1.353992in}}%
\pgfpathlineto{\pgfqpoint{2.409694in}{1.354273in}}%
\pgfpathlineto{\pgfqpoint{2.410465in}{1.354557in}}%
\pgfpathlineto{\pgfqpoint{2.411235in}{1.354844in}}%
\pgfpathlineto{\pgfqpoint{2.412005in}{1.355134in}}%
\pgfpathlineto{\pgfqpoint{2.412776in}{1.355427in}}%
\pgfpathlineto{\pgfqpoint{2.413546in}{1.355724in}}%
\pgfpathlineto{\pgfqpoint{2.414316in}{1.356024in}}%
\pgfpathlineto{\pgfqpoint{2.415086in}{1.356327in}}%
\pgfpathlineto{\pgfqpoint{2.415857in}{1.356633in}}%
\pgfpathlineto{\pgfqpoint{2.416627in}{1.356943in}}%
\pgfpathlineto{\pgfqpoint{2.417397in}{1.357256in}}%
\pgfpathlineto{\pgfqpoint{2.418168in}{1.357573in}}%
\pgfpathlineto{\pgfqpoint{2.418938in}{1.357893in}}%
\pgfpathlineto{\pgfqpoint{2.419708in}{1.358216in}}%
\pgfpathlineto{\pgfqpoint{2.420478in}{1.358543in}}%
\pgfpathlineto{\pgfqpoint{2.421249in}{1.358873in}}%
\pgfpathlineto{\pgfqpoint{2.422019in}{1.359208in}}%
\pgfpathlineto{\pgfqpoint{2.422789in}{1.359545in}}%
\pgfpathlineto{\pgfqpoint{2.423560in}{1.359886in}}%
\pgfpathlineto{\pgfqpoint{2.424330in}{1.360231in}}%
\pgfpathlineto{\pgfqpoint{2.425100in}{1.360580in}}%
\pgfpathlineto{\pgfqpoint{2.425870in}{1.360932in}}%
\pgfpathlineto{\pgfqpoint{2.426641in}{1.361288in}}%
\pgfpathlineto{\pgfqpoint{2.427411in}{1.361648in}}%
\pgfpathlineto{\pgfqpoint{2.428181in}{1.362012in}}%
\pgfpathlineto{\pgfqpoint{2.428952in}{1.362379in}}%
\pgfpathlineto{\pgfqpoint{2.429722in}{1.362751in}}%
\pgfpathlineto{\pgfqpoint{2.430492in}{1.363126in}}%
\pgfpathlineto{\pgfqpoint{2.431263in}{1.363505in}}%
\pgfpathlineto{\pgfqpoint{2.432033in}{1.363888in}}%
\pgfpathlineto{\pgfqpoint{2.432803in}{1.364276in}}%
\pgfpathlineto{\pgfqpoint{2.433573in}{1.364667in}}%
\pgfpathlineto{\pgfqpoint{2.434344in}{1.365062in}}%
\pgfpathlineto{\pgfqpoint{2.435114in}{1.365462in}}%
\pgfpathlineto{\pgfqpoint{2.435884in}{1.365866in}}%
\pgfpathlineto{\pgfqpoint{2.436655in}{1.366273in}}%
\pgfpathlineto{\pgfqpoint{2.437425in}{1.366686in}}%
\pgfpathlineto{\pgfqpoint{2.438195in}{1.367102in}}%
\pgfpathlineto{\pgfqpoint{2.438965in}{1.367523in}}%
\pgfpathlineto{\pgfqpoint{2.439736in}{1.367948in}}%
\pgfpathlineto{\pgfqpoint{2.440506in}{1.368377in}}%
\pgfpathlineto{\pgfqpoint{2.441276in}{1.368811in}}%
\pgfpathlineto{\pgfqpoint{2.442047in}{1.369249in}}%
\pgfpathlineto{\pgfqpoint{2.442817in}{1.369691in}}%
\pgfpathlineto{\pgfqpoint{2.443587in}{1.370138in}}%
\pgfpathlineto{\pgfqpoint{2.444357in}{1.370590in}}%
\pgfpathlineto{\pgfqpoint{2.445128in}{1.371046in}}%
\pgfpathlineto{\pgfqpoint{2.445898in}{1.371507in}}%
\pgfpathlineto{\pgfqpoint{2.446668in}{1.371973in}}%
\pgfpathlineto{\pgfqpoint{2.447439in}{1.372443in}}%
\pgfpathlineto{\pgfqpoint{2.448209in}{1.372918in}}%
\pgfpathlineto{\pgfqpoint{2.448979in}{1.373397in}}%
\pgfpathlineto{\pgfqpoint{2.449749in}{1.373881in}}%
\pgfpathlineto{\pgfqpoint{2.450520in}{1.374371in}}%
\pgfpathlineto{\pgfqpoint{2.451290in}{1.374864in}}%
\pgfpathlineto{\pgfqpoint{2.452060in}{1.375363in}}%
\pgfpathlineto{\pgfqpoint{2.452831in}{1.375867in}}%
\pgfpathlineto{\pgfqpoint{2.453601in}{1.376376in}}%
\pgfpathlineto{\pgfqpoint{2.454371in}{1.376889in}}%
\pgfpathlineto{\pgfqpoint{2.455142in}{1.377408in}}%
\pgfpathlineto{\pgfqpoint{2.455912in}{1.377932in}}%
\pgfpathlineto{\pgfqpoint{2.456682in}{1.378461in}}%
\pgfpathlineto{\pgfqpoint{2.457452in}{1.378995in}}%
\pgfpathlineto{\pgfqpoint{2.458223in}{1.379534in}}%
\pgfpathlineto{\pgfqpoint{2.458993in}{1.380078in}}%
\pgfpathlineto{\pgfqpoint{2.459763in}{1.380627in}}%
\pgfpathlineto{\pgfqpoint{2.460534in}{1.381182in}}%
\pgfpathlineto{\pgfqpoint{2.461304in}{1.381742in}}%
\pgfpathlineto{\pgfqpoint{2.462074in}{1.382307in}}%
\pgfpathlineto{\pgfqpoint{2.462844in}{1.382878in}}%
\pgfpathlineto{\pgfqpoint{2.463615in}{1.383454in}}%
\pgfpathlineto{\pgfqpoint{2.464385in}{1.384035in}}%
\pgfpathlineto{\pgfqpoint{2.465155in}{1.384622in}}%
\pgfpathlineto{\pgfqpoint{2.465926in}{1.385214in}}%
\pgfpathlineto{\pgfqpoint{2.466696in}{1.385812in}}%
\pgfpathlineto{\pgfqpoint{2.467466in}{1.386416in}}%
\pgfpathlineto{\pgfqpoint{2.468236in}{1.387024in}}%
\pgfpathlineto{\pgfqpoint{2.469007in}{1.387639in}}%
\pgfpathlineto{\pgfqpoint{2.469777in}{1.388259in}}%
\pgfpathlineto{\pgfqpoint{2.470547in}{1.388885in}}%
\pgfpathlineto{\pgfqpoint{2.471318in}{1.389516in}}%
\pgfpathlineto{\pgfqpoint{2.472088in}{1.390153in}}%
\pgfpathlineto{\pgfqpoint{2.472858in}{1.390796in}}%
\pgfpathlineto{\pgfqpoint{2.473629in}{1.391445in}}%
\pgfpathlineto{\pgfqpoint{2.474399in}{1.392099in}}%
\pgfpathlineto{\pgfqpoint{2.475169in}{1.392760in}}%
\pgfpathlineto{\pgfqpoint{2.475939in}{1.393426in}}%
\pgfpathlineto{\pgfqpoint{2.476710in}{1.394098in}}%
\pgfpathlineto{\pgfqpoint{2.477480in}{1.394776in}}%
\pgfpathlineto{\pgfqpoint{2.478250in}{1.395459in}}%
\pgfpathlineto{\pgfqpoint{2.479021in}{1.396149in}}%
\pgfpathlineto{\pgfqpoint{2.479791in}{1.396845in}}%
\pgfpathlineto{\pgfqpoint{2.480561in}{1.397547in}}%
\pgfpathlineto{\pgfqpoint{2.481331in}{1.398254in}}%
\pgfpathlineto{\pgfqpoint{2.482102in}{1.398968in}}%
\pgfpathlineto{\pgfqpoint{2.482872in}{1.399688in}}%
\pgfpathlineto{\pgfqpoint{2.483642in}{1.400414in}}%
\pgfpathlineto{\pgfqpoint{2.484413in}{1.401146in}}%
\pgfpathlineto{\pgfqpoint{2.485183in}{1.401885in}}%
\pgfpathlineto{\pgfqpoint{2.485953in}{1.402629in}}%
\pgfpathlineto{\pgfqpoint{2.486723in}{1.403380in}}%
\pgfpathlineto{\pgfqpoint{2.487494in}{1.404136in}}%
\pgfpathlineto{\pgfqpoint{2.488264in}{1.404899in}}%
\pgfpathlineto{\pgfqpoint{2.489034in}{1.405669in}}%
\pgfpathlineto{\pgfqpoint{2.489805in}{1.406444in}}%
\pgfpathlineto{\pgfqpoint{2.490575in}{1.407226in}}%
\pgfpathlineto{\pgfqpoint{2.491345in}{1.408014in}}%
\pgfpathlineto{\pgfqpoint{2.492115in}{1.408809in}}%
\pgfpathlineto{\pgfqpoint{2.492886in}{1.409609in}}%
\pgfpathlineto{\pgfqpoint{2.493656in}{1.410416in}}%
\pgfpathlineto{\pgfqpoint{2.494426in}{1.411230in}}%
\pgfpathlineto{\pgfqpoint{2.495197in}{1.412049in}}%
\pgfpathlineto{\pgfqpoint{2.495967in}{1.412875in}}%
\pgfpathlineto{\pgfqpoint{2.496737in}{1.413708in}}%
\pgfpathlineto{\pgfqpoint{2.497508in}{1.414547in}}%
\pgfpathlineto{\pgfqpoint{2.498278in}{1.415392in}}%
\pgfpathlineto{\pgfqpoint{2.499048in}{1.416244in}}%
\pgfpathlineto{\pgfqpoint{2.499818in}{1.417102in}}%
\pgfpathlineto{\pgfqpoint{2.500589in}{1.417967in}}%
\pgfpathlineto{\pgfqpoint{2.501359in}{1.418838in}}%
\pgfpathlineto{\pgfqpoint{2.502129in}{1.419715in}}%
\pgfpathlineto{\pgfqpoint{2.502900in}{1.420599in}}%
\pgfpathlineto{\pgfqpoint{2.503670in}{1.421489in}}%
\pgfpathlineto{\pgfqpoint{2.504440in}{1.422386in}}%
\pgfpathlineto{\pgfqpoint{2.505210in}{1.423289in}}%
\pgfpathlineto{\pgfqpoint{2.505981in}{1.424199in}}%
\pgfpathlineto{\pgfqpoint{2.506751in}{1.425115in}}%
\pgfpathlineto{\pgfqpoint{2.507521in}{1.426037in}}%
\pgfpathlineto{\pgfqpoint{2.508292in}{1.426966in}}%
\pgfpathlineto{\pgfqpoint{2.509062in}{1.427902in}}%
\pgfpathlineto{\pgfqpoint{2.509832in}{1.428844in}}%
\pgfpathlineto{\pgfqpoint{2.510602in}{1.429792in}}%
\pgfpathlineto{\pgfqpoint{2.511373in}{1.430747in}}%
\pgfpathlineto{\pgfqpoint{2.512143in}{1.431708in}}%
\pgfpathlineto{\pgfqpoint{2.512913in}{1.432675in}}%
\pgfpathlineto{\pgfqpoint{2.513684in}{1.433649in}}%
\pgfpathlineto{\pgfqpoint{2.514454in}{1.434629in}}%
\pgfpathlineto{\pgfqpoint{2.515224in}{1.435616in}}%
\pgfpathlineto{\pgfqpoint{2.515994in}{1.436609in}}%
\pgfpathlineto{\pgfqpoint{2.516765in}{1.437609in}}%
\pgfpathlineto{\pgfqpoint{2.517535in}{1.438614in}}%
\pgfpathlineto{\pgfqpoint{2.518305in}{1.439626in}}%
\pgfpathlineto{\pgfqpoint{2.519076in}{1.440645in}}%
\pgfpathlineto{\pgfqpoint{2.519846in}{1.441669in}}%
\pgfpathlineto{\pgfqpoint{2.520616in}{1.442700in}}%
\pgfpathlineto{\pgfqpoint{2.521387in}{1.443737in}}%
\pgfpathlineto{\pgfqpoint{2.522157in}{1.444781in}}%
\pgfpathlineto{\pgfqpoint{2.522927in}{1.445830in}}%
\pgfpathlineto{\pgfqpoint{2.523697in}{1.446886in}}%
\pgfpathlineto{\pgfqpoint{2.524468in}{1.447948in}}%
\pgfpathlineto{\pgfqpoint{2.525238in}{1.449015in}}%
\pgfpathlineto{\pgfqpoint{2.526008in}{1.450089in}}%
\pgfpathlineto{\pgfqpoint{2.526779in}{1.451169in}}%
\pgfpathlineto{\pgfqpoint{2.527549in}{1.452255in}}%
\pgfpathlineto{\pgfqpoint{2.528319in}{1.453348in}}%
\pgfpathlineto{\pgfqpoint{2.529089in}{1.454446in}}%
\pgfpathlineto{\pgfqpoint{2.529860in}{1.455549in}}%
\pgfpathlineto{\pgfqpoint{2.530630in}{1.456659in}}%
\pgfpathlineto{\pgfqpoint{2.531400in}{1.457775in}}%
\pgfpathlineto{\pgfqpoint{2.532171in}{1.458896in}}%
\pgfpathlineto{\pgfqpoint{2.532941in}{1.460023in}}%
\pgfpathlineto{\pgfqpoint{2.533711in}{1.461156in}}%
\pgfpathlineto{\pgfqpoint{2.534481in}{1.462295in}}%
\pgfpathlineto{\pgfqpoint{2.535252in}{1.463439in}}%
\pgfpathlineto{\pgfqpoint{2.536022in}{1.464589in}}%
\pgfpathlineto{\pgfqpoint{2.536792in}{1.465744in}}%
\pgfpathlineto{\pgfqpoint{2.537563in}{1.466905in}}%
\pgfpathlineto{\pgfqpoint{2.538333in}{1.468072in}}%
\pgfpathlineto{\pgfqpoint{2.539103in}{1.469243in}}%
\pgfpathlineto{\pgfqpoint{2.539873in}{1.470420in}}%
\pgfpathlineto{\pgfqpoint{2.540644in}{1.471602in}}%
\pgfpathlineto{\pgfqpoint{2.541414in}{1.472790in}}%
\pgfpathlineto{\pgfqpoint{2.542184in}{1.473983in}}%
\pgfpathlineto{\pgfqpoint{2.542955in}{1.475181in}}%
\pgfpathlineto{\pgfqpoint{2.543725in}{1.476383in}}%
\pgfpathlineto{\pgfqpoint{2.544495in}{1.477591in}}%
\pgfpathlineto{\pgfqpoint{2.545266in}{1.478804in}}%
\pgfpathlineto{\pgfqpoint{2.546036in}{1.480022in}}%
\pgfpathlineto{\pgfqpoint{2.546806in}{1.481244in}}%
\pgfpathlineto{\pgfqpoint{2.547576in}{1.482472in}}%
\pgfpathlineto{\pgfqpoint{2.548347in}{1.483704in}}%
\pgfpathlineto{\pgfqpoint{2.549117in}{1.484940in}}%
\pgfpathlineto{\pgfqpoint{2.549887in}{1.486181in}}%
\pgfpathlineto{\pgfqpoint{2.550658in}{1.487427in}}%
\pgfpathlineto{\pgfqpoint{2.551428in}{1.488677in}}%
\pgfpathlineto{\pgfqpoint{2.552198in}{1.489931in}}%
\pgfpathlineto{\pgfqpoint{2.552968in}{1.491190in}}%
\pgfpathlineto{\pgfqpoint{2.553739in}{1.492452in}}%
\pgfpathlineto{\pgfqpoint{2.554509in}{1.493719in}}%
\pgfpathlineto{\pgfqpoint{2.555279in}{1.494990in}}%
\pgfpathlineto{\pgfqpoint{2.556050in}{1.496265in}}%
\pgfpathlineto{\pgfqpoint{2.556820in}{1.497544in}}%
\pgfpathlineto{\pgfqpoint{2.557590in}{1.498826in}}%
\pgfpathlineto{\pgfqpoint{2.558360in}{1.500113in}}%
\pgfpathlineto{\pgfqpoint{2.559131in}{1.501403in}}%
\pgfpathlineto{\pgfqpoint{2.559901in}{1.502696in}}%
\pgfpathlineto{\pgfqpoint{2.560671in}{1.503993in}}%
\pgfpathlineto{\pgfqpoint{2.561442in}{1.505293in}}%
\pgfpathlineto{\pgfqpoint{2.562212in}{1.506597in}}%
\pgfpathlineto{\pgfqpoint{2.562982in}{1.507904in}}%
\pgfpathlineto{\pgfqpoint{2.563753in}{1.509214in}}%
\pgfpathlineto{\pgfqpoint{2.564523in}{1.510527in}}%
\pgfpathlineto{\pgfqpoint{2.565293in}{1.511843in}}%
\pgfpathlineto{\pgfqpoint{2.566063in}{1.513161in}}%
\pgfpathlineto{\pgfqpoint{2.566834in}{1.514483in}}%
\pgfpathlineto{\pgfqpoint{2.567604in}{1.515807in}}%
\pgfpathlineto{\pgfqpoint{2.568374in}{1.517134in}}%
\pgfpathlineto{\pgfqpoint{2.569145in}{1.518463in}}%
\pgfpathlineto{\pgfqpoint{2.569915in}{1.519795in}}%
\pgfpathlineto{\pgfqpoint{2.570685in}{1.521129in}}%
\pgfpathlineto{\pgfqpoint{2.571455in}{1.522465in}}%
\pgfpathlineto{\pgfqpoint{2.572226in}{1.523803in}}%
\pgfpathlineto{\pgfqpoint{2.572996in}{1.525143in}}%
\pgfpathlineto{\pgfqpoint{2.573766in}{1.526485in}}%
\pgfpathlineto{\pgfqpoint{2.574537in}{1.527829in}}%
\pgfpathlineto{\pgfqpoint{2.575307in}{1.529175in}}%
\pgfpathlineto{\pgfqpoint{2.576077in}{1.530522in}}%
\pgfpathlineto{\pgfqpoint{2.576847in}{1.531871in}}%
\pgfpathlineto{\pgfqpoint{2.577618in}{1.533222in}}%
\pgfpathlineto{\pgfqpoint{2.578388in}{1.534573in}}%
\pgfpathlineto{\pgfqpoint{2.579158in}{1.535926in}}%
\pgfpathlineto{\pgfqpoint{2.579929in}{1.537280in}}%
\pgfpathlineto{\pgfqpoint{2.580699in}{1.538635in}}%
\pgfpathlineto{\pgfqpoint{2.581469in}{1.539990in}}%
\pgfpathlineto{\pgfqpoint{2.582239in}{1.541347in}}%
\pgfpathlineto{\pgfqpoint{2.583010in}{1.542704in}}%
\pgfpathlineto{\pgfqpoint{2.583780in}{1.544062in}}%
\pgfpathlineto{\pgfqpoint{2.584550in}{1.545421in}}%
\pgfpathlineto{\pgfqpoint{2.585321in}{1.546779in}}%
\pgfpathlineto{\pgfqpoint{2.586091in}{1.548138in}}%
\pgfpathlineto{\pgfqpoint{2.586861in}{1.549498in}}%
\pgfpathlineto{\pgfqpoint{2.587632in}{1.550857in}}%
\pgfpathlineto{\pgfqpoint{2.588402in}{1.552216in}}%
\pgfpathlineto{\pgfqpoint{2.589172in}{1.553576in}}%
\pgfpathlineto{\pgfqpoint{2.589942in}{1.554935in}}%
\pgfpathlineto{\pgfqpoint{2.590713in}{1.556293in}}%
\pgfpathlineto{\pgfqpoint{2.591483in}{1.557652in}}%
\pgfpathlineto{\pgfqpoint{2.592253in}{1.559009in}}%
\pgfpathlineto{\pgfqpoint{2.593024in}{1.560366in}}%
\pgfpathlineto{\pgfqpoint{2.593794in}{1.561723in}}%
\pgfpathlineto{\pgfqpoint{2.594564in}{1.563078in}}%
\pgfpathlineto{\pgfqpoint{2.595334in}{1.564433in}}%
\pgfpathlineto{\pgfqpoint{2.596105in}{1.565786in}}%
\pgfpathlineto{\pgfqpoint{2.596875in}{1.567139in}}%
\pgfpathlineto{\pgfqpoint{2.596875in}{1.575000in}}%
\pgfpathlineto{\pgfqpoint{2.596875in}{1.575000in}}%
\pgfpathlineto{\pgfqpoint{2.596105in}{1.575000in}}%
\pgfpathlineto{\pgfqpoint{2.595334in}{1.575000in}}%
\pgfpathlineto{\pgfqpoint{2.594564in}{1.575000in}}%
\pgfpathlineto{\pgfqpoint{2.593794in}{1.575000in}}%
\pgfpathlineto{\pgfqpoint{2.593024in}{1.575000in}}%
\pgfpathlineto{\pgfqpoint{2.592253in}{1.575000in}}%
\pgfpathlineto{\pgfqpoint{2.591483in}{1.575000in}}%
\pgfpathlineto{\pgfqpoint{2.590713in}{1.575000in}}%
\pgfpathlineto{\pgfqpoint{2.589942in}{1.575000in}}%
\pgfpathlineto{\pgfqpoint{2.589172in}{1.575000in}}%
\pgfpathlineto{\pgfqpoint{2.588402in}{1.575000in}}%
\pgfpathlineto{\pgfqpoint{2.587632in}{1.575000in}}%
\pgfpathlineto{\pgfqpoint{2.586861in}{1.575000in}}%
\pgfpathlineto{\pgfqpoint{2.586091in}{1.575000in}}%
\pgfpathlineto{\pgfqpoint{2.585321in}{1.575000in}}%
\pgfpathlineto{\pgfqpoint{2.584550in}{1.575000in}}%
\pgfpathlineto{\pgfqpoint{2.583780in}{1.575000in}}%
\pgfpathlineto{\pgfqpoint{2.583010in}{1.575000in}}%
\pgfpathlineto{\pgfqpoint{2.582239in}{1.575000in}}%
\pgfpathlineto{\pgfqpoint{2.581469in}{1.575000in}}%
\pgfpathlineto{\pgfqpoint{2.580699in}{1.575000in}}%
\pgfpathlineto{\pgfqpoint{2.579929in}{1.575000in}}%
\pgfpathlineto{\pgfqpoint{2.579158in}{1.575000in}}%
\pgfpathlineto{\pgfqpoint{2.578388in}{1.575000in}}%
\pgfpathlineto{\pgfqpoint{2.577618in}{1.575000in}}%
\pgfpathlineto{\pgfqpoint{2.576847in}{1.575000in}}%
\pgfpathlineto{\pgfqpoint{2.576077in}{1.575000in}}%
\pgfpathlineto{\pgfqpoint{2.575307in}{1.575000in}}%
\pgfpathlineto{\pgfqpoint{2.574537in}{1.575000in}}%
\pgfpathlineto{\pgfqpoint{2.573766in}{1.575000in}}%
\pgfpathlineto{\pgfqpoint{2.572996in}{1.575000in}}%
\pgfpathlineto{\pgfqpoint{2.572226in}{1.575000in}}%
\pgfpathlineto{\pgfqpoint{2.571455in}{1.575000in}}%
\pgfpathlineto{\pgfqpoint{2.570685in}{1.575000in}}%
\pgfpathlineto{\pgfqpoint{2.569915in}{1.575000in}}%
\pgfpathlineto{\pgfqpoint{2.569145in}{1.575000in}}%
\pgfpathlineto{\pgfqpoint{2.568374in}{1.575000in}}%
\pgfpathlineto{\pgfqpoint{2.567604in}{1.575000in}}%
\pgfpathlineto{\pgfqpoint{2.566834in}{1.575000in}}%
\pgfpathlineto{\pgfqpoint{2.566063in}{1.575000in}}%
\pgfpathlineto{\pgfqpoint{2.565293in}{1.575000in}}%
\pgfpathlineto{\pgfqpoint{2.564523in}{1.575000in}}%
\pgfpathlineto{\pgfqpoint{2.563753in}{1.575000in}}%
\pgfpathlineto{\pgfqpoint{2.562982in}{1.575000in}}%
\pgfpathlineto{\pgfqpoint{2.562212in}{1.575000in}}%
\pgfpathlineto{\pgfqpoint{2.561442in}{1.575000in}}%
\pgfpathlineto{\pgfqpoint{2.560671in}{1.575000in}}%
\pgfpathlineto{\pgfqpoint{2.559901in}{1.575000in}}%
\pgfpathlineto{\pgfqpoint{2.559131in}{1.575000in}}%
\pgfpathlineto{\pgfqpoint{2.558360in}{1.575000in}}%
\pgfpathlineto{\pgfqpoint{2.557590in}{1.575000in}}%
\pgfpathlineto{\pgfqpoint{2.556820in}{1.575000in}}%
\pgfpathlineto{\pgfqpoint{2.556050in}{1.575000in}}%
\pgfpathlineto{\pgfqpoint{2.555279in}{1.575000in}}%
\pgfpathlineto{\pgfqpoint{2.554509in}{1.575000in}}%
\pgfpathlineto{\pgfqpoint{2.553739in}{1.575000in}}%
\pgfpathlineto{\pgfqpoint{2.552968in}{1.575000in}}%
\pgfpathlineto{\pgfqpoint{2.552198in}{1.575000in}}%
\pgfpathlineto{\pgfqpoint{2.551428in}{1.575000in}}%
\pgfpathlineto{\pgfqpoint{2.550658in}{1.575000in}}%
\pgfpathlineto{\pgfqpoint{2.549887in}{1.575000in}}%
\pgfpathlineto{\pgfqpoint{2.549117in}{1.575000in}}%
\pgfpathlineto{\pgfqpoint{2.548347in}{1.575000in}}%
\pgfpathlineto{\pgfqpoint{2.547576in}{1.575000in}}%
\pgfpathlineto{\pgfqpoint{2.546806in}{1.575000in}}%
\pgfpathlineto{\pgfqpoint{2.546036in}{1.575000in}}%
\pgfpathlineto{\pgfqpoint{2.545266in}{1.575000in}}%
\pgfpathlineto{\pgfqpoint{2.544495in}{1.575000in}}%
\pgfpathlineto{\pgfqpoint{2.543725in}{1.575000in}}%
\pgfpathlineto{\pgfqpoint{2.542955in}{1.575000in}}%
\pgfpathlineto{\pgfqpoint{2.542184in}{1.575000in}}%
\pgfpathlineto{\pgfqpoint{2.541414in}{1.575000in}}%
\pgfpathlineto{\pgfqpoint{2.540644in}{1.575000in}}%
\pgfpathlineto{\pgfqpoint{2.539873in}{1.575000in}}%
\pgfpathlineto{\pgfqpoint{2.539103in}{1.575000in}}%
\pgfpathlineto{\pgfqpoint{2.538333in}{1.575000in}}%
\pgfpathlineto{\pgfqpoint{2.537563in}{1.575000in}}%
\pgfpathlineto{\pgfqpoint{2.536792in}{1.575000in}}%
\pgfpathlineto{\pgfqpoint{2.536022in}{1.575000in}}%
\pgfpathlineto{\pgfqpoint{2.535252in}{1.575000in}}%
\pgfpathlineto{\pgfqpoint{2.534481in}{1.575000in}}%
\pgfpathlineto{\pgfqpoint{2.533711in}{1.575000in}}%
\pgfpathlineto{\pgfqpoint{2.532941in}{1.575000in}}%
\pgfpathlineto{\pgfqpoint{2.532171in}{1.575000in}}%
\pgfpathlineto{\pgfqpoint{2.531400in}{1.575000in}}%
\pgfpathlineto{\pgfqpoint{2.530630in}{1.575000in}}%
\pgfpathlineto{\pgfqpoint{2.529860in}{1.575000in}}%
\pgfpathlineto{\pgfqpoint{2.529089in}{1.575000in}}%
\pgfpathlineto{\pgfqpoint{2.528319in}{1.575000in}}%
\pgfpathlineto{\pgfqpoint{2.527549in}{1.575000in}}%
\pgfpathlineto{\pgfqpoint{2.526779in}{1.575000in}}%
\pgfpathlineto{\pgfqpoint{2.526008in}{1.575000in}}%
\pgfpathlineto{\pgfqpoint{2.525238in}{1.575000in}}%
\pgfpathlineto{\pgfqpoint{2.524468in}{1.575000in}}%
\pgfpathlineto{\pgfqpoint{2.523697in}{1.575000in}}%
\pgfpathlineto{\pgfqpoint{2.522927in}{1.575000in}}%
\pgfpathlineto{\pgfqpoint{2.522157in}{1.575000in}}%
\pgfpathlineto{\pgfqpoint{2.521387in}{1.575000in}}%
\pgfpathlineto{\pgfqpoint{2.520616in}{1.575000in}}%
\pgfpathlineto{\pgfqpoint{2.519846in}{1.575000in}}%
\pgfpathlineto{\pgfqpoint{2.519076in}{1.575000in}}%
\pgfpathlineto{\pgfqpoint{2.518305in}{1.575000in}}%
\pgfpathlineto{\pgfqpoint{2.517535in}{1.575000in}}%
\pgfpathlineto{\pgfqpoint{2.516765in}{1.575000in}}%
\pgfpathlineto{\pgfqpoint{2.515994in}{1.575000in}}%
\pgfpathlineto{\pgfqpoint{2.515224in}{1.575000in}}%
\pgfpathlineto{\pgfqpoint{2.514454in}{1.575000in}}%
\pgfpathlineto{\pgfqpoint{2.513684in}{1.575000in}}%
\pgfpathlineto{\pgfqpoint{2.512913in}{1.575000in}}%
\pgfpathlineto{\pgfqpoint{2.512143in}{1.575000in}}%
\pgfpathlineto{\pgfqpoint{2.511373in}{1.575000in}}%
\pgfpathlineto{\pgfqpoint{2.510602in}{1.575000in}}%
\pgfpathlineto{\pgfqpoint{2.509832in}{1.575000in}}%
\pgfpathlineto{\pgfqpoint{2.509062in}{1.575000in}}%
\pgfpathlineto{\pgfqpoint{2.508292in}{1.575000in}}%
\pgfpathlineto{\pgfqpoint{2.507521in}{1.575000in}}%
\pgfpathlineto{\pgfqpoint{2.506751in}{1.575000in}}%
\pgfpathlineto{\pgfqpoint{2.505981in}{1.575000in}}%
\pgfpathlineto{\pgfqpoint{2.505210in}{1.575000in}}%
\pgfpathlineto{\pgfqpoint{2.504440in}{1.575000in}}%
\pgfpathlineto{\pgfqpoint{2.503670in}{1.575000in}}%
\pgfpathlineto{\pgfqpoint{2.502900in}{1.575000in}}%
\pgfpathlineto{\pgfqpoint{2.502129in}{1.575000in}}%
\pgfpathlineto{\pgfqpoint{2.501359in}{1.575000in}}%
\pgfpathlineto{\pgfqpoint{2.500589in}{1.575000in}}%
\pgfpathlineto{\pgfqpoint{2.499818in}{1.575000in}}%
\pgfpathlineto{\pgfqpoint{2.499048in}{1.575000in}}%
\pgfpathlineto{\pgfqpoint{2.498278in}{1.575000in}}%
\pgfpathlineto{\pgfqpoint{2.497508in}{1.575000in}}%
\pgfpathlineto{\pgfqpoint{2.496737in}{1.575000in}}%
\pgfpathlineto{\pgfqpoint{2.495967in}{1.575000in}}%
\pgfpathlineto{\pgfqpoint{2.495197in}{1.575000in}}%
\pgfpathlineto{\pgfqpoint{2.494426in}{1.575000in}}%
\pgfpathlineto{\pgfqpoint{2.493656in}{1.575000in}}%
\pgfpathlineto{\pgfqpoint{2.492886in}{1.575000in}}%
\pgfpathlineto{\pgfqpoint{2.492115in}{1.575000in}}%
\pgfpathlineto{\pgfqpoint{2.491345in}{1.575000in}}%
\pgfpathlineto{\pgfqpoint{2.490575in}{1.575000in}}%
\pgfpathlineto{\pgfqpoint{2.489805in}{1.575000in}}%
\pgfpathlineto{\pgfqpoint{2.489034in}{1.575000in}}%
\pgfpathlineto{\pgfqpoint{2.488264in}{1.575000in}}%
\pgfpathlineto{\pgfqpoint{2.487494in}{1.575000in}}%
\pgfpathlineto{\pgfqpoint{2.486723in}{1.575000in}}%
\pgfpathlineto{\pgfqpoint{2.485953in}{1.575000in}}%
\pgfpathlineto{\pgfqpoint{2.485183in}{1.575000in}}%
\pgfpathlineto{\pgfqpoint{2.484413in}{1.575000in}}%
\pgfpathlineto{\pgfqpoint{2.483642in}{1.575000in}}%
\pgfpathlineto{\pgfqpoint{2.482872in}{1.575000in}}%
\pgfpathlineto{\pgfqpoint{2.482102in}{1.575000in}}%
\pgfpathlineto{\pgfqpoint{2.481331in}{1.575000in}}%
\pgfpathlineto{\pgfqpoint{2.480561in}{1.575000in}}%
\pgfpathlineto{\pgfqpoint{2.479791in}{1.575000in}}%
\pgfpathlineto{\pgfqpoint{2.479021in}{1.575000in}}%
\pgfpathlineto{\pgfqpoint{2.478250in}{1.575000in}}%
\pgfpathlineto{\pgfqpoint{2.477480in}{1.575000in}}%
\pgfpathlineto{\pgfqpoint{2.476710in}{1.575000in}}%
\pgfpathlineto{\pgfqpoint{2.475939in}{1.575000in}}%
\pgfpathlineto{\pgfqpoint{2.475169in}{1.575000in}}%
\pgfpathlineto{\pgfqpoint{2.474399in}{1.575000in}}%
\pgfpathlineto{\pgfqpoint{2.473629in}{1.575000in}}%
\pgfpathlineto{\pgfqpoint{2.472858in}{1.575000in}}%
\pgfpathlineto{\pgfqpoint{2.472088in}{1.575000in}}%
\pgfpathlineto{\pgfqpoint{2.471318in}{1.575000in}}%
\pgfpathlineto{\pgfqpoint{2.470547in}{1.575000in}}%
\pgfpathlineto{\pgfqpoint{2.469777in}{1.575000in}}%
\pgfpathlineto{\pgfqpoint{2.469007in}{1.575000in}}%
\pgfpathlineto{\pgfqpoint{2.468236in}{1.575000in}}%
\pgfpathlineto{\pgfqpoint{2.467466in}{1.575000in}}%
\pgfpathlineto{\pgfqpoint{2.466696in}{1.575000in}}%
\pgfpathlineto{\pgfqpoint{2.465926in}{1.575000in}}%
\pgfpathlineto{\pgfqpoint{2.465155in}{1.575000in}}%
\pgfpathlineto{\pgfqpoint{2.464385in}{1.575000in}}%
\pgfpathlineto{\pgfqpoint{2.463615in}{1.575000in}}%
\pgfpathlineto{\pgfqpoint{2.462844in}{1.575000in}}%
\pgfpathlineto{\pgfqpoint{2.462074in}{1.575000in}}%
\pgfpathlineto{\pgfqpoint{2.461304in}{1.575000in}}%
\pgfpathlineto{\pgfqpoint{2.460534in}{1.575000in}}%
\pgfpathlineto{\pgfqpoint{2.459763in}{1.575000in}}%
\pgfpathlineto{\pgfqpoint{2.458993in}{1.575000in}}%
\pgfpathlineto{\pgfqpoint{2.458223in}{1.575000in}}%
\pgfpathlineto{\pgfqpoint{2.457452in}{1.575000in}}%
\pgfpathlineto{\pgfqpoint{2.456682in}{1.575000in}}%
\pgfpathlineto{\pgfqpoint{2.455912in}{1.575000in}}%
\pgfpathlineto{\pgfqpoint{2.455142in}{1.575000in}}%
\pgfpathlineto{\pgfqpoint{2.454371in}{1.575000in}}%
\pgfpathlineto{\pgfqpoint{2.453601in}{1.575000in}}%
\pgfpathlineto{\pgfqpoint{2.452831in}{1.575000in}}%
\pgfpathlineto{\pgfqpoint{2.452060in}{1.575000in}}%
\pgfpathlineto{\pgfqpoint{2.451290in}{1.575000in}}%
\pgfpathlineto{\pgfqpoint{2.450520in}{1.575000in}}%
\pgfpathlineto{\pgfqpoint{2.449749in}{1.575000in}}%
\pgfpathlineto{\pgfqpoint{2.448979in}{1.575000in}}%
\pgfpathlineto{\pgfqpoint{2.448209in}{1.575000in}}%
\pgfpathlineto{\pgfqpoint{2.447439in}{1.575000in}}%
\pgfpathlineto{\pgfqpoint{2.446668in}{1.575000in}}%
\pgfpathlineto{\pgfqpoint{2.445898in}{1.575000in}}%
\pgfpathlineto{\pgfqpoint{2.445128in}{1.575000in}}%
\pgfpathlineto{\pgfqpoint{2.444357in}{1.575000in}}%
\pgfpathlineto{\pgfqpoint{2.443587in}{1.575000in}}%
\pgfpathlineto{\pgfqpoint{2.442817in}{1.575000in}}%
\pgfpathlineto{\pgfqpoint{2.442047in}{1.575000in}}%
\pgfpathlineto{\pgfqpoint{2.441276in}{1.575000in}}%
\pgfpathlineto{\pgfqpoint{2.440506in}{1.575000in}}%
\pgfpathlineto{\pgfqpoint{2.439736in}{1.575000in}}%
\pgfpathlineto{\pgfqpoint{2.438965in}{1.575000in}}%
\pgfpathlineto{\pgfqpoint{2.438195in}{1.575000in}}%
\pgfpathlineto{\pgfqpoint{2.437425in}{1.575000in}}%
\pgfpathlineto{\pgfqpoint{2.436655in}{1.575000in}}%
\pgfpathlineto{\pgfqpoint{2.435884in}{1.575000in}}%
\pgfpathlineto{\pgfqpoint{2.435114in}{1.575000in}}%
\pgfpathlineto{\pgfqpoint{2.434344in}{1.575000in}}%
\pgfpathlineto{\pgfqpoint{2.433573in}{1.575000in}}%
\pgfpathlineto{\pgfqpoint{2.432803in}{1.575000in}}%
\pgfpathlineto{\pgfqpoint{2.432033in}{1.575000in}}%
\pgfpathlineto{\pgfqpoint{2.431263in}{1.575000in}}%
\pgfpathlineto{\pgfqpoint{2.430492in}{1.575000in}}%
\pgfpathlineto{\pgfqpoint{2.429722in}{1.575000in}}%
\pgfpathlineto{\pgfqpoint{2.428952in}{1.575000in}}%
\pgfpathlineto{\pgfqpoint{2.428181in}{1.575000in}}%
\pgfpathlineto{\pgfqpoint{2.427411in}{1.575000in}}%
\pgfpathlineto{\pgfqpoint{2.426641in}{1.575000in}}%
\pgfpathlineto{\pgfqpoint{2.425870in}{1.575000in}}%
\pgfpathlineto{\pgfqpoint{2.425100in}{1.575000in}}%
\pgfpathlineto{\pgfqpoint{2.424330in}{1.575000in}}%
\pgfpathlineto{\pgfqpoint{2.423560in}{1.575000in}}%
\pgfpathlineto{\pgfqpoint{2.422789in}{1.575000in}}%
\pgfpathlineto{\pgfqpoint{2.422019in}{1.575000in}}%
\pgfpathlineto{\pgfqpoint{2.421249in}{1.575000in}}%
\pgfpathlineto{\pgfqpoint{2.420478in}{1.575000in}}%
\pgfpathlineto{\pgfqpoint{2.419708in}{1.575000in}}%
\pgfpathlineto{\pgfqpoint{2.418938in}{1.575000in}}%
\pgfpathlineto{\pgfqpoint{2.418168in}{1.575000in}}%
\pgfpathlineto{\pgfqpoint{2.417397in}{1.575000in}}%
\pgfpathlineto{\pgfqpoint{2.416627in}{1.575000in}}%
\pgfpathlineto{\pgfqpoint{2.415857in}{1.575000in}}%
\pgfpathlineto{\pgfqpoint{2.415086in}{1.575000in}}%
\pgfpathlineto{\pgfqpoint{2.414316in}{1.575000in}}%
\pgfpathlineto{\pgfqpoint{2.413546in}{1.575000in}}%
\pgfpathlineto{\pgfqpoint{2.412776in}{1.575000in}}%
\pgfpathlineto{\pgfqpoint{2.412005in}{1.575000in}}%
\pgfpathlineto{\pgfqpoint{2.411235in}{1.575000in}}%
\pgfpathlineto{\pgfqpoint{2.410465in}{1.575000in}}%
\pgfpathlineto{\pgfqpoint{2.409694in}{1.575000in}}%
\pgfpathlineto{\pgfqpoint{2.408924in}{1.575000in}}%
\pgfpathlineto{\pgfqpoint{2.408154in}{1.575000in}}%
\pgfpathlineto{\pgfqpoint{2.407384in}{1.575000in}}%
\pgfpathlineto{\pgfqpoint{2.406613in}{1.575000in}}%
\pgfpathlineto{\pgfqpoint{2.405843in}{1.575000in}}%
\pgfpathlineto{\pgfqpoint{2.405073in}{1.575000in}}%
\pgfpathlineto{\pgfqpoint{2.404302in}{1.575000in}}%
\pgfpathlineto{\pgfqpoint{2.403532in}{1.575000in}}%
\pgfpathlineto{\pgfqpoint{2.402762in}{1.575000in}}%
\pgfpathlineto{\pgfqpoint{2.401991in}{1.575000in}}%
\pgfpathlineto{\pgfqpoint{2.401221in}{1.575000in}}%
\pgfpathlineto{\pgfqpoint{2.400451in}{1.575000in}}%
\pgfpathlineto{\pgfqpoint{2.399681in}{1.575000in}}%
\pgfpathlineto{\pgfqpoint{2.398910in}{1.575000in}}%
\pgfpathlineto{\pgfqpoint{2.398140in}{1.575000in}}%
\pgfpathlineto{\pgfqpoint{2.397370in}{1.575000in}}%
\pgfpathlineto{\pgfqpoint{2.396599in}{1.575000in}}%
\pgfpathlineto{\pgfqpoint{2.395829in}{1.575000in}}%
\pgfpathlineto{\pgfqpoint{2.395059in}{1.575000in}}%
\pgfpathlineto{\pgfqpoint{2.394289in}{1.575000in}}%
\pgfpathlineto{\pgfqpoint{2.393518in}{1.575000in}}%
\pgfpathlineto{\pgfqpoint{2.392748in}{1.575000in}}%
\pgfpathlineto{\pgfqpoint{2.391978in}{1.575000in}}%
\pgfpathlineto{\pgfqpoint{2.391207in}{1.575000in}}%
\pgfpathlineto{\pgfqpoint{2.390437in}{1.575000in}}%
\pgfpathlineto{\pgfqpoint{2.389667in}{1.575000in}}%
\pgfpathlineto{\pgfqpoint{2.388897in}{1.575000in}}%
\pgfpathlineto{\pgfqpoint{2.388126in}{1.575000in}}%
\pgfpathlineto{\pgfqpoint{2.387356in}{1.575000in}}%
\pgfpathlineto{\pgfqpoint{2.386586in}{1.575000in}}%
\pgfpathlineto{\pgfqpoint{2.385815in}{1.575000in}}%
\pgfpathlineto{\pgfqpoint{2.385045in}{1.575000in}}%
\pgfpathlineto{\pgfqpoint{2.384275in}{1.575000in}}%
\pgfpathlineto{\pgfqpoint{2.383505in}{1.575000in}}%
\pgfpathlineto{\pgfqpoint{2.382734in}{1.575000in}}%
\pgfpathlineto{\pgfqpoint{2.381964in}{1.575000in}}%
\pgfpathlineto{\pgfqpoint{2.381194in}{1.575000in}}%
\pgfpathlineto{\pgfqpoint{2.380423in}{1.575000in}}%
\pgfpathlineto{\pgfqpoint{2.379653in}{1.575000in}}%
\pgfpathlineto{\pgfqpoint{2.378883in}{1.575000in}}%
\pgfpathlineto{\pgfqpoint{2.378112in}{1.575000in}}%
\pgfpathlineto{\pgfqpoint{2.377342in}{1.575000in}}%
\pgfpathlineto{\pgfqpoint{2.376572in}{1.575000in}}%
\pgfpathlineto{\pgfqpoint{2.375802in}{1.575000in}}%
\pgfpathlineto{\pgfqpoint{2.375031in}{1.575000in}}%
\pgfpathlineto{\pgfqpoint{2.374261in}{1.575000in}}%
\pgfpathlineto{\pgfqpoint{2.373491in}{1.575000in}}%
\pgfpathlineto{\pgfqpoint{2.372720in}{1.575000in}}%
\pgfpathlineto{\pgfqpoint{2.371950in}{1.575000in}}%
\pgfpathlineto{\pgfqpoint{2.371180in}{1.575000in}}%
\pgfpathlineto{\pgfqpoint{2.370410in}{1.575000in}}%
\pgfpathlineto{\pgfqpoint{2.369639in}{1.575000in}}%
\pgfpathlineto{\pgfqpoint{2.368869in}{1.575000in}}%
\pgfpathlineto{\pgfqpoint{2.368099in}{1.575000in}}%
\pgfpathlineto{\pgfqpoint{2.367328in}{1.575000in}}%
\pgfpathlineto{\pgfqpoint{2.366558in}{1.575000in}}%
\pgfpathlineto{\pgfqpoint{2.365788in}{1.575000in}}%
\pgfpathlineto{\pgfqpoint{2.365018in}{1.575000in}}%
\pgfpathlineto{\pgfqpoint{2.364247in}{1.575000in}}%
\pgfpathlineto{\pgfqpoint{2.363477in}{1.575000in}}%
\pgfpathlineto{\pgfqpoint{2.362707in}{1.575000in}}%
\pgfpathlineto{\pgfqpoint{2.361936in}{1.575000in}}%
\pgfpathlineto{\pgfqpoint{2.361166in}{1.575000in}}%
\pgfpathlineto{\pgfqpoint{2.360396in}{1.575000in}}%
\pgfpathlineto{\pgfqpoint{2.359626in}{1.575000in}}%
\pgfpathlineto{\pgfqpoint{2.358855in}{1.575000in}}%
\pgfpathlineto{\pgfqpoint{2.358085in}{1.575000in}}%
\pgfpathlineto{\pgfqpoint{2.357315in}{1.575000in}}%
\pgfpathlineto{\pgfqpoint{2.356544in}{1.575000in}}%
\pgfpathlineto{\pgfqpoint{2.355774in}{1.575000in}}%
\pgfpathlineto{\pgfqpoint{2.355004in}{1.575000in}}%
\pgfpathlineto{\pgfqpoint{2.354233in}{1.575000in}}%
\pgfpathlineto{\pgfqpoint{2.353463in}{1.575000in}}%
\pgfpathlineto{\pgfqpoint{2.352693in}{1.575000in}}%
\pgfpathlineto{\pgfqpoint{2.351923in}{1.575000in}}%
\pgfpathlineto{\pgfqpoint{2.351152in}{1.575000in}}%
\pgfpathlineto{\pgfqpoint{2.350382in}{1.575000in}}%
\pgfpathlineto{\pgfqpoint{2.349612in}{1.575000in}}%
\pgfpathlineto{\pgfqpoint{2.348841in}{1.575000in}}%
\pgfpathlineto{\pgfqpoint{2.348071in}{1.575000in}}%
\pgfpathlineto{\pgfqpoint{2.347301in}{1.575000in}}%
\pgfpathlineto{\pgfqpoint{2.346531in}{1.575000in}}%
\pgfpathlineto{\pgfqpoint{2.345760in}{1.575000in}}%
\pgfpathlineto{\pgfqpoint{2.344990in}{1.575000in}}%
\pgfpathlineto{\pgfqpoint{2.344220in}{1.575000in}}%
\pgfpathlineto{\pgfqpoint{2.343449in}{1.575000in}}%
\pgfpathlineto{\pgfqpoint{2.342679in}{1.575000in}}%
\pgfpathlineto{\pgfqpoint{2.341909in}{1.575000in}}%
\pgfpathlineto{\pgfqpoint{2.341139in}{1.575000in}}%
\pgfpathlineto{\pgfqpoint{2.340368in}{1.575000in}}%
\pgfpathlineto{\pgfqpoint{2.339598in}{1.575000in}}%
\pgfpathlineto{\pgfqpoint{2.338828in}{1.575000in}}%
\pgfpathlineto{\pgfqpoint{2.338057in}{1.575000in}}%
\pgfpathlineto{\pgfqpoint{2.337287in}{1.575000in}}%
\pgfpathlineto{\pgfqpoint{2.336517in}{1.575000in}}%
\pgfpathlineto{\pgfqpoint{2.335746in}{1.575000in}}%
\pgfpathlineto{\pgfqpoint{2.334976in}{1.575000in}}%
\pgfpathlineto{\pgfqpoint{2.334206in}{1.575000in}}%
\pgfpathlineto{\pgfqpoint{2.333436in}{1.575000in}}%
\pgfpathlineto{\pgfqpoint{2.332665in}{1.575000in}}%
\pgfpathlineto{\pgfqpoint{2.331895in}{1.575000in}}%
\pgfpathlineto{\pgfqpoint{2.331125in}{1.575000in}}%
\pgfpathlineto{\pgfqpoint{2.330354in}{1.575000in}}%
\pgfpathlineto{\pgfqpoint{2.329584in}{1.575000in}}%
\pgfpathlineto{\pgfqpoint{2.328814in}{1.575000in}}%
\pgfpathlineto{\pgfqpoint{2.328044in}{1.575000in}}%
\pgfpathlineto{\pgfqpoint{2.327273in}{1.575000in}}%
\pgfpathlineto{\pgfqpoint{2.326503in}{1.575000in}}%
\pgfpathlineto{\pgfqpoint{2.325733in}{1.575000in}}%
\pgfpathlineto{\pgfqpoint{2.324962in}{1.575000in}}%
\pgfpathlineto{\pgfqpoint{2.324192in}{1.575000in}}%
\pgfpathlineto{\pgfqpoint{2.323422in}{1.575000in}}%
\pgfpathlineto{\pgfqpoint{2.322652in}{1.575000in}}%
\pgfpathlineto{\pgfqpoint{2.321881in}{1.575000in}}%
\pgfpathlineto{\pgfqpoint{2.321111in}{1.575000in}}%
\pgfpathlineto{\pgfqpoint{2.320341in}{1.575000in}}%
\pgfpathlineto{\pgfqpoint{2.319570in}{1.575000in}}%
\pgfpathlineto{\pgfqpoint{2.318800in}{1.575000in}}%
\pgfpathlineto{\pgfqpoint{2.318030in}{1.575000in}}%
\pgfpathlineto{\pgfqpoint{2.317260in}{1.575000in}}%
\pgfpathlineto{\pgfqpoint{2.316489in}{1.575000in}}%
\pgfpathlineto{\pgfqpoint{2.315719in}{1.575000in}}%
\pgfpathlineto{\pgfqpoint{2.314949in}{1.575000in}}%
\pgfpathlineto{\pgfqpoint{2.314178in}{1.575000in}}%
\pgfpathlineto{\pgfqpoint{2.313408in}{1.575000in}}%
\pgfpathlineto{\pgfqpoint{2.312638in}{1.575000in}}%
\pgfpathlineto{\pgfqpoint{2.311867in}{1.575000in}}%
\pgfpathlineto{\pgfqpoint{2.311097in}{1.575000in}}%
\pgfpathlineto{\pgfqpoint{2.310327in}{1.575000in}}%
\pgfpathlineto{\pgfqpoint{2.309557in}{1.575000in}}%
\pgfpathlineto{\pgfqpoint{2.308786in}{1.575000in}}%
\pgfpathlineto{\pgfqpoint{2.308016in}{1.575000in}}%
\pgfpathlineto{\pgfqpoint{2.307246in}{1.575000in}}%
\pgfpathlineto{\pgfqpoint{2.306475in}{1.575000in}}%
\pgfpathlineto{\pgfqpoint{2.305705in}{1.575000in}}%
\pgfpathlineto{\pgfqpoint{2.304935in}{1.575000in}}%
\pgfpathlineto{\pgfqpoint{2.304165in}{1.575000in}}%
\pgfpathlineto{\pgfqpoint{2.303394in}{1.575000in}}%
\pgfpathlineto{\pgfqpoint{2.302624in}{1.575000in}}%
\pgfpathlineto{\pgfqpoint{2.301854in}{1.575000in}}%
\pgfpathlineto{\pgfqpoint{2.301083in}{1.575000in}}%
\pgfpathlineto{\pgfqpoint{2.300313in}{1.575000in}}%
\pgfpathlineto{\pgfqpoint{2.299543in}{1.575000in}}%
\pgfpathlineto{\pgfqpoint{2.298773in}{1.575000in}}%
\pgfpathlineto{\pgfqpoint{2.298002in}{1.575000in}}%
\pgfpathlineto{\pgfqpoint{2.297232in}{1.575000in}}%
\pgfpathlineto{\pgfqpoint{2.296462in}{1.575000in}}%
\pgfpathlineto{\pgfqpoint{2.295691in}{1.575000in}}%
\pgfpathlineto{\pgfqpoint{2.294921in}{1.575000in}}%
\pgfpathlineto{\pgfqpoint{2.294151in}{1.575000in}}%
\pgfpathlineto{\pgfqpoint{2.293381in}{1.575000in}}%
\pgfpathlineto{\pgfqpoint{2.292610in}{1.575000in}}%
\pgfpathlineto{\pgfqpoint{2.291840in}{1.575000in}}%
\pgfpathlineto{\pgfqpoint{2.291070in}{1.575000in}}%
\pgfpathlineto{\pgfqpoint{2.290299in}{1.575000in}}%
\pgfpathlineto{\pgfqpoint{2.289529in}{1.575000in}}%
\pgfpathlineto{\pgfqpoint{2.288759in}{1.575000in}}%
\pgfpathlineto{\pgfqpoint{2.287988in}{1.575000in}}%
\pgfpathlineto{\pgfqpoint{2.287218in}{1.575000in}}%
\pgfpathlineto{\pgfqpoint{2.286448in}{1.575000in}}%
\pgfpathlineto{\pgfqpoint{2.285678in}{1.575000in}}%
\pgfpathlineto{\pgfqpoint{2.284907in}{1.575000in}}%
\pgfpathlineto{\pgfqpoint{2.284137in}{1.575000in}}%
\pgfpathlineto{\pgfqpoint{2.283367in}{1.575000in}}%
\pgfpathlineto{\pgfqpoint{2.282596in}{1.575000in}}%
\pgfpathlineto{\pgfqpoint{2.281826in}{1.575000in}}%
\pgfpathlineto{\pgfqpoint{2.281056in}{1.575000in}}%
\pgfpathlineto{\pgfqpoint{2.280286in}{1.575000in}}%
\pgfpathlineto{\pgfqpoint{2.279515in}{1.575000in}}%
\pgfpathlineto{\pgfqpoint{2.278745in}{1.575000in}}%
\pgfpathlineto{\pgfqpoint{2.277975in}{1.575000in}}%
\pgfpathlineto{\pgfqpoint{2.277204in}{1.575000in}}%
\pgfpathlineto{\pgfqpoint{2.276434in}{1.575000in}}%
\pgfpathlineto{\pgfqpoint{2.275664in}{1.575000in}}%
\pgfpathlineto{\pgfqpoint{2.274894in}{1.575000in}}%
\pgfpathlineto{\pgfqpoint{2.274123in}{1.575000in}}%
\pgfpathlineto{\pgfqpoint{2.273353in}{1.575000in}}%
\pgfpathlineto{\pgfqpoint{2.272583in}{1.575000in}}%
\pgfpathlineto{\pgfqpoint{2.271812in}{1.575000in}}%
\pgfpathlineto{\pgfqpoint{2.271042in}{1.575000in}}%
\pgfpathlineto{\pgfqpoint{2.270272in}{1.575000in}}%
\pgfpathlineto{\pgfqpoint{2.269502in}{1.575000in}}%
\pgfpathlineto{\pgfqpoint{2.268731in}{1.575000in}}%
\pgfpathlineto{\pgfqpoint{2.267961in}{1.575000in}}%
\pgfpathlineto{\pgfqpoint{2.267191in}{1.575000in}}%
\pgfpathlineto{\pgfqpoint{2.266420in}{1.575000in}}%
\pgfpathlineto{\pgfqpoint{2.265650in}{1.575000in}}%
\pgfpathlineto{\pgfqpoint{2.264880in}{1.575000in}}%
\pgfpathlineto{\pgfqpoint{2.264109in}{1.575000in}}%
\pgfpathlineto{\pgfqpoint{2.263339in}{1.575000in}}%
\pgfpathlineto{\pgfqpoint{2.262569in}{1.575000in}}%
\pgfpathlineto{\pgfqpoint{2.261799in}{1.575000in}}%
\pgfpathlineto{\pgfqpoint{2.261028in}{1.575000in}}%
\pgfpathlineto{\pgfqpoint{2.260258in}{1.575000in}}%
\pgfpathlineto{\pgfqpoint{2.259488in}{1.575000in}}%
\pgfpathlineto{\pgfqpoint{2.258717in}{1.575000in}}%
\pgfpathlineto{\pgfqpoint{2.257947in}{1.575000in}}%
\pgfpathlineto{\pgfqpoint{2.257177in}{1.575000in}}%
\pgfpathlineto{\pgfqpoint{2.256407in}{1.575000in}}%
\pgfpathlineto{\pgfqpoint{2.255636in}{1.575000in}}%
\pgfpathlineto{\pgfqpoint{2.254866in}{1.575000in}}%
\pgfpathlineto{\pgfqpoint{2.254096in}{1.575000in}}%
\pgfpathlineto{\pgfqpoint{2.253325in}{1.575000in}}%
\pgfpathlineto{\pgfqpoint{2.252555in}{1.575000in}}%
\pgfpathlineto{\pgfqpoint{2.251785in}{1.575000in}}%
\pgfpathlineto{\pgfqpoint{2.251015in}{1.575000in}}%
\pgfpathlineto{\pgfqpoint{2.250244in}{1.575000in}}%
\pgfpathlineto{\pgfqpoint{2.249474in}{1.575000in}}%
\pgfpathlineto{\pgfqpoint{2.248704in}{1.575000in}}%
\pgfpathlineto{\pgfqpoint{2.247933in}{1.575000in}}%
\pgfpathlineto{\pgfqpoint{2.247163in}{1.575000in}}%
\pgfpathlineto{\pgfqpoint{2.246393in}{1.575000in}}%
\pgfpathlineto{\pgfqpoint{2.245622in}{1.575000in}}%
\pgfpathlineto{\pgfqpoint{2.244852in}{1.575000in}}%
\pgfpathlineto{\pgfqpoint{2.244082in}{1.575000in}}%
\pgfpathlineto{\pgfqpoint{2.243312in}{1.575000in}}%
\pgfpathlineto{\pgfqpoint{2.242541in}{1.575000in}}%
\pgfpathlineto{\pgfqpoint{2.241771in}{1.575000in}}%
\pgfpathlineto{\pgfqpoint{2.241001in}{1.575000in}}%
\pgfpathlineto{\pgfqpoint{2.240230in}{1.575000in}}%
\pgfpathlineto{\pgfqpoint{2.239460in}{1.575000in}}%
\pgfpathlineto{\pgfqpoint{2.238690in}{1.575000in}}%
\pgfpathlineto{\pgfqpoint{2.237920in}{1.575000in}}%
\pgfpathlineto{\pgfqpoint{2.237149in}{1.575000in}}%
\pgfpathlineto{\pgfqpoint{2.236379in}{1.575000in}}%
\pgfpathlineto{\pgfqpoint{2.235609in}{1.575000in}}%
\pgfpathlineto{\pgfqpoint{2.234838in}{1.575000in}}%
\pgfpathlineto{\pgfqpoint{2.234068in}{1.575000in}}%
\pgfpathlineto{\pgfqpoint{2.233298in}{1.575000in}}%
\pgfpathlineto{\pgfqpoint{2.232528in}{1.575000in}}%
\pgfpathlineto{\pgfqpoint{2.231757in}{1.575000in}}%
\pgfpathlineto{\pgfqpoint{2.230987in}{1.575000in}}%
\pgfpathlineto{\pgfqpoint{2.230217in}{1.575000in}}%
\pgfpathlineto{\pgfqpoint{2.229446in}{1.575000in}}%
\pgfpathlineto{\pgfqpoint{2.228676in}{1.575000in}}%
\pgfpathlineto{\pgfqpoint{2.227906in}{1.575000in}}%
\pgfpathlineto{\pgfqpoint{2.227136in}{1.575000in}}%
\pgfpathlineto{\pgfqpoint{2.226365in}{1.575000in}}%
\pgfpathlineto{\pgfqpoint{2.225595in}{1.575000in}}%
\pgfpathlineto{\pgfqpoint{2.224825in}{1.575000in}}%
\pgfpathlineto{\pgfqpoint{2.224054in}{1.575000in}}%
\pgfpathlineto{\pgfqpoint{2.223284in}{1.575000in}}%
\pgfpathlineto{\pgfqpoint{2.222514in}{1.575000in}}%
\pgfpathlineto{\pgfqpoint{2.221743in}{1.575000in}}%
\pgfpathlineto{\pgfqpoint{2.220973in}{1.575000in}}%
\pgfpathlineto{\pgfqpoint{2.220203in}{1.575000in}}%
\pgfpathlineto{\pgfqpoint{2.219433in}{1.575000in}}%
\pgfpathlineto{\pgfqpoint{2.218662in}{1.575000in}}%
\pgfpathlineto{\pgfqpoint{2.217892in}{1.575000in}}%
\pgfpathlineto{\pgfqpoint{2.217122in}{1.575000in}}%
\pgfpathlineto{\pgfqpoint{2.216351in}{1.575000in}}%
\pgfpathlineto{\pgfqpoint{2.215581in}{1.575000in}}%
\pgfpathlineto{\pgfqpoint{2.214811in}{1.575000in}}%
\pgfpathlineto{\pgfqpoint{2.214041in}{1.575000in}}%
\pgfpathlineto{\pgfqpoint{2.213270in}{1.575000in}}%
\pgfpathlineto{\pgfqpoint{2.212500in}{1.575000in}}%
\pgfpathclose%
\pgfusepath{fill}%
\end{pgfscope}%
\begin{pgfscope}%
\pgfpathrectangle{\pgfqpoint{0.900000in}{0.600000in}}{\pgfqpoint{3.375000in}{1.950000in}} %
\pgfusepath{clip}%
\pgfsetbuttcap%
\pgfsetroundjoin%
\definecolor{currentfill}{rgb}{1.000000,0.000000,1.000000}%
\pgfsetfillcolor{currentfill}%
\pgfsetfillopacity{0.500000}%
\pgfsetlinewidth{0.000000pt}%
\definecolor{currentstroke}{rgb}{1.000000,0.000000,1.000000}%
\pgfsetstrokecolor{currentstroke}%
\pgfsetstrokeopacity{0.500000}%
\pgfsetdash{}{0pt}%
\pgfpathmoveto{\pgfqpoint{1.518750in}{1.575000in}}%
\pgfpathlineto{\pgfqpoint{1.518750in}{1.250871in}}%
\pgfpathlineto{\pgfqpoint{1.520141in}{1.252306in}}%
\pgfpathlineto{\pgfqpoint{1.521532in}{1.253716in}}%
\pgfpathlineto{\pgfqpoint{1.522923in}{1.255101in}}%
\pgfpathlineto{\pgfqpoint{1.524314in}{1.256461in}}%
\pgfpathlineto{\pgfqpoint{1.525705in}{1.257796in}}%
\pgfpathlineto{\pgfqpoint{1.527096in}{1.259108in}}%
\pgfpathlineto{\pgfqpoint{1.528487in}{1.260397in}}%
\pgfpathlineto{\pgfqpoint{1.529878in}{1.261662in}}%
\pgfpathlineto{\pgfqpoint{1.531269in}{1.262905in}}%
\pgfpathlineto{\pgfqpoint{1.532660in}{1.264125in}}%
\pgfpathlineto{\pgfqpoint{1.534051in}{1.265324in}}%
\pgfpathlineto{\pgfqpoint{1.535442in}{1.266502in}}%
\pgfpathlineto{\pgfqpoint{1.536833in}{1.267658in}}%
\pgfpathlineto{\pgfqpoint{1.538224in}{1.268794in}}%
\pgfpathlineto{\pgfqpoint{1.539615in}{1.269909in}}%
\pgfpathlineto{\pgfqpoint{1.541007in}{1.271005in}}%
\pgfpathlineto{\pgfqpoint{1.542398in}{1.272080in}}%
\pgfpathlineto{\pgfqpoint{1.543789in}{1.273137in}}%
\pgfpathlineto{\pgfqpoint{1.545180in}{1.274175in}}%
\pgfpathlineto{\pgfqpoint{1.546571in}{1.275194in}}%
\pgfpathlineto{\pgfqpoint{1.547962in}{1.276195in}}%
\pgfpathlineto{\pgfqpoint{1.549353in}{1.277179in}}%
\pgfpathlineto{\pgfqpoint{1.550744in}{1.278144in}}%
\pgfpathlineto{\pgfqpoint{1.552135in}{1.279093in}}%
\pgfpathlineto{\pgfqpoint{1.553526in}{1.280024in}}%
\pgfpathlineto{\pgfqpoint{1.554917in}{1.280939in}}%
\pgfpathlineto{\pgfqpoint{1.556308in}{1.281838in}}%
\pgfpathlineto{\pgfqpoint{1.557699in}{1.282720in}}%
\pgfpathlineto{\pgfqpoint{1.559090in}{1.283587in}}%
\pgfpathlineto{\pgfqpoint{1.560481in}{1.284438in}}%
\pgfpathlineto{\pgfqpoint{1.561872in}{1.285274in}}%
\pgfpathlineto{\pgfqpoint{1.563263in}{1.286095in}}%
\pgfpathlineto{\pgfqpoint{1.564654in}{1.286902in}}%
\pgfpathlineto{\pgfqpoint{1.566045in}{1.287694in}}%
\pgfpathlineto{\pgfqpoint{1.567436in}{1.288472in}}%
\pgfpathlineto{\pgfqpoint{1.568827in}{1.289236in}}%
\pgfpathlineto{\pgfqpoint{1.570218in}{1.289986in}}%
\pgfpathlineto{\pgfqpoint{1.571609in}{1.290723in}}%
\pgfpathlineto{\pgfqpoint{1.573000in}{1.291447in}}%
\pgfpathlineto{\pgfqpoint{1.574391in}{1.292158in}}%
\pgfpathlineto{\pgfqpoint{1.575782in}{1.292856in}}%
\pgfpathlineto{\pgfqpoint{1.577173in}{1.293541in}}%
\pgfpathlineto{\pgfqpoint{1.578564in}{1.294215in}}%
\pgfpathlineto{\pgfqpoint{1.579955in}{1.294876in}}%
\pgfpathlineto{\pgfqpoint{1.581346in}{1.295526in}}%
\pgfpathlineto{\pgfqpoint{1.582737in}{1.296164in}}%
\pgfpathlineto{\pgfqpoint{1.584129in}{1.296790in}}%
\pgfpathlineto{\pgfqpoint{1.585520in}{1.297406in}}%
\pgfpathlineto{\pgfqpoint{1.586911in}{1.298010in}}%
\pgfpathlineto{\pgfqpoint{1.588302in}{1.298604in}}%
\pgfpathlineto{\pgfqpoint{1.589693in}{1.299187in}}%
\pgfpathlineto{\pgfqpoint{1.591084in}{1.299760in}}%
\pgfpathlineto{\pgfqpoint{1.592475in}{1.300322in}}%
\pgfpathlineto{\pgfqpoint{1.593866in}{1.300874in}}%
\pgfpathlineto{\pgfqpoint{1.595257in}{1.301417in}}%
\pgfpathlineto{\pgfqpoint{1.596648in}{1.301950in}}%
\pgfpathlineto{\pgfqpoint{1.598039in}{1.302473in}}%
\pgfpathlineto{\pgfqpoint{1.599430in}{1.302987in}}%
\pgfpathlineto{\pgfqpoint{1.600821in}{1.303492in}}%
\pgfpathlineto{\pgfqpoint{1.602212in}{1.303987in}}%
\pgfpathlineto{\pgfqpoint{1.603603in}{1.304474in}}%
\pgfpathlineto{\pgfqpoint{1.604994in}{1.304953in}}%
\pgfpathlineto{\pgfqpoint{1.606385in}{1.305422in}}%
\pgfpathlineto{\pgfqpoint{1.607776in}{1.305883in}}%
\pgfpathlineto{\pgfqpoint{1.609167in}{1.306336in}}%
\pgfpathlineto{\pgfqpoint{1.610558in}{1.306781in}}%
\pgfpathlineto{\pgfqpoint{1.611949in}{1.307218in}}%
\pgfpathlineto{\pgfqpoint{1.613340in}{1.307648in}}%
\pgfpathlineto{\pgfqpoint{1.614731in}{1.308069in}}%
\pgfpathlineto{\pgfqpoint{1.616122in}{1.308483in}}%
\pgfpathlineto{\pgfqpoint{1.617513in}{1.308890in}}%
\pgfpathlineto{\pgfqpoint{1.618904in}{1.309289in}}%
\pgfpathlineto{\pgfqpoint{1.620295in}{1.309681in}}%
\pgfpathlineto{\pgfqpoint{1.621686in}{1.310066in}}%
\pgfpathlineto{\pgfqpoint{1.623077in}{1.310445in}}%
\pgfpathlineto{\pgfqpoint{1.624468in}{1.310816in}}%
\pgfpathlineto{\pgfqpoint{1.625859in}{1.311181in}}%
\pgfpathlineto{\pgfqpoint{1.627251in}{1.311540in}}%
\pgfpathlineto{\pgfqpoint{1.628642in}{1.311892in}}%
\pgfpathlineto{\pgfqpoint{1.630033in}{1.312237in}}%
\pgfpathlineto{\pgfqpoint{1.631424in}{1.312577in}}%
\pgfpathlineto{\pgfqpoint{1.632815in}{1.312910in}}%
\pgfpathlineto{\pgfqpoint{1.634206in}{1.313238in}}%
\pgfpathlineto{\pgfqpoint{1.635597in}{1.313560in}}%
\pgfpathlineto{\pgfqpoint{1.636988in}{1.313876in}}%
\pgfpathlineto{\pgfqpoint{1.638379in}{1.314186in}}%
\pgfpathlineto{\pgfqpoint{1.639770in}{1.314491in}}%
\pgfpathlineto{\pgfqpoint{1.641161in}{1.314790in}}%
\pgfpathlineto{\pgfqpoint{1.642552in}{1.315084in}}%
\pgfpathlineto{\pgfqpoint{1.643943in}{1.315373in}}%
\pgfpathlineto{\pgfqpoint{1.645334in}{1.315656in}}%
\pgfpathlineto{\pgfqpoint{1.646725in}{1.315935in}}%
\pgfpathlineto{\pgfqpoint{1.648116in}{1.316208in}}%
\pgfpathlineto{\pgfqpoint{1.649507in}{1.316477in}}%
\pgfpathlineto{\pgfqpoint{1.650898in}{1.316741in}}%
\pgfpathlineto{\pgfqpoint{1.652289in}{1.317000in}}%
\pgfpathlineto{\pgfqpoint{1.653680in}{1.317254in}}%
\pgfpathlineto{\pgfqpoint{1.655071in}{1.317504in}}%
\pgfpathlineto{\pgfqpoint{1.656462in}{1.317750in}}%
\pgfpathlineto{\pgfqpoint{1.657853in}{1.317991in}}%
\pgfpathlineto{\pgfqpoint{1.659244in}{1.318228in}}%
\pgfpathlineto{\pgfqpoint{1.660635in}{1.318460in}}%
\pgfpathlineto{\pgfqpoint{1.662026in}{1.318689in}}%
\pgfpathlineto{\pgfqpoint{1.663417in}{1.318913in}}%
\pgfpathlineto{\pgfqpoint{1.664808in}{1.319133in}}%
\pgfpathlineto{\pgfqpoint{1.666199in}{1.319350in}}%
\pgfpathlineto{\pgfqpoint{1.667590in}{1.319562in}}%
\pgfpathlineto{\pgfqpoint{1.668981in}{1.319771in}}%
\pgfpathlineto{\pgfqpoint{1.670372in}{1.319976in}}%
\pgfpathlineto{\pgfqpoint{1.671764in}{1.320177in}}%
\pgfpathlineto{\pgfqpoint{1.673155in}{1.320375in}}%
\pgfpathlineto{\pgfqpoint{1.674546in}{1.320569in}}%
\pgfpathlineto{\pgfqpoint{1.675937in}{1.320760in}}%
\pgfpathlineto{\pgfqpoint{1.677328in}{1.320947in}}%
\pgfpathlineto{\pgfqpoint{1.678719in}{1.321131in}}%
\pgfpathlineto{\pgfqpoint{1.680110in}{1.321312in}}%
\pgfpathlineto{\pgfqpoint{1.681501in}{1.321490in}}%
\pgfpathlineto{\pgfqpoint{1.682892in}{1.321664in}}%
\pgfpathlineto{\pgfqpoint{1.684283in}{1.321835in}}%
\pgfpathlineto{\pgfqpoint{1.685674in}{1.322003in}}%
\pgfpathlineto{\pgfqpoint{1.687065in}{1.322168in}}%
\pgfpathlineto{\pgfqpoint{1.688456in}{1.322331in}}%
\pgfpathlineto{\pgfqpoint{1.689847in}{1.322490in}}%
\pgfpathlineto{\pgfqpoint{1.691238in}{1.322646in}}%
\pgfpathlineto{\pgfqpoint{1.692629in}{1.322800in}}%
\pgfpathlineto{\pgfqpoint{1.694020in}{1.322951in}}%
\pgfpathlineto{\pgfqpoint{1.695411in}{1.323099in}}%
\pgfpathlineto{\pgfqpoint{1.696802in}{1.323245in}}%
\pgfpathlineto{\pgfqpoint{1.698193in}{1.323388in}}%
\pgfpathlineto{\pgfqpoint{1.699584in}{1.323528in}}%
\pgfpathlineto{\pgfqpoint{1.700975in}{1.323666in}}%
\pgfpathlineto{\pgfqpoint{1.702366in}{1.323801in}}%
\pgfpathlineto{\pgfqpoint{1.703757in}{1.323934in}}%
\pgfpathlineto{\pgfqpoint{1.705148in}{1.324065in}}%
\pgfpathlineto{\pgfqpoint{1.706539in}{1.324193in}}%
\pgfpathlineto{\pgfqpoint{1.707930in}{1.324319in}}%
\pgfpathlineto{\pgfqpoint{1.709321in}{1.324443in}}%
\pgfpathlineto{\pgfqpoint{1.710712in}{1.324565in}}%
\pgfpathlineto{\pgfqpoint{1.712103in}{1.324684in}}%
\pgfpathlineto{\pgfqpoint{1.713494in}{1.324801in}}%
\pgfpathlineto{\pgfqpoint{1.714886in}{1.324917in}}%
\pgfpathlineto{\pgfqpoint{1.716277in}{1.325030in}}%
\pgfpathlineto{\pgfqpoint{1.717668in}{1.325141in}}%
\pgfpathlineto{\pgfqpoint{1.719059in}{1.325250in}}%
\pgfpathlineto{\pgfqpoint{1.720450in}{1.325357in}}%
\pgfpathlineto{\pgfqpoint{1.721841in}{1.325462in}}%
\pgfpathlineto{\pgfqpoint{1.723232in}{1.325566in}}%
\pgfpathlineto{\pgfqpoint{1.724623in}{1.325667in}}%
\pgfpathlineto{\pgfqpoint{1.726014in}{1.325767in}}%
\pgfpathlineto{\pgfqpoint{1.727405in}{1.325865in}}%
\pgfpathlineto{\pgfqpoint{1.728796in}{1.325961in}}%
\pgfpathlineto{\pgfqpoint{1.730187in}{1.326056in}}%
\pgfpathlineto{\pgfqpoint{1.731578in}{1.326148in}}%
\pgfpathlineto{\pgfqpoint{1.732969in}{1.326239in}}%
\pgfpathlineto{\pgfqpoint{1.734360in}{1.326329in}}%
\pgfpathlineto{\pgfqpoint{1.735751in}{1.326417in}}%
\pgfpathlineto{\pgfqpoint{1.737142in}{1.326503in}}%
\pgfpathlineto{\pgfqpoint{1.738533in}{1.326588in}}%
\pgfpathlineto{\pgfqpoint{1.739924in}{1.326671in}}%
\pgfpathlineto{\pgfqpoint{1.741315in}{1.326753in}}%
\pgfpathlineto{\pgfqpoint{1.742706in}{1.326833in}}%
\pgfpathlineto{\pgfqpoint{1.744097in}{1.326912in}}%
\pgfpathlineto{\pgfqpoint{1.745488in}{1.326990in}}%
\pgfpathlineto{\pgfqpoint{1.746879in}{1.327066in}}%
\pgfpathlineto{\pgfqpoint{1.748270in}{1.327140in}}%
\pgfpathlineto{\pgfqpoint{1.749661in}{1.327214in}}%
\pgfpathlineto{\pgfqpoint{1.751052in}{1.327286in}}%
\pgfpathlineto{\pgfqpoint{1.752443in}{1.327357in}}%
\pgfpathlineto{\pgfqpoint{1.753834in}{1.327426in}}%
\pgfpathlineto{\pgfqpoint{1.755225in}{1.327495in}}%
\pgfpathlineto{\pgfqpoint{1.756616in}{1.327562in}}%
\pgfpathlineto{\pgfqpoint{1.758008in}{1.327627in}}%
\pgfpathlineto{\pgfqpoint{1.759399in}{1.327692in}}%
\pgfpathlineto{\pgfqpoint{1.760790in}{1.327756in}}%
\pgfpathlineto{\pgfqpoint{1.762181in}{1.327818in}}%
\pgfpathlineto{\pgfqpoint{1.763572in}{1.327879in}}%
\pgfpathlineto{\pgfqpoint{1.764963in}{1.327940in}}%
\pgfpathlineto{\pgfqpoint{1.766354in}{1.327999in}}%
\pgfpathlineto{\pgfqpoint{1.767745in}{1.328057in}}%
\pgfpathlineto{\pgfqpoint{1.769136in}{1.328114in}}%
\pgfpathlineto{\pgfqpoint{1.770527in}{1.328170in}}%
\pgfpathlineto{\pgfqpoint{1.771918in}{1.328225in}}%
\pgfpathlineto{\pgfqpoint{1.773309in}{1.328279in}}%
\pgfpathlineto{\pgfqpoint{1.774700in}{1.328332in}}%
\pgfpathlineto{\pgfqpoint{1.776091in}{1.328384in}}%
\pgfpathlineto{\pgfqpoint{1.777482in}{1.328435in}}%
\pgfpathlineto{\pgfqpoint{1.778873in}{1.328485in}}%
\pgfpathlineto{\pgfqpoint{1.780264in}{1.328535in}}%
\pgfpathlineto{\pgfqpoint{1.781655in}{1.328583in}}%
\pgfpathlineto{\pgfqpoint{1.783046in}{1.328631in}}%
\pgfpathlineto{\pgfqpoint{1.784437in}{1.328678in}}%
\pgfpathlineto{\pgfqpoint{1.785828in}{1.328724in}}%
\pgfpathlineto{\pgfqpoint{1.787219in}{1.328769in}}%
\pgfpathlineto{\pgfqpoint{1.788610in}{1.328813in}}%
\pgfpathlineto{\pgfqpoint{1.790001in}{1.328857in}}%
\pgfpathlineto{\pgfqpoint{1.791392in}{1.328899in}}%
\pgfpathlineto{\pgfqpoint{1.792783in}{1.328941in}}%
\pgfpathlineto{\pgfqpoint{1.794174in}{1.328983in}}%
\pgfpathlineto{\pgfqpoint{1.795565in}{1.329023in}}%
\pgfpathlineto{\pgfqpoint{1.796956in}{1.329063in}}%
\pgfpathlineto{\pgfqpoint{1.798347in}{1.329102in}}%
\pgfpathlineto{\pgfqpoint{1.799738in}{1.329140in}}%
\pgfpathlineto{\pgfqpoint{1.801130in}{1.329178in}}%
\pgfpathlineto{\pgfqpoint{1.802521in}{1.329215in}}%
\pgfpathlineto{\pgfqpoint{1.803912in}{1.329251in}}%
\pgfpathlineto{\pgfqpoint{1.805303in}{1.329287in}}%
\pgfpathlineto{\pgfqpoint{1.806694in}{1.329322in}}%
\pgfpathlineto{\pgfqpoint{1.808085in}{1.329357in}}%
\pgfpathlineto{\pgfqpoint{1.809476in}{1.329390in}}%
\pgfpathlineto{\pgfqpoint{1.810867in}{1.329424in}}%
\pgfpathlineto{\pgfqpoint{1.812258in}{1.329456in}}%
\pgfpathlineto{\pgfqpoint{1.813649in}{1.329488in}}%
\pgfpathlineto{\pgfqpoint{1.815040in}{1.329520in}}%
\pgfpathlineto{\pgfqpoint{1.816431in}{1.329551in}}%
\pgfpathlineto{\pgfqpoint{1.817822in}{1.329581in}}%
\pgfpathlineto{\pgfqpoint{1.819213in}{1.329611in}}%
\pgfpathlineto{\pgfqpoint{1.820604in}{1.329640in}}%
\pgfpathlineto{\pgfqpoint{1.821995in}{1.329669in}}%
\pgfpathlineto{\pgfqpoint{1.823386in}{1.329697in}}%
\pgfpathlineto{\pgfqpoint{1.824777in}{1.329725in}}%
\pgfpathlineto{\pgfqpoint{1.826168in}{1.329752in}}%
\pgfpathlineto{\pgfqpoint{1.827559in}{1.329779in}}%
\pgfpathlineto{\pgfqpoint{1.828950in}{1.329805in}}%
\pgfpathlineto{\pgfqpoint{1.830341in}{1.329831in}}%
\pgfpathlineto{\pgfqpoint{1.831732in}{1.329856in}}%
\pgfpathlineto{\pgfqpoint{1.833123in}{1.329881in}}%
\pgfpathlineto{\pgfqpoint{1.834514in}{1.329906in}}%
\pgfpathlineto{\pgfqpoint{1.835905in}{1.329930in}}%
\pgfpathlineto{\pgfqpoint{1.837296in}{1.329953in}}%
\pgfpathlineto{\pgfqpoint{1.838687in}{1.329976in}}%
\pgfpathlineto{\pgfqpoint{1.840078in}{1.329999in}}%
\pgfpathlineto{\pgfqpoint{1.841469in}{1.330021in}}%
\pgfpathlineto{\pgfqpoint{1.842860in}{1.330043in}}%
\pgfpathlineto{\pgfqpoint{1.844252in}{1.330065in}}%
\pgfpathlineto{\pgfqpoint{1.845643in}{1.330086in}}%
\pgfpathlineto{\pgfqpoint{1.847034in}{1.330107in}}%
\pgfpathlineto{\pgfqpoint{1.848425in}{1.330127in}}%
\pgfpathlineto{\pgfqpoint{1.849816in}{1.330147in}}%
\pgfpathlineto{\pgfqpoint{1.851207in}{1.330167in}}%
\pgfpathlineto{\pgfqpoint{1.852598in}{1.330186in}}%
\pgfpathlineto{\pgfqpoint{1.853989in}{1.330205in}}%
\pgfpathlineto{\pgfqpoint{1.855380in}{1.330224in}}%
\pgfpathlineto{\pgfqpoint{1.856771in}{1.330242in}}%
\pgfpathlineto{\pgfqpoint{1.858162in}{1.330260in}}%
\pgfpathlineto{\pgfqpoint{1.859553in}{1.330278in}}%
\pgfpathlineto{\pgfqpoint{1.860944in}{1.330295in}}%
\pgfpathlineto{\pgfqpoint{1.862335in}{1.330312in}}%
\pgfpathlineto{\pgfqpoint{1.863726in}{1.330329in}}%
\pgfpathlineto{\pgfqpoint{1.865117in}{1.330346in}}%
\pgfpathlineto{\pgfqpoint{1.866508in}{1.330362in}}%
\pgfpathlineto{\pgfqpoint{1.867899in}{1.330378in}}%
\pgfpathlineto{\pgfqpoint{1.869290in}{1.330393in}}%
\pgfpathlineto{\pgfqpoint{1.870681in}{1.330408in}}%
\pgfpathlineto{\pgfqpoint{1.872072in}{1.330424in}}%
\pgfpathlineto{\pgfqpoint{1.873463in}{1.330438in}}%
\pgfpathlineto{\pgfqpoint{1.874854in}{1.330453in}}%
\pgfpathlineto{\pgfqpoint{1.876245in}{1.330467in}}%
\pgfpathlineto{\pgfqpoint{1.877636in}{1.330481in}}%
\pgfpathlineto{\pgfqpoint{1.879027in}{1.330495in}}%
\pgfpathlineto{\pgfqpoint{1.880418in}{1.330508in}}%
\pgfpathlineto{\pgfqpoint{1.881809in}{1.330521in}}%
\pgfpathlineto{\pgfqpoint{1.883200in}{1.330534in}}%
\pgfpathlineto{\pgfqpoint{1.884591in}{1.330547in}}%
\pgfpathlineto{\pgfqpoint{1.885982in}{1.330560in}}%
\pgfpathlineto{\pgfqpoint{1.887373in}{1.330572in}}%
\pgfpathlineto{\pgfqpoint{1.888765in}{1.330584in}}%
\pgfpathlineto{\pgfqpoint{1.890156in}{1.330596in}}%
\pgfpathlineto{\pgfqpoint{1.891547in}{1.330608in}}%
\pgfpathlineto{\pgfqpoint{1.892938in}{1.330619in}}%
\pgfpathlineto{\pgfqpoint{1.894329in}{1.330631in}}%
\pgfpathlineto{\pgfqpoint{1.895720in}{1.330642in}}%
\pgfpathlineto{\pgfqpoint{1.897111in}{1.330652in}}%
\pgfpathlineto{\pgfqpoint{1.898502in}{1.330663in}}%
\pgfpathlineto{\pgfqpoint{1.899893in}{1.330674in}}%
\pgfpathlineto{\pgfqpoint{1.901284in}{1.330684in}}%
\pgfpathlineto{\pgfqpoint{1.902675in}{1.330694in}}%
\pgfpathlineto{\pgfqpoint{1.904066in}{1.330704in}}%
\pgfpathlineto{\pgfqpoint{1.905457in}{1.330714in}}%
\pgfpathlineto{\pgfqpoint{1.906848in}{1.330723in}}%
\pgfpathlineto{\pgfqpoint{1.908239in}{1.330733in}}%
\pgfpathlineto{\pgfqpoint{1.909630in}{1.330742in}}%
\pgfpathlineto{\pgfqpoint{1.911021in}{1.330751in}}%
\pgfpathlineto{\pgfqpoint{1.912412in}{1.330760in}}%
\pgfpathlineto{\pgfqpoint{1.913803in}{1.330769in}}%
\pgfpathlineto{\pgfqpoint{1.915194in}{1.330777in}}%
\pgfpathlineto{\pgfqpoint{1.916585in}{1.330786in}}%
\pgfpathlineto{\pgfqpoint{1.917976in}{1.330794in}}%
\pgfpathlineto{\pgfqpoint{1.919367in}{1.330802in}}%
\pgfpathlineto{\pgfqpoint{1.920758in}{1.330810in}}%
\pgfpathlineto{\pgfqpoint{1.922149in}{1.330818in}}%
\pgfpathlineto{\pgfqpoint{1.923540in}{1.330826in}}%
\pgfpathlineto{\pgfqpoint{1.924931in}{1.330833in}}%
\pgfpathlineto{\pgfqpoint{1.926322in}{1.330841in}}%
\pgfpathlineto{\pgfqpoint{1.927713in}{1.330848in}}%
\pgfpathlineto{\pgfqpoint{1.929104in}{1.330855in}}%
\pgfpathlineto{\pgfqpoint{1.930495in}{1.330862in}}%
\pgfpathlineto{\pgfqpoint{1.931887in}{1.330869in}}%
\pgfpathlineto{\pgfqpoint{1.933278in}{1.330876in}}%
\pgfpathlineto{\pgfqpoint{1.934669in}{1.330883in}}%
\pgfpathlineto{\pgfqpoint{1.936060in}{1.330889in}}%
\pgfpathlineto{\pgfqpoint{1.937451in}{1.330896in}}%
\pgfpathlineto{\pgfqpoint{1.938842in}{1.330902in}}%
\pgfpathlineto{\pgfqpoint{1.940233in}{1.330908in}}%
\pgfpathlineto{\pgfqpoint{1.941624in}{1.330914in}}%
\pgfpathlineto{\pgfqpoint{1.943015in}{1.330920in}}%
\pgfpathlineto{\pgfqpoint{1.944406in}{1.330926in}}%
\pgfpathlineto{\pgfqpoint{1.945797in}{1.330932in}}%
\pgfpathlineto{\pgfqpoint{1.947188in}{1.330938in}}%
\pgfpathlineto{\pgfqpoint{1.948579in}{1.330943in}}%
\pgfpathlineto{\pgfqpoint{1.949970in}{1.330949in}}%
\pgfpathlineto{\pgfqpoint{1.951361in}{1.330954in}}%
\pgfpathlineto{\pgfqpoint{1.952752in}{1.330959in}}%
\pgfpathlineto{\pgfqpoint{1.954143in}{1.330965in}}%
\pgfpathlineto{\pgfqpoint{1.955534in}{1.330970in}}%
\pgfpathlineto{\pgfqpoint{1.956925in}{1.330975in}}%
\pgfpathlineto{\pgfqpoint{1.958316in}{1.330980in}}%
\pgfpathlineto{\pgfqpoint{1.959707in}{1.330984in}}%
\pgfpathlineto{\pgfqpoint{1.961098in}{1.330989in}}%
\pgfpathlineto{\pgfqpoint{1.962489in}{1.330994in}}%
\pgfpathlineto{\pgfqpoint{1.963880in}{1.330998in}}%
\pgfpathlineto{\pgfqpoint{1.965271in}{1.331003in}}%
\pgfpathlineto{\pgfqpoint{1.966662in}{1.331007in}}%
\pgfpathlineto{\pgfqpoint{1.968053in}{1.331012in}}%
\pgfpathlineto{\pgfqpoint{1.969444in}{1.331016in}}%
\pgfpathlineto{\pgfqpoint{1.970835in}{1.331020in}}%
\pgfpathlineto{\pgfqpoint{1.972226in}{1.331024in}}%
\pgfpathlineto{\pgfqpoint{1.973617in}{1.331028in}}%
\pgfpathlineto{\pgfqpoint{1.975009in}{1.331032in}}%
\pgfpathlineto{\pgfqpoint{1.976400in}{1.331036in}}%
\pgfpathlineto{\pgfqpoint{1.977791in}{1.331040in}}%
\pgfpathlineto{\pgfqpoint{1.979182in}{1.331044in}}%
\pgfpathlineto{\pgfqpoint{1.980573in}{1.331047in}}%
\pgfpathlineto{\pgfqpoint{1.981964in}{1.331051in}}%
\pgfpathlineto{\pgfqpoint{1.983355in}{1.331055in}}%
\pgfpathlineto{\pgfqpoint{1.984746in}{1.331058in}}%
\pgfpathlineto{\pgfqpoint{1.986137in}{1.331061in}}%
\pgfpathlineto{\pgfqpoint{1.987528in}{1.331065in}}%
\pgfpathlineto{\pgfqpoint{1.988919in}{1.331068in}}%
\pgfpathlineto{\pgfqpoint{1.990310in}{1.331071in}}%
\pgfpathlineto{\pgfqpoint{1.991701in}{1.331075in}}%
\pgfpathlineto{\pgfqpoint{1.993092in}{1.331078in}}%
\pgfpathlineto{\pgfqpoint{1.994483in}{1.331081in}}%
\pgfpathlineto{\pgfqpoint{1.995874in}{1.331084in}}%
\pgfpathlineto{\pgfqpoint{1.997265in}{1.331087in}}%
\pgfpathlineto{\pgfqpoint{1.998656in}{1.331090in}}%
\pgfpathlineto{\pgfqpoint{2.000047in}{1.331093in}}%
\pgfpathlineto{\pgfqpoint{2.001438in}{1.331095in}}%
\pgfpathlineto{\pgfqpoint{2.002829in}{1.331098in}}%
\pgfpathlineto{\pgfqpoint{2.004220in}{1.331101in}}%
\pgfpathlineto{\pgfqpoint{2.005611in}{1.331104in}}%
\pgfpathlineto{\pgfqpoint{2.007002in}{1.331106in}}%
\pgfpathlineto{\pgfqpoint{2.008393in}{1.331109in}}%
\pgfpathlineto{\pgfqpoint{2.009784in}{1.331111in}}%
\pgfpathlineto{\pgfqpoint{2.011175in}{1.331114in}}%
\pgfpathlineto{\pgfqpoint{2.012566in}{1.331116in}}%
\pgfpathlineto{\pgfqpoint{2.013957in}{1.331119in}}%
\pgfpathlineto{\pgfqpoint{2.015348in}{1.331121in}}%
\pgfpathlineto{\pgfqpoint{2.016739in}{1.331123in}}%
\pgfpathlineto{\pgfqpoint{2.018131in}{1.331125in}}%
\pgfpathlineto{\pgfqpoint{2.019522in}{1.331128in}}%
\pgfpathlineto{\pgfqpoint{2.020913in}{1.331130in}}%
\pgfpathlineto{\pgfqpoint{2.022304in}{1.331132in}}%
\pgfpathlineto{\pgfqpoint{2.023695in}{1.331134in}}%
\pgfpathlineto{\pgfqpoint{2.025086in}{1.331136in}}%
\pgfpathlineto{\pgfqpoint{2.026477in}{1.331138in}}%
\pgfpathlineto{\pgfqpoint{2.027868in}{1.331140in}}%
\pgfpathlineto{\pgfqpoint{2.029259in}{1.331142in}}%
\pgfpathlineto{\pgfqpoint{2.030650in}{1.331144in}}%
\pgfpathlineto{\pgfqpoint{2.032041in}{1.331146in}}%
\pgfpathlineto{\pgfqpoint{2.033432in}{1.331148in}}%
\pgfpathlineto{\pgfqpoint{2.034823in}{1.331150in}}%
\pgfpathlineto{\pgfqpoint{2.036214in}{1.331152in}}%
\pgfpathlineto{\pgfqpoint{2.037605in}{1.331153in}}%
\pgfpathlineto{\pgfqpoint{2.038996in}{1.331155in}}%
\pgfpathlineto{\pgfqpoint{2.040387in}{1.331157in}}%
\pgfpathlineto{\pgfqpoint{2.041778in}{1.331158in}}%
\pgfpathlineto{\pgfqpoint{2.043169in}{1.331160in}}%
\pgfpathlineto{\pgfqpoint{2.044560in}{1.331162in}}%
\pgfpathlineto{\pgfqpoint{2.045951in}{1.331163in}}%
\pgfpathlineto{\pgfqpoint{2.047342in}{1.331165in}}%
\pgfpathlineto{\pgfqpoint{2.048733in}{1.331166in}}%
\pgfpathlineto{\pgfqpoint{2.050124in}{1.331168in}}%
\pgfpathlineto{\pgfqpoint{2.051515in}{1.331169in}}%
\pgfpathlineto{\pgfqpoint{2.052906in}{1.331171in}}%
\pgfpathlineto{\pgfqpoint{2.054297in}{1.331172in}}%
\pgfpathlineto{\pgfqpoint{2.055688in}{1.331173in}}%
\pgfpathlineto{\pgfqpoint{2.057079in}{1.331175in}}%
\pgfpathlineto{\pgfqpoint{2.058470in}{1.331176in}}%
\pgfpathlineto{\pgfqpoint{2.059861in}{1.331178in}}%
\pgfpathlineto{\pgfqpoint{2.061253in}{1.331179in}}%
\pgfpathlineto{\pgfqpoint{2.062644in}{1.331180in}}%
\pgfpathlineto{\pgfqpoint{2.064035in}{1.331181in}}%
\pgfpathlineto{\pgfqpoint{2.065426in}{1.331183in}}%
\pgfpathlineto{\pgfqpoint{2.066817in}{1.331184in}}%
\pgfpathlineto{\pgfqpoint{2.068208in}{1.331185in}}%
\pgfpathlineto{\pgfqpoint{2.069599in}{1.331186in}}%
\pgfpathlineto{\pgfqpoint{2.070990in}{1.331187in}}%
\pgfpathlineto{\pgfqpoint{2.072381in}{1.331188in}}%
\pgfpathlineto{\pgfqpoint{2.073772in}{1.331189in}}%
\pgfpathlineto{\pgfqpoint{2.075163in}{1.331191in}}%
\pgfpathlineto{\pgfqpoint{2.076554in}{1.331192in}}%
\pgfpathlineto{\pgfqpoint{2.077945in}{1.331193in}}%
\pgfpathlineto{\pgfqpoint{2.079336in}{1.331194in}}%
\pgfpathlineto{\pgfqpoint{2.080727in}{1.331195in}}%
\pgfpathlineto{\pgfqpoint{2.082118in}{1.331196in}}%
\pgfpathlineto{\pgfqpoint{2.083509in}{1.331197in}}%
\pgfpathlineto{\pgfqpoint{2.084900in}{1.331198in}}%
\pgfpathlineto{\pgfqpoint{2.086291in}{1.331199in}}%
\pgfpathlineto{\pgfqpoint{2.087682in}{1.331199in}}%
\pgfpathlineto{\pgfqpoint{2.089073in}{1.331200in}}%
\pgfpathlineto{\pgfqpoint{2.090464in}{1.331201in}}%
\pgfpathlineto{\pgfqpoint{2.091855in}{1.331202in}}%
\pgfpathlineto{\pgfqpoint{2.093246in}{1.331203in}}%
\pgfpathlineto{\pgfqpoint{2.094637in}{1.331204in}}%
\pgfpathlineto{\pgfqpoint{2.096028in}{1.331205in}}%
\pgfpathlineto{\pgfqpoint{2.097419in}{1.331205in}}%
\pgfpathlineto{\pgfqpoint{2.098810in}{1.331206in}}%
\pgfpathlineto{\pgfqpoint{2.100201in}{1.331207in}}%
\pgfpathlineto{\pgfqpoint{2.101592in}{1.331208in}}%
\pgfpathlineto{\pgfqpoint{2.102983in}{1.331209in}}%
\pgfpathlineto{\pgfqpoint{2.104374in}{1.331209in}}%
\pgfpathlineto{\pgfqpoint{2.105766in}{1.331210in}}%
\pgfpathlineto{\pgfqpoint{2.107157in}{1.331211in}}%
\pgfpathlineto{\pgfqpoint{2.108548in}{1.331211in}}%
\pgfpathlineto{\pgfqpoint{2.109939in}{1.331212in}}%
\pgfpathlineto{\pgfqpoint{2.111330in}{1.331213in}}%
\pgfpathlineto{\pgfqpoint{2.112721in}{1.331213in}}%
\pgfpathlineto{\pgfqpoint{2.114112in}{1.331214in}}%
\pgfpathlineto{\pgfqpoint{2.115503in}{1.331215in}}%
\pgfpathlineto{\pgfqpoint{2.116894in}{1.331215in}}%
\pgfpathlineto{\pgfqpoint{2.118285in}{1.331216in}}%
\pgfpathlineto{\pgfqpoint{2.119676in}{1.331217in}}%
\pgfpathlineto{\pgfqpoint{2.121067in}{1.331217in}}%
\pgfpathlineto{\pgfqpoint{2.122458in}{1.331218in}}%
\pgfpathlineto{\pgfqpoint{2.123849in}{1.331218in}}%
\pgfpathlineto{\pgfqpoint{2.125240in}{1.331219in}}%
\pgfpathlineto{\pgfqpoint{2.126631in}{1.331220in}}%
\pgfpathlineto{\pgfqpoint{2.128022in}{1.331220in}}%
\pgfpathlineto{\pgfqpoint{2.129413in}{1.331221in}}%
\pgfpathlineto{\pgfqpoint{2.130804in}{1.331221in}}%
\pgfpathlineto{\pgfqpoint{2.132195in}{1.331222in}}%
\pgfpathlineto{\pgfqpoint{2.133586in}{1.331222in}}%
\pgfpathlineto{\pgfqpoint{2.134977in}{1.331223in}}%
\pgfpathlineto{\pgfqpoint{2.136368in}{1.331223in}}%
\pgfpathlineto{\pgfqpoint{2.137759in}{1.331224in}}%
\pgfpathlineto{\pgfqpoint{2.139150in}{1.331224in}}%
\pgfpathlineto{\pgfqpoint{2.140541in}{1.331225in}}%
\pgfpathlineto{\pgfqpoint{2.141932in}{1.331225in}}%
\pgfpathlineto{\pgfqpoint{2.143323in}{1.331225in}}%
\pgfpathlineto{\pgfqpoint{2.144714in}{1.331226in}}%
\pgfpathlineto{\pgfqpoint{2.146105in}{1.331226in}}%
\pgfpathlineto{\pgfqpoint{2.147496in}{1.331227in}}%
\pgfpathlineto{\pgfqpoint{2.148888in}{1.331227in}}%
\pgfpathlineto{\pgfqpoint{2.150279in}{1.331228in}}%
\pgfpathlineto{\pgfqpoint{2.151670in}{1.331228in}}%
\pgfpathlineto{\pgfqpoint{2.153061in}{1.331228in}}%
\pgfpathlineto{\pgfqpoint{2.154452in}{1.331229in}}%
\pgfpathlineto{\pgfqpoint{2.155843in}{1.331229in}}%
\pgfpathlineto{\pgfqpoint{2.157234in}{1.331230in}}%
\pgfpathlineto{\pgfqpoint{2.158625in}{1.331230in}}%
\pgfpathlineto{\pgfqpoint{2.160016in}{1.331230in}}%
\pgfpathlineto{\pgfqpoint{2.161407in}{1.331231in}}%
\pgfpathlineto{\pgfqpoint{2.162798in}{1.331231in}}%
\pgfpathlineto{\pgfqpoint{2.164189in}{1.331231in}}%
\pgfpathlineto{\pgfqpoint{2.165580in}{1.331232in}}%
\pgfpathlineto{\pgfqpoint{2.166971in}{1.331232in}}%
\pgfpathlineto{\pgfqpoint{2.168362in}{1.331232in}}%
\pgfpathlineto{\pgfqpoint{2.169753in}{1.331233in}}%
\pgfpathlineto{\pgfqpoint{2.171144in}{1.331233in}}%
\pgfpathlineto{\pgfqpoint{2.172535in}{1.331233in}}%
\pgfpathlineto{\pgfqpoint{2.173926in}{1.331234in}}%
\pgfpathlineto{\pgfqpoint{2.175317in}{1.331234in}}%
\pgfpathlineto{\pgfqpoint{2.176708in}{1.331234in}}%
\pgfpathlineto{\pgfqpoint{2.178099in}{1.331234in}}%
\pgfpathlineto{\pgfqpoint{2.179490in}{1.331235in}}%
\pgfpathlineto{\pgfqpoint{2.180881in}{1.331235in}}%
\pgfpathlineto{\pgfqpoint{2.182272in}{1.331235in}}%
\pgfpathlineto{\pgfqpoint{2.183663in}{1.331235in}}%
\pgfpathlineto{\pgfqpoint{2.185054in}{1.331236in}}%
\pgfpathlineto{\pgfqpoint{2.186445in}{1.331236in}}%
\pgfpathlineto{\pgfqpoint{2.187836in}{1.331236in}}%
\pgfpathlineto{\pgfqpoint{2.189227in}{1.331236in}}%
\pgfpathlineto{\pgfqpoint{2.190618in}{1.331237in}}%
\pgfpathlineto{\pgfqpoint{2.192010in}{1.331237in}}%
\pgfpathlineto{\pgfqpoint{2.193401in}{1.331237in}}%
\pgfpathlineto{\pgfqpoint{2.194792in}{1.331237in}}%
\pgfpathlineto{\pgfqpoint{2.196183in}{1.331238in}}%
\pgfpathlineto{\pgfqpoint{2.197574in}{1.331238in}}%
\pgfpathlineto{\pgfqpoint{2.198965in}{1.331238in}}%
\pgfpathlineto{\pgfqpoint{2.200356in}{1.331238in}}%
\pgfpathlineto{\pgfqpoint{2.201747in}{1.331239in}}%
\pgfpathlineto{\pgfqpoint{2.203138in}{1.331239in}}%
\pgfpathlineto{\pgfqpoint{2.204529in}{1.331239in}}%
\pgfpathlineto{\pgfqpoint{2.205920in}{1.331239in}}%
\pgfpathlineto{\pgfqpoint{2.207311in}{1.331239in}}%
\pgfpathlineto{\pgfqpoint{2.208702in}{1.331240in}}%
\pgfpathlineto{\pgfqpoint{2.210093in}{1.331240in}}%
\pgfpathlineto{\pgfqpoint{2.211484in}{1.331240in}}%
\pgfpathlineto{\pgfqpoint{2.212875in}{1.331240in}}%
\pgfpathlineto{\pgfqpoint{2.212875in}{1.575000in}}%
\pgfpathlineto{\pgfqpoint{2.212875in}{1.575000in}}%
\pgfpathlineto{\pgfqpoint{2.211484in}{1.575000in}}%
\pgfpathlineto{\pgfqpoint{2.210093in}{1.575000in}}%
\pgfpathlineto{\pgfqpoint{2.208702in}{1.575000in}}%
\pgfpathlineto{\pgfqpoint{2.207311in}{1.575000in}}%
\pgfpathlineto{\pgfqpoint{2.205920in}{1.575000in}}%
\pgfpathlineto{\pgfqpoint{2.204529in}{1.575000in}}%
\pgfpathlineto{\pgfqpoint{2.203138in}{1.575000in}}%
\pgfpathlineto{\pgfqpoint{2.201747in}{1.575000in}}%
\pgfpathlineto{\pgfqpoint{2.200356in}{1.575000in}}%
\pgfpathlineto{\pgfqpoint{2.198965in}{1.575000in}}%
\pgfpathlineto{\pgfqpoint{2.197574in}{1.575000in}}%
\pgfpathlineto{\pgfqpoint{2.196183in}{1.575000in}}%
\pgfpathlineto{\pgfqpoint{2.194792in}{1.575000in}}%
\pgfpathlineto{\pgfqpoint{2.193401in}{1.575000in}}%
\pgfpathlineto{\pgfqpoint{2.192010in}{1.575000in}}%
\pgfpathlineto{\pgfqpoint{2.190618in}{1.575000in}}%
\pgfpathlineto{\pgfqpoint{2.189227in}{1.575000in}}%
\pgfpathlineto{\pgfqpoint{2.187836in}{1.575000in}}%
\pgfpathlineto{\pgfqpoint{2.186445in}{1.575000in}}%
\pgfpathlineto{\pgfqpoint{2.185054in}{1.575000in}}%
\pgfpathlineto{\pgfqpoint{2.183663in}{1.575000in}}%
\pgfpathlineto{\pgfqpoint{2.182272in}{1.575000in}}%
\pgfpathlineto{\pgfqpoint{2.180881in}{1.575000in}}%
\pgfpathlineto{\pgfqpoint{2.179490in}{1.575000in}}%
\pgfpathlineto{\pgfqpoint{2.178099in}{1.575000in}}%
\pgfpathlineto{\pgfqpoint{2.176708in}{1.575000in}}%
\pgfpathlineto{\pgfqpoint{2.175317in}{1.575000in}}%
\pgfpathlineto{\pgfqpoint{2.173926in}{1.575000in}}%
\pgfpathlineto{\pgfqpoint{2.172535in}{1.575000in}}%
\pgfpathlineto{\pgfqpoint{2.171144in}{1.575000in}}%
\pgfpathlineto{\pgfqpoint{2.169753in}{1.575000in}}%
\pgfpathlineto{\pgfqpoint{2.168362in}{1.575000in}}%
\pgfpathlineto{\pgfqpoint{2.166971in}{1.575000in}}%
\pgfpathlineto{\pgfqpoint{2.165580in}{1.575000in}}%
\pgfpathlineto{\pgfqpoint{2.164189in}{1.575000in}}%
\pgfpathlineto{\pgfqpoint{2.162798in}{1.575000in}}%
\pgfpathlineto{\pgfqpoint{2.161407in}{1.575000in}}%
\pgfpathlineto{\pgfqpoint{2.160016in}{1.575000in}}%
\pgfpathlineto{\pgfqpoint{2.158625in}{1.575000in}}%
\pgfpathlineto{\pgfqpoint{2.157234in}{1.575000in}}%
\pgfpathlineto{\pgfqpoint{2.155843in}{1.575000in}}%
\pgfpathlineto{\pgfqpoint{2.154452in}{1.575000in}}%
\pgfpathlineto{\pgfqpoint{2.153061in}{1.575000in}}%
\pgfpathlineto{\pgfqpoint{2.151670in}{1.575000in}}%
\pgfpathlineto{\pgfqpoint{2.150279in}{1.575000in}}%
\pgfpathlineto{\pgfqpoint{2.148888in}{1.575000in}}%
\pgfpathlineto{\pgfqpoint{2.147496in}{1.575000in}}%
\pgfpathlineto{\pgfqpoint{2.146105in}{1.575000in}}%
\pgfpathlineto{\pgfqpoint{2.144714in}{1.575000in}}%
\pgfpathlineto{\pgfqpoint{2.143323in}{1.575000in}}%
\pgfpathlineto{\pgfqpoint{2.141932in}{1.575000in}}%
\pgfpathlineto{\pgfqpoint{2.140541in}{1.575000in}}%
\pgfpathlineto{\pgfqpoint{2.139150in}{1.575000in}}%
\pgfpathlineto{\pgfqpoint{2.137759in}{1.575000in}}%
\pgfpathlineto{\pgfqpoint{2.136368in}{1.575000in}}%
\pgfpathlineto{\pgfqpoint{2.134977in}{1.575000in}}%
\pgfpathlineto{\pgfqpoint{2.133586in}{1.575000in}}%
\pgfpathlineto{\pgfqpoint{2.132195in}{1.575000in}}%
\pgfpathlineto{\pgfqpoint{2.130804in}{1.575000in}}%
\pgfpathlineto{\pgfqpoint{2.129413in}{1.575000in}}%
\pgfpathlineto{\pgfqpoint{2.128022in}{1.575000in}}%
\pgfpathlineto{\pgfqpoint{2.126631in}{1.575000in}}%
\pgfpathlineto{\pgfqpoint{2.125240in}{1.575000in}}%
\pgfpathlineto{\pgfqpoint{2.123849in}{1.575000in}}%
\pgfpathlineto{\pgfqpoint{2.122458in}{1.575000in}}%
\pgfpathlineto{\pgfqpoint{2.121067in}{1.575000in}}%
\pgfpathlineto{\pgfqpoint{2.119676in}{1.575000in}}%
\pgfpathlineto{\pgfqpoint{2.118285in}{1.575000in}}%
\pgfpathlineto{\pgfqpoint{2.116894in}{1.575000in}}%
\pgfpathlineto{\pgfqpoint{2.115503in}{1.575000in}}%
\pgfpathlineto{\pgfqpoint{2.114112in}{1.575000in}}%
\pgfpathlineto{\pgfqpoint{2.112721in}{1.575000in}}%
\pgfpathlineto{\pgfqpoint{2.111330in}{1.575000in}}%
\pgfpathlineto{\pgfqpoint{2.109939in}{1.575000in}}%
\pgfpathlineto{\pgfqpoint{2.108548in}{1.575000in}}%
\pgfpathlineto{\pgfqpoint{2.107157in}{1.575000in}}%
\pgfpathlineto{\pgfqpoint{2.105766in}{1.575000in}}%
\pgfpathlineto{\pgfqpoint{2.104374in}{1.575000in}}%
\pgfpathlineto{\pgfqpoint{2.102983in}{1.575000in}}%
\pgfpathlineto{\pgfqpoint{2.101592in}{1.575000in}}%
\pgfpathlineto{\pgfqpoint{2.100201in}{1.575000in}}%
\pgfpathlineto{\pgfqpoint{2.098810in}{1.575000in}}%
\pgfpathlineto{\pgfqpoint{2.097419in}{1.575000in}}%
\pgfpathlineto{\pgfqpoint{2.096028in}{1.575000in}}%
\pgfpathlineto{\pgfqpoint{2.094637in}{1.575000in}}%
\pgfpathlineto{\pgfqpoint{2.093246in}{1.575000in}}%
\pgfpathlineto{\pgfqpoint{2.091855in}{1.575000in}}%
\pgfpathlineto{\pgfqpoint{2.090464in}{1.575000in}}%
\pgfpathlineto{\pgfqpoint{2.089073in}{1.575000in}}%
\pgfpathlineto{\pgfqpoint{2.087682in}{1.575000in}}%
\pgfpathlineto{\pgfqpoint{2.086291in}{1.575000in}}%
\pgfpathlineto{\pgfqpoint{2.084900in}{1.575000in}}%
\pgfpathlineto{\pgfqpoint{2.083509in}{1.575000in}}%
\pgfpathlineto{\pgfqpoint{2.082118in}{1.575000in}}%
\pgfpathlineto{\pgfqpoint{2.080727in}{1.575000in}}%
\pgfpathlineto{\pgfqpoint{2.079336in}{1.575000in}}%
\pgfpathlineto{\pgfqpoint{2.077945in}{1.575000in}}%
\pgfpathlineto{\pgfqpoint{2.076554in}{1.575000in}}%
\pgfpathlineto{\pgfqpoint{2.075163in}{1.575000in}}%
\pgfpathlineto{\pgfqpoint{2.073772in}{1.575000in}}%
\pgfpathlineto{\pgfqpoint{2.072381in}{1.575000in}}%
\pgfpathlineto{\pgfqpoint{2.070990in}{1.575000in}}%
\pgfpathlineto{\pgfqpoint{2.069599in}{1.575000in}}%
\pgfpathlineto{\pgfqpoint{2.068208in}{1.575000in}}%
\pgfpathlineto{\pgfqpoint{2.066817in}{1.575000in}}%
\pgfpathlineto{\pgfqpoint{2.065426in}{1.575000in}}%
\pgfpathlineto{\pgfqpoint{2.064035in}{1.575000in}}%
\pgfpathlineto{\pgfqpoint{2.062644in}{1.575000in}}%
\pgfpathlineto{\pgfqpoint{2.061253in}{1.575000in}}%
\pgfpathlineto{\pgfqpoint{2.059861in}{1.575000in}}%
\pgfpathlineto{\pgfqpoint{2.058470in}{1.575000in}}%
\pgfpathlineto{\pgfqpoint{2.057079in}{1.575000in}}%
\pgfpathlineto{\pgfqpoint{2.055688in}{1.575000in}}%
\pgfpathlineto{\pgfqpoint{2.054297in}{1.575000in}}%
\pgfpathlineto{\pgfqpoint{2.052906in}{1.575000in}}%
\pgfpathlineto{\pgfqpoint{2.051515in}{1.575000in}}%
\pgfpathlineto{\pgfqpoint{2.050124in}{1.575000in}}%
\pgfpathlineto{\pgfqpoint{2.048733in}{1.575000in}}%
\pgfpathlineto{\pgfqpoint{2.047342in}{1.575000in}}%
\pgfpathlineto{\pgfqpoint{2.045951in}{1.575000in}}%
\pgfpathlineto{\pgfqpoint{2.044560in}{1.575000in}}%
\pgfpathlineto{\pgfqpoint{2.043169in}{1.575000in}}%
\pgfpathlineto{\pgfqpoint{2.041778in}{1.575000in}}%
\pgfpathlineto{\pgfqpoint{2.040387in}{1.575000in}}%
\pgfpathlineto{\pgfqpoint{2.038996in}{1.575000in}}%
\pgfpathlineto{\pgfqpoint{2.037605in}{1.575000in}}%
\pgfpathlineto{\pgfqpoint{2.036214in}{1.575000in}}%
\pgfpathlineto{\pgfqpoint{2.034823in}{1.575000in}}%
\pgfpathlineto{\pgfqpoint{2.033432in}{1.575000in}}%
\pgfpathlineto{\pgfqpoint{2.032041in}{1.575000in}}%
\pgfpathlineto{\pgfqpoint{2.030650in}{1.575000in}}%
\pgfpathlineto{\pgfqpoint{2.029259in}{1.575000in}}%
\pgfpathlineto{\pgfqpoint{2.027868in}{1.575000in}}%
\pgfpathlineto{\pgfqpoint{2.026477in}{1.575000in}}%
\pgfpathlineto{\pgfqpoint{2.025086in}{1.575000in}}%
\pgfpathlineto{\pgfqpoint{2.023695in}{1.575000in}}%
\pgfpathlineto{\pgfqpoint{2.022304in}{1.575000in}}%
\pgfpathlineto{\pgfqpoint{2.020913in}{1.575000in}}%
\pgfpathlineto{\pgfqpoint{2.019522in}{1.575000in}}%
\pgfpathlineto{\pgfqpoint{2.018131in}{1.575000in}}%
\pgfpathlineto{\pgfqpoint{2.016739in}{1.575000in}}%
\pgfpathlineto{\pgfqpoint{2.015348in}{1.575000in}}%
\pgfpathlineto{\pgfqpoint{2.013957in}{1.575000in}}%
\pgfpathlineto{\pgfqpoint{2.012566in}{1.575000in}}%
\pgfpathlineto{\pgfqpoint{2.011175in}{1.575000in}}%
\pgfpathlineto{\pgfqpoint{2.009784in}{1.575000in}}%
\pgfpathlineto{\pgfqpoint{2.008393in}{1.575000in}}%
\pgfpathlineto{\pgfqpoint{2.007002in}{1.575000in}}%
\pgfpathlineto{\pgfqpoint{2.005611in}{1.575000in}}%
\pgfpathlineto{\pgfqpoint{2.004220in}{1.575000in}}%
\pgfpathlineto{\pgfqpoint{2.002829in}{1.575000in}}%
\pgfpathlineto{\pgfqpoint{2.001438in}{1.575000in}}%
\pgfpathlineto{\pgfqpoint{2.000047in}{1.575000in}}%
\pgfpathlineto{\pgfqpoint{1.998656in}{1.575000in}}%
\pgfpathlineto{\pgfqpoint{1.997265in}{1.575000in}}%
\pgfpathlineto{\pgfqpoint{1.995874in}{1.575000in}}%
\pgfpathlineto{\pgfqpoint{1.994483in}{1.575000in}}%
\pgfpathlineto{\pgfqpoint{1.993092in}{1.575000in}}%
\pgfpathlineto{\pgfqpoint{1.991701in}{1.575000in}}%
\pgfpathlineto{\pgfqpoint{1.990310in}{1.575000in}}%
\pgfpathlineto{\pgfqpoint{1.988919in}{1.575000in}}%
\pgfpathlineto{\pgfqpoint{1.987528in}{1.575000in}}%
\pgfpathlineto{\pgfqpoint{1.986137in}{1.575000in}}%
\pgfpathlineto{\pgfqpoint{1.984746in}{1.575000in}}%
\pgfpathlineto{\pgfqpoint{1.983355in}{1.575000in}}%
\pgfpathlineto{\pgfqpoint{1.981964in}{1.575000in}}%
\pgfpathlineto{\pgfqpoint{1.980573in}{1.575000in}}%
\pgfpathlineto{\pgfqpoint{1.979182in}{1.575000in}}%
\pgfpathlineto{\pgfqpoint{1.977791in}{1.575000in}}%
\pgfpathlineto{\pgfqpoint{1.976400in}{1.575000in}}%
\pgfpathlineto{\pgfqpoint{1.975009in}{1.575000in}}%
\pgfpathlineto{\pgfqpoint{1.973617in}{1.575000in}}%
\pgfpathlineto{\pgfqpoint{1.972226in}{1.575000in}}%
\pgfpathlineto{\pgfqpoint{1.970835in}{1.575000in}}%
\pgfpathlineto{\pgfqpoint{1.969444in}{1.575000in}}%
\pgfpathlineto{\pgfqpoint{1.968053in}{1.575000in}}%
\pgfpathlineto{\pgfqpoint{1.966662in}{1.575000in}}%
\pgfpathlineto{\pgfqpoint{1.965271in}{1.575000in}}%
\pgfpathlineto{\pgfqpoint{1.963880in}{1.575000in}}%
\pgfpathlineto{\pgfqpoint{1.962489in}{1.575000in}}%
\pgfpathlineto{\pgfqpoint{1.961098in}{1.575000in}}%
\pgfpathlineto{\pgfqpoint{1.959707in}{1.575000in}}%
\pgfpathlineto{\pgfqpoint{1.958316in}{1.575000in}}%
\pgfpathlineto{\pgfqpoint{1.956925in}{1.575000in}}%
\pgfpathlineto{\pgfqpoint{1.955534in}{1.575000in}}%
\pgfpathlineto{\pgfqpoint{1.954143in}{1.575000in}}%
\pgfpathlineto{\pgfqpoint{1.952752in}{1.575000in}}%
\pgfpathlineto{\pgfqpoint{1.951361in}{1.575000in}}%
\pgfpathlineto{\pgfqpoint{1.949970in}{1.575000in}}%
\pgfpathlineto{\pgfqpoint{1.948579in}{1.575000in}}%
\pgfpathlineto{\pgfqpoint{1.947188in}{1.575000in}}%
\pgfpathlineto{\pgfqpoint{1.945797in}{1.575000in}}%
\pgfpathlineto{\pgfqpoint{1.944406in}{1.575000in}}%
\pgfpathlineto{\pgfqpoint{1.943015in}{1.575000in}}%
\pgfpathlineto{\pgfqpoint{1.941624in}{1.575000in}}%
\pgfpathlineto{\pgfqpoint{1.940233in}{1.575000in}}%
\pgfpathlineto{\pgfqpoint{1.938842in}{1.575000in}}%
\pgfpathlineto{\pgfqpoint{1.937451in}{1.575000in}}%
\pgfpathlineto{\pgfqpoint{1.936060in}{1.575000in}}%
\pgfpathlineto{\pgfqpoint{1.934669in}{1.575000in}}%
\pgfpathlineto{\pgfqpoint{1.933278in}{1.575000in}}%
\pgfpathlineto{\pgfqpoint{1.931887in}{1.575000in}}%
\pgfpathlineto{\pgfqpoint{1.930495in}{1.575000in}}%
\pgfpathlineto{\pgfqpoint{1.929104in}{1.575000in}}%
\pgfpathlineto{\pgfqpoint{1.927713in}{1.575000in}}%
\pgfpathlineto{\pgfqpoint{1.926322in}{1.575000in}}%
\pgfpathlineto{\pgfqpoint{1.924931in}{1.575000in}}%
\pgfpathlineto{\pgfqpoint{1.923540in}{1.575000in}}%
\pgfpathlineto{\pgfqpoint{1.922149in}{1.575000in}}%
\pgfpathlineto{\pgfqpoint{1.920758in}{1.575000in}}%
\pgfpathlineto{\pgfqpoint{1.919367in}{1.575000in}}%
\pgfpathlineto{\pgfqpoint{1.917976in}{1.575000in}}%
\pgfpathlineto{\pgfqpoint{1.916585in}{1.575000in}}%
\pgfpathlineto{\pgfqpoint{1.915194in}{1.575000in}}%
\pgfpathlineto{\pgfqpoint{1.913803in}{1.575000in}}%
\pgfpathlineto{\pgfqpoint{1.912412in}{1.575000in}}%
\pgfpathlineto{\pgfqpoint{1.911021in}{1.575000in}}%
\pgfpathlineto{\pgfqpoint{1.909630in}{1.575000in}}%
\pgfpathlineto{\pgfqpoint{1.908239in}{1.575000in}}%
\pgfpathlineto{\pgfqpoint{1.906848in}{1.575000in}}%
\pgfpathlineto{\pgfqpoint{1.905457in}{1.575000in}}%
\pgfpathlineto{\pgfqpoint{1.904066in}{1.575000in}}%
\pgfpathlineto{\pgfqpoint{1.902675in}{1.575000in}}%
\pgfpathlineto{\pgfqpoint{1.901284in}{1.575000in}}%
\pgfpathlineto{\pgfqpoint{1.899893in}{1.575000in}}%
\pgfpathlineto{\pgfqpoint{1.898502in}{1.575000in}}%
\pgfpathlineto{\pgfqpoint{1.897111in}{1.575000in}}%
\pgfpathlineto{\pgfqpoint{1.895720in}{1.575000in}}%
\pgfpathlineto{\pgfqpoint{1.894329in}{1.575000in}}%
\pgfpathlineto{\pgfqpoint{1.892938in}{1.575000in}}%
\pgfpathlineto{\pgfqpoint{1.891547in}{1.575000in}}%
\pgfpathlineto{\pgfqpoint{1.890156in}{1.575000in}}%
\pgfpathlineto{\pgfqpoint{1.888765in}{1.575000in}}%
\pgfpathlineto{\pgfqpoint{1.887373in}{1.575000in}}%
\pgfpathlineto{\pgfqpoint{1.885982in}{1.575000in}}%
\pgfpathlineto{\pgfqpoint{1.884591in}{1.575000in}}%
\pgfpathlineto{\pgfqpoint{1.883200in}{1.575000in}}%
\pgfpathlineto{\pgfqpoint{1.881809in}{1.575000in}}%
\pgfpathlineto{\pgfqpoint{1.880418in}{1.575000in}}%
\pgfpathlineto{\pgfqpoint{1.879027in}{1.575000in}}%
\pgfpathlineto{\pgfqpoint{1.877636in}{1.575000in}}%
\pgfpathlineto{\pgfqpoint{1.876245in}{1.575000in}}%
\pgfpathlineto{\pgfqpoint{1.874854in}{1.575000in}}%
\pgfpathlineto{\pgfqpoint{1.873463in}{1.575000in}}%
\pgfpathlineto{\pgfqpoint{1.872072in}{1.575000in}}%
\pgfpathlineto{\pgfqpoint{1.870681in}{1.575000in}}%
\pgfpathlineto{\pgfqpoint{1.869290in}{1.575000in}}%
\pgfpathlineto{\pgfqpoint{1.867899in}{1.575000in}}%
\pgfpathlineto{\pgfqpoint{1.866508in}{1.575000in}}%
\pgfpathlineto{\pgfqpoint{1.865117in}{1.575000in}}%
\pgfpathlineto{\pgfqpoint{1.863726in}{1.575000in}}%
\pgfpathlineto{\pgfqpoint{1.862335in}{1.575000in}}%
\pgfpathlineto{\pgfqpoint{1.860944in}{1.575000in}}%
\pgfpathlineto{\pgfqpoint{1.859553in}{1.575000in}}%
\pgfpathlineto{\pgfqpoint{1.858162in}{1.575000in}}%
\pgfpathlineto{\pgfqpoint{1.856771in}{1.575000in}}%
\pgfpathlineto{\pgfqpoint{1.855380in}{1.575000in}}%
\pgfpathlineto{\pgfqpoint{1.853989in}{1.575000in}}%
\pgfpathlineto{\pgfqpoint{1.852598in}{1.575000in}}%
\pgfpathlineto{\pgfqpoint{1.851207in}{1.575000in}}%
\pgfpathlineto{\pgfqpoint{1.849816in}{1.575000in}}%
\pgfpathlineto{\pgfqpoint{1.848425in}{1.575000in}}%
\pgfpathlineto{\pgfqpoint{1.847034in}{1.575000in}}%
\pgfpathlineto{\pgfqpoint{1.845643in}{1.575000in}}%
\pgfpathlineto{\pgfqpoint{1.844252in}{1.575000in}}%
\pgfpathlineto{\pgfqpoint{1.842860in}{1.575000in}}%
\pgfpathlineto{\pgfqpoint{1.841469in}{1.575000in}}%
\pgfpathlineto{\pgfqpoint{1.840078in}{1.575000in}}%
\pgfpathlineto{\pgfqpoint{1.838687in}{1.575000in}}%
\pgfpathlineto{\pgfqpoint{1.837296in}{1.575000in}}%
\pgfpathlineto{\pgfqpoint{1.835905in}{1.575000in}}%
\pgfpathlineto{\pgfqpoint{1.834514in}{1.575000in}}%
\pgfpathlineto{\pgfqpoint{1.833123in}{1.575000in}}%
\pgfpathlineto{\pgfqpoint{1.831732in}{1.575000in}}%
\pgfpathlineto{\pgfqpoint{1.830341in}{1.575000in}}%
\pgfpathlineto{\pgfqpoint{1.828950in}{1.575000in}}%
\pgfpathlineto{\pgfqpoint{1.827559in}{1.575000in}}%
\pgfpathlineto{\pgfqpoint{1.826168in}{1.575000in}}%
\pgfpathlineto{\pgfqpoint{1.824777in}{1.575000in}}%
\pgfpathlineto{\pgfqpoint{1.823386in}{1.575000in}}%
\pgfpathlineto{\pgfqpoint{1.821995in}{1.575000in}}%
\pgfpathlineto{\pgfqpoint{1.820604in}{1.575000in}}%
\pgfpathlineto{\pgfqpoint{1.819213in}{1.575000in}}%
\pgfpathlineto{\pgfqpoint{1.817822in}{1.575000in}}%
\pgfpathlineto{\pgfqpoint{1.816431in}{1.575000in}}%
\pgfpathlineto{\pgfqpoint{1.815040in}{1.575000in}}%
\pgfpathlineto{\pgfqpoint{1.813649in}{1.575000in}}%
\pgfpathlineto{\pgfqpoint{1.812258in}{1.575000in}}%
\pgfpathlineto{\pgfqpoint{1.810867in}{1.575000in}}%
\pgfpathlineto{\pgfqpoint{1.809476in}{1.575000in}}%
\pgfpathlineto{\pgfqpoint{1.808085in}{1.575000in}}%
\pgfpathlineto{\pgfqpoint{1.806694in}{1.575000in}}%
\pgfpathlineto{\pgfqpoint{1.805303in}{1.575000in}}%
\pgfpathlineto{\pgfqpoint{1.803912in}{1.575000in}}%
\pgfpathlineto{\pgfqpoint{1.802521in}{1.575000in}}%
\pgfpathlineto{\pgfqpoint{1.801130in}{1.575000in}}%
\pgfpathlineto{\pgfqpoint{1.799738in}{1.575000in}}%
\pgfpathlineto{\pgfqpoint{1.798347in}{1.575000in}}%
\pgfpathlineto{\pgfqpoint{1.796956in}{1.575000in}}%
\pgfpathlineto{\pgfqpoint{1.795565in}{1.575000in}}%
\pgfpathlineto{\pgfqpoint{1.794174in}{1.575000in}}%
\pgfpathlineto{\pgfqpoint{1.792783in}{1.575000in}}%
\pgfpathlineto{\pgfqpoint{1.791392in}{1.575000in}}%
\pgfpathlineto{\pgfqpoint{1.790001in}{1.575000in}}%
\pgfpathlineto{\pgfqpoint{1.788610in}{1.575000in}}%
\pgfpathlineto{\pgfqpoint{1.787219in}{1.575000in}}%
\pgfpathlineto{\pgfqpoint{1.785828in}{1.575000in}}%
\pgfpathlineto{\pgfqpoint{1.784437in}{1.575000in}}%
\pgfpathlineto{\pgfqpoint{1.783046in}{1.575000in}}%
\pgfpathlineto{\pgfqpoint{1.781655in}{1.575000in}}%
\pgfpathlineto{\pgfqpoint{1.780264in}{1.575000in}}%
\pgfpathlineto{\pgfqpoint{1.778873in}{1.575000in}}%
\pgfpathlineto{\pgfqpoint{1.777482in}{1.575000in}}%
\pgfpathlineto{\pgfqpoint{1.776091in}{1.575000in}}%
\pgfpathlineto{\pgfqpoint{1.774700in}{1.575000in}}%
\pgfpathlineto{\pgfqpoint{1.773309in}{1.575000in}}%
\pgfpathlineto{\pgfqpoint{1.771918in}{1.575000in}}%
\pgfpathlineto{\pgfqpoint{1.770527in}{1.575000in}}%
\pgfpathlineto{\pgfqpoint{1.769136in}{1.575000in}}%
\pgfpathlineto{\pgfqpoint{1.767745in}{1.575000in}}%
\pgfpathlineto{\pgfqpoint{1.766354in}{1.575000in}}%
\pgfpathlineto{\pgfqpoint{1.764963in}{1.575000in}}%
\pgfpathlineto{\pgfqpoint{1.763572in}{1.575000in}}%
\pgfpathlineto{\pgfqpoint{1.762181in}{1.575000in}}%
\pgfpathlineto{\pgfqpoint{1.760790in}{1.575000in}}%
\pgfpathlineto{\pgfqpoint{1.759399in}{1.575000in}}%
\pgfpathlineto{\pgfqpoint{1.758008in}{1.575000in}}%
\pgfpathlineto{\pgfqpoint{1.756616in}{1.575000in}}%
\pgfpathlineto{\pgfqpoint{1.755225in}{1.575000in}}%
\pgfpathlineto{\pgfqpoint{1.753834in}{1.575000in}}%
\pgfpathlineto{\pgfqpoint{1.752443in}{1.575000in}}%
\pgfpathlineto{\pgfqpoint{1.751052in}{1.575000in}}%
\pgfpathlineto{\pgfqpoint{1.749661in}{1.575000in}}%
\pgfpathlineto{\pgfqpoint{1.748270in}{1.575000in}}%
\pgfpathlineto{\pgfqpoint{1.746879in}{1.575000in}}%
\pgfpathlineto{\pgfqpoint{1.745488in}{1.575000in}}%
\pgfpathlineto{\pgfqpoint{1.744097in}{1.575000in}}%
\pgfpathlineto{\pgfqpoint{1.742706in}{1.575000in}}%
\pgfpathlineto{\pgfqpoint{1.741315in}{1.575000in}}%
\pgfpathlineto{\pgfqpoint{1.739924in}{1.575000in}}%
\pgfpathlineto{\pgfqpoint{1.738533in}{1.575000in}}%
\pgfpathlineto{\pgfqpoint{1.737142in}{1.575000in}}%
\pgfpathlineto{\pgfqpoint{1.735751in}{1.575000in}}%
\pgfpathlineto{\pgfqpoint{1.734360in}{1.575000in}}%
\pgfpathlineto{\pgfqpoint{1.732969in}{1.575000in}}%
\pgfpathlineto{\pgfqpoint{1.731578in}{1.575000in}}%
\pgfpathlineto{\pgfqpoint{1.730187in}{1.575000in}}%
\pgfpathlineto{\pgfqpoint{1.728796in}{1.575000in}}%
\pgfpathlineto{\pgfqpoint{1.727405in}{1.575000in}}%
\pgfpathlineto{\pgfqpoint{1.726014in}{1.575000in}}%
\pgfpathlineto{\pgfqpoint{1.724623in}{1.575000in}}%
\pgfpathlineto{\pgfqpoint{1.723232in}{1.575000in}}%
\pgfpathlineto{\pgfqpoint{1.721841in}{1.575000in}}%
\pgfpathlineto{\pgfqpoint{1.720450in}{1.575000in}}%
\pgfpathlineto{\pgfqpoint{1.719059in}{1.575000in}}%
\pgfpathlineto{\pgfqpoint{1.717668in}{1.575000in}}%
\pgfpathlineto{\pgfqpoint{1.716277in}{1.575000in}}%
\pgfpathlineto{\pgfqpoint{1.714886in}{1.575000in}}%
\pgfpathlineto{\pgfqpoint{1.713494in}{1.575000in}}%
\pgfpathlineto{\pgfqpoint{1.712103in}{1.575000in}}%
\pgfpathlineto{\pgfqpoint{1.710712in}{1.575000in}}%
\pgfpathlineto{\pgfqpoint{1.709321in}{1.575000in}}%
\pgfpathlineto{\pgfqpoint{1.707930in}{1.575000in}}%
\pgfpathlineto{\pgfqpoint{1.706539in}{1.575000in}}%
\pgfpathlineto{\pgfqpoint{1.705148in}{1.575000in}}%
\pgfpathlineto{\pgfqpoint{1.703757in}{1.575000in}}%
\pgfpathlineto{\pgfqpoint{1.702366in}{1.575000in}}%
\pgfpathlineto{\pgfqpoint{1.700975in}{1.575000in}}%
\pgfpathlineto{\pgfqpoint{1.699584in}{1.575000in}}%
\pgfpathlineto{\pgfqpoint{1.698193in}{1.575000in}}%
\pgfpathlineto{\pgfqpoint{1.696802in}{1.575000in}}%
\pgfpathlineto{\pgfqpoint{1.695411in}{1.575000in}}%
\pgfpathlineto{\pgfqpoint{1.694020in}{1.575000in}}%
\pgfpathlineto{\pgfqpoint{1.692629in}{1.575000in}}%
\pgfpathlineto{\pgfqpoint{1.691238in}{1.575000in}}%
\pgfpathlineto{\pgfqpoint{1.689847in}{1.575000in}}%
\pgfpathlineto{\pgfqpoint{1.688456in}{1.575000in}}%
\pgfpathlineto{\pgfqpoint{1.687065in}{1.575000in}}%
\pgfpathlineto{\pgfqpoint{1.685674in}{1.575000in}}%
\pgfpathlineto{\pgfqpoint{1.684283in}{1.575000in}}%
\pgfpathlineto{\pgfqpoint{1.682892in}{1.575000in}}%
\pgfpathlineto{\pgfqpoint{1.681501in}{1.575000in}}%
\pgfpathlineto{\pgfqpoint{1.680110in}{1.575000in}}%
\pgfpathlineto{\pgfqpoint{1.678719in}{1.575000in}}%
\pgfpathlineto{\pgfqpoint{1.677328in}{1.575000in}}%
\pgfpathlineto{\pgfqpoint{1.675937in}{1.575000in}}%
\pgfpathlineto{\pgfqpoint{1.674546in}{1.575000in}}%
\pgfpathlineto{\pgfqpoint{1.673155in}{1.575000in}}%
\pgfpathlineto{\pgfqpoint{1.671764in}{1.575000in}}%
\pgfpathlineto{\pgfqpoint{1.670372in}{1.575000in}}%
\pgfpathlineto{\pgfqpoint{1.668981in}{1.575000in}}%
\pgfpathlineto{\pgfqpoint{1.667590in}{1.575000in}}%
\pgfpathlineto{\pgfqpoint{1.666199in}{1.575000in}}%
\pgfpathlineto{\pgfqpoint{1.664808in}{1.575000in}}%
\pgfpathlineto{\pgfqpoint{1.663417in}{1.575000in}}%
\pgfpathlineto{\pgfqpoint{1.662026in}{1.575000in}}%
\pgfpathlineto{\pgfqpoint{1.660635in}{1.575000in}}%
\pgfpathlineto{\pgfqpoint{1.659244in}{1.575000in}}%
\pgfpathlineto{\pgfqpoint{1.657853in}{1.575000in}}%
\pgfpathlineto{\pgfqpoint{1.656462in}{1.575000in}}%
\pgfpathlineto{\pgfqpoint{1.655071in}{1.575000in}}%
\pgfpathlineto{\pgfqpoint{1.653680in}{1.575000in}}%
\pgfpathlineto{\pgfqpoint{1.652289in}{1.575000in}}%
\pgfpathlineto{\pgfqpoint{1.650898in}{1.575000in}}%
\pgfpathlineto{\pgfqpoint{1.649507in}{1.575000in}}%
\pgfpathlineto{\pgfqpoint{1.648116in}{1.575000in}}%
\pgfpathlineto{\pgfqpoint{1.646725in}{1.575000in}}%
\pgfpathlineto{\pgfqpoint{1.645334in}{1.575000in}}%
\pgfpathlineto{\pgfqpoint{1.643943in}{1.575000in}}%
\pgfpathlineto{\pgfqpoint{1.642552in}{1.575000in}}%
\pgfpathlineto{\pgfqpoint{1.641161in}{1.575000in}}%
\pgfpathlineto{\pgfqpoint{1.639770in}{1.575000in}}%
\pgfpathlineto{\pgfqpoint{1.638379in}{1.575000in}}%
\pgfpathlineto{\pgfqpoint{1.636988in}{1.575000in}}%
\pgfpathlineto{\pgfqpoint{1.635597in}{1.575000in}}%
\pgfpathlineto{\pgfqpoint{1.634206in}{1.575000in}}%
\pgfpathlineto{\pgfqpoint{1.632815in}{1.575000in}}%
\pgfpathlineto{\pgfqpoint{1.631424in}{1.575000in}}%
\pgfpathlineto{\pgfqpoint{1.630033in}{1.575000in}}%
\pgfpathlineto{\pgfqpoint{1.628642in}{1.575000in}}%
\pgfpathlineto{\pgfqpoint{1.627251in}{1.575000in}}%
\pgfpathlineto{\pgfqpoint{1.625859in}{1.575000in}}%
\pgfpathlineto{\pgfqpoint{1.624468in}{1.575000in}}%
\pgfpathlineto{\pgfqpoint{1.623077in}{1.575000in}}%
\pgfpathlineto{\pgfqpoint{1.621686in}{1.575000in}}%
\pgfpathlineto{\pgfqpoint{1.620295in}{1.575000in}}%
\pgfpathlineto{\pgfqpoint{1.618904in}{1.575000in}}%
\pgfpathlineto{\pgfqpoint{1.617513in}{1.575000in}}%
\pgfpathlineto{\pgfqpoint{1.616122in}{1.575000in}}%
\pgfpathlineto{\pgfqpoint{1.614731in}{1.575000in}}%
\pgfpathlineto{\pgfqpoint{1.613340in}{1.575000in}}%
\pgfpathlineto{\pgfqpoint{1.611949in}{1.575000in}}%
\pgfpathlineto{\pgfqpoint{1.610558in}{1.575000in}}%
\pgfpathlineto{\pgfqpoint{1.609167in}{1.575000in}}%
\pgfpathlineto{\pgfqpoint{1.607776in}{1.575000in}}%
\pgfpathlineto{\pgfqpoint{1.606385in}{1.575000in}}%
\pgfpathlineto{\pgfqpoint{1.604994in}{1.575000in}}%
\pgfpathlineto{\pgfqpoint{1.603603in}{1.575000in}}%
\pgfpathlineto{\pgfqpoint{1.602212in}{1.575000in}}%
\pgfpathlineto{\pgfqpoint{1.600821in}{1.575000in}}%
\pgfpathlineto{\pgfqpoint{1.599430in}{1.575000in}}%
\pgfpathlineto{\pgfqpoint{1.598039in}{1.575000in}}%
\pgfpathlineto{\pgfqpoint{1.596648in}{1.575000in}}%
\pgfpathlineto{\pgfqpoint{1.595257in}{1.575000in}}%
\pgfpathlineto{\pgfqpoint{1.593866in}{1.575000in}}%
\pgfpathlineto{\pgfqpoint{1.592475in}{1.575000in}}%
\pgfpathlineto{\pgfqpoint{1.591084in}{1.575000in}}%
\pgfpathlineto{\pgfqpoint{1.589693in}{1.575000in}}%
\pgfpathlineto{\pgfqpoint{1.588302in}{1.575000in}}%
\pgfpathlineto{\pgfqpoint{1.586911in}{1.575000in}}%
\pgfpathlineto{\pgfqpoint{1.585520in}{1.575000in}}%
\pgfpathlineto{\pgfqpoint{1.584129in}{1.575000in}}%
\pgfpathlineto{\pgfqpoint{1.582737in}{1.575000in}}%
\pgfpathlineto{\pgfqpoint{1.581346in}{1.575000in}}%
\pgfpathlineto{\pgfqpoint{1.579955in}{1.575000in}}%
\pgfpathlineto{\pgfqpoint{1.578564in}{1.575000in}}%
\pgfpathlineto{\pgfqpoint{1.577173in}{1.575000in}}%
\pgfpathlineto{\pgfqpoint{1.575782in}{1.575000in}}%
\pgfpathlineto{\pgfqpoint{1.574391in}{1.575000in}}%
\pgfpathlineto{\pgfqpoint{1.573000in}{1.575000in}}%
\pgfpathlineto{\pgfqpoint{1.571609in}{1.575000in}}%
\pgfpathlineto{\pgfqpoint{1.570218in}{1.575000in}}%
\pgfpathlineto{\pgfqpoint{1.568827in}{1.575000in}}%
\pgfpathlineto{\pgfqpoint{1.567436in}{1.575000in}}%
\pgfpathlineto{\pgfqpoint{1.566045in}{1.575000in}}%
\pgfpathlineto{\pgfqpoint{1.564654in}{1.575000in}}%
\pgfpathlineto{\pgfqpoint{1.563263in}{1.575000in}}%
\pgfpathlineto{\pgfqpoint{1.561872in}{1.575000in}}%
\pgfpathlineto{\pgfqpoint{1.560481in}{1.575000in}}%
\pgfpathlineto{\pgfqpoint{1.559090in}{1.575000in}}%
\pgfpathlineto{\pgfqpoint{1.557699in}{1.575000in}}%
\pgfpathlineto{\pgfqpoint{1.556308in}{1.575000in}}%
\pgfpathlineto{\pgfqpoint{1.554917in}{1.575000in}}%
\pgfpathlineto{\pgfqpoint{1.553526in}{1.575000in}}%
\pgfpathlineto{\pgfqpoint{1.552135in}{1.575000in}}%
\pgfpathlineto{\pgfqpoint{1.550744in}{1.575000in}}%
\pgfpathlineto{\pgfqpoint{1.549353in}{1.575000in}}%
\pgfpathlineto{\pgfqpoint{1.547962in}{1.575000in}}%
\pgfpathlineto{\pgfqpoint{1.546571in}{1.575000in}}%
\pgfpathlineto{\pgfqpoint{1.545180in}{1.575000in}}%
\pgfpathlineto{\pgfqpoint{1.543789in}{1.575000in}}%
\pgfpathlineto{\pgfqpoint{1.542398in}{1.575000in}}%
\pgfpathlineto{\pgfqpoint{1.541007in}{1.575000in}}%
\pgfpathlineto{\pgfqpoint{1.539615in}{1.575000in}}%
\pgfpathlineto{\pgfqpoint{1.538224in}{1.575000in}}%
\pgfpathlineto{\pgfqpoint{1.536833in}{1.575000in}}%
\pgfpathlineto{\pgfqpoint{1.535442in}{1.575000in}}%
\pgfpathlineto{\pgfqpoint{1.534051in}{1.575000in}}%
\pgfpathlineto{\pgfqpoint{1.532660in}{1.575000in}}%
\pgfpathlineto{\pgfqpoint{1.531269in}{1.575000in}}%
\pgfpathlineto{\pgfqpoint{1.529878in}{1.575000in}}%
\pgfpathlineto{\pgfqpoint{1.528487in}{1.575000in}}%
\pgfpathlineto{\pgfqpoint{1.527096in}{1.575000in}}%
\pgfpathlineto{\pgfqpoint{1.525705in}{1.575000in}}%
\pgfpathlineto{\pgfqpoint{1.524314in}{1.575000in}}%
\pgfpathlineto{\pgfqpoint{1.522923in}{1.575000in}}%
\pgfpathlineto{\pgfqpoint{1.521532in}{1.575000in}}%
\pgfpathlineto{\pgfqpoint{1.520141in}{1.575000in}}%
\pgfpathlineto{\pgfqpoint{1.518750in}{1.575000in}}%
\pgfpathclose%
\pgfusepath{fill}%
\end{pgfscope}%
\begin{pgfscope}%
\pgfpathrectangle{\pgfqpoint{0.900000in}{0.600000in}}{\pgfqpoint{3.375000in}{1.950000in}} %
\pgfusepath{clip}%
\pgfsetbuttcap%
\pgfsetmiterjoin%
\definecolor{currentfill}{rgb}{0.000000,0.000000,0.000000}%
\pgfsetfillcolor{currentfill}%
\pgfsetlinewidth{1.003750pt}%
\definecolor{currentstroke}{rgb}{0.000000,0.000000,0.000000}%
\pgfsetstrokecolor{currentstroke}%
\pgfsetdash{}{0pt}%
\pgfpathmoveto{\pgfqpoint{4.181250in}{1.575000in}}%
\pgfpathlineto{\pgfqpoint{4.087500in}{1.526250in}}%
\pgfpathlineto{\pgfqpoint{4.087500in}{1.574976in}}%
\pgfpathlineto{\pgfqpoint{3.900000in}{1.574976in}}%
\pgfpathlineto{\pgfqpoint{3.900000in}{1.575024in}}%
\pgfpathlineto{\pgfqpoint{4.087500in}{1.575024in}}%
\pgfpathlineto{\pgfqpoint{4.087500in}{1.623750in}}%
\pgfpathclose%
\pgfusepath{stroke,fill}%
\end{pgfscope}%
\begin{pgfscope}%
\pgfpathrectangle{\pgfqpoint{0.900000in}{0.600000in}}{\pgfqpoint{3.375000in}{1.950000in}} %
\pgfusepath{clip}%
\pgfsetbuttcap%
\pgfsetmiterjoin%
\definecolor{currentfill}{rgb}{0.000000,0.000000,0.000000}%
\pgfsetfillcolor{currentfill}%
\pgfsetlinewidth{1.003750pt}%
\definecolor{currentstroke}{rgb}{0.000000,0.000000,0.000000}%
\pgfsetstrokecolor{currentstroke}%
\pgfsetdash{}{0pt}%
\pgfpathmoveto{\pgfqpoint{2.596875in}{2.501250in}}%
\pgfpathlineto{\pgfqpoint{2.643750in}{2.403750in}}%
\pgfpathlineto{\pgfqpoint{2.597812in}{2.403750in}}%
\pgfpathlineto{\pgfqpoint{2.597812in}{2.306250in}}%
\pgfpathlineto{\pgfqpoint{2.595938in}{2.306250in}}%
\pgfpathlineto{\pgfqpoint{2.595938in}{2.403750in}}%
\pgfpathlineto{\pgfqpoint{2.550000in}{2.403750in}}%
\pgfpathclose%
\pgfusepath{stroke,fill}%
\end{pgfscope}%
\begin{pgfscope}%
\pgfpathrectangle{\pgfqpoint{0.900000in}{0.600000in}}{\pgfqpoint{3.375000in}{1.950000in}} %
\pgfusepath{clip}%
\pgfsetrectcap%
\pgfsetroundjoin%
\pgfsetlinewidth{0.501875pt}%
\definecolor{currentstroke}{rgb}{1.000000,0.000000,0.000000}%
\pgfsetstrokecolor{currentstroke}%
\pgfsetdash{}{0pt}%
\pgfpathmoveto{\pgfqpoint{3.150000in}{1.770026in}}%
\pgfpathlineto{\pgfqpoint{3.518729in}{1.771212in}}%
\pgfpathlineto{\pgfqpoint{3.624080in}{1.773612in}}%
\pgfpathlineto{\pgfqpoint{3.691806in}{1.777287in}}%
\pgfpathlineto{\pgfqpoint{3.740719in}{1.782098in}}%
\pgfpathlineto{\pgfqpoint{3.778344in}{1.787868in}}%
\pgfpathlineto{\pgfqpoint{3.812207in}{1.795380in}}%
\pgfpathlineto{\pgfqpoint{3.838545in}{1.803346in}}%
\pgfpathlineto{\pgfqpoint{3.861120in}{1.812136in}}%
\pgfpathlineto{\pgfqpoint{3.883696in}{1.823244in}}%
\pgfpathlineto{\pgfqpoint{3.902508in}{1.834706in}}%
\pgfpathlineto{\pgfqpoint{3.921321in}{1.848636in}}%
\pgfpathlineto{\pgfqpoint{3.940134in}{1.865564in}}%
\pgfpathlineto{\pgfqpoint{3.958946in}{1.886137in}}%
\pgfpathlineto{\pgfqpoint{3.973997in}{1.905741in}}%
\pgfpathlineto{\pgfqpoint{3.989047in}{1.928655in}}%
\pgfpathlineto{\pgfqpoint{4.004097in}{1.955436in}}%
\pgfpathlineto{\pgfqpoint{4.019147in}{1.986738in}}%
\pgfpathlineto{\pgfqpoint{4.034197in}{2.023323in}}%
\pgfpathlineto{\pgfqpoint{4.049247in}{2.066085in}}%
\pgfpathlineto{\pgfqpoint{4.064298in}{2.116064in}}%
\pgfpathlineto{\pgfqpoint{4.079348in}{2.174480in}}%
\pgfpathlineto{\pgfqpoint{4.094398in}{2.242757in}}%
\pgfpathlineto{\pgfqpoint{4.109448in}{2.322559in}}%
\pgfpathlineto{\pgfqpoint{4.124498in}{2.415832in}}%
\pgfpathlineto{\pgfqpoint{4.139548in}{2.524850in}}%
\pgfpathlineto{\pgfqpoint{4.143930in}{2.560000in}}%
\pgfpathlineto{\pgfqpoint{4.143930in}{2.560000in}}%
\pgfusepath{stroke}%
\end{pgfscope}%
\begin{pgfscope}%
\pgfpathrectangle{\pgfqpoint{0.900000in}{0.600000in}}{\pgfqpoint{3.375000in}{1.950000in}} %
\pgfusepath{clip}%
\pgfsetrectcap%
\pgfsetroundjoin%
\pgfsetlinewidth{0.501875pt}%
\definecolor{currentstroke}{rgb}{0.000000,0.000000,1.000000}%
\pgfsetstrokecolor{currentstroke}%
\pgfsetdash{}{0pt}%
\pgfpathmoveto{\pgfqpoint{3.150000in}{1.770114in}}%
\pgfpathlineto{\pgfqpoint{3.349415in}{1.771509in}}%
\pgfpathlineto{\pgfqpoint{3.428428in}{1.774201in}}%
\pgfpathlineto{\pgfqpoint{3.477341in}{1.777917in}}%
\pgfpathlineto{\pgfqpoint{3.514967in}{1.782890in}}%
\pgfpathlineto{\pgfqpoint{3.545067in}{1.789037in}}%
\pgfpathlineto{\pgfqpoint{3.571405in}{1.796778in}}%
\pgfpathlineto{\pgfqpoint{3.593980in}{1.805875in}}%
\pgfpathlineto{\pgfqpoint{3.612793in}{1.815777in}}%
\pgfpathlineto{\pgfqpoint{3.631605in}{1.828410in}}%
\pgfpathlineto{\pgfqpoint{3.646656in}{1.840985in}}%
\pgfpathlineto{\pgfqpoint{3.661706in}{1.856266in}}%
\pgfpathlineto{\pgfqpoint{3.676756in}{1.874837in}}%
\pgfpathlineto{\pgfqpoint{3.691806in}{1.897407in}}%
\pgfpathlineto{\pgfqpoint{3.706856in}{1.924835in}}%
\pgfpathlineto{\pgfqpoint{3.721906in}{1.958167in}}%
\pgfpathlineto{\pgfqpoint{3.736957in}{1.998676in}}%
\pgfpathlineto{\pgfqpoint{3.752007in}{2.047905in}}%
\pgfpathlineto{\pgfqpoint{3.763294in}{2.091665in}}%
\pgfpathlineto{\pgfqpoint{3.774582in}{2.142315in}}%
\pgfpathlineto{\pgfqpoint{3.785870in}{2.200940in}}%
\pgfpathlineto{\pgfqpoint{3.797157in}{2.268797in}}%
\pgfpathlineto{\pgfqpoint{3.808445in}{2.347339in}}%
\pgfpathlineto{\pgfqpoint{3.819732in}{2.438249in}}%
\pgfpathlineto{\pgfqpoint{3.832630in}{2.560000in}}%
\pgfpathlineto{\pgfqpoint{3.832630in}{2.560000in}}%
\pgfusepath{stroke}%
\end{pgfscope}%
\begin{pgfscope}%
\pgfpathrectangle{\pgfqpoint{0.900000in}{0.600000in}}{\pgfqpoint{3.375000in}{1.950000in}} %
\pgfusepath{clip}%
\pgfsetrectcap%
\pgfsetroundjoin%
\pgfsetlinewidth{0.501875pt}%
\definecolor{currentstroke}{rgb}{0.000000,0.501961,0.000000}%
\pgfsetstrokecolor{currentstroke}%
\pgfsetdash{}{0pt}%
\pgfpathmoveto{\pgfqpoint{3.150000in}{1.771139in}}%
\pgfpathlineto{\pgfqpoint{3.236538in}{1.773497in}}%
\pgfpathlineto{\pgfqpoint{3.289214in}{1.776919in}}%
\pgfpathlineto{\pgfqpoint{3.330602in}{1.781829in}}%
\pgfpathlineto{\pgfqpoint{3.360702in}{1.787470in}}%
\pgfpathlineto{\pgfqpoint{3.387040in}{1.794575in}}%
\pgfpathlineto{\pgfqpoint{3.409615in}{1.802924in}}%
\pgfpathlineto{\pgfqpoint{3.428428in}{1.812010in}}%
\pgfpathlineto{\pgfqpoint{3.447241in}{1.823605in}}%
\pgfpathlineto{\pgfqpoint{3.462291in}{1.835145in}}%
\pgfpathlineto{\pgfqpoint{3.477341in}{1.849169in}}%
\pgfpathlineto{\pgfqpoint{3.492391in}{1.866213in}}%
\pgfpathlineto{\pgfqpoint{3.507441in}{1.886925in}}%
\pgfpathlineto{\pgfqpoint{3.522492in}{1.912097in}}%
\pgfpathlineto{\pgfqpoint{3.537542in}{1.942688in}}%
\pgfpathlineto{\pgfqpoint{3.552592in}{1.979864in}}%
\pgfpathlineto{\pgfqpoint{3.567642in}{2.025044in}}%
\pgfpathlineto{\pgfqpoint{3.578930in}{2.065204in}}%
\pgfpathlineto{\pgfqpoint{3.590217in}{2.111687in}}%
\pgfpathlineto{\pgfqpoint{3.601505in}{2.165490in}}%
\pgfpathlineto{\pgfqpoint{3.612793in}{2.227765in}}%
\pgfpathlineto{\pgfqpoint{3.624080in}{2.299847in}}%
\pgfpathlineto{\pgfqpoint{3.635368in}{2.383278in}}%
\pgfpathlineto{\pgfqpoint{3.646656in}{2.479846in}}%
\pgfpathlineto{\pgfqpoint{3.654899in}{2.560000in}}%
\pgfpathlineto{\pgfqpoint{3.654899in}{2.560000in}}%
\pgfusepath{stroke}%
\end{pgfscope}%
\begin{pgfscope}%
\pgfpathrectangle{\pgfqpoint{0.900000in}{0.600000in}}{\pgfqpoint{3.375000in}{1.950000in}} %
\pgfusepath{clip}%
\pgfsetrectcap%
\pgfsetroundjoin%
\pgfsetlinewidth{0.501875pt}%
\definecolor{currentstroke}{rgb}{0.000000,0.000000,1.000000}%
\pgfsetstrokecolor{currentstroke}%
\pgfsetdash{}{0pt}%
\pgfpathmoveto{\pgfqpoint{1.347279in}{0.590000in}}%
\pgfpathlineto{\pgfqpoint{1.357776in}{0.684367in}}%
\pgfpathlineto{\pgfqpoint{1.370318in}{0.781377in}}%
\pgfpathlineto{\pgfqpoint{1.382860in}{0.863839in}}%
\pgfpathlineto{\pgfqpoint{1.395401in}{0.933935in}}%
\pgfpathlineto{\pgfqpoint{1.407943in}{0.993518in}}%
\pgfpathlineto{\pgfqpoint{1.420485in}{1.044166in}}%
\pgfpathlineto{\pgfqpoint{1.433027in}{1.087219in}}%
\pgfpathlineto{\pgfqpoint{1.445569in}{1.123815in}}%
\pgfpathlineto{\pgfqpoint{1.458110in}{1.154923in}}%
\pgfpathlineto{\pgfqpoint{1.473788in}{1.187332in}}%
\pgfpathlineto{\pgfqpoint{1.489465in}{1.213784in}}%
\pgfpathlineto{\pgfqpoint{1.505142in}{1.235375in}}%
\pgfpathlineto{\pgfqpoint{1.520819in}{1.252997in}}%
\pgfpathlineto{\pgfqpoint{1.536497in}{1.267380in}}%
\pgfpathlineto{\pgfqpoint{1.552174in}{1.279119in}}%
\pgfpathlineto{\pgfqpoint{1.570987in}{1.290395in}}%
\pgfpathlineto{\pgfqpoint{1.589799in}{1.299231in}}%
\pgfpathlineto{\pgfqpoint{1.611747in}{1.307156in}}%
\pgfpathlineto{\pgfqpoint{1.636831in}{1.313840in}}%
\pgfpathlineto{\pgfqpoint{1.668186in}{1.319652in}}%
\pgfpathlineto{\pgfqpoint{1.705811in}{1.324127in}}%
\pgfpathlineto{\pgfqpoint{1.752843in}{1.327377in}}%
\pgfpathlineto{\pgfqpoint{1.821823in}{1.329665in}}%
\pgfpathlineto{\pgfqpoint{1.837500in}{1.329957in}}%
\pgfpathlineto{\pgfqpoint{1.837500in}{1.329957in}}%
\pgfusepath{stroke}%
\end{pgfscope}%
\begin{pgfscope}%
\pgfpathrectangle{\pgfqpoint{0.900000in}{0.600000in}}{\pgfqpoint{3.375000in}{1.950000in}} %
\pgfusepath{clip}%
\pgfsetrectcap%
\pgfsetroundjoin%
\pgfsetlinewidth{0.501875pt}%
\definecolor{currentstroke}{rgb}{0.000000,0.000000,1.000000}%
\pgfsetstrokecolor{currentstroke}%
\pgfsetdash{}{0pt}%
\pgfpathmoveto{\pgfqpoint{1.837500in}{1.330034in}}%
\pgfpathlineto{\pgfqpoint{2.213754in}{1.331144in}}%
\pgfpathlineto{\pgfqpoint{2.283988in}{1.333437in}}%
\pgfpathlineto{\pgfqpoint{2.329139in}{1.336989in}}%
\pgfpathlineto{\pgfqpoint{2.364256in}{1.342091in}}%
\pgfpathlineto{\pgfqpoint{2.389339in}{1.347806in}}%
\pgfpathlineto{\pgfqpoint{2.409406in}{1.354167in}}%
\pgfpathlineto{\pgfqpoint{2.429473in}{1.362630in}}%
\pgfpathlineto{\pgfqpoint{2.449540in}{1.373749in}}%
\pgfpathlineto{\pgfqpoint{2.469607in}{1.388122in}}%
\pgfpathlineto{\pgfqpoint{2.484657in}{1.401380in}}%
\pgfpathlineto{\pgfqpoint{2.499707in}{1.416978in}}%
\pgfpathlineto{\pgfqpoint{2.519774in}{1.441574in}}%
\pgfpathlineto{\pgfqpoint{2.539841in}{1.470371in}}%
\pgfpathlineto{\pgfqpoint{2.564925in}{1.511213in}}%
\pgfpathlineto{\pgfqpoint{2.640176in}{1.638446in}}%
\pgfpathlineto{\pgfqpoint{2.660242in}{1.666232in}}%
\pgfpathlineto{\pgfqpoint{2.680309in}{1.689740in}}%
\pgfpathlineto{\pgfqpoint{2.700376in}{1.708958in}}%
\pgfpathlineto{\pgfqpoint{2.720443in}{1.724236in}}%
\pgfpathlineto{\pgfqpoint{2.740510in}{1.736112in}}%
\pgfpathlineto{\pgfqpoint{2.760577in}{1.745184in}}%
\pgfpathlineto{\pgfqpoint{2.780644in}{1.752023in}}%
\pgfpathlineto{\pgfqpoint{2.805727in}{1.758181in}}%
\pgfpathlineto{\pgfqpoint{2.835828in}{1.763072in}}%
\pgfpathlineto{\pgfqpoint{2.875962in}{1.766895in}}%
\pgfpathlineto{\pgfqpoint{2.931145in}{1.769420in}}%
\pgfpathlineto{\pgfqpoint{3.016430in}{1.770758in}}%
\pgfpathlineto{\pgfqpoint{3.232149in}{1.771204in}}%
\pgfpathlineto{\pgfqpoint{3.337500in}{1.771216in}}%
\pgfpathlineto{\pgfqpoint{3.337500in}{1.771216in}}%
\pgfusepath{stroke}%
\end{pgfscope}%
\begin{pgfscope}%
\pgfpathrectangle{\pgfqpoint{0.900000in}{0.600000in}}{\pgfqpoint{3.375000in}{1.950000in}} %
\pgfusepath{clip}%
\pgfsetrectcap%
\pgfsetroundjoin%
\pgfsetlinewidth{0.501875pt}%
\definecolor{currentstroke}{rgb}{0.000000,0.000000,0.000000}%
\pgfsetstrokecolor{currentstroke}%
\pgfsetdash{}{0pt}%
\pgfpathmoveto{\pgfqpoint{0.900000in}{1.575000in}}%
\pgfpathlineto{\pgfqpoint{4.275000in}{1.575000in}}%
\pgfusepath{stroke}%
\end{pgfscope}%
\begin{pgfscope}%
\pgfpathrectangle{\pgfqpoint{0.900000in}{0.600000in}}{\pgfqpoint{3.375000in}{1.950000in}} %
\pgfusepath{clip}%
\pgfsetrectcap%
\pgfsetroundjoin%
\pgfsetlinewidth{0.501875pt}%
\definecolor{currentstroke}{rgb}{0.000000,0.000000,0.000000}%
\pgfsetstrokecolor{currentstroke}%
\pgfsetdash{}{0pt}%
\pgfpathmoveto{\pgfqpoint{2.596875in}{0.600000in}}%
\pgfpathlineto{\pgfqpoint{2.596875in}{2.550000in}}%
\pgfusepath{stroke}%
\end{pgfscope}%
\begin{pgfscope}%
\pgfsetrectcap%
\pgfsetmiterjoin%
\pgfsetlinewidth{0.501875pt}%
\definecolor{currentstroke}{rgb}{0.000000,0.000000,0.000000}%
\pgfsetstrokecolor{currentstroke}%
\pgfsetdash{}{0pt}%
\pgfpathmoveto{\pgfqpoint{0.900000in}{0.600000in}}%
\pgfpathlineto{\pgfqpoint{4.275000in}{0.600000in}}%
\pgfusepath{stroke}%
\end{pgfscope}%
\begin{pgfscope}%
\pgfsetrectcap%
\pgfsetmiterjoin%
\pgfsetlinewidth{0.501875pt}%
\definecolor{currentstroke}{rgb}{0.000000,0.000000,0.000000}%
\pgfsetstrokecolor{currentstroke}%
\pgfsetdash{}{0pt}%
\pgfpathmoveto{\pgfqpoint{0.900000in}{2.550000in}}%
\pgfpathlineto{\pgfqpoint{4.275000in}{2.550000in}}%
\pgfusepath{stroke}%
\end{pgfscope}%
\begin{pgfscope}%
\pgfsetrectcap%
\pgfsetmiterjoin%
\pgfsetlinewidth{0.501875pt}%
\definecolor{currentstroke}{rgb}{0.000000,0.000000,0.000000}%
\pgfsetstrokecolor{currentstroke}%
\pgfsetdash{}{0pt}%
\pgfpathmoveto{\pgfqpoint{4.275000in}{0.600000in}}%
\pgfpathlineto{\pgfqpoint{4.275000in}{2.550000in}}%
\pgfusepath{stroke}%
\end{pgfscope}%
\begin{pgfscope}%
\pgfsetrectcap%
\pgfsetmiterjoin%
\pgfsetlinewidth{0.501875pt}%
\definecolor{currentstroke}{rgb}{0.000000,0.000000,0.000000}%
\pgfsetstrokecolor{currentstroke}%
\pgfsetdash{}{0pt}%
\pgfpathmoveto{\pgfqpoint{0.900000in}{0.600000in}}%
\pgfpathlineto{\pgfqpoint{0.900000in}{2.550000in}}%
\pgfusepath{stroke}%
\end{pgfscope}%
\begin{pgfscope}%
\pgftext[x=2.587500in,y=0.530556in,,top]{\rmfamily\fontsize{9.000000}{10.800000}\selectfont Spannung (V)}%
\end{pgfscope}%
\begin{pgfscope}%
\pgftext[x=0.830556in,y=1.575000in,,bottom,rotate=90.000000]{\rmfamily\fontsize{9.000000}{10.800000}\selectfont Strom (A)}%
\end{pgfscope}%
\begin{pgfscope}%
\pgftext[x=2.025000in,y=1.623750in,left,base]{\rmfamily\fontsize{9.000000}{10.800000}\selectfont \(\displaystyle I_S\)}%
\end{pgfscope}%
\begin{pgfscope}%
\pgftext[x=2.587500in,y=2.619444in,,base]{\rmfamily\fontsize{11.000000}{13.200000}\selectfont IV-Kurve einer Diode}%
\end{pgfscope}%
\end{pgfpicture}%
\makeatother%
\endgroup%

    \caption{%
        Vereinfachte Strom-Spannungs-Kurve einer Diode. Der Reverse Saturation
        Current   $I_{\mathrm{S}}$   tritt   im   \textcolor{magenta}{magenta}
        eingef\"arbten Bereich auf. Der angegebene Zahlenwert bezieht sich auf
        den  Bereich  der  Kurve,  in der  $I_{\mathrm{S}}$  relativ  konstant
        ist.\protect\\
        Die    \textcolor{blue}{blaue   Kurve}    dient   als    Referenz. Die
        \textcolor{red}{rote  Kurve}  zeigt  den Einfluss  eines  ansteigenden
        Idealit\"atsfaktors  (also  st\"arkeres  Abweichen von  einer  idealen
        Diode) auf die IV-Kennlinie.\protect\\
        Die \textcolor{green!50!black}{gr\"une Kurve} zeigt den Einfluss eines
        ansteigenden  Reverse Saturation  Current: Der Diodenstrom  steigt bei
        einer gegebenen Diodenspannung.\protect\\
        Zur  Verbesserung  der  \"Ubersichtlichkeit  wurden  die  abweichenden
        Kurven nur  im positiven Bereich geplottet,  nat\"urlich \"andert sich
        auch das Verhalten im Reverse-Betrieb analog.%
    }
    \label{fig:diodeVI:IS}
\end{figure}

\todo{Reverse-Betrieb-Kurven: analog?}

Zur  Simulation  in  \code{LTspice}  wird das  Diodenmodell  auf  den  Reverse
Saturation  Current   und  den  Idealit\"atsfaktor  reduziert,   und  folgende
gesch\"atzte Parameter als Ausgangslage f\"ur den Iterationsprozess benutzt:

\begin{center}
    \code{.model diode1 D(IS=1e-6 N=2)}
\end{center}

Es wird die Schaltung gem\"ass Abbildung \ref{fig:circuit:solarCell} von Seite
\pageref{fig:circuit:solarCell} aufgebaut und die  oben bestimmten Werte f\"ur
$R_{\mathrm{S}}$, $R_{\mathrm{P}}$ und $C$ eingesetzt.

Mit einer  Transientensimulation (\code{.tran  1m}) wird die  Zelle simuliert,
ihre  Leerlaufspannung  $V_{\mathrm{offen}}$  gemessen und  mit  dem  Zielwert
aus   Gleichung~\ref{eq:voffen}  verglichen\todo{``wird''   nur  einmal   oder
wiederholen?}.  Anschliessend  werden die  Werte f\"ur \code{IS}  und \code{N}
angepasst, bis die gew\"unschte Leerlaufspannung erreicht ist.

Nach einigen Iterationen liefert dieser Prozess\footnotemark:

\footnotetext{%
    Dies ist eine m\"ogliche L\"osung. Es gibt nat\"urlich noch beliebig viele
    weitere  Kombinationen von  \code{IS} und  \code{N}, welche  die gegebenen
    Bedingungen  erf\"ullen. Wir   sind  hier  lediglich  an   einer  L\"osung
    interessiert, von  der wir zuversichtlich  sind, dass sie  ein hinreichend
    gutes Modell liefert.%
}

\begin{align}
    \label{eq:cell:diode:IS:N:result}
    I_s &= \SI{4}{\micro\ampere} \\
    N   &= 1.65
\end{align}

Somit ist  das Modell  einer einzelnen  Zelle bestimmt  und kann  dazu benutzt
werden,  ein   Modul  aufzubauen.


% ---------------------------------------------------------------------------- %
\section{Modellierung eines PV-Moduls}
\label{sec:simu:model:module}
% ---------------------------------------------------------------------------- %

\todo{referenz auf vereinfachtes Modell}

Wie  in   Anhang  \ref{app:simu:module}  ab   Seite  \pageref{app:simu:module}
erw\"ahnt,     werden     zwei      verschiedene     Module     simuliert: Ein
Modul    mit    zwei    parallelen     Str\"angen    zu    je    36    seriell
geschalteten    Zellen     (\fref{fig:ltspice:module:cellBased:36x2},    Seite
\pageref{fig:ltspice:module:cellBased:36x2})    und   ein    Modul   mit    72
in   Serie   geschalteten  Zellen   (\fref{fig:ltspice:module:cellBased:72x1},
Seite     \pageref{fig:ltspice:module:cellBased:72x1}). Die     $36     \times
2$-Konfiguration\todo{Bindestrich?}   hat   eine  h\"ohere   Kapazit\"at   als
die   $72  \times   1$-Anordnung,   liefert  daf\"ur   aber   nur  die   halbe
Ausgangsspannung. Der Ausgangsstrom  beider Module wurde  auf \SI{10}{\ampere}
festgelegt,  die Stromquellen  im  $36 \times  2$-Modul  geben also  lediglich
\SI{5}{\ampere}  ab. Basierend auf  Tabelle \ref{tab:moduleData:IU}  von Seite
\pageref{tab:moduleData:IU}  mit Informationen  zu kommerziell  erh\"altlichen
Solarmodulen scheint uns dieser Ansatz gerechtfertigt.

\fref{fig:ltspice:solarCell}   zeigt   die  Implementation   unseres   Modells
einer     Solarzelle      gem\"ass     \fref{fig:circuit:solarCell}     (Seite
\pageref{fig:circuit:solarCell} in Abschnitt \ref{subsubsec:hw:ask:modell}) in
\code{LTspice}.

Ein MOSFET  wird benutzt, um  die Zelle  gesteuert (in unserer  Simulation bei
einer Tr\"agerfrequenz von \SI{10}{\kilo\hertz}) kurzschliessen zu k\"onnen.

Es    soll   das    f\"ur   den    MOSFET   schlimmere    Szenario   evaluiert
werden   (mehr    Strom/Leistung   durch    den   MOSFET\todo{Korrekt?}). Dazu
werden   aus  72   Zellen  zwei   verschiedene  Module   aufgebaut: Ein  Modul
\"ahnlich    zum    Sunset   Solargenerator    AS150    \cite{ref:solar:as150}
mit    ca. \SI{10}{\ampere}   Kurzschlussstrom    und   etwa    \SI{22}{\volt}
Leerlaufspannung  (\fref{fig:ltspice:module:cellBased:36x2}),  und  ein  Modul
\"ahnlich  zum Sunmodule  Pro-Series  XL SWB320  \cite{ref:solar:sunmodulePro}
mit   \SI{10}{\ampere}   Kurzschlussstrom    und   ungef\"ahr   \SI{45}{\volt}
Leerlaufspannung (\fref{fig:ltspice:module:cellBased:72x1}).

Weil die gesamte Kapazit\"at bei in Serie geschalteten Kondensatoren sinkt und
bei  paralleler  Anordnung steigt,  hat  die  $2 \times  36$-Parallelschaltung
\todo{Bindestrich und Spaces?} etwa die doppelte Kapazit\"at des Einzelstrangs
aus 72 Zellen. Der  durch den MOSFET fliessende Stroms, die  \"uber den MOSFET
abfallende Spannung und die im  MOSFET dissipierte Leistung beim Durchschalten
des Transistors sollen zwischen den beiden Szenarien verglichen werden.

Die zu  diesen Schaltungen geh\"orenden \code{.asc}-Dateien  sind elektronisch
verf\"ugbar  (Datentr\"ager  siehe Anhang  \ref{app:electronicStorage},  Seite
\pageref{app:electronicStorage}).

\todo{kleines Bild der gesamten Schaltung?}


% ---------------------------------------------------------------------------- %
\section{Frequenzumtastung: Kapazitive Einkopplung}
\label{sec:simu:fsk:capacitive}
% ---------------------------------------------------------------------------- %

% ---------------------------------------------------------------------------- %
\subsection{Sender}
\label{sec:simu:fsk:capacitive:transmitter}
% ---------------------------------------------------------------------------- %

% ---------------------------------------------------------------------------- %
\subsection{Empf\"anger}
\label{sec:simu:fsk:capacitive:receiver}
% ---------------------------------------------------------------------------- %

% ---------------------------------------------------------------------------- %
\subsection{Gesamtsystem}
\label{sec:simu:fsk:capacitive:system}
% ---------------------------------------------------------------------------- %

% ---------------------------------------------------------------------------- %
\section{Frequenzumtastung: Induktive Einkopplung}
\label{sec:simu:fsk:inductive}
% ---------------------------------------------------------------------------- %

Eine induktive Einkopplung legt eine  Spule um die DC-Leitung. Auf diese Spule
wird von der FSK-Schaltung das zu  \"ubertragende Signal gegeben und die Spule
induziert in  der DC-Leitung  entsprechende Spannungs-Rippel, die  vom \Master
ausgewertet werden k\"onnen. Der entsprechende  Schaltkreis ist schematisch in
Abbildung \ref{fig:circ:coupling:inductive} dargestellt.

Verglichen mit  Kondensatoren sind Spulen relativ  gross und teuer. Allerdings
ist das Prinzip  der induktiven Einkopplung gut dokumentiert  und die Aussicht
auf Erfolg (bei vern\"unftigem Aufwand) somit gut.

\begin{figure}[h!tb]
    \centering
    \begin{circuitikz}
    \draw
    (-1,0) to[empty photodiode,o-,l_=PV-Zelle] (1,0) to[short] (6,0)
    %(2,-2) to[short,o-] (2,-0.05) to[L=L] (4,-0.05) to[short,-o] (4,-2)
    (2,-2) -- (2,-0.05) to[L,l^=Kopplung] (4,-0.05) -- (4,-2) to[vco,l^=$U_{\mathrm{Signal}}$] (2,-2)
    ;
\end{circuitikz}

    \caption{Induktive Einkopplung}
    \label{fig:circ:coupling:inductive}
\end{figure}

% ---------------------------------------------------------------------------- %
\subsection{Sender}
\label{sec:simu:fsk:inductive:transmitter}
% ---------------------------------------------------------------------------- %

% ---------------------------------------------------------------------------- %
\subsection{Empf\"anger}
\label{sec:simu:fsk:inductive:receiver}
% ---------------------------------------------------------------------------- %

% ---------------------------------------------------------------------------- %
\subsection{Gesamtsystem}
\label{sec:simu:fsk:inductive:inductive}
% ---------------------------------------------------------------------------- %


% ---------------------------------------------------------------------------- %
\section{Amplitudenumtastung}
\label{sec:simu:ask}
% ---------------------------------------------------------------------------- %

Wie in Abschnitt \todo{reference} erw\"ahnt, wird bei dieser L\"osungsvariante
jeweils   ein   Modul   gesteuert   kurzgeschlossen. Dies   verursacht   kurze
Spannungseinbr\"uche auf  der DC-Leitung,  welche vom  Empf\"anger ausgewertet
werden k\"onnen, wie in Abbildung \todo{reference}vereinfact dargestellt.

Potentielle Probleme sind bei  dieser L\"osungsvariante in folgenden Bereichen
zu suchen:

\begin{symbols}
    \firmlist
    \item[\textbf{Induktivit\"at der Leitung:}]
        Der  Spannungsabfall   auf  der  DC-Leitung  wird   bei  geschlossenem
        Stromkreis zu Strom\"anderungen auf der DC-Leitung f\"uhren. Dies wird
        eine Spannungs\"anderung auf der DC-Leitung bewirken\footnotemark.
        \footnotetext{%
            Spannung in Abh\"angigkeit der Strom\"anderung:
            $v = L \cdot \frac{\mathrm{d}i}{\mathrm{d}t}$%
        }

        Gem\"ass    Lenz'scher     Regel    \todo{reference}     wird    diese
        Spannungs\"anderung  so gerichtet  sein,  dass  sich der  zugeh\"orige
        Strom der aufgezwungenen \"Anderung widersetzt.

        Es  kann also  sein, dass  die  Induktivit\"at der  DC-Leitung das  zu
        \"ubertragende  Signal  so  stark  kompensiert,  dass  es  nicht  mehr
        detektierbar  ist. Je   h\"oher  die  Frequenz,  mit   der  das  Modul
        kurzgeschlossen  und  wieder  ge\"offnet   wird,  um  so  h\"oher  die
        zugeh\"origen Strom\"anderungen
        $\frac{\mathrm{d}i}{\mathrm{d}{t}}$
        und somit Impedanz der Induktivit\"at (gem\"ass
        $Z_L = j \omega L$).
    \item[\textbf{Kapazit\"at des Solarmoduls:}]
        Dem  Solarmodul   wird  bei   diesem  Vorgehen  das   Verhalten  einer
        Wechselstromquelle   aufgezwungen. Besitzt  es   interne  parasit\"are
        Kapazit\"aten,  k\"onnen  diese  bei  den  abrupten  \"Anderungen  der
        Spannung hohe Str\"ome im  kurzgeschlossenen Pfad und seinen Bauteilen
        verursachen.\footnotemark
        \footnotetext{%
            Strom\"anderung in Abh\"angigkeit der Spannungs\"anderung:
            $I(t) = C \cdot \frac{\mathrm{V(t)}}{\mathrm{d}t}$%
        }

        Besitzen    diese   Bauteile    Ohm'sche   Widerst\"ande,    entstehen
        entsprechende thermische  Verluste, welche die  Bauteile besch\"adigen
        k\"onnen.
\end{symbols}

Die    Tr\"agerfrequenz    in    den    folgenden    Simulationen    betr\"agt
\SI{10}{\kilo\hertz}.


% ---------------------------------------------------------------------------- %
\subsection{Sender}
\label{subsec:simu:ask:sensor}
% ---------------------------------------------------------------------------- %

Das  gesteuerte Kurzschliessen  des Moduls  wird mit  einem MOSFET  umgesetzt,
welcher zwischen  Eingang und  Ausgang des Moduls  durchschalten kann  und vom
Microcontroller auf dem Sensor gesteuert wird. \fref{fig:module:mosfet:simple}
zeigt diesen Aufbau schematisch.

Die  Ansteuerung des  Transistors  erfolgt mit  \SI{3.3}{\volt},  da dies  die
maximale Spannung  ist, welche der  auf dem Sensor  platzierte Microcontroller
ausgeben kann.

\begin{figure}[h!tb]
    \centering
    \input{images/circuits/moduleMOSFET.tex}
    \caption{%
        Gesteuerter     Kurzschluss     eines    Solarmoduls     mit     einem
        microcontroller-gesteuerten       Transistor. Die      vollst\"andigen
        \code{LTspice}-Schaltungen  sind  in Anhang  \ref{app:simu:module}  ab
        Seite \pageref{app:simu:module} dokumentiert.%
    }
    \label{fig:module:mosfet:simple}
\end{figure}

%TODO: line width
\begin{figure}
    \begin{tikzpicture}
       \begin{scope}[x={(0mm,0mm)},y={(120mm,\textwidth)}]
           \begin{axis}[%
                   height=40mm,
                   width=\textwidth,
                   at={(0,70mm)},
                   grid=both,
                   xlabel=Zeit (\si{\micro\second}),
                   ylabel=Strom (\si{\ampere}),
                   %x unit=u,
                   change x base=true,
                   %line width = 1pt,
                   thick,
                   x SI prefix=micro,
               ]
               \addplot[-,color=blue] table {data/module-72cells--I-MOSFET--0005u.dat};
           \end{axis}
           \begin{axis}[%
                   height=40mm,
                   width=\textwidth,
                   at={(0,35mm)},
                   grid=both,
                   xlabel=Zeit (\si{\micro\second}),
                   ylabel=Spannung (\si{\volt}),
                   %x unit=u,
                   change x base=true,
                   x SI prefix=micro,
               ]
               \addplot[-,color=magenta] table {data/module-72cells--U-MOSFET--0005u.dat};
           \end{axis}
           \begin{axis}[%
                   height=40mm,
                   width=\textwidth,
                   at={(0,0)},
                   grid=both,
                   xlabel=Zeit (\si{\micro\second}),
                   ylabel=Leistung (\si{\watt}),
                   %x unit=u,
                   change x base=true,
                   x SI prefix=micro,
               ]
               \addplot[-,color=red] table {data/module-72cells--P-MOSFET--0005u.dat};
           \end{axis}
       \end{scope}
   \end{tikzpicture}
   \caption{%
       Verlauf    von    Strom,    Spannung    und    Leistung    am    MOSFET
       bei   einer   Schaltfrequenz   von   \SI{10}{\kilo\hertz}   bei   einer
       Modulkonfiguration    von    $36     \times    2$    Zellen    gem\"ass
       Schema    in    \fref{fig:ltspice:module:cellBased:36x2}   auf    Seite
       \ref{fig:ltspice:module:cellBased:36x2}.\protect\\
       Anhang       \ref{app:sec:simu:complementary:36x2}      auf       Seite
       \pageref{app:sec:simu:complementary:36x2} beinhaltet zum Vergleich noch
       Simulationen f\"ur den Zeitraum von einer Millisekunde.%
   }
   \label{fig:simu:results:36x2:3u}
\end{figure}

\begin{figure}
    \begin{tikzpicture}
       \begin{scope}[x={(0mm,0mm)},y={(120mm,\textwidth)}]
           \begin{axis}[%
                   height=40mm,
                   width=\textwidth,
                   at={(0,70mm)},
                   grid=both,
                   xlabel=Zeit (\si{\micro\second}),
                   ylabel=Strom (\si{\ampere}),
                   %x unit=u,
                   change x base=true,
                   x SI prefix=micro,
               ]
               \addplot[-,color=blue] table {data/module-72cells-series--I-MOSFET--0003u.dat};
           \end{axis}
           \begin{axis}[%
                   height=40mm,
                   width=\textwidth,
                   at={(0,35mm)},
                   grid=both,
                   xlabel=Zeit (\si{\micro\second}),
                   ylabel=Spannung (\si{\volt}),
                   %x unit=u,
                   change x base=true,
                   x SI prefix=micro,
               ]
               \addplot[-,color=magenta] table {data/module-72cells-series--U-MOSFET--0003u.dat};
           \end{axis}
           \begin{axis}[%
                   height=40mm,
                   width=\textwidth,
                   at={(0,0)},
                   grid=both,
                   xlabel=Zeit (\si{\micro\second}),
                   ylabel=Leistung (\si{\watt}),
                   %x unit=u,
                   change x base=true,
                   x SI prefix=micro,
               ]
               \addplot[-,color=red] table {data/module-72cells-series--P-MOSFET--0003u.dat};
           \end{axis}
       \end{scope}
   \end{tikzpicture}
   \caption{%
       Verlauf    von    Strom,    Spannung    und    Leistung    am    MOSFET
       bei   einer   Schaltfrequenz   von   \SI{10}{\kilo\hertz}   bei   einer
       Modulkonfiguration    von    $72     \times    1$    Zellen    gem\"ass
       Schema    in    \fref{fig:ltspice:module:cellBased:36x2}   auf    Seite
       \ref{fig:ltspice:module:cellBased:72x1}. \protect\\
       Anhang     \ref{app:sec:simu:complementary:72x1}     auf     Seite
       \pageref{app:sec:simu:complementary:72x1} beinhaltet zum Vergleich noch
       Simulationen f\"ur den Zeitraum von einer Millisekunde.%
   }
   \label{fig:simu:results:72x1:3u}
\end{figure}

Die    Resultate   der    Simulation   f\"ur    ein   $36    \times   2$-Modul
sind   in   \fref{fig:simu:results:36x2:3u}   f\"ur  einen   Zeitbereich   von
\SI{3}{\micro\second}  dargestellt,  die  Ergebnisse   f\"ur  das  $72  \times
1$-Modul in \fref{fig:simu:results:72x1:3u}.  Tabelle \ref{tab:36x2:72x1:heat}
fasst   die  wichtigsten   Eckdaten  der   Simulationen  zusammen,   inklusive
Durchschnittswerte f\"ur Strom und Leistung.

\begin{table}[h!tb]
    \centering
    \caption{%
        Eckdaten zur Simulation des Kurzschlussverfahrens f\"ur ein Solarpanel
        mit  $36   \times  2$  Zellen  und   ein  Panel  mit  $72   \times  1$
        Zellen.\protect\\
        Bei     den     Durchschnittswerten     sind     sowohl     Ergebnisse
        f\"ur      \SI{1}{\milli\second}     wie      auch     \SI{1}{\second}
        angegeben,    um   zu    zeigen,   dass    sich   die    Konfiguration
        bereits    bei    einer    Millisekunde   stabilisiert    hat    (auch
        zu     sehen    in     Anhang    \ref{app:sec:simu:complementary:36x2}
        auf      Seite      \pageref{app:sec:simu:complementary:36x2}      und
        Anhang      \ref{app:sec:simu:complementary:72x1}       auf      Seite
        \pageref{app:sec:simu:complementary:72x1}).\protect\\
        Die     Dauer      der     Spitze     ist     beim      $36     \times
        2$-Modul     \SI{2}{\micro\second}    und     \SI{1.78}{\micro\second}
        beim    $72    \times    1$-Modul    (siehe    unterste    Plots    in
        den     Abbildungen      \ref{fig:simu:results:36x2:3u}     respektive
        \ref{fig:simu:results:72x1:3u}).%
    }
    \label{tab:36x2:72x1:heat}
    \begin{tabular}{lrr}

    \toprule
    Kriterium                                      & $36 \times 2$-Modul & $72 \times 1$-Modul \\
    \midrule
     Spitzenwert Strom:                            & \SI{32}{\ampere}    & \SI{24}{\ampere}    \\
     Spitzenwert Leistung:                         & \SI{310}{\watt}     & \SI{535}{\watt}     \\
     $\overline{P}$ f\"ur Dauer der Spitze         & \SI{158.77}{\watt}  & \SI{267.46}{\watt}  \\
     $\overline{P}$ f\"ur \SI{1}{\milli\second}    & \SI{4.312}{\watt}   & \SI{6.0577}{\watt}  \\
     $\overline{P}$ f\"ur \SI{1}{\second}          & \SI{4.3047}{\watt}  & \SI{6.0433}{\watt}  \\
    \bottomrule
    \end{tabular}
\end{table}

Aus \fref{fig:simu:results:36x2:5u}  und \tref{tab:36x2:72x1:heat}  ziehen wir
folgende Schlussfolgerungen:

\begin{enumerate}
    \firmlist
    \item
        Der  Spitzenwert f\"ur  den  Strom ist  hoch,  aber Transistoren,  die
        solche Str\"ome  verkraften k\"onnen,  sind zu  vern\"unftigen Preisen
        erh\"altlich.
    \item
        Die    Spitzenwerte    der    Leistungen   zwar    nur    sehr    kurz
        (etwa   \SI{2}{\micro\second}),   aber   sehr  hoch. Auch   wenn   der
        Durchschnittliche Leistungswert  weit unter  der Grenze  von g\"unstig
        erh\"altlichen  MOSFETs   liegt,  k\"onnte  die   Leistungsspitze  den
        Transistor irreversibel sch\"adigen.
    \item
        Der  durchschnittliche  Leistungsverbrauch  ist  weit  \"uber  dem  im
        Pflichtenheft angestrebten Wert von \SI{100}{\milli\watt}.
\end{enumerate}
\todo{korrekte Schlussfolgerungen?}


% ---------------------------------------------------------------------------- %
\clearpage
\subsection{\Master (Empf\"anger)}
\label{subsec:simu:ask:recv}
% ---------------------------------------------------------------------------- %


% ---------------------------------------------------------------------------- %
\subsection{Gesamtsystem}
\label{subsec:simu:ask:total}
% ---------------------------------------------------------------------------- %
