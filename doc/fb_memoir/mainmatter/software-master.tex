% ---------------------------------------------------------------------------- %
\section{Software \Master}
\label{sec:software:master}
% ---------------------------------------------------------------------------- %

Es wird  im Folgenden auf  die Software des \Master~  genauer eingegangen. Die
verwendeten  Komponenten  werden beschrieben  und  es  wird auf  die  Lizenzen
ebendieser  Komponenten eingegangen. Anschliessend  werden der  Aufbau unseres
Software-Stacks und die Funktionsprinzipien dokumentiert.


% ---------------------------------------------------------------------------- %
\subsection{Komponenten}
\label{subsec:software:master:components}
% ---------------------------------------------------------------------------- %

Als Betriebssystem kommt Raspbian zum  Einsatz. Raspbian ist eine Variante von
Debian-Linux mit  einigen Erweiterungen,  welche das  System auf  einem \Raspi
lauff\"ahig machen. Als  graphische Oberfl\"ache wird LXDE  benutzt, aufbauend
auf  X11. Die wichtigsten  Komponenten  des gesamten  Software-Stackn sind  in
Abbildung \ref{fig:softwarestack} dargestellt.

Die  Funktionalit\"at  unserer  Software  wird  mit  einigen  Python-Libraries
implementiert. PyQt  wird  benutzt,  um  die  graphische  Benutzeroberfl\"ache
zu   programmieren,  SQLAlchemy   ist  daf\"ur   dient  als   Datenbanktreiber
verantwortlich  und \emph{WiringPi  for  Python}  ist daf\"ur  verantwortlich,
die  Hardware-Schnittstellen  des  \Raspi~   zu  abstrahieren  und  in  Python
bereitzustellen. Tabelle  \ref{tab:pythonLibs} listet  die Libraries  und ihre
Aufgaben in einer \"Ubersicht auf.

\begin{figure}[h!tb]
    \centering
    \begin{bytefield}{32}
        %\begin{rightwordgroup}{Adresse}
        \colorbitbox{solarized-base2}  {solarized-base02}{32}{Unsere Software} \\
        \colorbitbox{solarized-blue}   {solarized-base3}  {8}{PyQt}
        \colorbitbox{solarized-blue}   {solarized-base3}  {8}{SQLAlchemy}
        \colorbitbox{solarized-blue}   {solarized-base3}  {8}{WiringPi}
        \colorbitbox{solarized-magenta}{solarized-base3}  {8}{LXDE} \\
        \colorbitbox{solarized-blue}   {solarized-base3} {24}{Python 3}
        \colorbitbox{solarized-magenta}{solarized-base3}  {8}{X11} \\
        \colorbitbox{solarized-base01} {solarized-base3} {32}{Linux} \\
        %\end{rightwordgroup} \\
  \end{bytefield}
  \caption{Software-Stack f\"ur unser Projekt}
  \label{fig:softwarestack}
\end{figure}

\begin{table}[h!tb]
    \centering
    \caption{Liste der verwendeten Python-Libraries}
    \label{tab:pythonLibs}
    \small
    \begin{tabular}{lrp{50mm}r}
        \toprule
        \textsc{Library} & \textsc{Version} & \textsc{Zweck} & \textsc{Website} \\
        \midrule
        PyQt & 5 & Erstellen und verwalten der graphischen Bedienelemente & \cite{ref:pyqt} \\
        [2mm]
        \rowcolor{solarized-base2}
        SQLAlchemy & 1.0 & Datenbankabstraktion                           & \cite{ref:sqlalchemy} \\
        [2mm]
        WiringPi for Python & 2 & Abstraktion der Hardware-Schnittstellen & \cite{ref:wiringpi} \\
        \bottomrule
    \end{tabular}
\end{table}

% ---------------------------------------------------------------------------- %
\clearpage
\subsection{Lizenzen}
\label{subsec:software:master:licenses}
% ---------------------------------------------------------------------------- %

Bei    der   Auswahl    von    Drittsoftware   wird    auf   die    jeweiligen
Lizenzbedingungen   geachtet,   um   keine   Konflikte   zu   verursachen. Die
wichtigsten  drei Lizenz-Bereiche  und ihre  Charakteristiken sind  in Tabelle
\ref{tab:lincenseAreas} aufgef\"uhrt.

\begin{table}[h!tb]
    \centering
    \caption{Lizenzbereiche}
    \label{tab:licenseAreas}
    \small
    \begin{tabular}{>{\raggedright}p{30mm}>{\raggedright}p{30mm}p{50mm}}
        \toprule
        \textsc{Bereich} &
        \textsc{Lizenz} &
        \textsc{Bedingungen} \\
        \midrule
        Linux-Kernel &
        GPL &
        Quellcode und \"Anderungen m\"ussen \"offentlich sein. \\
        [2mm]

        \rowcolor{solarized-base2}
        Treiber f\"ur Raspberry Pi und Display &
        Restricted &
        Quellcode wird vom Hersteller geheim gehalten \\
        [2mm]

        Raspbian &
        DFSG (Sammlung diverser Lizenzen) &
        Darf frei verwendet, aber nicht unbedingt verkauft, werden. \\
        \bottomrule
    \end{tabular}
\end{table}

Grundlage  f\"ur die  Software  bildet  das angepasste  Betriebssystem. Dieses
wird  von  der Raspberry  Pi  Foundation  frei  zur Verf\"ugung  gestellt  und
unterliegt den  Bedingungen den  DFSG (\emph{Debian Free  Software Guidelines}
\cite{ref:socialContract}). Die  darauf  aufbauenden Programmbibliotheken  zur
Abstraktion  von Betriebsystemfunktionen  und weiteren  Hardware-Aufrufen sind
alle aus den Raspbian-Repositories verf\"ugbar und unterliegen daher ebenfalls
den  DFSG.   Da  die  Mastersoftware zwar  auf  diesen  Komponenten  aufsetzt,
sie  aber  nicht ver\"andert  oder  statisch  verlinkt wird,  entstehen  keine
Lizenzkonflikte. Zu beachten ist hier, dass diese Drittsoftware im Allgemeinen
nicht als Eigenwerk  verkauft werden darf. Das heisst, dass  sie zwar beliebig
verbreitet werden darf, nicht aber zum Produkt hinzugez\"ahlt werden kann.

Die DFSG stellen  insbesondere folgende Anforderung an  alle Programme, welche
Teil von Raspbian sind:

\begin{itemize}
    \tightlist
\item
    Die Software darf frei verbreitet werden (Regel 1)
\item
    Die Software darf f\"ur beliebige Zwecke eingesetzt werden (Regel 6)
\item
    Die Software beschr\"ankt unzusammenh\"angende Software nicht (Regel 9)
\end{itemize}

Der eigentliche  Mastersoftware-Quellcode dagegen  ist nicht  \"offentlich und
kann als Bestandteil des Produkts verkauft werden.



% ---------------------------------------------------------------------------- %
\subsection{Threads}
\label{subsec:software:master:threads}
% ---------------------------------------------------------------------------- %


Die  Mastersoftware  ist in  mehrere  Threads  gegliedert, unter  welchen  die
Funktionen aufgeteild  sind. Sie werden alle von  einem Hauptthread gestartet,
welcher die Koordination mittels Semaphoren \"ubernimmt. Dazu initialisiert er
alle  Resourcen,  auf  welche  aus  mehreren  Threads  zugegriffen  sind. Dies
sind  Datenbank und  Logging-System, welche  beide Multithreading  beherrschen
und  threadsafe  sind. Die zentrale  Koordination  bedeutet  zudem, dass  alle
Arbeitsschritte nur sofern n\"otig und nicht mit veralteten Daten ausgef\"uhrt
werden.

Um  die Threadsicherheit  zu  gew\"ahrleisten verwenden  die beiden  geteilten
Resourcen spezielle Mechanismen:
\begin{itemize}
    \tightlist
    \item
        Die   Loggingfunktion   setzt  eine   Queue   eim   um  Meldungen   zu
        zwischenzuspeichern und asynchron hintereinander abzuarbeiten.
     \item
        SQLAlchemy  beinhaltet   mit  der  "ScopedSession"  eine   Methode  um
        aus  mehreren  Threads  parallel  aufgerufen  zu  werden. Dazu  werden
        die  Anfragen  aus jeweils  einem  Thread  gruppiert und  dann  atomar
        ausgef\"uhrt.
 \end{itemize}

Alle anderen  externen Ressourcen, dies  sind \ISC-bus, UART und  GPIO, werden
jeweils  von  nur einem  Thread  genutzt,  wodurch keine  Konflikte  auftreten
k\"onnen.

Im Detail sieht die Aufgabenteilung folgendermassen aus:

\begin{itemize}
    \tightlist
    \item
        \textbf{Prozesssteuerung:} Der  Hauptthread  verwaltet  Resourcen  und
        alle weiteren Threads.
    \item
        \textbf{Datensammlung:} Ein  Thread  empf\"angt   die  Spannungs-  und
        Strommesswerte von den Sensoren und speichert sie in der Datenbank ab.
    \item
        \textbf{Datenauswertung:} Ein   separater    Thread   untersucht   die
        gespeicherten  Messwerte   auf  defekte   Panels  und   speichert  die
        Ergebnisse ebenfalls in der Datenbank ab.
    \item
        \textbf{Graphisches  Benutzerinterface:} Der   GUI-Thread  stellt  ein
        Fenster  dar,   mit  welchem   der  Benutzer  die   Einstellungen  von
        Alarmierung  und   Telefonnummer  konfigurieren  kann   und  speichert
        \"Anderungen in der Datenbank.
    \item
        \textbf{Ausgabe:} Die  Umsetzung  der definierten  Massnahmen  obliegt
        einem Thread, welcher das Modem verwaltet und bei Bedarf die digitalen
        Ausg\"ange zur Steuerung der Relais bet\"atigt
\end{itemize}


% ---------------------------------------------------------------------------- %
\subsection{Benutzeroberfl\"ache}
\label{subsec:software:master:GUI}
% ---------------------------------------------------------------------------- %

\todo{unterliegender aufbau statt benutzerf\"uhrung?}
Die Benutzeroberfl\"ache wurde bewusst sehr schlicht gehalten. Sie besteht aus
folgenden  Ansichten: Hauptmen\"u,  Einstellungen, Eingabe  der  Telefonnummer
und   dem   Fehlerverlauf. Im   regul\"aren  Betrieb   ist   das   Hauptmen\"u
ersichtlich. Es  informiert  \"uber  den  Anlagestatus  sowie  den  gemessenen
Strom  der Str\"ange. Vom  Hauptmen\"u  aus kann  durch  dr\"ucken des  Feldes
\emph{Einstellungen} zu den Einstellungen gelangt werden. In den Einstellungen
k\"onnen die  Relaiskontakte 1  und 2  aktiviert oder  deaktiviert werden. Sie
dienen  zur Bet\"atigung  einer externen  Signalisation im  Fehlerfall. Weiter
kann   die  SMS   Benachrichtigung   ebenfalls   aktiviert  oder   deaktiviert
werden. Durch  bet\"atigen  des  Feldes  \emph{Speichern}  wird  die  aktuelle
Auswahl  gespeichert. Falls  die  SMS Benachrichtigung  gew\"ahlt  wird,  muss
eine  g\"ultige Telefonnummer  hinterlegt werden.   Die Ansicht  Eingabe einer
Telefonnummer  wird vom  Feld \emph{Telefonnr.  hinzuf\"ugen} aufgerufen. Nach
der  Eingabe  einer  g\"ultigen  Nummer,  wird  mittels  \emph{Speichern}  die
Telefonnummer  gespeichert und  es kann  mit dem  Feld \emph{Zur\"uck}  zu den
Einstellungen  zur\"uckgekehrt  werden. Diese  Ansichten sind  im  regul\"aren
Betrieb ersichtlich und k\"onnen nach Belieben eingestellt werden.

Folgend werden  die St\"orbetrieb Ansichten erl\"autert. Wird  ein Modulfehler
erkannt,  signalisiert   dies  die  Ansicht  \emph{Modulfeher}. Es   wird  die
Modulseriennummer und die Zeit mit Datum angezeigt. Damit ist es m\"oglich ein
fehlerhaftes Modul zu erkennen  und entsprechend zu handeln. Durch bet\"atigen
des  Feldes  \emph{OK} wird  der  externe  Alarm der  Relais  1  und 2  (falls
aktiviert)  quittiert  und das  Hauptmen\"u  wird  wieder angezeigt. Das  Feld
\emph{Fehlerverlauf}  f\"uhrt  zu  einer   \"Ubersicht  aller  bis  zu  diesem
Zeitpunkt aufgetretenen  Modulfehler. Jeweils wird  die Modulnummer,  die Zeit
und das Datum in einer Zeile angezeigt.


% ---------------------------------------------------------------------------- %
\subsection{Datenbank}
\label{subsec:software:master:database}
% ---------------------------------------------------------------------------- %


Um  die  gesammelten  Daten  optimal  zu speichern  und  zu  einem  sp\"ateren
Zeitpunkt wieder  verwenden zu  k\"onnen, wird eine  Datenbank verwendet. Dies
hat gegen\"uber  der Verwendung von internen  Datenstrukturen wie Dictionaries
und Arrays  zwar die Nachteile von  gr\"osserem (Erst-)Implementationsaufwand,
h\"oherem  Arbeitsspeicherbedarf  und   gr\"osserem  Rechenaufwand  f\"ur  die
CPU. Jedoch   wird  die   Wartung  der   Software  und   das  Erg\"anzen   von
zus\"atzlicher   Funktionalit\"at   stark   vereinfacht. Ebenfalls   ist   die
Auswertung einfacher und  flexibler, da schon beim Auslesen der  Werte aus der
Datenbank nach verschiedenen Kriterien gefiltert werden kann.

\begin{figure}[h!tb]
    \centering
    \begin{tikzpicture}
    \sffamily
    \small
    %\graph [left anchor=east,grow right sep,right anchor=west,grow right,branch down,nodes={rounded rectangle,draw}]
    \begin{scope}[
            every node/.style = {
                draw = solarized-base02,
                rounded rectangle,
                fill=solarized-base2,
                text=solarized-base02,
                inner sep=1mm,
                text height=0.7em,
            },
            node distance=3mm,
            table/.style={
                fill=solarized-cyan,
                text=solarized-base2,
                inner sep=2mm,
            },
            column/.style={
                minimum width=25mm,
            }
        ]
        \node (database)[
            fill=solarized-blue,
            text=solarized-base2,
            inner sep=2mm
        ] at (0,0) {Datenbank};

        \node (strings)   [table,below left=15mm of database] {strings};
        \node (sID)       [column,below right=12mm of strings] {ID};
        \node (sStringNo) [column,below=of sID]            {stringnumber};
        \node (sCurr)     [column,below=of sStringNo]      {stringcurrent};
        \node (sTs)       [column,below=of sCurr]          {timestamp};
        \node (sDev)      [column,below=of sTs]            {deviation};
        \node (sReported) [column,below=of sDev]           {flag\_reported};

        \node (panels)    [table,left=of strings]  {panels};
        \node (pID)       [column,below left=12mm of panels]    {ID};
        \node (pSerial)   [column,below=of pID]       {serialnumber};
        \node (pVoltage)  [column,below=of pSerial]   {voltage};
        \node (pStringNo) [column,below=of pVoltage]  {stringnumber};
        \node (pTs)       [column,below=of pStringNo] {timestamp};
        \node (pDev)      [column,below=of pTs]       {deviation};
        \node (pReported) [column,below=of pDev]      {flag\_reported};

        \node (string1)  [table,below right=15mm of database] {string\{1,2,3\}};
        \node (s1ID)     [column,below right=12mm of string1]        {ID};
        \node (s1Serial) [column,below=of s1ID]           {serialnumber};
    \end{scope}
    \begin{scope}[on background layer]
    \end{scope}

    \begin{scope}[
            rounded corners,
            draw=solarized-base02
        ] % connections
        \draw[-latex] (database) -| (panels);
        \draw[-latex] (database) -| (strings);
        \draw[-latex] (database) -| (string1);

        \draw[-latex] (panels) |- (pID);
        \draw[-latex] (panels) |- (pSerial);
        \draw[-latex] (panels) |-  (pVoltage);
        \draw[-latex] (panels) |-  (pStringNo);
        \draw[-latex] (panels) |-  (pTs);
        \draw[-latex] (panels) |-  (pDev);
        \draw[-latex] (panels) |-  (pReported);

        \draw[-latex] (strings) |- (sID);
        \draw[-latex] (strings) |- (sStringNo);
        \draw[-latex] (strings) |- (sCurr);
        \draw[-latex] (strings) |- (sTs);
        \draw[-latex] (strings) |- (sDev);
        \draw[-latex] (strings) |- (sReported);

        \draw[-latex] (string1) |- (s1ID);
        \draw[-latex] (string1) |- (s1Serial);
    \end{scope}
\end{tikzpicture}

    \caption{Datenbank-Layout}
  \label{fig:database:layout}
\end{figure}


%stringX-tables:
Unsere Datenbank  umfasst 5 Tabellen  (oder \emph{Tables} im  Fachjargon), wie
auch in Abbildung \ref{fig:database:layout} gezeigt:

\begin{itemize}
    \tightlist
    \item
        \code{panels:} Speichert die Spannungen der PV-Module
    \item
        \code{strings:} Speichert die Str\"ome in den Strings
    \item
        \code{string1:} Speichert Sensor-IDs in String 1
    \item
        \code{string2:} Speichert Sensor-IDs in String 2
    \item
        \code{string3:} Speichert Sensor-IDs in String 3
\end{itemize}

Letztere drei  (in Abbildung \ref{fig:database:layout} auf  der rechten Seite)
davon  dienen  ausschliesslich  der  Zuordnung  aller  im  System  vorhandenen
Seriennummern  zu  den  einzelnen  Strings. Dies ist  notwendig,  um  bei  der
Auswertung  zu wissen,  welche Sensoren  sich  im selben  String befinden  und
dementsprechend miteinander verglichen  werden m\"ussen. Diese Tables bestehen
lediglich aus 2  Spalten, eine f\"ur die Seriennummer sowie  eine f\"ur die ID
des Eintrages, welcher ben\"otigt wird um doppelte Zeilen zu umgehen.

%strings-table:
Das Table  \code{strings} (Mitte in Abbildung  \ref{fig:database:layout} dient
der Speicherung  der gemessenen String-Str\"ome. Diese Werte  werden als Paket
mit einigen  wichtigen zus\"atzlichen Daten gespeichert. Dazu  geh\"oren neben
dem gemessenen Stromwert noch die  String-Nummer, um zu wissen, welcher String
gemessen wurde,  ein Zeitstempel, um  nachvollziehen zu k\"onnen, wie  alt die
Eintr\"age  sind,  ein leeres  Feld,  um  bei  der Auswertung  die  Abweichung
zum  Durchschnitt eintragen  zu k\"onnen,  sowie  ein Feld,  welches ein  Flag
beinhaltet, das  anzeigt, ob  der String bereits  als ausserhalb  der Toleranz
gemeldet wurde. Zudem wird auch hier  wieder eine Spalte f\"ur die Eintrags-ID
ben\"otigt.

%panels-table:
Das  letzte  und  gr\"osste  Table   ist  \code{panels}  (links  in  Abbildung
\ref{fig:database:layout}),  in  welchem   die  Modulspannungen  abgespeichert
werden. Auch  hier   wird  dies   wieder  als   Paket  mit   relevanten  Daten
realisiert. Die  Zeile in  diesem Table  ist grunds\"atzlich  gleich aufgebaut
wie  jene  des  Strings-Table,  nur   dass  hier  anstelle  des  Stringstromes
die  Modulspannung  eingetragen,  sowie  ein  zus\"atzliches  Feld  f\"ur  die
Seriennummer  des  gemessenen  Modul-Sensors verwendet  wird. Dies  wird  hier
zus\"atzlich ben\"otigt, um die gemessenen Daten in  direktem Zusammenhang mit
den Seriennummern der Sensoren zu bringen,  was f\"ur die Auswertung der Daten
zwingend n\"otig ist.


% ---------------------------------------------------------------------------- %
\subsection{Implementation}
\label{subsec:software:master:implementation}
% ---------------------------------------------------------------------------- %

Die  Master-Software besteht  aus  5 einzelnen  Python-Files,  auf welche  die
anfallenden  Aufgaben sinnvoll  verteilt werden. Im  Folgenden wird  auf jedes
dieser Files etwas genauer eingegangen.


% ---------------------------------------------------------------------------- %
\subsubsection{\code{database.py}}
\label{subsubsec:software:master:implementation:database}
% ---------------------------------------------------------------------------- %

Dieses File ist  dazu da, die Datenbank zu erzeugen  und zu verwalten. Mittels
der  vordefinierten   Library  SQL-Alchemy  wird  zuerst   das  Datenbank-File
erstellt  und die  einzelnen Tables  gem\"ass unseren  Vorlagen aufgebaut  und
hinzugef\"ugt. Dabei  wird  vor dem  Erzeugen  der  einzelnen Tabellen  zuerst
\"uberpr\"uft, ob eine  solche bereits vorhanden ist. Das  stellt sicher, dass
auch bei  einem Neustart  des Masterger\"ates  die vorhandene  Datenbank nicht
\"uberschrieben wird. Nachdem  die Datenbank  erstellt worden ist,  wird zudem
f\"ur jedes Table eine eigene Klasse mit zugeh\"origem Mapper definiert. Diese
werden  ben\"otigt,  damit   gleichzeitig  ausgel\"oste  Datenbank-Operationen
sich nicht  gegenseitig blockieren,  sondern nacheinander  abgearbeitet werden
k\"onnen.


{\begin{a3pages}
\setlength{\parindentbak}{\parindent}
    \noindent\adjustbox{valign=t}{\begin{minipage}{135mm}
% ---------------------------------------------------------------------------- %
\subsubsection{\code{input\_handler.py}}
\label{subsubsec:software:master:implementation:inputHandler}
% ---------------------------------------------------------------------------- %


    Dieses  sehr  umfangreiche  File  hat  den  Zweck,  s\"amtliche  Messwerte
    abzufragen,  einen   ersten  Teil  der  Auswertung   zu  \"ubernehmen  und
    schlussendlich  die Eintr\"age  in  der  Datenbank vorzunehmen. F\"ur  die
    Abfrage  der Messdaten  wird hier  die gesamte  Kommunikation \"uber  \ISC
    implementiert.  Da dieses File sehr  umfangreich ist, ist der zugeh\"orige
    Prozess in Abbildung \ref{fig:inputHandler} grafisch dargestellt.

    \setlength{\parindent}{\parindentbak} % restore paragraph indentation
    Der        Hauptablauf        startet        mit        der        Methode
    \code{stringcurrents\_request}. Diese   ruft   nacheinander  mittels   der
    implementierten  Methoden  \code{read\_stringcurrents\_i2c} die  aktuellen
    Werte  der  Stringstr\"ome ab  und  speichert  diese auch  gleich  mittels
    \code{write\_string\_into\_database}   an  den   richtigen   Ort  in   der
    Datenbank. Anschliessend   sollen  die   soeben  ausgelesenen   Stromwerte
    auch  gleich  ausgewertet  werden. Da   dies  hier  kein  grosser  Aufwand
    darstellt  wird  dies  ebenfalls  im  selben  File  implementiert. Daf\"ur
    wird  im Hauptablauf  als  n\"achstes  die Methode  \code{string\_compare}
    ausgef\"uhrt. Diese berechnet  den Durchschnitt der aktuellen  Werte sowie
    die Abweichungen  der einzelnen Strings  zu diesem. Ist die  Abweichung zu
    goss,  wird  der  String  dem  Benutzer  gemeldet. Dies  dient  vor  allem
    dazu,  grobe  Probleme  festzustellen  und diese  relativ  z\"ugig  melden
    zu  k\"onnen. Ausf\"alle   einzelner  Module   dagegen  werden   vom  File
    \code{evaluator.py} detektiert.

    Nach  den   Str\"omen  widmet   sich  das  File   den  Modulspannungen. Im
    Hauptablauf  des  Files  wird  die  Methode  \code{modulevoltage\_request}
    aufgerufen,  welche   in  jedem   String  die   vorhandenen  Seriennummern
    durchgeht  und dabei  f\"ur  jede mittels  \code{read\_modulevoltage\_i2c}
    einen   Request  aussendet. Wird   keine  Antwort   empfangen,  wird   die
    History   dieser   Seriennummer    \"uberpr\"uft   –   wurde   w\"ahrend
    24   Stunden   kein   Eintrag   gemacht  wird   das   Modul   als   defekt
    gemeldet. Bekommt  die Software  eine Antwort  auf den  Request, wird  der
    Wert  unter  Verwendung  der  Methode  \code{write\_panel\_into\_database}
    in   die  Datenbank   geschrieben. Zus\"atzlich  wird   \"uberpr\"uft,  ob
    die   Seriennummer   bereits   im  zugeh\"origen   Stringpanel   vorhanden
    ist. So  wird   ein  neu   eingesetztes  Solarmodul   automatisch  erkannt
    und   per   \code{write\_stringX\_into\_database}  dem   richtigen   Table
    hinzugef\"ugt. Nachdem  alle  Spannungen  abgefragt sind  wird  auch  hier
    bereits die Grundlage f\"ur die Auswertung gelegt. So wird in jedem String
    der  h\"ochste  Eintrag der  Messreihe  gesucht  und bei  jedem  einzelnen
    Modul  im  deviation-Feld  die  Abweichung  zu  diesem  Eingetragen. Somit
    wird immer  der aktuelle  Wert mit  dem einer  funktionierenden Solarzelle
    verglichen. Bei einer  funktionierenden Zelle  treten somit  lediglich bei
    Abschattungen kurzzeitig  grosse Differenzen  auf, welche aber  \"uber den
    Tag gemittelt  keinen sehr  grossen Einfluss  haben. So kann  gut zwischen
    tempor\"ar  abgeschatteten  Modulen  welche  aber  eigentlich  einwandfrei
    funktionieren und defekten Modulen die l\"angerfristig grosse Abweichungen
    aufzeigen, unterschieden werden.

    \vspace*{5mm}

    \hfill\adjustbox{valign=t}{\begin{minipage}{52mm}
        \figcaption
        [Ablaufdiagramm \code{input\_handler.py}]
        {Ablaufdiagramm \protect\\\code{input\_handler.py}}
        \label{fig:inputHandler}
    \end{minipage}}
\end{minipage}}
\hspace*{25mm}
\adjustbox{valign=t}{\begin{minipage}{195mm}
    \begin{tikzpicture}[%
        align=center,
        text=solarized-base02,
        draw=solarized-base02,
        remember picture,
        overlay,
    ]
    \small
    \ttfamily

    \begin{scope}[
        every node/.style = draw,
        terminal/.append style={
            rounded rectangle,
            fill=solarized-violet,
            text=solarized-base3,
            inner sep=2mm,
        }, % data packages
        sign/.style={
            inner sep=2mm,
            rounded corners=1mm,
            fill=solarized-magenta,
            text=solarized-base3,
        },         % custom signal style
        circ/.style={
            inner sep=2mm,
            rounded corners=1mm,
            double,
            fill=solarized-base02,
            draw=solarized-base02,
            text=solarized-base2,
        }, % circuitry
        proc/.style={
            inner sep=2mm,
            rounded corners=1mm,
            fill=solarized-cyan,
            text=solarized-base3,
        },       % process/activity
        stor/.style={
            fill=cyan!30
        },         % storage
        dbtable/.style={
            text=solarized-base3,
            draw=solarized-base3,
            rounded corners=1mm,
            inner sep=2mm,
        } % database tables
    ]
        %\node (stringcurrents-request) [
        %    sign,
        %    signal,
        %    signal to=east
        %] at (10mm,0mm) {stringcurrents\_request};

        \node (stringcurrents-request) [
            proc,
        ] at (20mm,-20mm) {stringcurrents\_request};

        %\node (string-support) [
        %    right=30mm of stringcurrents-request,
        %    draw=white,
        %] { };

        \node (stringnumber) [
            sign,
            signal,
            signal to=east,
            signal from=west,
            above right=20mm of stringcurrents-request,
        ] {stringnumber};

        \node (stringcurrentget) [
            sign,
            signal,
            signal to=west,
            signal from=east,
            right=20mm of stringcurrents-request,
        ] {stringcurrent};

        \node (read-stringcurrents-i2c) [
            proc,
            right=of stringcurrentget,
        ] {read\_stringcurrents\_i2c};

        \node (write-string-into-database) [
            proc,
            below=of read-stringcurrents-i2c,
        ] {write\_string\_into\_database};

        \node (stringcurrentput) [
            sign,
            signal,
            signal to=east,
            signal from=west,
            left=of write-string-into-database,
        ] {stringcurrent};

        %\node (string-compare) [
        %    terminal,
        %    below=20mm of stringcurrents-request,
        %] {string\_compare};

        \node (string-compare) [
            proc,
            below=20mm of stringcurrents-request,
        ] {string\_compare};

        \node (modulevoltage-request) [
            proc,
            below=20mm of string-compare,
        ] {modulevoltage\_request};

        \node (stringNo-serialNo) [
            sign,
            signal,
            signal to=east,
            signal from=west,
            right=20mm of modulevoltage-request,
            align=center,
        ] {stringnumber\\serialnumber};

        \node (serialNo-voltage) [
            sign,
            signal,
            signal to=west,
            signal from=east,
            below=of stringNo-serialNo,
            align=center,
        ] {serialnumber\\voltage};

        \node (read-modulevoltage-i2c) [
            proc,
            right=of stringNo-serialNo,
        ] {read\_modulevoltage\_i2c};

        \node (serialNo-stringNo-voltage) [
            sign,
            signal,
            signal to=south,
            signal from=north,
            below=60mm of modulevoltage-request,
            align=center,
        ] {serialnumber\\stringnumber\\voltage};

        \node (support1) [
            fill=white,
            draw=white,
            above=of serialNo-stringNo-voltage,
        ] { };

        \node (write-panel-into-database) [
            proc,
            below=of serialNo-stringNo-voltage,
        ] {write\_panel\_into\_database};

        \node (serialNo1) [
            sign,
            signal,
            signal to=east,
            signal from=west,
            right=of write-panel-into-database,
        ] {serialnumber};

        \node (write-string1-into-database) [
            proc,
            above right=of serialNo1,
        ] {write\_string1\_into\_database};

        \node (write-string2-into-database) [
            proc,
            below=of write-string1-into-database,
        ] {write\_string2\_into\_database};

        \node (write-string3-into-database) [
            proc,
            below=of write-string2-into-database,
        ] {write\_string3\_into\_database};
    \end{scope}

    \begin{scope}[on background layer]
        \node (getStringCurrents) [%
            draw,
            double,
            rounded corners=5mm,
            fill=solarized-base3,
            inner sep=1.75em,
            fit=(read-stringcurrents-i2c) (write-string-into-database) (stringnumber) (stringcurrentget) (stringcurrentput),
            text height=-0.5em,
            align=right,
        ] {\sffamily wird pro String $1\times$ ausgef\"uhrt, also total $3\times$};

        \node (getModuleVoltages) [%
            draw,
            double,
            rounded corners=5mm,
            fill=solarized-base3,
            inner sep=1.75em,
            fit=(read-modulevoltage-i2c) (stringNo-serialNo) (serialNo-voltage),
            text height=-0.5em,
            align=right,
        ] {\sffamily mit Schlaufe f\"ur jedes Modul ausgef\"uhrt};

        \node (checkSN) [%
            draw,
            double,
            rounded corners=5mm,
            fill=solarized-base3,
            inner sep=1.75em,
            fit=(serialNo-stringNo-voltage) (write-panel-into-database) (serialNo1) (write-string1-into-database) (write-string2-into-database) (write-string3-into-database) (support1),
            text height=-0.5em,
            align=right,
        ] {\sffamily SN wird \"uberpr\"uft und in das entsprechende Table\\eingetragen, falls noch kein Eintrag vorhanden ist.};

        \node (mainProcess) [%
            draw,
            double,
            rounded corners=5mm,
            fill=solarized-base2,
            inner sep=1.5em,
            fit=(stringcurrents-request) (string-compare) (modulevoltage-request),
            text height=-0.5em,
            align=right,
            text=solarized-base02,
        ] {\sffamily Hauptablauf des Files};
    \end{scope}


    \begin{scope}[
            rounded corners,
            every path/.append style={draw=solarized-base03,},
    ]
        \draw[-latex] (stringcurrents-request) |- (stringnumber);
        \draw[-latex] (stringnumber) -| (read-stringcurrents-i2c);
        \draw[-latex] (read-stringcurrents-i2c) -- (stringcurrentget);
        \draw[-latex] (stringcurrentget) -- (stringcurrents-request);
        \draw[-latex] (stringcurrents-request) |- (stringcurrentput);
        \draw[-latex] (stringcurrentput) -- (write-string-into-database);

        \draw[-latex] (stringcurrents-request) -- (string-compare);
        \draw[-latex] (string-compare) -- (modulevoltage-request);

        \draw[-latex] (modulevoltage-request) -- (stringNo-serialNo);
        \draw[-latex] (stringNo-serialNo) -- (read-modulevoltage-i2c);
        \draw[-latex] (read-modulevoltage-i2c) |- (serialNo-voltage);
        \draw[-latex] (serialNo-voltage) -| (modulevoltage-request);


        \draw[-latex] (modulevoltage-request) -- (serialNo-stringNo-voltage);
        \draw[-latex] (serialNo-stringNo-voltage) -- (write-panel-into-database);
        \draw[-latex] (write-panel-into-database) -- (serialNo1);
        \draw[dashed,-latex] (serialNo1) |- (write-string1-into-database);
        \draw[dashed,-latex] (serialNo1) -- (write-string2-into-database);
        \draw[dashed,-latex] (serialNo1) |- (write-string3-into-database);
    \end{scope}

\end{tikzpicture}

\end{minipage}}

\end{a3pages}}



% ---------------------------------------------------------------------------- %
\subsubsection{\code{evaluator.py}}
\label{subsubsec:software:master:implementation:evaluator}
% ---------------------------------------------------------------------------- %

Der  Evaluator ist  f\"ur die  Auswertung der  Modulspannungen zust\"andig. Er
wird  einmal  t\"aglich ausgef\"uhrt  und  kontrolliert  die Spannungen  aller
Module des Systems innerhalb letzten 24 Stunden.

Daf\"ur  wird zuerst  f\"ur  jeden String  eine Liste  mit  den aktuell  darin
vorhandenen  Seriennummern  erstellt. Innerhalb  jedes Strings  wird  wiederum
f\"ur  jedes Modul  eine Liste  mit den  Abweichungen jeder  einzelnen Messung
w\"ahrend der  letzten 24 Stunden  erstellt. Nun wird \"uber die  gesamte Zeit
der quadratische  Mittelwert der Abweichungen gezogen. Mit  allen Mittelwerten
der   Module  innerhalb   eines  Strings   wird  nun   die  Standartabweichung
berechnet. Liegt ein  Modul ausserhalb  dieser Abweichung  wird es  als defekt
gemeldet. Dieses Vorgehen  ist zwar relativ kompliziert,  garantiert aber dass
nur Module  gemeldet werden, welche  \"uber l\"angere Dauer zu  wenig Leistung
bringen. Dies  erh\"oht die  Zuverl\"assigkeit des  Systems und  reduziert die
Wahrscheinlichkeit  auf  einen  Fehlalarm   welcher  den  Benutzer  unn\"otige
Umst\"ande bereiten w\"urde.


% ---------------------------------------------------------------------------- %
\subsubsection{\code{reporter.py}}
\label{subsubsec:software:master:implementation:reporter}
% ---------------------------------------------------------------------------- %

Der  Reporter hat  die Aufgabe,  bei  einem detektierten  Fehler den  Benutzer
\"uber  diesen  zu informieren. Dies  geschieht  indem  er einerseits  daf\"ur
sorgt,  dass   im  GUI  das  Fehlermeldungs-Display   mit  der  entsprechenden
Seriennummer angezeigt  wird. Zudem werden die Relaiskontakte  angesprochen um
einen beliebigen  externen Alarm  auszul\"osen. Und zu  guter Letzt  wird auch
noch das  GSM-Modul aktiviert um  auf die hinterlegte  Mobiltelefonnummer eine
Textnachricht zu senden.
