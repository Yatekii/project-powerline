% **************************************************************************** %
\chapter{Einleitung}
\label{chap:einleitung}
% **************************************************************************** %

In Zeiten der Energiewende  sind Photovoltaikanlagen kein Nischenprodukt mehr.
Um die Abh\"angigkeit vom Erd\"ol  zu verringern, werden vielerorts Anlagen in
allen Gr\"ossen  gebaut; von kleinen Installationen  f\"ur Einfamilienh\"auser
bis zu Anlagen,  welche einem normalen Kraftwerk in Sachen  Leistung die Stirn
bieten k\"onnen.  Die  Sonnenstrahlung, welche kostenlos von  der Sonne kommt,
wird  in elektrische  Energie  umgewandelt  und kann  gleich  vor Ort  genutzt
werden.   PV-Anlagen sind  teuer,  weshalb  Anlagenbesitzer darauf  angewiesen
sind,  dass diese  den maximalen  Ertrag liefern. Das  ist in  der Regel  ohne
grossen  Aufwand  der  Fall.   Doch  es gibt  Umst\"ande,  unter  welchen  die
Effizienz  einer  Photovoltaikanlage  sich   erheblich  verringern  kann,  und
h\"aufig wird dies  nicht oder nur mit bedeutender  Verz\"ogerung bemerkt.  In
einer  Photovoltaikanlage werden  \"ublicherweise mehrere  PV-Module zu  einem
String zusammengefasst, indem sie  in Serie geschaltet werden. Dabei reduziert
ein abgeschattetes,  verschmutztes oder  gar defektes  Modul den  Strom dieser
Serieschaltung  und  beeintr\"achtigt somit  auch  die  Leistung des  gesamten
Strings und der Anlage stark. Dies hat grosse finanzielle Einbussen zur Folge.

Das Ziel des  Projektes P4 ist es, ein  PV-\"Uberwachungssystem zu entwickeln,
um bei  Szenarien wie dem  oben geschilderten Abhilfe zu  schaffen. Das System
soll aus  einer Sensorplatine  und einem zentralen  Meldeger\"at bestehen. Die
Sensorplatine wird  beim PV-Modul  eingebaut und \"uberwacht  dessen Spannung,
das   zentrale  Meldeger\"at   wird  im   Schaltschrank  beim   Wechselrichter
installiert. Die Kommunikation zwischen den  beiden Komponenten erfolgt \"uber
die  zur  Energie\"ubertragung  benutzte  DC-Leitung;  es  werden  also  keine
zus\"atzlichen Datenleitungen ben\"otigt.

Das Master-Ger\"at wertet die gemessenen  Spannungen der Module aus und l\"ost
beim Erkennen eines  fehlerhaften Moduls einen Alarm am  Ger\"at selbst, lokal
in der PV-Anlage via Relais (z.B. Sirene oder Warnlampe) und per SMS aus.  Das
System  soll m\"oglichst  energieeffizient  und kosteng\"unstig  sein, um  die
Wirtschaftlichkeit einer Photovoltaikanlage nicht zu verschlechtern.

Das  Hauptproblem  eines solchen  Systems  liegt  bei der  Signal\"ubertragung
\"uber  die  DC-Leitung  der   Photovoltaikanlage. Denn  die  Spannung  darauf
schwankt  zwischen 12  und  60  Volt an  der  Sensorplatine  und betr\"agt  am
Masterger\"at bis zu 1000 Volt. Auf dieser Leitung ein Signal zu \"ubertragen,
ist schwierig  und wird heutzutage  kaum gemacht. Unsere L\"osung  koppelt mit
einer Spule  und dem Effekt  der Induktivit\"at  ein Signal in  die DC-Leitung
ein, was uns das \"Ubertragen von  Daten erlaubt. Die Spule wird von einem auf
dem  Sensor  platzierten Mikrocontroller  gesteuert,  der  die Spannung  misst
und  die  Messdaten  zum  Versenden  vorbereitet. Das  Master-Ger\"at  besitzt
eine  grafische  Benutzeroberfl\"ache zur  Interaktion  mit  dem Benutzer  und
benutzt  eine Datenbank  zum Verwalten  der Messdaten.   Zudem ist  das System
darauf optimiert, m\"oglichst  wenig Leistung zu verbrauchen,  um nicht selbst
bedeutende Zusatzkosten im Betrieb zu verursachen.

Der   vorliegende  Bericht   stellt  die   technische  Dokumentation   unseres
Systems   dar. Das   Kapitel  \emph{\titleref{chap:uberblick}}   bietet   eine
kurze  Einf\"uhrung  in   die  Thematik  von  PV-Anlagen;   der  Aufbau  einer
\"ublichen  PV-Anlage  wird  kurz   erl\"autert,  um  anschliessend  die  oben
Problematik  von Leistungsverlusten  zu  untersuchen. Ebenfalls werden  einige
Grundinformationen  zur  Signalmodulation  gegeben.  Anschliessend  werden  in
den  Kapiteln  \emph{\titleref{chap:models}}  und  \emph{\titleref{chap:simu}}
verschiedene     verfolgte    L\"osungsvarianten     modelliert    und     mit
Simulationen  in  \code{LTspice}   untersucht. Die  schlussendlich  gew\"ahlte
L\"osung   mit    induktiver   Kopplung   ist   in    den   darauf   folgenden
Kapiteln  \emph{\titleref{chap:hardware}} und  \emph{\titleref{chap:software}}
dokumentiert.   Im   Kapitel  \emph{\titleref{chap:validierung}}   werden  das
entwickelte  System   und  seine   Komponenten  mit   Labormessungen  kritisch
untersucht, um letztlich  im \emph{\titleref{chap:fazit}} unseren Gesamterfolg
zu beurteilen und auch  Empfehlungen f\"ur zuk\"unftige Weiterentwicklungen zu
geben.
