% ---------------------------------------------------------------------------- %
\subsection{Aufbau}
% ---------------------------------------------------------------------------- %

Das System besteht  aus zwei Teilen. Aus Sensoren, welche sich  direkt bei den
einzelnen Solarmodulen befinden und einer Zentrale (Master), welche die
Informationen der Sensoren auswertet.


% ---------------------------------------------------------------------------- %
\subsection{Spezifikationen}
\label{subsec:specs}
% ---------------------------------------------------------------------------- %

Die Spezifikationen  werden aufgegliedert gem\"ass den  beiden Hauptteilen des
Gesamtsystems: Sensor und Master.

% ---------------------------------------------------------------------------- %
\subsubsection{Sensor}
% ---------------------------------------------------------------------------- %

\begin{minipage}[c][][t]{.49\textwidth}
	Der Sensor besteht  aus einer m\"oglichst kleinen  Leiterplatte, welche direkt
	bei den  einzelnen Modulen installiert  werden kann.

    Jeder  Sensor \"uberwacht  dabei  die Spannung  der  jeweiligen Zelle  und
    \"ubermittelt  diese in  vorgegebenen Zeitabst\"anden  an den  Master. Die
    Spannungsmessung und  die Kommunikation werden von  einem energiesparenden
    Mikrokontroller geregelt.

	F\"ur  die  Daten\"ubertragung   werden  keine  zus\"atzlichen  Installationen
	ben\"otigt. Die   Kommunikation   findet   \"uber   die   Stromleitungen   der
	Solaranlage   statt. Daf\"ur  ist   eine   Modulation  vorgesehen. F\"ur   die
	Realisierung werden  folgende zwei M\"oglichkeiten evaluiert  (siehe Abschnitt
	\ref{subsec:losungsvarianten}).

	\begin{itemize}
		\item
			kapazitiv eingekoppelte Frequenzumtastung (FSK)
		\item
			serielle \"Ubertragung durch kurzzeitiges Kurzschliessen der Zelle
	\end{itemize}

	Der     schematische    Aufbau     des     Sensors     ist    in     Abbildung
	\ref{fig:blockdiag:sensor} zu sehen.
\end{minipage}
\hspace*{0.02\textwidth}
\begin{minipage}[c][][t]{.49\textwidth}
		\centering
		\includegraphics[width=\textwidth]{images/blockdiag-sensor.png}
		\captionof{figure}{Blockdiagramm des Sensors}
		\label{fig:blockdiag:sensor}
\end{minipage}


\clearpage
% ---------------------------------------------------------------------------- %
\subsubsection{Master}
% ---------------------------------------------------------------------------- %

Die Aufgabe des Masters besteht  im Empfangen der Messwerte der Panel-Sensoren
sowie  der  Messung des  Stromes  der  einzelnen Strings. Diese  Informationen
sollen gespeichert und  verarbeitet werden, um bei Bedarf  den Anwender \"uber
ein Display und/oder  eine Textnachricht zu informieren  (falls das Wunschziel
\emph{GSM-Modul}  realisiert worden  ist) und  elektrische Signale  via Relais
auszugeben.

Die   Grundlage  dieses   Ger\"ats   bildet  ein   Computer,  welcher   \"uber
die   ben\"otigten  Schnittstellen   und   die  erforderliche   Rechenleistung
verf\"ugt. Aufgrund der guten Verf\"ugbarkeit, des g\"unstigen Preises und der
grossen Auswahl  and kompatibler  Peripherie wird  hierf\"ur ein  Raspberry Pi
Modell  B+  oder  neuer  eingesetzt. Es kann  direkt  mit  einem  Anzeigemodul
verbunden werden, hat  USB-Anschl\"usse f\"ur ein USB-GSM Modem  und ist klein
genug, um in einem DIN-Schienengeh\"ause untergebracht werden zu k\"onnen.

Funktionen,   welche   nicht   Bestandteil  des   kommerziell   erh\"altlichen
Computermoduls sind,  werden auf  einer im  Verlauf des  Projekts entwickelten
Tochterplatine realisiert. Dies sind inbesondere  die Strommessung der Strings
bis  \SI{10}{\ampere}, der  Empfang  von Nachrichten  der  Sensoren sowie  die
Ausgabe von Signalen via Relais.

Die    Stromversorgung   des    Ger\"ats   wird    von   einem    eingekauften
\SI{5}{\volt}-Modul \"ubernommen.

Falls  das  Wunschziel  \emph{GSM-Modul}  realisiert  werden  kann,  wird  zum
Versenden  der  Textnachrichten  im  Testbetrieb w\"ahrend  des  Projekts  die
SIM-Karte  eines Teammitglieds  benutzt. Im Praxisbetrieb  sollte vorzugsweise
eine  SIM-Karte mit  unbeschr\"anktem Budget  f\"ur Textnachrichten  verwendet
werden.
