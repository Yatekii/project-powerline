% **************************************************************************** %
\chapter{Einleitung}
\label{chap:einleitung}
% **************************************************************************** %

Photovoltaikanlagen   sind  heutzutage   kein   Nischenprodukt  mehr. Um   die
Abhängigkeit  vom Erdöl  zu verringern,  werden vielerorts  kleine, aber  auch
grosse  Anlagen gebaut. Die  W\"arme-Energie, welche  kostenlos von  der Sonne
kommt,  wird  in elektrische  Energie  umgewandelt  und  kann gleich  vor  Ort
genutzt werden. Anlagenbesitzer  investieren meistens einen grossen  Betrag in
eine  neue  Anlage  und  sind  darauf angewiesen,  dass  diese  den  maximalen
Ertrag  liefert. Das ist  in der  Regel  ohne grossen  Aufwand der  Fall. Doch
es  gibt Umstände,  welche  die Effizienz  einer Photovoltaikanlage  erheblich
verringern  können und  dies meist  ohne, dass  es jemand  bemerkt.  In  einer
Photovoltaikanlage  werden üblicherweise  mehrere  PV-Module  zu einem  String
zusammengefasst,  indem  sie  in   Serie  geschaltet  werden. Dabei  kann  ein
abgeschattetes,  verschmutztes  oder  gar  defektes  Modul  den  Strom  dieser
Serieschaltung und somit auch die Leistung des gesamten Strings und der Anlage
stark beeinträchtigen. Was grosse finanzielle Einbussen zur Folge haben kann.

Das  Ziel des  Projektes P4  war es,  ein PV-Überwachungssystem  bestehend aus
einer Sensorplatine für den Einbau in  die Anschlussbox jedes Moduls und einem
zentralen Meldegerät  für den Einbau  im Schaltschrank beim  Wechselrichter zu
entwickeln, aufzubauen und zu testen. Die  Sensorplatine soll die Spannung des
jeweiligen PV-Modules  messen und sie  an das Mastergerät über  die bestehende
DC-Leitung  der  Anlage  übermitteln. Im  Mastergerät  werden  die  gemessenen
Spannungen der  einzelnen PV-Module  gespeichert und  ausgewertet. Erkennt das
Mastergerät ein fehlerhaftes  PV-Modul, soll eine Alarmierung  am Gerät selbst
und  per  SMS  ausgegeben  werden. Zusätzlich wird  ein  Relais  zur  externen
Signalisation  betätigt.   Das  System  soll  möglichst  energieeffizient  und
kostengünstig sein,  um die wirtschaftlichkeit einer  Photovoltaikanlage nicht
zu verschlechtern.

Das  Hauptproblem liegt  bei  der Signalübertragung  über  die DC-Leitung  der
Photovoltaikanlage. Denn die Spannung darauf schwankt  zwischen 12 und 60 Volt
an der Sensorplatine  und beträgt am Mastergerät bis zu  1000 Volt. Auf dieser
Leitung  ein Signal  zu  übertragen  ist schwierig  und  wird heutzutage  kaum
gemacht.  Zudem muss auf kleine Leistung beim gesamten System geachtet werden,
um keine wertfolle Energie zu verschwenden.

\todo{Beschreibung unseres Produkts}

Der vorliegende  Bericht stellt  die technische Dokumentation  unseres Systems
dar.  Zuerst wird das Konzept unserer L\"osung beschrieben, zusammen mit einer
Benutzerf\"uhrung.  Anschliessend  wird auf  das Hardware-  und Firmwaredesign
eingegangen,  und   zuletzt  werden  die  am   System  durchgef\"uhrten  Tests
dokumentiert.

\lipsum
