% ---------------------------------------------------------------------------- %
\chapter{Vereinfachtes Modell f\"ur ein PV-Modul}
\label{app:models:develop:module:simple}
% ---------------------------------------------------------------------------- %

Zur  Reduktion des  Rechenaufwands im  Vergleich zu  dem im  vorigen Abschnitt
entwickelten Modell soll ein vereinfachtes Modell eines Solarmoduls entwickelt
werden.

Ausgangspunkt   ist   das   Eindiodenmodell    einer   Zelle   aus   Abbildung
\todo{reference}, f\"ur  welches die Parameter  so angepasst werden,  dass das
Verhalten des aus Zellen aufgebauten Modells aus dem vorigen Abschnitt und des
vereinfachten Modells zufriedenstellend \"ubereinstimmen.

Das Vorgehen ist dabei wie folgt: \todo{Eigentliche Werte}

\begin{itemize}
    \item
        Die parasit\"are  Kapazit\"at des  Moduls wird angepasst  gem\"ass dem
        Gesetz zur Serieschaltung von Kapazit\"aten:
        $C_{\mathrm{p, Modul}} = C_{\mathrm{p, Zelle}} \div \text{Anzahl Zellen}$
    \item
        Der Seriewiderstand des Moduls ist die Summe der Seriewiderst\"ande aller
        Zellen
    \item
        Der  Parallele   Widerstand  wird   entsprechend  der   Anzahl  Zellen
        hochgerechnet.
        \todo{kein Einfluss, verifikation}
    \item
        Der Reverse  Saturation Current  des Moduls ist  von der  Fl\"ache des
        Moduls abh\"angig und  wird als Startwert vorerst  gem\"ass der Anzahl
        Module hochgerechnet.
    \item
        Der  Idealit\"atsfaktor ist  ein Indikator  f\"ur den  Spannungsabfall
        \"uber   einer   Diode   bei   einem   bestimmten   Strom   \todo{ref:
        abbildung}. Bei  einer Serieschaltung  von Dioden  sollte entsprechend
        mehr   Spannung    \"uber   dem   gesamten   Strang    abfallen   (bei
        gleichbleibendem  Strom). Der Startwert  f\"ur den  Idealit\"atsfaktor
        des vereinfachten Modells wird  deshalb entsprechend der Anzahl Zellen
        im Modul skaliert.
\end{itemize}

Gem\"ass dieser Methodik werden die Startwerte festgelegt auf:

\begin{itemize}
    \firmlist
    \item
        $C_{\mathrm{p, Modul}} = \SI{167}{\nano\farad}$
    \item
        $R_{\mathrm{S}} = \SI{12}{\milli\ohm}$
    \item
        $R_{\mathrm{Shunt}} = \SI{60}{\kilo\ohm}$
    \item
        $I_{\mathrm{S}} = \SI{288}{\milli\ampere}$
    \item
        $n = 118$
\end{itemize}

Mit diesen  Werten werden verschiedene  Szenarien simuliert und  die Resultate
mit den  Ergebnissen des komplexeren Modells  verglichen. Durch Iteration wird
$I_{\mathrm{S}}$ schlussendlich  auf \SI{2.88}{\micro\ampere}  festgelegt, was
kurioserweise kleiner ist als der Wert einer einzelnen Diode.

Abbildungen
\ref{fig:model:simpel:verif:current:mosfet:freq},
\ref{fig:model:simpel:verif:voltage:mosfet:freq},
\ref{fig:model:simpel:verif:current:RL:freq} und
\ref{fig:model:simpel:verif:current:mosfet:time} stellen die Resultate
verschiedener Simulationen mit und ohne Last f\"ur das vereinfachte Modell und
das zellenbasierte  Modell graphisch dar.   Wie zu erkennen ist,  besteht eine
gute \"Ubereinstimmung.

%Parasit\"are Kapazit\"at: Kapazit\"at einer Zelle / anzahl Zellen \\
%Serie widerstand: widerstand einzelzelle * anzahl zellen \\
%shunt widerstand: widerstand einer zelle / anzahl zellen \\
%IS: IS einzelzelle * anzahl zellen (0.288m) -> anpassen, bis frequenzgang und zeitplot passend \\
%N: N einzelzelle * anzahl zellen \\
%Shockley diode model \\
\todo{Einheiten achsen, auflistung parameter, schaltkreise zu plots, LTspice diagramme}


\begin{figure}
    \begin{tikzpicture}
       \begin{scope}[x={(0mm,0mm)},y={(90mm,0.9\textwidth)}]
           \begin{axis}[%
                   xmode=log,
                   height=45mm,
                   width=0.9\textwidth,
                   at={(0,45mm)},
                   grid=both,
                   xlabel=Frequenz (\si{\hertz}),
                   ylabel=Strom (\si{\deci\bel}),
               ]
                \addplot[color=blue] table {data/simple-model-verification/module-72cells-series--reference--freq-sweep-over-mosfet-no-load--I--magn.dat};
                \addplot[color=magenta] table {data/simple-model-verification/module-simple-72x1--freq-sqeep-over-mosfet-no-load--I--magn.dat};
           \end{axis}
           \begin{axis}[%
                   xmode=log,
                   height=45mm,
                   width=0.9\textwidth,
                   at={(0,0)},
                   grid=both,
                   xlabel=Frequenz (\si{\hertz}),
                   ylabel=Phase (\si{\degree}),
               ]
                \addplot[color=blue] table {data/simple-model-verification/module-72cells-series--reference--freq-sweep-over-mosfet-no-load--I--phase.dat};
                \addplot[color=magenta] table {data/simple-model-verification/module-simple-72x1--freq-sqeep-over-mosfet-no-load--I--phase.dat};
           \end{axis}
       \end{scope}
   \end{tikzpicture}
    \caption{Frequenzgang des Stroms durch den MOSFET bei einer Beschaltung ohne zus\"atzliche Last}
    \label{fig:model:simpel:verif:current:mosfet:freq}
\end{figure}

\begin{figure}
    \begin{tikzpicture}
       \begin{scope}[x={(0mm,0mm)},y={(90mm,0.9\textwidth)}]
           \begin{axis}[%
                   xmode=log,
                   height=45mm,
                   width=0.9\textwidth,
                   at={(0,45mm)},
                   grid=both,
                   xlabel=Frequenz (\si{\hertz}),
                   ylabel=Spannung (\si{\deci\bel}),
               ]
                \addplot[color=blue] table {data/simple-model-verification/module-72cells-series--reference--freq-sweep-over-mosfet-no-load--U--magn.dat};
                \addplot[color=magenta] table {data/simple-model-verification/module-simple-72x1--freq-sweep-over-mosfet-no-load--U--magn.dat};
           \end{axis}
           \begin{axis}[%
                   xmode=log,
                   height=45mm,
                   width=0.9\textwidth,
                   at={(0,0)},
                   grid=both,
                   xlabel=Frequenz (\si{\hertz}),
                   ylabel=Phase (\si{\degree}),
               ]
                \addplot[color=blue] table {data/simple-model-verification/module-72cells-series--reference--freq-sweep-over-mosfet-no-load--U--phase.dat};
                \addplot[color=magenta] table {data/simple-model-verification/module-simple-72x1--freq-sweep-over-mosfet-no-load--U--phase.dat};
           \end{axis}
       \end{scope}
   \end{tikzpicture}
    \caption{Frequenzgang der Spannung \"uber dem MOSFET bei einer Beschaltung ohne zus\"atzliche Last}
    \label{fig:model:simpel:verif:voltage:mosfet:freq}
\end{figure}

\begin{figure}
    \begin{tikzpicture}
       \begin{scope}[x={(0mm,0mm)},y={(90mm,0.9\textwidth)}]
           \begin{axis}[%
                   xmode=log,
                   height=45mm,
                   width=0.9\textwidth,
                   at={(0,45mm)},
                   grid=both,
                   xlabel=Frequenz (\si{\hertz}),
                   ylabel=Spannung (\si{\deci\bel}),
               ]
                \addplot[color=blue] table {data/simple-model-verification/module-72cells-series--reference--freq-sweep-over-Rload--100ohm-100uF--I--magn.dat};
                \addplot[color=magenta] table {data/simple-model-verification/module-simple-72x1--freq-sweep-over-Rload--100ohm-100uF--I--magn.dat};
           \end{axis}
           \begin{axis}[%
                   xmode=log,
                   height=45mm,
                   width=0.9\textwidth,
                   at={(0,0)},
                   grid=both,
                   xlabel=Frequenz (\si{\hertz}),
                   ylabel=Phase (\si{\degree}),
               ]
                \addplot[color=blue] table {data/simple-model-verification/module-72cells-series--reference--freq-sweep-over-Rload--100ohm-100uF--I--phase.dat};
                \addplot[color=magenta] table {data/simple-model-verification/module-simple-72x1--freq-sweep-over-Rload--100ohm-100uF--I--phase.dat};
           \end{axis}
       \end{scope}
   \end{tikzpicture}
   \caption{Frequenzgang des Stroms durch den Lastwiderstand bei einer Last von \SI{100}{\ohm} parallel zu \SI{100}{\micro\farad}}
    \label{fig:model:simpel:verif:current:RL:freq}
   %\label{fig:freqresponse:module:simple}
\end{figure}


\begin{figure}
    \begin{tikzpicture}
       \begin{scope}[x={(0mm,0mm)},y={(60mm,0.9\textwidth)}]
           \begin{axis}[%
                   height=60mm,
                   width=0.9\textwidth,
                   at={(0,0)},
                   grid=both,
                   xlabel=Zeit (\si{\second}),
                   ylabel=Strom (\si{\ampere}),
               ]
                \addplot[color=blue] table {data/simple-model-verification/module-72cells-series--reference--time-domain-over-mosfet-10kHz--I.dat};
                \addplot[color=magenta] table {data/simple-model-verification/module-simple-72x1--time-domain-over-mosfet--10kHz--I.dat};
           \end{axis}
       \end{scope}
   \end{tikzpicture}
    \caption{Zeitlicher Verlauf des Stroms durch den MOSFET bei einer Beschaltung ohne zus\"atzliche Last}
    \label{fig:model:simpel:verif:current:mosfet:time}
\end{figure}
