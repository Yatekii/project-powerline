% **************************************************************************** %
\chapter{Einleitung}
\label{chap:einleitung}
% **************************************************************************** %

Im  Projekt  4 vom  Fr\"uhlingssemester  2016  des Studiengangs  Elektro-  und
Informationstechnik wurde  die Aufgabe gestellt, ein  System zur \"Uberwachung
einer  Photovoltaikanlage zu  entwickeln. In  einer Photovoltaikanlage  werden
\"ublicherweise mehrere  PV-Module zu einem Strang  zusammengefasst, indem sie
in Serie geschaltet werden. Dabei  kann ein abgeschattetes, verschmutztes oder
gar defektes Modul den Strom dieser Serieschaltung und somit auch die Leistung
des  gesamten  Strangs und  der  Anlage  stark beeintr\"achtigen. Der  dadurch
verursachte  Energieverlust kann  zwar erkannt  werden, jedoch  nur mit  einem
betr\"achtlichen Zeitaufwand.

Das zu  entwickelnde System  soll diese aufw\"andige  Arbeit \"ubernehmen. Die
konkreten    Ziele    des    Projektes   sind: Ein    kosteng\"unstiges    und
energieeffizientes  System  zu entwickeln,  welches  die  Daten der  einzelnen
Module an ein  zentrales Master-Ger\"at sendet, um  sie dort auszuwerten. Dies
soll   \"uber  die   bestehende   Gleichstromleitung  der   Photovoltaikanlage
geschehen, um den Installationsaufwand so  gering wie m\"oglich zu halten. Das
Master-Ger\"at  soll  alle  Daten  der   Module  sammeln  und  die  Auswertung
\"ubernehmen. Ergibt sich aus  der Auswertung ein Modulfehler,  wird ein Alarm
\"uber GSM  und zus\"atzlich  \"uber zwei Relais  ausgel\"ost. Das fehlerhafte
Modul wird angezeigt, um die Fehlerquelle schnell lokalisieren zu k\"onnen und
so Leistungseinbussen der Anlage zu verhindern.

Das Projektteam  besteht aus  sieben Studenten  des Studiengangs  Elektro- und
Informationstechnik.  Es ist zu erw\"ahnen, dass das Team eine herausfordernde
Zusammenstellung aufweist. Es besteht aus drei  Abg\"anger der ETH, welche nur
geringe Projekterfahrung mit sich bringen  und zudem aus drei berufsbegleitend
Studierenden, welche  ebenfalls nur beim  Projekt 1  dabei waren und  ein sehr
begrenztes Zeitpensum zur Verf\"ugung haben.

Die   Aufteilung  des   Teams   wurde  in   die  Ressorts   Projektmanagement,
Hardware, Software  und nachtr\"aglich auch in  Dokumentation vorgenommen. Das
Projetmanagement  besteht  aus  Reto   Nussbaumer  und  seinem  Stellvertreter
Marco  Koch,  das Ressort  Hardware  besteht  aus  Noah H\"usser  und  Dominik
Keller. Weiter gibt es  das Ressort Software mit Marcel  Heymann und Francesco
Rovelli und zu guter Letzt, das nachtr\"aglich erstellte Ressort Dokumentation
mit Raphael Frey.

Die  Kommunikation  mit dem  Auftraggeber  und  die Abgabe  der  Lieferobjekte
erfolgten  grunds\"atzlich \"uber  den  Projektleiter. Jedoch wurden  wichtige
Entscheidungen bei den w\"ochentlichen Sitzungen gemeinsam getroffen.

Das Projekt stellte eine grosse Herausforderung f\"ur das gesamte Team dar. Es
konnten viele  neue Erfahrungen gesammelt  werden, aus technischer,  vor allem
aber  auch  aus methodischer  Sicht. Das  Klima  im Team  war  gr\"osstenteils
sehr  gut und  alle konnten  positive Erfahrungen  sammeln. Es wurde  viel mit
unbekannten und neuen Mitteln gearbeitet und trotzdem waren Erfolge schnell zu
sehen. Auf der Projektmanagement-Ebene  wurden einige Arbeiten untersch\"atzt,
dies hat  sich gegen Ende  des Projektes  angeh\"auft und f\"uhrte  dazu, dass
nicht alle Arbeiten fertig gestellt werden konnten.

Auf  der  Produkt-Ebene  konnten   viele  Arbeiten  erfolgreich  abgeschlossen
werden. Obwohl  das Produkt  am Ende  nicht wie  geplant funktionierte,  waren
die  Simulationen und  die  Berechnungen  sehr erfolgversprechend. Die  fertig
designten Platinen waren  bereit f\"ur die Bestellung,  leider reichte daf\"ur
jedoch die Zeit nicht mehr aus.

Die  wohl wichtigste  Erkenntnis aus  diesem Projekt  war, dass  auf Bekanntes
und  Bew\"ahrtes  gesetzt werden  sollte. Wir  versuchten  mit uns  bis  dahin
unbekannten  Mitteln  das   Ziel  zu  erreichen,  was   sehr  zeitraubend  und
nervt\"otend war. Wir verbrachten  zu viel Zeit mit neuem  Lernen, anstatt das
Gelernte anzuwenden. Am Schluss fehlte uns schlicht und einfach diese Zeit.

Dieser Bericht ist in drei  Kapitel unterteilt; Markante Ereignisse, Reflexion
und das  Schlusswort. Im ersten  Kapitel Markante  Ereignisse werden  die drei
wichtigsten positiven und negativen Ereignisse des Projektmanagements in einem
Diagramm  dargestellt und  kurz erl\"autert. Anschliessend  folgt das  Kapitel
Reflexion mit der detaillierten  Analyse einiger ausgew\"ahlter Ereignisse. Am
Ende folgt  das Schlusswort, welches  die wichtigsten Punkte  dieses Berichtes
zusammenfasst.
