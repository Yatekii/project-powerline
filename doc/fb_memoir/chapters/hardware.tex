% **************************************************************************** %
\chapter{Hardware}
\label{chap:hardware}
% **************************************************************************** %

Das folgende Kapitel dokumentiert  unseren L\"osungsfindungsprozess im Bereich
Hardware. Zuerst  wird  auf  die  Problematik  der  Kommunikation  \"uber  die
DC-Leitung  zwischen  Solarmodul  und \Master  eingegangen. Es  werden  unsere
L\"osungsans\"atze erl\"autert  und auf St\"arken und  Schw\"achen untersucht,
um daraus anschliessend eine Entscheidung abzuleiten.

In   einem  n\"achsten   Schritt  werden   die  L\"osungskonzepte   f\"ur  die
Sensor-Platine und das \Master genauer erkl\"art.

\anweisung Verifizierbare  Angaben machen, nicht  allgemeine Floskeln. Zahlen,
Fakten,  theoretische Hintergrundinformationen,  daraus gezogene  Konsequenzen
etc. Bezug auf das Pflichtenheft, wo m\"oglich.

\anweisung Bilder, Diagramme, Formeln etc.


% ---------------------------------------------------------------------------- %
\clearpage
\section{Sensorplatine}
\label{sec:hw:sensorplatine}
% ---------------------------------------------------------------------------- %

Was  ist der  Zweck  der Sensorplatine? Welche  Aufgaben  muss sie  erf\"ullen
k\"onnen? Wie ist sie aufgebaut? PCB-Layout? Energieversorgung?

\anweisung Dimensionierung der Bauteile: Berechnungen


% ---------------------------------------------------------------------------- %
\subsection{Energieversorgung}
\label{subsec:sensor:pcb}
% ---------------------------------------------------------------------------- %

Energiebezug,  Leistungsanforderungen, Standby,  Verhalten bei  ungen\"ugender
Leistungszufuhr, ...


% ---------------------------------------------------------------------------- %
\subsection{PCB}
\label{subsec:sensor:pcb}
% ---------------------------------------------------------------------------- %

Wie sieht das PCB aus, und weshalb?


% ---------------------------------------------------------------------------- %
\section{Master-Ger\"at}
\label{sec:hw:mastergerat}
% ---------------------------------------------------------------------------- %

Was ist der Zweck des Master-Ger\"ats? Aufgaben? Aufbau?

\anweisung Dimensionierung der Bauteile: Berechnungen


% ---------------------------------------------------------------------------- %
\subsection{Speisung}
\label{subsec:mastergerat:speisung}
% ---------------------------------------------------------------------------- %

Woher bezieht das Master-Ger\"at  seine Energie, weshalb? Wieviel Energie wird
ben\"otigt?


% ---------------------------------------------------------------------------- %
\subsection{Benutzer-Interface}
\label{subsec:mastergerat:interface}
% ---------------------------------------------------------------------------- %

\anweisung Das Benutzerinterface ist schon in der Anleitung im vorigen Kapitel
beschrieben worden. Es soll  hier um die Komponentenwahl gehen,  nicht um eine
Rekapitulation bereits gemachter Informationen.


% ---------------------------------------------------------------------------- %
\subsection{PCB}
\label{subsec:mastergerat:pcb}
% ---------------------------------------------------------------------------- %

Wie sieht das PCB aus, und weshalb?


% ---------------------------------------------------------------------------- %
\subsection{Montage/Geh\"ause}
\label{subsec:mastergerat:pcb}
% ---------------------------------------------------------------------------- %

Einbindung in die Umgebung, Wahl des Geh\"auses.


% ---------------------------------------------------------------------------- %
\section{Kommunikation}
\label{sec:kommunikation}
% ---------------------------------------------------------------------------- %

Die   Kommunikation  zwischen   Sensor  und   Master-Ger\"at  ist   eines  der
Herzst\"ucke des ganzen Systems und betrifft beide Sub-Systeme. Daher in einem
eigenen Abschnitt.

\begin{itemize}
    \item
        Grunds\"atzliches Problem
    \item
        Verfolgte L\"osungsans\"atze
    \item
        Ausgew\"ahlter L\"osungsansatz (mit Begr\"undung(en))
    \item
        Genauere Beschreibung dieses L\"osungsansatzes
\end{itemize}
