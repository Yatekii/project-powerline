% **************************************************************************** %
\chapter{Schlusswort}
\label{chap:schlusswort}
% **************************************************************************** %

Leider  konnte  kein   funktionierender  Prototyp  entwickelt  werden. Daf\"ur
wurden  einige Teilsysteme  fertiggestellt. Die gr\"osste  Herausforderung war
wohl  die  \"Ubertragung  der  Messdaten  \"uber  die  Gleichstromleitung  der
Photovoltaikanlage. Diese konnte  getestet werden und funktionierte  sehr gut,
was  sehr erfreulich  war. Zudem wurde  das Masterger\"at  in einem  Geh\"ause
untergebracht, wobei die Men\"uf\"uhrung  voll funktionierte. Ein Prototyp der
Sensorplatine war ebenfalls  funktionsf\"ahig und zu all dem  waren auch viele
Simulationen sehr erfolgsversprechend.

Im   Projektmanagement   wurden   einige   Fehleinsch\"atzungen   gemacht. Zum
Beispiel   fehlten  getrennte   Arbeitspakete  f\"ur   die  Sensor-   und  die
Masterplatine. Weiter  wurde  der  Arbeitsaufwand bei  einigen  Arbeitspaketen
untersch\"atzt. Denn  obwohl  die  ganze  Planung in  der  Gruppe  detailliert
besprochen   wurde,   konnten   die  Fehleinsch\"atzungen   nicht   korrigiert
werden. Dies ist  ganz klar  auf die  mangelnde Projekterfahrung  des gesamten
Teams  zur\"uckzuf\"uhren. Positiv   zu  erw\"ahnen  ist,  dass   das  gesamte
Projektmanagement  mit allen  Elementen  konsequent  durchgef\"uhrt wurde  und
auch  Erfolge mit  sich brachte. Es  konnten einige  organisatorische Probleme
wie  zum  Beispiel  unklare Aufgabenstellungen,  Zust\"andigkeitsprobleme  und
zwischenmenschliche Konflikte mit den gelernten Methoden entsch\"arft werden.

Die wichtigste  Erfahrung \"uber das  ganze Projekt war, dass  auf Bew\"ahrtes
gesetzt  werden   sollte.   R\"uckblickend   war  es  illusorisch   f\"ur  das
Projekt  eine  unbekannte  Programmiersprache,  ein  neues  Layout-Tool,  neue
Mikrocontroller zu  verwende und alles  im kleinstm\"oglichen Format  bauen zu
wollen.   All diese  neuen Teile  f\"uhrten zu  einem enormen  Zeitaufwand und
schlussendlich zu einem nicht funktionierenden Prototyp.

Es  wurden jedoch  auch  sehr positive  Erfahrungen  gemacht. Obwohl das  Team
kaum  unterschiedlicher sein  konnte, wurde  sehr fleissig  und zielorientiert
zusammengearbeitet. Die  fertigen   Teile  des   Projektes  waren   von  guter
Qualit\"at und jedes Mitglied konnte sein Bestes dazu beitragen.
