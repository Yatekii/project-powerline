\begin{tikzpicture}
    \begin{scope}[x={(0mm,0.95\textwidth)},y={(0mm,150mm)}]
        \begin{axis}[%
            title=Spannung am Knoten \code{MEASURE},
            height=45mm,
            width=0.95\textwidth,
            at={(0,100mm)},
            grid=both,
            xlabel=Zeit (\si{\micro\second}),
            ylabel=Spannung (\si{\volt}),
            xmin = 0,
            xmax = 2e-4,
            change x base=true,
            x SI prefix=micro,
        ]
            \addplot[-,blue] table[x=time, y=V(measure)] {data/short-circuit/shortcircuit-transmitter-0.2ms.dat};
        \end{axis}
        \begin{axis}[%
            title=Strom durch MOSFET und Modul,
            height=45mm,
            width=0.95\textwidth,
            at={(0,50mm)},
            grid=both,
            xlabel=Zeit (\si{\micro\second}),
            ylabel=Strom (\si{\ampere}),
            xmin = 0,
            xmax = 2e-4,
            change x base=true,
            x SI prefix=micro,
        ]
            \addplot[-,blue] table[x=time, y=Id(MOS)] {data/short-circuit/shortcircuit-transmitter-0.2ms.dat};
            \addplot[-,magenta] table[x=time, y=I(module)] {data/short-circuit/shortcircuit-transmitter-0.2ms.dat};
            \legend{MOSFET,PV-Modul};
        \end{axis}
        \begin{axis}[%
            title=Leistung im MOSFET und PV-Modul,
            height=45mm,
            width=0.95\textwidth,
            at={(0,0)},
            grid=both,
            xlabel=Zeit (\si{\micro\second}),
            ylabel=Leistung (\si{\watt}),
            xmin = 0,
            xmax = 1.5e-6,
            change x base=true,
            x SI prefix=micro,
        ]
            \addplot[-,blue] table[x=time, y=powerMOS] {data/short-circuit/shortcircuit-transmitter-PPeak.dat};
            \addplot[-,magenta] table[x=time, y=powerModule] {data/short-circuit/shortcircuit-transmitter-PPeak.dat};
            \legend{MOSFET,PV-Modul};
        \end{axis}
   \end{scope}
\end{tikzpicture}
