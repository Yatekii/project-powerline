Die Bedienung wird sinnvollerweise gem\"ass Sensor und Master unterteilt.

% Initialisierung/Inbetriebnahme


% ---------------------------------------------------------------------------- %
\subsection{Sensor}
% ---------------------------------------------------------------------------- %

Jeder   Sensor   ist   werkseitig   mit   einem   Kleber   mit   vorgedruckter
Identifikationsnummer  ausgestattet. Bei der  Installation wird  dieser Kleber
abgezogen  und  auf  dem  Installationsplan  an  der  entsprechenden  Position
aufgeklebt. So beh\"alt der Installateur  automatisch den \"Uberblick, welcher
Sensor wo  platziert ist. Damit  der Sensor permanent  identifizierbar bleibt,
ist seine Identifikationsnummer auch aufgedruckt.

Sobald der Sensor mit Strom versorgt ist, schaltet er sich automatisch ein und
sendet ein Datenpaket mit seiner  Identifikationsnummer an den Master, welcher
dann die ankommenden Daten auswertet.

Zum Erzwingen  einer erneuten Initialisierung  ist am Sensor  ein Reset-Button
angebracht; das Debuggen wird durch LEDs erleichtert.


% ---------------------------------------------------------------------------- %
\subsection{Master}
% ---------------------------------------------------------------------------- %


Auch  der Master  schaltet  sich bei  vorhandener Stromversorgung  automatisch
ein.  Ohne weitere Inputs der  Aussenwelt (Sensoren oder Benutzer) werden nach
dem  Hochfahren  die  konfigurierten  Sensoren  ausgewertet  (bzw.  ankommende
Datenpakete) und bei Bedarf die konfigurierten Reaktionen ausgel\"ost.

Falls ein  GSM-Modul (Wunschziel)  vorhanden ist,  kann die  Telefonnummer, an
welche  Fehler-  und  Statusmeldungen versendet  werden  sollen,  konfiguriert
werden. Das  gew\"unschte Verhalten  f\"ur verschiedene  Fehlerzust\"ande kann
ebenfalls  vom  Benutzer  definiert  werden. Ansonsten  sind  bei  der  ersten
Inbetriebnahme   keine   speziellen   Massnahmen   erforderlich. Im   normalen
Betriebsfall ist keine menschliche Intervention n\"otig.

Die  Bedienung  soll  m\"oglichst einfach  sein. Ein  integriertes  Touchpanel
im  Master  erlaubt ein  komfortables  Interagieren  mit dem  Ger\"at.   F\"ur
einen  allf\"alligen  Systemneustart  ist ein  Reset-Button  angebracht.   Zum
erleichterten  Debuggen  und  allenfalls  sonstige  Informationsausgabe  (z.B.
defektes Solarpanel) sind LEDs vorhanden.
