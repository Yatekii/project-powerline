% **************************************************************************** %
\chapter{Fazit}
\label{chap:fazit}
% **************************************************************************** %
\enlargethispage{4em}

\vspace*{-2em}
Nach  Abschluss  des   Projektes  steht  dem  Auftraggeber   ein  Produkt  zur
Verf\"ugung, das  als Gesamtsystem die  Spannungen an PV-Modulen  erfassen und
auswerten  kann. Die einzelnen  Teilsysteme sind  mehrheitlich als  Prototypen
vorhanden, im Labor getestet und funktionsf\"ahig.

Der  Sensor kann  Spannungen messen  und Werte  ausgeben. Mit einer  Abmessung
von  knapp  $\SI{5}{\centi\meter}\times\SI{5}{\centi\meter}$  bieten  wir  dem
Auftraggeber eine  Sensorplatine, die  garantiert in  jedem Anschlussgeh\"ause
Platz findet. Die  Firmware zusammen  mit den  gew\"ahlten Hardwarekomponenten
erm\"oglichen eine saubere Messung  der Modulspannung. Die Hauptproblematik im
Projekt bestand darin, eine optimale L\"osung f\"ur die Kommunikation zwischen
Sensor und Master-Ger\"at  zu finden. Mit induktiver Einkopplung  und OOSK ist
eine eine Variante gew\"ahlt worden, die es erm\"oglicht, die gemessenen Daten
der  PV-Module dem  Masterger\"at  zu  \"ubermitteln. Das Grundprinzip  dieser
\"Ubertragungsmethode ist validiert.

Mit einem Testaufbau k\"onnen Signale  vom Master-Ger\"at empfangen und an den
Raspberry  Pi  weitergeleitet  werden. Mithilfe  der  Mastersoftware  ist  der
Raspberry Pi  im Stande, die  empfangenen Messwerte, auszuwerten,  mit anderen
Messdaten zu vergleichen und n\"otigenfalls eine Fehlermeldung auf dem Display
auszugeben. Das  Master-Ger\"at  bietet  einerseits  eine  benutzerfreundliche
Oberfl\"ache,  andererseits l\"asst  sich  das schlichte  Geh\"ause bequem  in
einen bestehenden  Generatoranschlusskasten installieren, auch wenn  nur wenig
Platz  vorhanden  ist. Die  Bedienung  und  Inbetriebnahme  \"uber  das  2.5``
Touch-Display ist \"ausserst komfortabel und selbsterkl\"arend.  Ein GSM-Modul
f\"ur  das Versenden  einer Textnachricht  an ein  Mobiltelefon ist  ebenfalls
vorhanden,  lediglich  dessen  Ansteuerungsfirmware  muss  noch  implementiert
werden. Anschliessend  ist  das  Master-Ger\"at  im  Stande,  im  Falle  eines
Modulfehlers,  neben der  Bet\"atigung  der beiden  Relaiskontakte, auch  eine
Fehlermeldung per  SMS zu  versenden. Die geplante Zusatzfunktion  zur Messung
der Strangstr\"ome am Master-Ger\"at ist wegen Zeitmangel noch nicht getested,
jedoch in die Entwicklung integriert.

Als  n\"achster   Schritt  sollte  die  Daten\"ubermittlung   vom  Sensor  zum
Master-Ger\"at mittels  Powerline getestet werden, wie  auch das Zusammenspiel
von Tochterboard und  Raspberry Pi. Letztlich sollte das  gesamten Systems als
Ganzes \"uberpr\"uft werden.
