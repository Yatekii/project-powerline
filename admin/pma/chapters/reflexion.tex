% **************************************************************************** %
\chapter{Reflexion}
\label{chap:reflexion}
% **************************************************************************** %

In diesem Kapitel werden von den  zuvor genannten Ereignissen je zwei negative
und  zwei  positive  Punkte  reflektiert. Die einzelnen  Reflexionen  sind  in
Abschnitte  gegliedert,  die zuerst  die  Ausgangslage  und anschliessend  die
Planabweichung  erl\"autert. Folgend werden  Massnahmen  und deren  Auswirkung
genannt. Am Schluss  wird ein Fazit  gezogen, um die gemachten  Erfahrungen im
n\"achsten Projekt verbessern zu k\"onnen.

% ---------------------------------------------------------------------------- %
\section{Erster Prototyp Sensor fr\"uh fertiggestellt}
\label{sec:firstSensorprototype}
% ---------------------------------------------------------------------------- %

Am  Anfang des  Projektes war  eine grosse  Unsicherheit vorhanden,  dass kein
Hardware-Teil entstehen  w\"urde, denn  im Team befand  sich nur  ein Mitglied
mit  Hardware-Erfahrung. Die  Motivation  war  dementsprechend  klein  und  es
musste  nach  einer  L\"osung  gesucht  werden. Nach  einer  Sitzung  mit  der
Projektbetreuerin  konnte  eine  L\"osung  f\"ur  die  fehlende  Erfahrung  im
Bereich  Hardware gefunden  werden. Das Team  erhielt noch  ein zus\"atzliches
Mitglied, welches \"uber die n\"otige  Erfahrung verf\"ugte.  Damit war bereit
ein  fr\"uhes Ereignis  erfolgreich  behandelt worden. Denn  der Zeitplan  war
\"ausserst ehrgeizig kalkuliert.

Gem\"ass   Planung  wurde   erwartet,  dass   beide  Hardware-Teile   (Sensor-
und    Masterplatine)    bis    zur   Projektwoche    fertiggestellt    werden
k\"onnten. Anschliessend  sollte   die  Validierung  und   die  dazugeh\"orige
Software  erarbeitet  werden. Mit  diesem   Zeitplan  war  gen\"ugend  Reserve
eingerechnet um allf\"allige Probleme aufzufangen.

Dank  der  guten  Arbeit  des  Hardware-Teams wurde  der  erste  Prototyp  der
Sensorplatine noch vor dem  geplanten Zeitpunkt fertiggestellt. Das ermunterte
das ganze  Team und zeigte, dass  etwas erreicht werden konnte  mit gen\"ugend
Einsatz.

Dank der M\"oglichkeit  die Sensorplatine in einem  privaten Bastelraum selbst
herzustellen,  konnte  rund  eine Woche  Wartezeit  weggelassen  werden. Diese
Wartezeit w\"are  entstanden, wenn das  Printed Circuit Board (PCB)  bei einem
Hersteller im Ausland hergestellt worden w\"are. So konnte dieses Arbeitspaket
viel fr\"uher als erwartet abgeschlossen werden.

F\"ur   folgende   Projekte  soll   mitgenommen   werden,   dass  nach   allen
M\"oglichkeiten  gesucht werden  soll, um  Zeit einzusparen. Die  Zeit ist  im
Projekt  die  knappste Ressource  und  meistens  liegt  ein Scheitern  an  der
Zeitknappheit.

% ---------------------------------------------------------------------------- %
\section{Fachbericht planm\"assig}
\label{sec:reportAsPlanned}
% ---------------------------------------------------------------------------- %

Der  Fachbericht ist  beim  Projekt  4 ein  zentrales  Element, welches  stark
gewichtet wird. Aus diesem Grund war  es wichtig den Fachbericht w\"ahrend dem
gesamten Projekt  immer im Hinterkopf  zu behalten. Die Erfahrung  zeigt, dass
der Fachbericht  immer in letzter Minute  fertig gestellt wird oder  sogar nur
einen Teil seiner zwingenden Inhalte besitzt.

Im  Team   wurde  f\"ur   diese  wichtige   Aufgabe  ein   Chef  Dokumentation
ernannt. Damit  konnte  sichergestellt  werden,  dass eine  Person  f\"ur  den
Bericht   verantwortlich  war. Die   Planung  der   Dokumentation  konnte   so
zuverl\"assig  erstellt  werden  und   allf\"allige  Probleme  wurden  schnell
erkannt.   Die  zust\"andige  Person  war   sehr  kompetent  und  mahnte  alle
Teammitglieder regelm\"assig,  ihre Teile  vom Fachbericht rechtzeitig  und in
geforderter Qualit\"at zusammenzustellen.

Das Resultat  dieses Vorgehens war  schnell zu sehen. Der Fachbericht  nahm ab
der Projektwoche Formen an, was gem\"ass Planung zu erwarten war. So kam keine
Hektik  beim Schreiben  auf und  ein qualitativ  hochwertiges Dokument  konnte
zusammengestellt werden. Es war zum Projektende \"ubersichtlich, vollst\"andig
und vor allem sch\"on anzusehen.

Der  Fachbericht  ist   bei  jedem  Projekt  ein   zentrales  Element. In  ihm
werden  alle wichtigen  Erkenntnisse zusammengefasst. Ebenfalls  enth\"alt der
Fachbericht  auch  alle  Entscheidungen,   welche  im  Verlauf  des  Projektes
getroffen wurden. So kann nachvollzogen werden, welche Entscheidungen zum Ziel
f\"uhrten und welche nicht. Im Falle eines  Scheiterns kann das Projekt ab der
letzten  richtigen  Entscheidung  fortgesetzt  und doch  noch  zu  einem  Ziel
gef\"uhrt  werden. Darum war  die  Entscheidung, einen  Chef Dokumentation  zu
ernennen,  richtig und  sollte bei  weiteren Projekten  auf jeden  Fall wieder
getroffen werden.

% ---------------------------------------------------------------------------- %
\section{Bestellung der Printplatine muss storniert werden}
\label{sec:orderCancelled}
% ---------------------------------------------------------------------------- %

Das  Ereignis  trat  in  der  drittletzten Woche  des  Semesters  auf. Es  war
beabsichtigt  worden die  finalen Platinen  bei einem  Platinen Hersteller  zu
bestellen,  damit professionelle  Platinen  best\"uckt werden  konnten. Leider
wurde schlussendlich  entschieden, diese  Bestellung nicht  auszul\"osen, weil
die  verbleibende  Zeit   nicht  ausgereicht  h\"atte,  um   die  Platinen  zu
best\"ucken und zu testen.

Es  war  geplant,  die  zwei  Platinen  vom  Sensor  und  vom  Master  in  der
zehnten Semesterwoche  zu bestellen. Damit  h\"atte die Zeit  ausgereicht alle
Bauelemente auf die Platine zu  l\"oten und anschliessend ausf\"uhrliche Tests
durchzuf\"uhren. Dieses  Vorhaben verz\"ogerte  sich  um zwei  Wochen, da  der
Zeitaufwand f\"ur das Platinen-Design h\"oher als erwartet war.

Die Abweichung vom  Zeitplan waren zu sp\"at  erkannt worden. Das Arbeitspaket
Platinen-Layout war einerseits  zu knapp bemessen, was auf  Grund der geringen
Projekterfahrung  des Teams  zur\"uckzuf\"uhren  war. Zus\"atzlich wurde  erst
am  eigentliche  Bestelldatum  darauf  aufmerksam gemacht,  dass  die  Platine
noch  nicht  fertig  designt  worden   war,  um  bestellt  zu  werden. Nachdem
rund  eineinhalb  Wochen  sp\"ater  als geplant  das  Design  fertig  gestellt
wurde, wurde  die Bestellung  via Ansprechperson  f\"ur Bestellungen  der FHNW
ausgel\"ost. Leider wurde vom  Hersteller mitgeteilt, dass es  ein Problem mit
der  zugestellten Datei  gab. Diese Mitteilung  wurde im  Team nicht  nach der
Vereinbarung der internen Kommunikation weitergegeben, sodass ein weiterer Tag
verging bis etwas unternommen werden konnte. Zu diesem Zeitpunkt war das halbe
Hardwareteam unerwartet  krank, was eine schnelle  L\"osung verhinderte. Durch
das  gemeinsame Auftreten  der  verschiedenen Probleme,  musste eine  radikale
Entscheidung  getroffen werden. Im  Plenum wurden  schlussendlich Entschieden,
dass   die  Bestellung   der  Platinen   storniert  wird. Damit   konnte  eine
\"Uberbelastung des  Hardware-Teams in  den letzten  Semesterwochen verhindert
werden und der Fokus konnte auf die \"ubrigen Aufgaben gelegt werden.

Dank  dieser Entscheidung  konnten  alle  verbleibenden Arbeiten  sorgf\"altig
erledigt werden. Es wurde  nicht auf Kosten anderer  Arbeiten ein halbherziges
Produkt zusammengebastelt. Die Entscheidung war auf jeden Fall richtig. Da sie
vom ganzen  Team unterst\"utzt  wurde, arbeiteten  alle motiviert  weiter, was
nicht  selbstverst\"andlich war. Es  wurden viele  Stunden in  das Design  der
Platinen investiert, welche leider nicht hergestellt werden konnten.

Was bleibt  ist die Erfahrung,  wie in solchen Situationen  vorgegangen werden
soll. Tritt  ein  kleines  Problem  auf, kann  meist  eine  schnelle  L\"osung
gefunden werden. Treten jedoch mehrere  gravierende Probleme am gleichen Punkt
auf und  herrscht zus\"atzlich  Zeitnot, so  ist eine  gesamthafte Beurteilung
gefragt. Die L\"osung  muss auf jeden  Fall realisierbar sein und  darf keinen
negativen Einfluss auf andere Arbeiten  haben. Nicht, dass am Ende das gesamte
Projekt  in Mitleidenschaft  gezogen wird,  weil  nur ein  kleiner Teil  nicht
planm\"assig funktioniert.


% ---------------------------------------------------------------------------- %
\section{Software lange im Verzug}
\label{sec:softwareBehind}
% ---------------------------------------------------------------------------- %

Der Softwareteil  war lange  Zeit in Verzug. Die  einzelnen Teile  waren nicht
rechtzeitig fertiggestellt worden und oder nur mangelhaft funktionsf\"ahig. Es
konnte  festgestellt  werden,  dass   nicht  gen\"ugend  Fachwissen  vorhanden
war. Wie im  Risikomanagement aufgef\"uhrt, waren die  Auswirkungen erheblich.
Viele folgende Arbeiten verz\"ogerten sich oder konnten nicht erledigt werden.

Urspr\"unglich  waren  die  Arbeiten  an der  Software  vor  der  Projektwoche
geplant. Auf  Antrag  des  gesamten  Teams  wurden  die  Softwarearbeiten  auf
die  Projektwoche   und  die   darauffolgenden  Wochen   verschoben. Denn  die
Evaluation  der Bauelemente,  respektive des  Mikrocontrollers war  noch nicht
abgeschlossen. Dieser  war   die  Basis  der   Software. Diese  Plan\"anderung
war  kurz  nach  Beginn  des  Projektes  entstanden  und  zu  jenem  Zeitpunkt
zu  verantworten. Leider  konnten die  Arbeiten  nicht  in gew\"unschter  Zeit
abgeschlossen  werden, welches  zwar  fr\"uhzeitig erkannt  wurde, aber  nicht
behoben werden konnte.

Die   Ursache  dieses   Ereignisses   ist  auf   den   Anfang  des   Projektes
zur\"uckzuf\"uhren. Es wurde am Anfang  definiert welche Komponenten verwendet
werden und in welcher Programmiersprache die einzelnen Teilsysteme geschrieben
werden. Durch  die  grosse  Erfahrung  von  zwei  Teammitglieder  mit  Python,
wurde diese  Programmiersprache f\"ur  das Masterger\"at  gew\"ahlt. Zudem war
ein  Raspberry Pi  als  zentrales Element  f\"ur  das Masterger\"at  gew\"ahlt
worden,  aus  den  gleichen  Gr\"unden  wie  die  Programmiersprache. Es  soll
erw\"ahnt werden,  dass auf Raspberry  Pi die  Sprache Python sehr  einfach zu
implementieren war. Diese Entscheidung war zu dieser Zeit einleuchtend. Leider
musste  schnell  festgestellt werden,  dass  es  sehr zeitintensiv  war,  eine
komplett neue Programmiersprache zu  erlernen und damit ein funktionsf\"ahiges
Programm zu schreiben.

Die  fehlende  Erfahrung  konnte  nur schwer  behoben  werden. Die  erfahrenen
Mitglieder hatten auch ihre eigenen  Arbeiten zu erledigen und waren ebenfalls
unter Zeitdruck. Dadurch entstand eine immer gr\"osser werdende Verz\"ogerung,
die  nur  mit  sehr  viel   Zeitaufwand  der  Beteiligten  verkleinert  werden
konnte. Dies  ben\"otigte unn\"otig  viel Zeit  und h\"atte  verhindert werden
k\"onnen.

Als  Erfahrung   f\"ur  das  n\"achste   Projekt  soll  auf   bekannte  Mittel
zur\"uckgegriffen werden. Es  wird viel  zu viel  Zeit ben\"otigt  neue Mittel
(z.B.  eine  neue Programmiersprache,  ein  neues  Programm usw.)  ausreichend
effizient  zu verwenden. Bekanntes  aus  dem Unterricht  ist  meist mit  wenig
Aufwand wieder verf\"ugbar und vor allem kann einfach beim Fachdozenten um Rat
gefragt werden. Die Idee  vom Projekt ist es, das erlernte  Anzuwenden und bei
Unklarheiten nachzufragen. Es soll nicht das Rad neu erfunden werden, denn das
ist enorm zeitintensiv.
