\subsection{Ausgangslage}
Mit   dem  Thema   Regelungstechn ik  ist  jede   Person  t\"aglich   indirekt
konfrontiert. Beispielsweise beginnt f\"ur die  meisten Menschen jeder Tag mit
einer  warmen  Dusche  und  einer  Tasse Kaffee. Beides  w\"are  ohne  die  im
Hintergrund  ablaufende  Regelung  bei  Dusche und  Kaffemaschine  nur  schwer
m\"oglich. Ein Kaffee ist nur geniessbar wenn das Gleichgewicht aus Temperatur
und  Br\"uhzeit  (gesteuert \"uber  die  Durchflussgeschwindigkeit  bzw.  den
Pumpendruck) genau stimmt.
Um solche Regelungsprozesse zuverlässig realisieren zu können, ist eine genaue
Dimensionierung der Regler notwendig. Für dies werden oft mathemathische
oder graphische Methoden zu Hilfe genommen.

Ziel  dieses  Projektes  ist  es  ein  m\"oglichst  benutzerfreundliches  Tool
f\"ur Regelungstechniker zu realisieren,  welches anhand der Phasengang-Methode
(siehe Abschnitt  Theoretische Grundlagen) die bestmöglichen  Parameter f\"ur eine
Reglerdimensionierung berechnet. Dabei soll  von der PTn-Schrittantwort (siehe
Abschnitt Theoretische Grundlagen) der Regelstrecke ausgegangen werden.  F\"ur
Vergleichswerte  werden bekannte  Faustformeln verwendet. Anschliessend  folgt
eine Simulation der Schrittantwort des geschlossenen Regelkreises, wodurch die
Reglerdimensionierung kontrolliert und verbessert werden kann.

%\begin{center}
%    \begin{figure}[h! width=\pagewidth]
%		%\hspace{-1in}
%		\includegraphics[width=\textwidth,clip=true,trim=22mm 40mm 18mm 10mm]{images/zieltabelle.pdf}
%        \captionof{table}{Zieltabelle}
%    \end{figure}
%\end{center}

\clearpage
\subsection{Projektziele}

\clearpage
\subsection{Lieferobjekte}

\begin{itemize}
    \item{18.02.2015}: Projektauftrag
    \item{25.03.2015}: Pflichtenheft
    \item{01.04.2015}: Statusbericht 1
    \item{29.04.2015}: Statusbericht 2
    \item{13.05.2015}: Statusbericht 3
    \item{03.06.2015}: Statusbericht 4
    \item{10.06.2015}: Fachbericht und PA-Bericht
\end{itemize}

\clearpage
\subsection{Rahmenbedingungen}
