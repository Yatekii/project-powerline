% **************************************************************************** %
\chapter{Software}
\label{chap:software}
% **************************************************************************** %

An  dieser  Stelle wird  genauer  erl\"autert,  welche \"Uberlegungen  in  die
Firmware  unseres  Systems  geflossen  sind. Weil  Sensor  und  Master-Ger\"at
jeweils  eine   eigene  Firmware   ben\"otigen,  ist  auch   dieser  Abschnitt
entsprechend aufgeteilt.

\anweisung   Diagramme    zur   Erkl\"arung    des   Aufbaus    der   Firmware
verwenden. Trockener    Text    \"uber    Software   ist    nicht    besonders
leserfreundlich.

\anweisung Bei der Verwendung von  Libraries sollen diese kurz beschrieben und
der Grund ihrer Wahl erl\"autert werden.

\anweisung   Unterkapitel  entsprechend   Gliederung  der   Firmware  sinnvoll
erg\"anzen.


% ---------------------------------------------------------------------------- %
\section{Firmware \Sensor}
\label{sec:firmware:sensor}
% ---------------------------------------------------------------------------- %

% ---------------------------------------------------------------------------- %
\section{Software \Master}
\label{sec:software:master}
% ---------------------------------------------------------------------------- %

Es wird  im Folgenden auf  die Software des \Master~  genauer eingegangen. Die
verwendeten  Komponenten  werden beschrieben  und  es  wird auf  die  Lizenzen
ebendieser  Komponenten eingegangen. Anschliessend  werden der  Aufbau unseres
Software-Stacks und die Funktionsprinzipien dokumentiert.


% ---------------------------------------------------------------------------- %
\subsection{Komponenten}
\label{subsec:software:master:components}
% ---------------------------------------------------------------------------- %

Als   Betriebssystem   kommt   Raspbian   zum   Einsatz. Raspbian   ist   eine
Linux-Distribution,  welche auf  Debian  \todo{link auf  website} aufbaut  und
Erweiterungen f\"ur  zus\"atzliche Display-Funktionalit\"at f\"ur  den \Raspi~
besitzt.

Unsere eigene Software basiert auf  Python3 und QT5. In Python3 stehen diverse
Bibliotheken  \todo{welche?} zur  Verf\"ugung,  welche  es uns  erm\"oglichen,
das  Rad  nicht   neu  erfinden  zu  m\"ussen. Sattdessen   k\"onnen  wir  uns
darauf  konzentrieren, die  eigentliche Funktionalit\"at  unseres Ger\"ats  zu
implementieren.

\todo{layering diagramm}


% ---------------------------------------------------------------------------- %
\subsection{Lizenzen}
\label{subsec:software:master:licenses}
% ---------------------------------------------------------------------------- %


% ---------------------------------------------------------------------------- %
\section{Kommunikation/Protokoll}
\label{sec:fw:sensorplatine}
% ---------------------------------------------------------------------------- %

Wie im Abschnitt  zur Hardware wird in einem  separaten Abschnitt beschrieben,
welche \"Uberlegungen in die Entwicklung  des Protokolls eingeflossen sind und
wie das Ergebnis aussieht.
