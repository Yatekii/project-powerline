%Zu erf\"ullende Aufgaben:
%
%\begin{itemize}
%    \item
%        Kommunikation Sensoren <-> Hub
%        \begin{itemize}
%            \item
%                Establishing contact
%            \item
%                \"ubertragung von messdaten
%            \item
%                fehlerhandhabung
%            \item
%                addressieren/verwalten
%        \end{itemize}
%    \item
%        Hub
%        \begin{itemize}
%            \item
%                Sammeln von messdaten (anfordern oder push?)
%            \item
%                speichern von messdaten
%            \item
%                auswerten von gespeicherten messdaten: definition von failure modes, statistische auswertung der moduldaten, adaptiv?
%            \item
%                kommunikation des Zustands der Anlage nach aussen (relay, LEDs, SMS, etc.)
%        \end{itemize}
%    \item
%        sensoren
%        \begin{itemize}
%            \item
%                sammeln von daten
%            \item
%                failure handling
%            \item
%                senden von daten
%        \end{itemize}
%\end{itemize}

Da die Hardware  aus zwei einzelnen Boards besteht, wird  nat\"urlich auch pro
Board eine separate Software ben\"otigt.

Der  in der  Sensorplatine  verbaute  Atmel-Chip wird  in  C programmiert,  da
sich  diese  Programmiersprache  durch   die  Hardwaren\"ahe  optimal  daf\"ur
eignet. Die Aufgabe der  Software besteht hier darin, die  gemessenen Daten zu
verarbeiten, falls ben\"otigt Pr\"ufsummen zu berechnen sowie die zu sendenden
Daten  als  Pakete  bereitzustellen, um  sie  auf  die  DC-Leitung  modulieren
zu  k\"onnen. Zus\"atzlich muss  zur  Identifikation  beim Masterger\"at  eine
Sensorspezifische  Nummer  eingelesen  und ebenfalls  \"uber  die  Netzleitung
kommuniziert werden k\"onnen.

Die  Programmierung  des  Master-Ger\"ates  erfolgt in  Python,  da  hier  die
zur  Verf\"ugung  stehenden Libraries  besser  mit  dem verwendeten  Raspberry
Pi  kombiniert  werden k\"onnen. Dabei  ist  die  Software einerseits  daf\"ur
zust\"andig  die  von den  Sensoren  \"ubermittelten,  demodulierten Daten  zu
erfassen und  zu verarbeiten. Dabei m\"ussen die  erhaltenen Daten verglichen,
eventuel  defekte Module  evaluiert, Fehler  ber\"ucksichtigt sowie  bei einem
erkannten  Defekt dies  gemeldet  werden. Andererseits  m\"ussen die  Sensoren
anhand  der  \"ubermittelten   Erkennungsnummern  den  einzelnen  Solarmodulen
zugewiesen werden,  damit die Messwerte korrekt  miteinander verglichen werden
k\"onnen.
