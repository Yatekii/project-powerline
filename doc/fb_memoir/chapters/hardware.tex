% Plot from tabular data
%\begin{figure}
%    \begin{tikzpicture}
%       \begin{scope}[x={(0mm,0mm)},y={(170mm,\textwidth)}]
%           \begin{axis}[%
%                   xmode=log,
%                   height=80mm,
%                   width=\textwidth,
%                   at={(0,80mm)},
%                   %grid=both,
%               ]
%               \addplot[-,color=blue] table {data/measurement-inductor-1layer-inductance.dat};
%               %\addplot[-,color=blue] table {data/fetsim.dat};
%           \end{axis}
%           %\begin{axis}[%
%           %        height=80mm,
%           %        width=\textwidth,
%           %        at={(0,0mm)},
%           %        %grid=both,
%           %    ]
%           %    \addplot[-,color=red] table {data/fetsim.dat};
%           %\end{axis}
%       \end{scope}
%   \end{tikzpicture}
%\end{figure}

% **************************************************************************** %
\chapter{Hardware}
\label{chap:hardware}
% **************************************************************************** %

Das folgende Kapitel dokumentiert  unseren L\"osungsfindungsprozess im Bereich
Hardware. Zuerst  wird  auf  die  Problematik  der  Kommunikation  \"uber  die
DC-Leitung  zwischen  Solarmodul  und \Master  eingegangen. Es  werden  unsere
L\"osungsans\"atze erl\"autert  und auf St\"arken und  Schw\"achen untersucht,
um daraus anschliessend eine Entscheidung abzuleiten.

In   einem  n\"achsten   Schritt  werden   die  L\"osungskonzepte   f\"ur  die
Sensor-Platine und das \Master genauer erkl\"art.

\anweisung Verifizierbare  Angaben machen, nicht  allgemeine Floskeln. Zahlen,
Fakten,  theoretische Hintergrundinformationen,  daraus gezogene  Konsequenzen
etc. Bezug auf das Pflichtenheft, wo m\"oglich.

\anweisung Bilder, Diagramme, Formeln etc.


% ---------------------------------------------------------------------------- %
\section{Kommunikation \"uber DC-Leitung}
\label{sec:hardware:dcLeitung}
% ---------------------------------------------------------------------------- %

W\"ahrend  des   Projekts  wurden   zwei  L\"osungsans\"atze   untersucht,  um
ein  Signal  \"uber   die  Gleichstromleitung  vom  \Sensor   zum  \Master  zu
\"ubertragen. Der  Gleichstromanteil   bel\"auft  sich  in  der   Leitung  auf
ca. \SI{960}{\volt} \todo{Berechnung/Begr\"undung}.

\myfancybreak

\emph{Frequency-shift    keying}   (\emph{Frequenzumtastung}): Dem    in   der
Leitung    fliessenden    Gleichstrom     wird    ein    (verh\"altnism\"assig
kleines\todo{Amplitudenverh\"altnisse})    Wechselstromsignal    aufmoduliert,
welches   die   zu   \"ubertragenden   Informationen   enth\"alt. Dabei   wird
die    Frequenz   dieses    Wechselstromsignals    in   diskreten    Schritten
variiert,  \todo{korrekt?}   und  diese   \"Anderung  der  Frequenz   ist  der
Informationstr\"ager.  F\"ur einen kurzen \"Uberblick  zum Thema sie an dieser
Stelle auf Wikipedia verwiesen~\cite{ref:fsk:wikipedia}.

Das    Verfahren    ist   schematisch    in    Abbildung~\ref{fig:fsk:concept}
dargestellt.\todo{Achsen: Einheiten}

\begin{figure}[h!tb]
    \centering
    %% Creator: Matplotlib, PGF backend
%%
%% To include the figure in your LaTeX document, write
%%   \input{<filename>.pgf}
%%
%% Make sure the required packages are loaded in your preamble
%%   \usepackage{pgf}
%%
%% Figures using additional raster images can only be included by \input if
%% they are in the same directory as the main LaTeX file. For loading figures
%% from other directories you can use the `import` package
%%   \usepackage{import}
%% and then include the figures with
%%   \import{<path to file>}{<filename>.pgf}
%%
%% Matplotlib used the following preamble
%%   \usepackage{fontspec}
%%   \setmainfont{Bitstream Vera Serif}
%%   \setsansfont{Bitstream Vera Sans}
%%   \setmonofont{Bitstream Vera Sans Mono}
%%
\begingroup%
\makeatletter%
\begin{pgfpicture}%
\pgfpathrectangle{\pgfpointorigin}{\pgfqpoint{4.500000in}{3.500000in}}%
\pgfusepath{use as bounding box, clip}%
\begin{pgfscope}%
\pgfsetbuttcap%
\pgfsetmiterjoin%
\pgfsetlinewidth{0.000000pt}%
\definecolor{currentstroke}{rgb}{0.000000,0.000000,0.000000}%
\pgfsetstrokecolor{currentstroke}%
\pgfsetstrokeopacity{0.000000}%
\pgfsetdash{}{0pt}%
\pgfpathmoveto{\pgfqpoint{0.000000in}{0.000000in}}%
\pgfpathlineto{\pgfqpoint{4.500000in}{0.000000in}}%
\pgfpathlineto{\pgfqpoint{4.500000in}{3.500000in}}%
\pgfpathlineto{\pgfqpoint{0.000000in}{3.500000in}}%
\pgfpathclose%
\pgfusepath{}%
\end{pgfscope}%
\begin{pgfscope}%
\pgfsetbuttcap%
\pgfsetmiterjoin%
\pgfsetlinewidth{0.000000pt}%
\definecolor{currentstroke}{rgb}{0.000000,0.000000,0.000000}%
\pgfsetstrokecolor{currentstroke}%
\pgfsetstrokeopacity{0.000000}%
\pgfsetdash{}{0pt}%
\pgfpathmoveto{\pgfqpoint{0.225000in}{2.065000in}}%
\pgfpathlineto{\pgfqpoint{4.275000in}{2.065000in}}%
\pgfpathlineto{\pgfqpoint{4.275000in}{3.325000in}}%
\pgfpathlineto{\pgfqpoint{0.225000in}{3.325000in}}%
\pgfpathclose%
\pgfusepath{}%
\end{pgfscope}%
\begin{pgfscope}%
\pgfpathrectangle{\pgfqpoint{0.225000in}{2.065000in}}{\pgfqpoint{4.050000in}{1.260000in}} %
\pgfusepath{clip}%
\pgfsetrectcap%
\pgfsetroundjoin%
\pgfsetlinewidth{1.003750pt}%
\definecolor{currentstroke}{rgb}{0.000000,0.000000,1.000000}%
\pgfsetstrokecolor{currentstroke}%
\pgfsetdash{}{0pt}%
\pgfpathmoveto{\pgfqpoint{0.225000in}{2.170000in}}%
\pgfpathlineto{\pgfqpoint{1.188898in}{2.170000in}}%
\pgfusepath{stroke}%
\end{pgfscope}%
\begin{pgfscope}%
\pgfpathrectangle{\pgfqpoint{0.225000in}{2.065000in}}{\pgfqpoint{4.050000in}{1.260000in}} %
\pgfusepath{clip}%
\pgfsetrectcap%
\pgfsetroundjoin%
\pgfsetlinewidth{1.003750pt}%
\definecolor{currentstroke}{rgb}{0.501961,0.501961,0.501961}%
\pgfsetstrokecolor{currentstroke}%
\pgfsetdash{}{0pt}%
\pgfpathmoveto{\pgfqpoint{1.188898in}{2.170000in}}%
\pgfpathlineto{\pgfqpoint{1.188898in}{3.220000in}}%
\pgfusepath{stroke}%
\end{pgfscope}%
\begin{pgfscope}%
\pgfpathrectangle{\pgfqpoint{0.225000in}{2.065000in}}{\pgfqpoint{4.050000in}{1.260000in}} %
\pgfusepath{clip}%
\pgfsetrectcap%
\pgfsetroundjoin%
\pgfsetlinewidth{1.003750pt}%
\definecolor{currentstroke}{rgb}{1.000000,0.000000,1.000000}%
\pgfsetstrokecolor{currentstroke}%
\pgfsetdash{}{0pt}%
\pgfpathmoveto{\pgfqpoint{1.188898in}{3.220000in}}%
\pgfpathlineto{\pgfqpoint{2.152795in}{3.220000in}}%
\pgfusepath{stroke}%
\end{pgfscope}%
\begin{pgfscope}%
\pgfpathrectangle{\pgfqpoint{0.225000in}{2.065000in}}{\pgfqpoint{4.050000in}{1.260000in}} %
\pgfusepath{clip}%
\pgfsetrectcap%
\pgfsetroundjoin%
\pgfsetlinewidth{1.003750pt}%
\definecolor{currentstroke}{rgb}{0.501961,0.501961,0.501961}%
\pgfsetstrokecolor{currentstroke}%
\pgfsetdash{}{0pt}%
\pgfpathmoveto{\pgfqpoint{2.152795in}{3.220000in}}%
\pgfpathlineto{\pgfqpoint{2.152795in}{2.170000in}}%
\pgfusepath{stroke}%
\end{pgfscope}%
\begin{pgfscope}%
\pgfpathrectangle{\pgfqpoint{0.225000in}{2.065000in}}{\pgfqpoint{4.050000in}{1.260000in}} %
\pgfusepath{clip}%
\pgfsetrectcap%
\pgfsetroundjoin%
\pgfsetlinewidth{1.003750pt}%
\definecolor{currentstroke}{rgb}{0.000000,0.000000,1.000000}%
\pgfsetstrokecolor{currentstroke}%
\pgfsetdash{}{0pt}%
\pgfpathmoveto{\pgfqpoint{2.152795in}{2.170000in}}%
\pgfpathlineto{\pgfqpoint{3.116693in}{2.170000in}}%
\pgfusepath{stroke}%
\end{pgfscope}%
\begin{pgfscope}%
\pgfpathrectangle{\pgfqpoint{0.225000in}{2.065000in}}{\pgfqpoint{4.050000in}{1.260000in}} %
\pgfusepath{clip}%
\pgfsetrectcap%
\pgfsetroundjoin%
\pgfsetlinewidth{1.003750pt}%
\definecolor{currentstroke}{rgb}{0.501961,0.501961,0.501961}%
\pgfsetstrokecolor{currentstroke}%
\pgfsetdash{}{0pt}%
\pgfpathmoveto{\pgfqpoint{3.116693in}{2.170000in}}%
\pgfpathlineto{\pgfqpoint{3.116693in}{3.220000in}}%
\pgfusepath{stroke}%
\end{pgfscope}%
\begin{pgfscope}%
\pgfpathrectangle{\pgfqpoint{0.225000in}{2.065000in}}{\pgfqpoint{4.050000in}{1.260000in}} %
\pgfusepath{clip}%
\pgfsetrectcap%
\pgfsetroundjoin%
\pgfsetlinewidth{1.003750pt}%
\definecolor{currentstroke}{rgb}{1.000000,0.000000,1.000000}%
\pgfsetstrokecolor{currentstroke}%
\pgfsetdash{}{0pt}%
\pgfpathmoveto{\pgfqpoint{3.116693in}{3.220000in}}%
\pgfpathlineto{\pgfqpoint{4.080591in}{3.220000in}}%
\pgfusepath{stroke}%
\end{pgfscope}%
\begin{pgfscope}%
\pgfsetrectcap%
\pgfsetmiterjoin%
\pgfsetlinewidth{1.003750pt}%
\definecolor{currentstroke}{rgb}{0.000000,0.000000,0.000000}%
\pgfsetstrokecolor{currentstroke}%
\pgfsetdash{}{0pt}%
\pgfpathmoveto{\pgfqpoint{4.275000in}{2.065000in}}%
\pgfpathlineto{\pgfqpoint{4.275000in}{3.325000in}}%
\pgfusepath{stroke}%
\end{pgfscope}%
\begin{pgfscope}%
\pgfsetrectcap%
\pgfsetmiterjoin%
\pgfsetlinewidth{1.003750pt}%
\definecolor{currentstroke}{rgb}{0.000000,0.000000,0.000000}%
\pgfsetstrokecolor{currentstroke}%
\pgfsetdash{}{0pt}%
\pgfpathmoveto{\pgfqpoint{0.225000in}{2.065000in}}%
\pgfpathlineto{\pgfqpoint{4.275000in}{2.065000in}}%
\pgfusepath{stroke}%
\end{pgfscope}%
\begin{pgfscope}%
\pgfsetrectcap%
\pgfsetmiterjoin%
\pgfsetlinewidth{1.003750pt}%
\definecolor{currentstroke}{rgb}{0.000000,0.000000,0.000000}%
\pgfsetstrokecolor{currentstroke}%
\pgfsetdash{}{0pt}%
\pgfpathmoveto{\pgfqpoint{0.225000in}{3.325000in}}%
\pgfpathlineto{\pgfqpoint{4.275000in}{3.325000in}}%
\pgfusepath{stroke}%
\end{pgfscope}%
\begin{pgfscope}%
\pgfsetrectcap%
\pgfsetmiterjoin%
\pgfsetlinewidth{1.003750pt}%
\definecolor{currentstroke}{rgb}{0.000000,0.000000,0.000000}%
\pgfsetstrokecolor{currentstroke}%
\pgfsetdash{}{0pt}%
\pgfpathmoveto{\pgfqpoint{0.225000in}{2.065000in}}%
\pgfpathlineto{\pgfqpoint{0.225000in}{3.325000in}}%
\pgfusepath{stroke}%
\end{pgfscope}%
\begin{pgfscope}%
\pgfsetbuttcap%
\pgfsetroundjoin%
\definecolor{currentfill}{rgb}{0.000000,0.000000,0.000000}%
\pgfsetfillcolor{currentfill}%
\pgfsetlinewidth{0.501875pt}%
\definecolor{currentstroke}{rgb}{0.000000,0.000000,0.000000}%
\pgfsetstrokecolor{currentstroke}%
\pgfsetdash{}{0pt}%
\pgfsys@defobject{currentmarker}{\pgfqpoint{0.000000in}{0.000000in}}{\pgfqpoint{0.000000in}{0.055556in}}{%
\pgfpathmoveto{\pgfqpoint{0.000000in}{0.000000in}}%
\pgfpathlineto{\pgfqpoint{0.000000in}{0.055556in}}%
\pgfusepath{stroke,fill}%
}%
\begin{pgfscope}%
\pgfsys@transformshift{0.225000in}{2.065000in}%
\pgfsys@useobject{currentmarker}{}%
\end{pgfscope}%
\end{pgfscope}%
\begin{pgfscope}%
\pgfsetbuttcap%
\pgfsetroundjoin%
\definecolor{currentfill}{rgb}{0.000000,0.000000,0.000000}%
\pgfsetfillcolor{currentfill}%
\pgfsetlinewidth{0.501875pt}%
\definecolor{currentstroke}{rgb}{0.000000,0.000000,0.000000}%
\pgfsetstrokecolor{currentstroke}%
\pgfsetdash{}{0pt}%
\pgfsys@defobject{currentmarker}{\pgfqpoint{0.000000in}{-0.055556in}}{\pgfqpoint{0.000000in}{0.000000in}}{%
\pgfpathmoveto{\pgfqpoint{0.000000in}{0.000000in}}%
\pgfpathlineto{\pgfqpoint{0.000000in}{-0.055556in}}%
\pgfusepath{stroke,fill}%
}%
\begin{pgfscope}%
\pgfsys@transformshift{0.225000in}{3.325000in}%
\pgfsys@useobject{currentmarker}{}%
\end{pgfscope}%
\end{pgfscope}%
\begin{pgfscope}%
\pgftext[x=0.225000in,y=2.009444in,,top]{\rmfamily\fontsize{9.000000}{10.800000}\selectfont \(\displaystyle 0.0000\)}%
\end{pgfscope}%
\begin{pgfscope}%
\pgfsetbuttcap%
\pgfsetroundjoin%
\definecolor{currentfill}{rgb}{0.000000,0.000000,0.000000}%
\pgfsetfillcolor{currentfill}%
\pgfsetlinewidth{0.501875pt}%
\definecolor{currentstroke}{rgb}{0.000000,0.000000,0.000000}%
\pgfsetstrokecolor{currentstroke}%
\pgfsetdash{}{0pt}%
\pgfsys@defobject{currentmarker}{\pgfqpoint{0.000000in}{0.000000in}}{\pgfqpoint{0.000000in}{0.055556in}}{%
\pgfpathmoveto{\pgfqpoint{0.000000in}{0.000000in}}%
\pgfpathlineto{\pgfqpoint{0.000000in}{0.055556in}}%
\pgfusepath{stroke,fill}%
}%
\begin{pgfscope}%
\pgfsys@transformshift{0.731250in}{2.065000in}%
\pgfsys@useobject{currentmarker}{}%
\end{pgfscope}%
\end{pgfscope}%
\begin{pgfscope}%
\pgfsetbuttcap%
\pgfsetroundjoin%
\definecolor{currentfill}{rgb}{0.000000,0.000000,0.000000}%
\pgfsetfillcolor{currentfill}%
\pgfsetlinewidth{0.501875pt}%
\definecolor{currentstroke}{rgb}{0.000000,0.000000,0.000000}%
\pgfsetstrokecolor{currentstroke}%
\pgfsetdash{}{0pt}%
\pgfsys@defobject{currentmarker}{\pgfqpoint{0.000000in}{-0.055556in}}{\pgfqpoint{0.000000in}{0.000000in}}{%
\pgfpathmoveto{\pgfqpoint{0.000000in}{0.000000in}}%
\pgfpathlineto{\pgfqpoint{0.000000in}{-0.055556in}}%
\pgfusepath{stroke,fill}%
}%
\begin{pgfscope}%
\pgfsys@transformshift{0.731250in}{3.325000in}%
\pgfsys@useobject{currentmarker}{}%
\end{pgfscope}%
\end{pgfscope}%
\begin{pgfscope}%
\pgftext[x=0.731250in,y=2.009444in,,top]{\rmfamily\fontsize{9.000000}{10.800000}\selectfont \(\displaystyle 0.0002\)}%
\end{pgfscope}%
\begin{pgfscope}%
\pgfsetbuttcap%
\pgfsetroundjoin%
\definecolor{currentfill}{rgb}{0.000000,0.000000,0.000000}%
\pgfsetfillcolor{currentfill}%
\pgfsetlinewidth{0.501875pt}%
\definecolor{currentstroke}{rgb}{0.000000,0.000000,0.000000}%
\pgfsetstrokecolor{currentstroke}%
\pgfsetdash{}{0pt}%
\pgfsys@defobject{currentmarker}{\pgfqpoint{0.000000in}{0.000000in}}{\pgfqpoint{0.000000in}{0.055556in}}{%
\pgfpathmoveto{\pgfqpoint{0.000000in}{0.000000in}}%
\pgfpathlineto{\pgfqpoint{0.000000in}{0.055556in}}%
\pgfusepath{stroke,fill}%
}%
\begin{pgfscope}%
\pgfsys@transformshift{1.237500in}{2.065000in}%
\pgfsys@useobject{currentmarker}{}%
\end{pgfscope}%
\end{pgfscope}%
\begin{pgfscope}%
\pgfsetbuttcap%
\pgfsetroundjoin%
\definecolor{currentfill}{rgb}{0.000000,0.000000,0.000000}%
\pgfsetfillcolor{currentfill}%
\pgfsetlinewidth{0.501875pt}%
\definecolor{currentstroke}{rgb}{0.000000,0.000000,0.000000}%
\pgfsetstrokecolor{currentstroke}%
\pgfsetdash{}{0pt}%
\pgfsys@defobject{currentmarker}{\pgfqpoint{0.000000in}{-0.055556in}}{\pgfqpoint{0.000000in}{0.000000in}}{%
\pgfpathmoveto{\pgfqpoint{0.000000in}{0.000000in}}%
\pgfpathlineto{\pgfqpoint{0.000000in}{-0.055556in}}%
\pgfusepath{stroke,fill}%
}%
\begin{pgfscope}%
\pgfsys@transformshift{1.237500in}{3.325000in}%
\pgfsys@useobject{currentmarker}{}%
\end{pgfscope}%
\end{pgfscope}%
\begin{pgfscope}%
\pgftext[x=1.237500in,y=2.009444in,,top]{\rmfamily\fontsize{9.000000}{10.800000}\selectfont \(\displaystyle 0.0004\)}%
\end{pgfscope}%
\begin{pgfscope}%
\pgfsetbuttcap%
\pgfsetroundjoin%
\definecolor{currentfill}{rgb}{0.000000,0.000000,0.000000}%
\pgfsetfillcolor{currentfill}%
\pgfsetlinewidth{0.501875pt}%
\definecolor{currentstroke}{rgb}{0.000000,0.000000,0.000000}%
\pgfsetstrokecolor{currentstroke}%
\pgfsetdash{}{0pt}%
\pgfsys@defobject{currentmarker}{\pgfqpoint{0.000000in}{0.000000in}}{\pgfqpoint{0.000000in}{0.055556in}}{%
\pgfpathmoveto{\pgfqpoint{0.000000in}{0.000000in}}%
\pgfpathlineto{\pgfqpoint{0.000000in}{0.055556in}}%
\pgfusepath{stroke,fill}%
}%
\begin{pgfscope}%
\pgfsys@transformshift{1.743750in}{2.065000in}%
\pgfsys@useobject{currentmarker}{}%
\end{pgfscope}%
\end{pgfscope}%
\begin{pgfscope}%
\pgfsetbuttcap%
\pgfsetroundjoin%
\definecolor{currentfill}{rgb}{0.000000,0.000000,0.000000}%
\pgfsetfillcolor{currentfill}%
\pgfsetlinewidth{0.501875pt}%
\definecolor{currentstroke}{rgb}{0.000000,0.000000,0.000000}%
\pgfsetstrokecolor{currentstroke}%
\pgfsetdash{}{0pt}%
\pgfsys@defobject{currentmarker}{\pgfqpoint{0.000000in}{-0.055556in}}{\pgfqpoint{0.000000in}{0.000000in}}{%
\pgfpathmoveto{\pgfqpoint{0.000000in}{0.000000in}}%
\pgfpathlineto{\pgfqpoint{0.000000in}{-0.055556in}}%
\pgfusepath{stroke,fill}%
}%
\begin{pgfscope}%
\pgfsys@transformshift{1.743750in}{3.325000in}%
\pgfsys@useobject{currentmarker}{}%
\end{pgfscope}%
\end{pgfscope}%
\begin{pgfscope}%
\pgftext[x=1.743750in,y=2.009444in,,top]{\rmfamily\fontsize{9.000000}{10.800000}\selectfont \(\displaystyle 0.0006\)}%
\end{pgfscope}%
\begin{pgfscope}%
\pgfsetbuttcap%
\pgfsetroundjoin%
\definecolor{currentfill}{rgb}{0.000000,0.000000,0.000000}%
\pgfsetfillcolor{currentfill}%
\pgfsetlinewidth{0.501875pt}%
\definecolor{currentstroke}{rgb}{0.000000,0.000000,0.000000}%
\pgfsetstrokecolor{currentstroke}%
\pgfsetdash{}{0pt}%
\pgfsys@defobject{currentmarker}{\pgfqpoint{0.000000in}{0.000000in}}{\pgfqpoint{0.000000in}{0.055556in}}{%
\pgfpathmoveto{\pgfqpoint{0.000000in}{0.000000in}}%
\pgfpathlineto{\pgfqpoint{0.000000in}{0.055556in}}%
\pgfusepath{stroke,fill}%
}%
\begin{pgfscope}%
\pgfsys@transformshift{2.250000in}{2.065000in}%
\pgfsys@useobject{currentmarker}{}%
\end{pgfscope}%
\end{pgfscope}%
\begin{pgfscope}%
\pgfsetbuttcap%
\pgfsetroundjoin%
\definecolor{currentfill}{rgb}{0.000000,0.000000,0.000000}%
\pgfsetfillcolor{currentfill}%
\pgfsetlinewidth{0.501875pt}%
\definecolor{currentstroke}{rgb}{0.000000,0.000000,0.000000}%
\pgfsetstrokecolor{currentstroke}%
\pgfsetdash{}{0pt}%
\pgfsys@defobject{currentmarker}{\pgfqpoint{0.000000in}{-0.055556in}}{\pgfqpoint{0.000000in}{0.000000in}}{%
\pgfpathmoveto{\pgfqpoint{0.000000in}{0.000000in}}%
\pgfpathlineto{\pgfqpoint{0.000000in}{-0.055556in}}%
\pgfusepath{stroke,fill}%
}%
\begin{pgfscope}%
\pgfsys@transformshift{2.250000in}{3.325000in}%
\pgfsys@useobject{currentmarker}{}%
\end{pgfscope}%
\end{pgfscope}%
\begin{pgfscope}%
\pgftext[x=2.250000in,y=2.009444in,,top]{\rmfamily\fontsize{9.000000}{10.800000}\selectfont \(\displaystyle 0.0008\)}%
\end{pgfscope}%
\begin{pgfscope}%
\pgfsetbuttcap%
\pgfsetroundjoin%
\definecolor{currentfill}{rgb}{0.000000,0.000000,0.000000}%
\pgfsetfillcolor{currentfill}%
\pgfsetlinewidth{0.501875pt}%
\definecolor{currentstroke}{rgb}{0.000000,0.000000,0.000000}%
\pgfsetstrokecolor{currentstroke}%
\pgfsetdash{}{0pt}%
\pgfsys@defobject{currentmarker}{\pgfqpoint{0.000000in}{0.000000in}}{\pgfqpoint{0.000000in}{0.055556in}}{%
\pgfpathmoveto{\pgfqpoint{0.000000in}{0.000000in}}%
\pgfpathlineto{\pgfqpoint{0.000000in}{0.055556in}}%
\pgfusepath{stroke,fill}%
}%
\begin{pgfscope}%
\pgfsys@transformshift{2.756250in}{2.065000in}%
\pgfsys@useobject{currentmarker}{}%
\end{pgfscope}%
\end{pgfscope}%
\begin{pgfscope}%
\pgfsetbuttcap%
\pgfsetroundjoin%
\definecolor{currentfill}{rgb}{0.000000,0.000000,0.000000}%
\pgfsetfillcolor{currentfill}%
\pgfsetlinewidth{0.501875pt}%
\definecolor{currentstroke}{rgb}{0.000000,0.000000,0.000000}%
\pgfsetstrokecolor{currentstroke}%
\pgfsetdash{}{0pt}%
\pgfsys@defobject{currentmarker}{\pgfqpoint{0.000000in}{-0.055556in}}{\pgfqpoint{0.000000in}{0.000000in}}{%
\pgfpathmoveto{\pgfqpoint{0.000000in}{0.000000in}}%
\pgfpathlineto{\pgfqpoint{0.000000in}{-0.055556in}}%
\pgfusepath{stroke,fill}%
}%
\begin{pgfscope}%
\pgfsys@transformshift{2.756250in}{3.325000in}%
\pgfsys@useobject{currentmarker}{}%
\end{pgfscope}%
\end{pgfscope}%
\begin{pgfscope}%
\pgftext[x=2.756250in,y=2.009444in,,top]{\rmfamily\fontsize{9.000000}{10.800000}\selectfont \(\displaystyle 0.0010\)}%
\end{pgfscope}%
\begin{pgfscope}%
\pgfsetbuttcap%
\pgfsetroundjoin%
\definecolor{currentfill}{rgb}{0.000000,0.000000,0.000000}%
\pgfsetfillcolor{currentfill}%
\pgfsetlinewidth{0.501875pt}%
\definecolor{currentstroke}{rgb}{0.000000,0.000000,0.000000}%
\pgfsetstrokecolor{currentstroke}%
\pgfsetdash{}{0pt}%
\pgfsys@defobject{currentmarker}{\pgfqpoint{0.000000in}{0.000000in}}{\pgfqpoint{0.000000in}{0.055556in}}{%
\pgfpathmoveto{\pgfqpoint{0.000000in}{0.000000in}}%
\pgfpathlineto{\pgfqpoint{0.000000in}{0.055556in}}%
\pgfusepath{stroke,fill}%
}%
\begin{pgfscope}%
\pgfsys@transformshift{3.262500in}{2.065000in}%
\pgfsys@useobject{currentmarker}{}%
\end{pgfscope}%
\end{pgfscope}%
\begin{pgfscope}%
\pgfsetbuttcap%
\pgfsetroundjoin%
\definecolor{currentfill}{rgb}{0.000000,0.000000,0.000000}%
\pgfsetfillcolor{currentfill}%
\pgfsetlinewidth{0.501875pt}%
\definecolor{currentstroke}{rgb}{0.000000,0.000000,0.000000}%
\pgfsetstrokecolor{currentstroke}%
\pgfsetdash{}{0pt}%
\pgfsys@defobject{currentmarker}{\pgfqpoint{0.000000in}{-0.055556in}}{\pgfqpoint{0.000000in}{0.000000in}}{%
\pgfpathmoveto{\pgfqpoint{0.000000in}{0.000000in}}%
\pgfpathlineto{\pgfqpoint{0.000000in}{-0.055556in}}%
\pgfusepath{stroke,fill}%
}%
\begin{pgfscope}%
\pgfsys@transformshift{3.262500in}{3.325000in}%
\pgfsys@useobject{currentmarker}{}%
\end{pgfscope}%
\end{pgfscope}%
\begin{pgfscope}%
\pgftext[x=3.262500in,y=2.009444in,,top]{\rmfamily\fontsize{9.000000}{10.800000}\selectfont \(\displaystyle 0.0012\)}%
\end{pgfscope}%
\begin{pgfscope}%
\pgfsetbuttcap%
\pgfsetroundjoin%
\definecolor{currentfill}{rgb}{0.000000,0.000000,0.000000}%
\pgfsetfillcolor{currentfill}%
\pgfsetlinewidth{0.501875pt}%
\definecolor{currentstroke}{rgb}{0.000000,0.000000,0.000000}%
\pgfsetstrokecolor{currentstroke}%
\pgfsetdash{}{0pt}%
\pgfsys@defobject{currentmarker}{\pgfqpoint{0.000000in}{0.000000in}}{\pgfqpoint{0.000000in}{0.055556in}}{%
\pgfpathmoveto{\pgfqpoint{0.000000in}{0.000000in}}%
\pgfpathlineto{\pgfqpoint{0.000000in}{0.055556in}}%
\pgfusepath{stroke,fill}%
}%
\begin{pgfscope}%
\pgfsys@transformshift{3.768750in}{2.065000in}%
\pgfsys@useobject{currentmarker}{}%
\end{pgfscope}%
\end{pgfscope}%
\begin{pgfscope}%
\pgfsetbuttcap%
\pgfsetroundjoin%
\definecolor{currentfill}{rgb}{0.000000,0.000000,0.000000}%
\pgfsetfillcolor{currentfill}%
\pgfsetlinewidth{0.501875pt}%
\definecolor{currentstroke}{rgb}{0.000000,0.000000,0.000000}%
\pgfsetstrokecolor{currentstroke}%
\pgfsetdash{}{0pt}%
\pgfsys@defobject{currentmarker}{\pgfqpoint{0.000000in}{-0.055556in}}{\pgfqpoint{0.000000in}{0.000000in}}{%
\pgfpathmoveto{\pgfqpoint{0.000000in}{0.000000in}}%
\pgfpathlineto{\pgfqpoint{0.000000in}{-0.055556in}}%
\pgfusepath{stroke,fill}%
}%
\begin{pgfscope}%
\pgfsys@transformshift{3.768750in}{3.325000in}%
\pgfsys@useobject{currentmarker}{}%
\end{pgfscope}%
\end{pgfscope}%
\begin{pgfscope}%
\pgftext[x=3.768750in,y=2.009444in,,top]{\rmfamily\fontsize{9.000000}{10.800000}\selectfont \(\displaystyle 0.0014\)}%
\end{pgfscope}%
\begin{pgfscope}%
\pgfsetbuttcap%
\pgfsetroundjoin%
\definecolor{currentfill}{rgb}{0.000000,0.000000,0.000000}%
\pgfsetfillcolor{currentfill}%
\pgfsetlinewidth{0.501875pt}%
\definecolor{currentstroke}{rgb}{0.000000,0.000000,0.000000}%
\pgfsetstrokecolor{currentstroke}%
\pgfsetdash{}{0pt}%
\pgfsys@defobject{currentmarker}{\pgfqpoint{0.000000in}{0.000000in}}{\pgfqpoint{0.000000in}{0.055556in}}{%
\pgfpathmoveto{\pgfqpoint{0.000000in}{0.000000in}}%
\pgfpathlineto{\pgfqpoint{0.000000in}{0.055556in}}%
\pgfusepath{stroke,fill}%
}%
\begin{pgfscope}%
\pgfsys@transformshift{4.275000in}{2.065000in}%
\pgfsys@useobject{currentmarker}{}%
\end{pgfscope}%
\end{pgfscope}%
\begin{pgfscope}%
\pgfsetbuttcap%
\pgfsetroundjoin%
\definecolor{currentfill}{rgb}{0.000000,0.000000,0.000000}%
\pgfsetfillcolor{currentfill}%
\pgfsetlinewidth{0.501875pt}%
\definecolor{currentstroke}{rgb}{0.000000,0.000000,0.000000}%
\pgfsetstrokecolor{currentstroke}%
\pgfsetdash{}{0pt}%
\pgfsys@defobject{currentmarker}{\pgfqpoint{0.000000in}{-0.055556in}}{\pgfqpoint{0.000000in}{0.000000in}}{%
\pgfpathmoveto{\pgfqpoint{0.000000in}{0.000000in}}%
\pgfpathlineto{\pgfqpoint{0.000000in}{-0.055556in}}%
\pgfusepath{stroke,fill}%
}%
\begin{pgfscope}%
\pgfsys@transformshift{4.275000in}{3.325000in}%
\pgfsys@useobject{currentmarker}{}%
\end{pgfscope}%
\end{pgfscope}%
\begin{pgfscope}%
\pgftext[x=4.275000in,y=2.009444in,,top]{\rmfamily\fontsize{9.000000}{10.800000}\selectfont \(\displaystyle 0.0016\)}%
\end{pgfscope}%
\begin{pgfscope}%
\pgfsetbuttcap%
\pgfsetroundjoin%
\definecolor{currentfill}{rgb}{0.000000,0.000000,0.000000}%
\pgfsetfillcolor{currentfill}%
\pgfsetlinewidth{0.501875pt}%
\definecolor{currentstroke}{rgb}{0.000000,0.000000,0.000000}%
\pgfsetstrokecolor{currentstroke}%
\pgfsetdash{}{0pt}%
\pgfsys@defobject{currentmarker}{\pgfqpoint{0.000000in}{0.000000in}}{\pgfqpoint{0.055556in}{0.000000in}}{%
\pgfpathmoveto{\pgfqpoint{0.000000in}{0.000000in}}%
\pgfpathlineto{\pgfqpoint{0.055556in}{0.000000in}}%
\pgfusepath{stroke,fill}%
}%
\begin{pgfscope}%
\pgfsys@transformshift{0.225000in}{2.170000in}%
\pgfsys@useobject{currentmarker}{}%
\end{pgfscope}%
\end{pgfscope}%
\begin{pgfscope}%
\pgfsetbuttcap%
\pgfsetroundjoin%
\definecolor{currentfill}{rgb}{0.000000,0.000000,0.000000}%
\pgfsetfillcolor{currentfill}%
\pgfsetlinewidth{0.501875pt}%
\definecolor{currentstroke}{rgb}{0.000000,0.000000,0.000000}%
\pgfsetstrokecolor{currentstroke}%
\pgfsetdash{}{0pt}%
\pgfsys@defobject{currentmarker}{\pgfqpoint{-0.055556in}{0.000000in}}{\pgfqpoint{0.000000in}{0.000000in}}{%
\pgfpathmoveto{\pgfqpoint{0.000000in}{0.000000in}}%
\pgfpathlineto{\pgfqpoint{-0.055556in}{0.000000in}}%
\pgfusepath{stroke,fill}%
}%
\begin{pgfscope}%
\pgfsys@transformshift{4.275000in}{2.170000in}%
\pgfsys@useobject{currentmarker}{}%
\end{pgfscope}%
\end{pgfscope}%
\begin{pgfscope}%
\pgftext[x=0.169444in,y=2.170000in,right,]{\rmfamily\fontsize{9.000000}{10.800000}\selectfont \(\displaystyle 0.0\)}%
\end{pgfscope}%
\begin{pgfscope}%
\pgfsetbuttcap%
\pgfsetroundjoin%
\definecolor{currentfill}{rgb}{0.000000,0.000000,0.000000}%
\pgfsetfillcolor{currentfill}%
\pgfsetlinewidth{0.501875pt}%
\definecolor{currentstroke}{rgb}{0.000000,0.000000,0.000000}%
\pgfsetstrokecolor{currentstroke}%
\pgfsetdash{}{0pt}%
\pgfsys@defobject{currentmarker}{\pgfqpoint{0.000000in}{0.000000in}}{\pgfqpoint{0.055556in}{0.000000in}}{%
\pgfpathmoveto{\pgfqpoint{0.000000in}{0.000000in}}%
\pgfpathlineto{\pgfqpoint{0.055556in}{0.000000in}}%
\pgfusepath{stroke,fill}%
}%
\begin{pgfscope}%
\pgfsys@transformshift{0.225000in}{2.380000in}%
\pgfsys@useobject{currentmarker}{}%
\end{pgfscope}%
\end{pgfscope}%
\begin{pgfscope}%
\pgfsetbuttcap%
\pgfsetroundjoin%
\definecolor{currentfill}{rgb}{0.000000,0.000000,0.000000}%
\pgfsetfillcolor{currentfill}%
\pgfsetlinewidth{0.501875pt}%
\definecolor{currentstroke}{rgb}{0.000000,0.000000,0.000000}%
\pgfsetstrokecolor{currentstroke}%
\pgfsetdash{}{0pt}%
\pgfsys@defobject{currentmarker}{\pgfqpoint{-0.055556in}{0.000000in}}{\pgfqpoint{0.000000in}{0.000000in}}{%
\pgfpathmoveto{\pgfqpoint{0.000000in}{0.000000in}}%
\pgfpathlineto{\pgfqpoint{-0.055556in}{0.000000in}}%
\pgfusepath{stroke,fill}%
}%
\begin{pgfscope}%
\pgfsys@transformshift{4.275000in}{2.380000in}%
\pgfsys@useobject{currentmarker}{}%
\end{pgfscope}%
\end{pgfscope}%
\begin{pgfscope}%
\pgftext[x=0.169444in,y=2.380000in,right,]{\rmfamily\fontsize{9.000000}{10.800000}\selectfont \(\displaystyle 0.2\)}%
\end{pgfscope}%
\begin{pgfscope}%
\pgfsetbuttcap%
\pgfsetroundjoin%
\definecolor{currentfill}{rgb}{0.000000,0.000000,0.000000}%
\pgfsetfillcolor{currentfill}%
\pgfsetlinewidth{0.501875pt}%
\definecolor{currentstroke}{rgb}{0.000000,0.000000,0.000000}%
\pgfsetstrokecolor{currentstroke}%
\pgfsetdash{}{0pt}%
\pgfsys@defobject{currentmarker}{\pgfqpoint{0.000000in}{0.000000in}}{\pgfqpoint{0.055556in}{0.000000in}}{%
\pgfpathmoveto{\pgfqpoint{0.000000in}{0.000000in}}%
\pgfpathlineto{\pgfqpoint{0.055556in}{0.000000in}}%
\pgfusepath{stroke,fill}%
}%
\begin{pgfscope}%
\pgfsys@transformshift{0.225000in}{2.590000in}%
\pgfsys@useobject{currentmarker}{}%
\end{pgfscope}%
\end{pgfscope}%
\begin{pgfscope}%
\pgfsetbuttcap%
\pgfsetroundjoin%
\definecolor{currentfill}{rgb}{0.000000,0.000000,0.000000}%
\pgfsetfillcolor{currentfill}%
\pgfsetlinewidth{0.501875pt}%
\definecolor{currentstroke}{rgb}{0.000000,0.000000,0.000000}%
\pgfsetstrokecolor{currentstroke}%
\pgfsetdash{}{0pt}%
\pgfsys@defobject{currentmarker}{\pgfqpoint{-0.055556in}{0.000000in}}{\pgfqpoint{0.000000in}{0.000000in}}{%
\pgfpathmoveto{\pgfqpoint{0.000000in}{0.000000in}}%
\pgfpathlineto{\pgfqpoint{-0.055556in}{0.000000in}}%
\pgfusepath{stroke,fill}%
}%
\begin{pgfscope}%
\pgfsys@transformshift{4.275000in}{2.590000in}%
\pgfsys@useobject{currentmarker}{}%
\end{pgfscope}%
\end{pgfscope}%
\begin{pgfscope}%
\pgftext[x=0.169444in,y=2.590000in,right,]{\rmfamily\fontsize{9.000000}{10.800000}\selectfont \(\displaystyle 0.4\)}%
\end{pgfscope}%
\begin{pgfscope}%
\pgfsetbuttcap%
\pgfsetroundjoin%
\definecolor{currentfill}{rgb}{0.000000,0.000000,0.000000}%
\pgfsetfillcolor{currentfill}%
\pgfsetlinewidth{0.501875pt}%
\definecolor{currentstroke}{rgb}{0.000000,0.000000,0.000000}%
\pgfsetstrokecolor{currentstroke}%
\pgfsetdash{}{0pt}%
\pgfsys@defobject{currentmarker}{\pgfqpoint{0.000000in}{0.000000in}}{\pgfqpoint{0.055556in}{0.000000in}}{%
\pgfpathmoveto{\pgfqpoint{0.000000in}{0.000000in}}%
\pgfpathlineto{\pgfqpoint{0.055556in}{0.000000in}}%
\pgfusepath{stroke,fill}%
}%
\begin{pgfscope}%
\pgfsys@transformshift{0.225000in}{2.800000in}%
\pgfsys@useobject{currentmarker}{}%
\end{pgfscope}%
\end{pgfscope}%
\begin{pgfscope}%
\pgfsetbuttcap%
\pgfsetroundjoin%
\definecolor{currentfill}{rgb}{0.000000,0.000000,0.000000}%
\pgfsetfillcolor{currentfill}%
\pgfsetlinewidth{0.501875pt}%
\definecolor{currentstroke}{rgb}{0.000000,0.000000,0.000000}%
\pgfsetstrokecolor{currentstroke}%
\pgfsetdash{}{0pt}%
\pgfsys@defobject{currentmarker}{\pgfqpoint{-0.055556in}{0.000000in}}{\pgfqpoint{0.000000in}{0.000000in}}{%
\pgfpathmoveto{\pgfqpoint{0.000000in}{0.000000in}}%
\pgfpathlineto{\pgfqpoint{-0.055556in}{0.000000in}}%
\pgfusepath{stroke,fill}%
}%
\begin{pgfscope}%
\pgfsys@transformshift{4.275000in}{2.800000in}%
\pgfsys@useobject{currentmarker}{}%
\end{pgfscope}%
\end{pgfscope}%
\begin{pgfscope}%
\pgftext[x=0.169444in,y=2.800000in,right,]{\rmfamily\fontsize{9.000000}{10.800000}\selectfont \(\displaystyle 0.6\)}%
\end{pgfscope}%
\begin{pgfscope}%
\pgfsetbuttcap%
\pgfsetroundjoin%
\definecolor{currentfill}{rgb}{0.000000,0.000000,0.000000}%
\pgfsetfillcolor{currentfill}%
\pgfsetlinewidth{0.501875pt}%
\definecolor{currentstroke}{rgb}{0.000000,0.000000,0.000000}%
\pgfsetstrokecolor{currentstroke}%
\pgfsetdash{}{0pt}%
\pgfsys@defobject{currentmarker}{\pgfqpoint{0.000000in}{0.000000in}}{\pgfqpoint{0.055556in}{0.000000in}}{%
\pgfpathmoveto{\pgfqpoint{0.000000in}{0.000000in}}%
\pgfpathlineto{\pgfqpoint{0.055556in}{0.000000in}}%
\pgfusepath{stroke,fill}%
}%
\begin{pgfscope}%
\pgfsys@transformshift{0.225000in}{3.010000in}%
\pgfsys@useobject{currentmarker}{}%
\end{pgfscope}%
\end{pgfscope}%
\begin{pgfscope}%
\pgfsetbuttcap%
\pgfsetroundjoin%
\definecolor{currentfill}{rgb}{0.000000,0.000000,0.000000}%
\pgfsetfillcolor{currentfill}%
\pgfsetlinewidth{0.501875pt}%
\definecolor{currentstroke}{rgb}{0.000000,0.000000,0.000000}%
\pgfsetstrokecolor{currentstroke}%
\pgfsetdash{}{0pt}%
\pgfsys@defobject{currentmarker}{\pgfqpoint{-0.055556in}{0.000000in}}{\pgfqpoint{0.000000in}{0.000000in}}{%
\pgfpathmoveto{\pgfqpoint{0.000000in}{0.000000in}}%
\pgfpathlineto{\pgfqpoint{-0.055556in}{0.000000in}}%
\pgfusepath{stroke,fill}%
}%
\begin{pgfscope}%
\pgfsys@transformshift{4.275000in}{3.010000in}%
\pgfsys@useobject{currentmarker}{}%
\end{pgfscope}%
\end{pgfscope}%
\begin{pgfscope}%
\pgftext[x=0.169444in,y=3.010000in,right,]{\rmfamily\fontsize{9.000000}{10.800000}\selectfont \(\displaystyle 0.8\)}%
\end{pgfscope}%
\begin{pgfscope}%
\pgfsetbuttcap%
\pgfsetroundjoin%
\definecolor{currentfill}{rgb}{0.000000,0.000000,0.000000}%
\pgfsetfillcolor{currentfill}%
\pgfsetlinewidth{0.501875pt}%
\definecolor{currentstroke}{rgb}{0.000000,0.000000,0.000000}%
\pgfsetstrokecolor{currentstroke}%
\pgfsetdash{}{0pt}%
\pgfsys@defobject{currentmarker}{\pgfqpoint{0.000000in}{0.000000in}}{\pgfqpoint{0.055556in}{0.000000in}}{%
\pgfpathmoveto{\pgfqpoint{0.000000in}{0.000000in}}%
\pgfpathlineto{\pgfqpoint{0.055556in}{0.000000in}}%
\pgfusepath{stroke,fill}%
}%
\begin{pgfscope}%
\pgfsys@transformshift{0.225000in}{3.220000in}%
\pgfsys@useobject{currentmarker}{}%
\end{pgfscope}%
\end{pgfscope}%
\begin{pgfscope}%
\pgfsetbuttcap%
\pgfsetroundjoin%
\definecolor{currentfill}{rgb}{0.000000,0.000000,0.000000}%
\pgfsetfillcolor{currentfill}%
\pgfsetlinewidth{0.501875pt}%
\definecolor{currentstroke}{rgb}{0.000000,0.000000,0.000000}%
\pgfsetstrokecolor{currentstroke}%
\pgfsetdash{}{0pt}%
\pgfsys@defobject{currentmarker}{\pgfqpoint{-0.055556in}{0.000000in}}{\pgfqpoint{0.000000in}{0.000000in}}{%
\pgfpathmoveto{\pgfqpoint{0.000000in}{0.000000in}}%
\pgfpathlineto{\pgfqpoint{-0.055556in}{0.000000in}}%
\pgfusepath{stroke,fill}%
}%
\begin{pgfscope}%
\pgfsys@transformshift{4.275000in}{3.220000in}%
\pgfsys@useobject{currentmarker}{}%
\end{pgfscope}%
\end{pgfscope}%
\begin{pgfscope}%
\pgftext[x=0.169444in,y=3.220000in,right,]{\rmfamily\fontsize{9.000000}{10.800000}\selectfont \(\displaystyle 1.0\)}%
\end{pgfscope}%
\begin{pgfscope}%
\pgftext[x=2.250000in,y=3.394444in,,base]{\rmfamily\fontsize{11.000000}{13.200000}\selectfont Daten}%
\end{pgfscope}%
\begin{pgfscope}%
\pgfsetbuttcap%
\pgfsetmiterjoin%
\pgfsetlinewidth{0.000000pt}%
\definecolor{currentstroke}{rgb}{0.000000,0.000000,0.000000}%
\pgfsetstrokecolor{currentstroke}%
\pgfsetstrokeopacity{0.000000}%
\pgfsetdash{}{0pt}%
\pgfpathmoveto{\pgfqpoint{0.225000in}{0.175000in}}%
\pgfpathlineto{\pgfqpoint{4.275000in}{0.175000in}}%
\pgfpathlineto{\pgfqpoint{4.275000in}{1.435000in}}%
\pgfpathlineto{\pgfqpoint{0.225000in}{1.435000in}}%
\pgfpathclose%
\pgfusepath{}%
\end{pgfscope}%
\begin{pgfscope}%
\pgfpathrectangle{\pgfqpoint{0.225000in}{0.175000in}}{\pgfqpoint{4.050000in}{1.260000in}} %
\pgfusepath{clip}%
\pgfsetrectcap%
\pgfsetroundjoin%
\pgfsetlinewidth{1.003750pt}%
\definecolor{currentstroke}{rgb}{0.000000,0.000000,1.000000}%
\pgfsetstrokecolor{currentstroke}%
\pgfsetdash{}{0pt}%
\pgfpathmoveto{\pgfqpoint{0.225000in}{0.232273in}}%
\pgfpathlineto{\pgfqpoint{0.229824in}{0.233405in}}%
\pgfpathlineto{\pgfqpoint{0.234649in}{0.236798in}}%
\pgfpathlineto{\pgfqpoint{0.240438in}{0.243833in}}%
\pgfpathlineto{\pgfqpoint{0.247192in}{0.256076in}}%
\pgfpathlineto{\pgfqpoint{0.254911in}{0.275268in}}%
\pgfpathlineto{\pgfqpoint{0.263595in}{0.303254in}}%
\pgfpathlineto{\pgfqpoint{0.273243in}{0.341866in}}%
\pgfpathlineto{\pgfqpoint{0.284821in}{0.397798in}}%
\pgfpathlineto{\pgfqpoint{0.298330in}{0.474655in}}%
\pgfpathlineto{\pgfqpoint{0.314732in}{0.581466in}}%
\pgfpathlineto{\pgfqpoint{0.337889in}{0.748359in}}%
\pgfpathlineto{\pgfqpoint{0.381308in}{1.062844in}}%
\pgfpathlineto{\pgfqpoint{0.398675in}{1.171068in}}%
\pgfpathlineto{\pgfqpoint{0.412183in}{1.242575in}}%
\pgfpathlineto{\pgfqpoint{0.423762in}{1.293166in}}%
\pgfpathlineto{\pgfqpoint{0.434375in}{1.329812in}}%
\pgfpathlineto{\pgfqpoint{0.443059in}{1.352358in}}%
\pgfpathlineto{\pgfqpoint{0.450778in}{1.366524in}}%
\pgfpathlineto{\pgfqpoint{0.457532in}{1.374262in}}%
\pgfpathlineto{\pgfqpoint{0.463321in}{1.377385in}}%
\pgfpathlineto{\pgfqpoint{0.468145in}{1.377498in}}%
\pgfpathlineto{\pgfqpoint{0.472970in}{1.375347in}}%
\pgfpathlineto{\pgfqpoint{0.478759in}{1.369791in}}%
\pgfpathlineto{\pgfqpoint{0.484548in}{1.361018in}}%
\pgfpathlineto{\pgfqpoint{0.491302in}{1.346788in}}%
\pgfpathlineto{\pgfqpoint{0.499021in}{1.325391in}}%
\pgfpathlineto{\pgfqpoint{0.508670in}{1.291272in}}%
\pgfpathlineto{\pgfqpoint{0.519283in}{1.244890in}}%
\pgfpathlineto{\pgfqpoint{0.531826in}{1.179314in}}%
\pgfpathlineto{\pgfqpoint{0.546299in}{1.091364in}}%
\pgfpathlineto{\pgfqpoint{0.565597in}{0.958879in}}%
\pgfpathlineto{\pgfqpoint{0.635067in}{0.465887in}}%
\pgfpathlineto{\pgfqpoint{0.649540in}{0.385336in}}%
\pgfpathlineto{\pgfqpoint{0.662083in}{0.327485in}}%
\pgfpathlineto{\pgfqpoint{0.672696in}{0.288434in}}%
\pgfpathlineto{\pgfqpoint{0.681380in}{0.263799in}}%
\pgfpathlineto{\pgfqpoint{0.689099in}{0.247711in}}%
\pgfpathlineto{\pgfqpoint{0.695853in}{0.238255in}}%
\pgfpathlineto{\pgfqpoint{0.701642in}{0.233643in}}%
\pgfpathlineto{\pgfqpoint{0.706466in}{0.232284in}}%
\pgfpathlineto{\pgfqpoint{0.711291in}{0.233190in}}%
\pgfpathlineto{\pgfqpoint{0.716115in}{0.236357in}}%
\pgfpathlineto{\pgfqpoint{0.721904in}{0.243124in}}%
\pgfpathlineto{\pgfqpoint{0.728658in}{0.255059in}}%
\pgfpathlineto{\pgfqpoint{0.736377in}{0.273909in}}%
\pgfpathlineto{\pgfqpoint{0.745061in}{0.301527in}}%
\pgfpathlineto{\pgfqpoint{0.754710in}{0.339756in}}%
\pgfpathlineto{\pgfqpoint{0.766288in}{0.395273in}}%
\pgfpathlineto{\pgfqpoint{0.779796in}{0.471719in}}%
\pgfpathlineto{\pgfqpoint{0.796199in}{0.578154in}}%
\pgfpathlineto{\pgfqpoint{0.819355in}{0.744775in}}%
\pgfpathlineto{\pgfqpoint{0.862774in}{1.059622in}}%
\pgfpathlineto{\pgfqpoint{0.880142in}{1.168291in}}%
\pgfpathlineto{\pgfqpoint{0.893650in}{1.240242in}}%
\pgfpathlineto{\pgfqpoint{0.905228in}{1.291272in}}%
\pgfpathlineto{\pgfqpoint{0.915842in}{1.328359in}}%
\pgfpathlineto{\pgfqpoint{0.924525in}{1.351287in}}%
\pgfpathlineto{\pgfqpoint{0.932244in}{1.365804in}}%
\pgfpathlineto{\pgfqpoint{0.938998in}{1.373855in}}%
\pgfpathlineto{\pgfqpoint{0.944788in}{1.377249in}}%
\pgfpathlineto{\pgfqpoint{0.949612in}{1.377589in}}%
\pgfpathlineto{\pgfqpoint{0.954436in}{1.375664in}}%
\pgfpathlineto{\pgfqpoint{0.959260in}{1.371483in}}%
\pgfpathlineto{\pgfqpoint{0.965050in}{1.363511in}}%
\pgfpathlineto{\pgfqpoint{0.971804in}{1.350195in}}%
\pgfpathlineto{\pgfqpoint{0.979523in}{1.329812in}}%
\pgfpathlineto{\pgfqpoint{0.988206in}{1.300545in}}%
\pgfpathlineto{\pgfqpoint{0.998820in}{1.256204in}}%
\pgfpathlineto{\pgfqpoint{1.010398in}{1.198029in}}%
\pgfpathlineto{\pgfqpoint{1.024871in}{1.112916in}}%
\pgfpathlineto{\pgfqpoint{1.042239in}{0.996647in}}%
\pgfpathlineto{\pgfqpoint{1.070220in}{0.791493in}}%
\pgfpathlineto{\pgfqpoint{1.102060in}{0.561731in}}%
\pgfpathlineto{\pgfqpoint{1.119428in}{0.451543in}}%
\pgfpathlineto{\pgfqpoint{1.133901in}{0.373289in}}%
\pgfpathlineto{\pgfqpoint{1.145479in}{0.321604in}}%
\pgfpathlineto{\pgfqpoint{1.156092in}{0.283859in}}%
\pgfpathlineto{\pgfqpoint{1.164776in}{0.260360in}}%
\pgfpathlineto{\pgfqpoint{1.172495in}{0.245318in}}%
\pgfpathlineto{\pgfqpoint{1.179249in}{0.236798in}}%
\pgfpathlineto{\pgfqpoint{1.185038in}{0.232998in}}%
\pgfpathlineto{\pgfqpoint{1.188898in}{0.232273in}}%
\pgfpathlineto{\pgfqpoint{1.188898in}{0.232273in}}%
\pgfusepath{stroke}%
\end{pgfscope}%
\begin{pgfscope}%
\pgfpathrectangle{\pgfqpoint{0.225000in}{0.175000in}}{\pgfqpoint{4.050000in}{1.260000in}} %
\pgfusepath{clip}%
\pgfsetrectcap%
\pgfsetroundjoin%
\pgfsetlinewidth{1.003750pt}%
\definecolor{currentstroke}{rgb}{1.000000,0.000000,1.000000}%
\pgfsetstrokecolor{currentstroke}%
\pgfsetdash{}{0pt}%
\pgfpathmoveto{\pgfqpoint{1.188898in}{0.232273in}}%
\pgfpathlineto{\pgfqpoint{1.190827in}{0.233903in}}%
\pgfpathlineto{\pgfqpoint{1.193722in}{0.242438in}}%
\pgfpathlineto{\pgfqpoint{1.197582in}{0.264988in}}%
\pgfpathlineto{\pgfqpoint{1.202406in}{0.310360in}}%
\pgfpathlineto{\pgfqpoint{1.208195in}{0.387796in}}%
\pgfpathlineto{\pgfqpoint{1.215914in}{0.523328in}}%
\pgfpathlineto{\pgfqpoint{1.228457in}{0.791493in}}%
\pgfpathlineto{\pgfqpoint{1.244860in}{1.136814in}}%
\pgfpathlineto{\pgfqpoint{1.253544in}{1.273375in}}%
\pgfpathlineto{\pgfqpoint{1.259333in}{1.335414in}}%
\pgfpathlineto{\pgfqpoint{1.264157in}{1.366524in}}%
\pgfpathlineto{\pgfqpoint{1.267052in}{1.375664in}}%
\pgfpathlineto{\pgfqpoint{1.268981in}{1.377702in}}%
\pgfpathlineto{\pgfqpoint{1.270911in}{1.376479in}}%
\pgfpathlineto{\pgfqpoint{1.272841in}{1.372002in}}%
\pgfpathlineto{\pgfqpoint{1.275735in}{1.359247in}}%
\pgfpathlineto{\pgfqpoint{1.279595in}{1.331243in}}%
\pgfpathlineto{\pgfqpoint{1.284419in}{1.279510in}}%
\pgfpathlineto{\pgfqpoint{1.291173in}{1.179314in}}%
\pgfpathlineto{\pgfqpoint{1.299857in}{1.013522in}}%
\pgfpathlineto{\pgfqpoint{1.331697in}{0.366265in}}%
\pgfpathlineto{\pgfqpoint{1.338451in}{0.285363in}}%
\pgfpathlineto{\pgfqpoint{1.343276in}{0.249416in}}%
\pgfpathlineto{\pgfqpoint{1.347135in}{0.234820in}}%
\pgfpathlineto{\pgfqpoint{1.349065in}{0.232375in}}%
\pgfpathlineto{\pgfqpoint{1.350995in}{0.233190in}}%
\pgfpathlineto{\pgfqpoint{1.352924in}{0.237261in}}%
\pgfpathlineto{\pgfqpoint{1.355819in}{0.249416in}}%
\pgfpathlineto{\pgfqpoint{1.359678in}{0.276648in}}%
\pgfpathlineto{\pgfqpoint{1.364503in}{0.327485in}}%
\pgfpathlineto{\pgfqpoint{1.371257in}{0.426614in}}%
\pgfpathlineto{\pgfqpoint{1.379941in}{0.591455in}}%
\pgfpathlineto{\pgfqpoint{1.411781in}{1.240242in}}%
\pgfpathlineto{\pgfqpoint{1.418535in}{1.322342in}}%
\pgfpathlineto{\pgfqpoint{1.423359in}{1.359247in}}%
\pgfpathlineto{\pgfqpoint{1.427219in}{1.374646in}}%
\pgfpathlineto{\pgfqpoint{1.429149in}{1.377498in}}%
\pgfpathlineto{\pgfqpoint{1.431078in}{1.377090in}}%
\pgfpathlineto{\pgfqpoint{1.433008in}{1.373425in}}%
\pgfpathlineto{\pgfqpoint{1.435903in}{1.361871in}}%
\pgfpathlineto{\pgfqpoint{1.439762in}{1.335414in}}%
\pgfpathlineto{\pgfqpoint{1.444586in}{1.285477in}}%
\pgfpathlineto{\pgfqpoint{1.451340in}{1.187426in}}%
\pgfpathlineto{\pgfqpoint{1.460024in}{1.023549in}}%
\pgfpathlineto{\pgfqpoint{1.492829in}{0.359398in}}%
\pgfpathlineto{\pgfqpoint{1.499584in}{0.280912in}}%
\pgfpathlineto{\pgfqpoint{1.504408in}{0.246891in}}%
\pgfpathlineto{\pgfqpoint{1.508267in}{0.233903in}}%
\pgfpathlineto{\pgfqpoint{1.510197in}{0.232273in}}%
\pgfpathlineto{\pgfqpoint{1.512127in}{0.233903in}}%
\pgfpathlineto{\pgfqpoint{1.515021in}{0.242438in}}%
\pgfpathlineto{\pgfqpoint{1.518881in}{0.264988in}}%
\pgfpathlineto{\pgfqpoint{1.523705in}{0.310360in}}%
\pgfpathlineto{\pgfqpoint{1.529494in}{0.387796in}}%
\pgfpathlineto{\pgfqpoint{1.537213in}{0.523328in}}%
\pgfpathlineto{\pgfqpoint{1.549756in}{0.791493in}}%
\pgfpathlineto{\pgfqpoint{1.566159in}{1.136814in}}%
\pgfpathlineto{\pgfqpoint{1.574843in}{1.273375in}}%
\pgfpathlineto{\pgfqpoint{1.580632in}{1.335414in}}%
\pgfpathlineto{\pgfqpoint{1.585456in}{1.366524in}}%
\pgfpathlineto{\pgfqpoint{1.588351in}{1.375664in}}%
\pgfpathlineto{\pgfqpoint{1.590281in}{1.377702in}}%
\pgfpathlineto{\pgfqpoint{1.592210in}{1.376479in}}%
\pgfpathlineto{\pgfqpoint{1.594140in}{1.372002in}}%
\pgfpathlineto{\pgfqpoint{1.597035in}{1.359247in}}%
\pgfpathlineto{\pgfqpoint{1.600894in}{1.331243in}}%
\pgfpathlineto{\pgfqpoint{1.605718in}{1.279510in}}%
\pgfpathlineto{\pgfqpoint{1.612472in}{1.179314in}}%
\pgfpathlineto{\pgfqpoint{1.621156in}{1.013522in}}%
\pgfpathlineto{\pgfqpoint{1.652997in}{0.366265in}}%
\pgfpathlineto{\pgfqpoint{1.659751in}{0.285363in}}%
\pgfpathlineto{\pgfqpoint{1.664575in}{0.249416in}}%
\pgfpathlineto{\pgfqpoint{1.668434in}{0.234820in}}%
\pgfpathlineto{\pgfqpoint{1.670364in}{0.232375in}}%
\pgfpathlineto{\pgfqpoint{1.672294in}{0.233190in}}%
\pgfpathlineto{\pgfqpoint{1.674224in}{0.237261in}}%
\pgfpathlineto{\pgfqpoint{1.677118in}{0.249416in}}%
\pgfpathlineto{\pgfqpoint{1.680978in}{0.276648in}}%
\pgfpathlineto{\pgfqpoint{1.685802in}{0.327485in}}%
\pgfpathlineto{\pgfqpoint{1.692556in}{0.426614in}}%
\pgfpathlineto{\pgfqpoint{1.701240in}{0.591455in}}%
\pgfpathlineto{\pgfqpoint{1.733080in}{1.240242in}}%
\pgfpathlineto{\pgfqpoint{1.739834in}{1.322342in}}%
\pgfpathlineto{\pgfqpoint{1.744659in}{1.359247in}}%
\pgfpathlineto{\pgfqpoint{1.748518in}{1.374646in}}%
\pgfpathlineto{\pgfqpoint{1.750448in}{1.377498in}}%
\pgfpathlineto{\pgfqpoint{1.752378in}{1.377090in}}%
\pgfpathlineto{\pgfqpoint{1.754307in}{1.373425in}}%
\pgfpathlineto{\pgfqpoint{1.757202in}{1.361871in}}%
\pgfpathlineto{\pgfqpoint{1.761061in}{1.335414in}}%
\pgfpathlineto{\pgfqpoint{1.765886in}{1.285477in}}%
\pgfpathlineto{\pgfqpoint{1.772640in}{1.187426in}}%
\pgfpathlineto{\pgfqpoint{1.781323in}{1.023549in}}%
\pgfpathlineto{\pgfqpoint{1.814129in}{0.359398in}}%
\pgfpathlineto{\pgfqpoint{1.820883in}{0.280912in}}%
\pgfpathlineto{\pgfqpoint{1.825707in}{0.246891in}}%
\pgfpathlineto{\pgfqpoint{1.829567in}{0.233903in}}%
\pgfpathlineto{\pgfqpoint{1.831496in}{0.232273in}}%
\pgfpathlineto{\pgfqpoint{1.833426in}{0.233903in}}%
\pgfpathlineto{\pgfqpoint{1.836321in}{0.242438in}}%
\pgfpathlineto{\pgfqpoint{1.840180in}{0.264988in}}%
\pgfpathlineto{\pgfqpoint{1.845004in}{0.310360in}}%
\pgfpathlineto{\pgfqpoint{1.850793in}{0.387796in}}%
\pgfpathlineto{\pgfqpoint{1.858512in}{0.523328in}}%
\pgfpathlineto{\pgfqpoint{1.871056in}{0.791493in}}%
\pgfpathlineto{\pgfqpoint{1.887458in}{1.136814in}}%
\pgfpathlineto{\pgfqpoint{1.896142in}{1.273375in}}%
\pgfpathlineto{\pgfqpoint{1.901931in}{1.335414in}}%
\pgfpathlineto{\pgfqpoint{1.906756in}{1.366524in}}%
\pgfpathlineto{\pgfqpoint{1.909650in}{1.375664in}}%
\pgfpathlineto{\pgfqpoint{1.911580in}{1.377702in}}%
\pgfpathlineto{\pgfqpoint{1.913510in}{1.376479in}}%
\pgfpathlineto{\pgfqpoint{1.915439in}{1.372002in}}%
\pgfpathlineto{\pgfqpoint{1.918334in}{1.359247in}}%
\pgfpathlineto{\pgfqpoint{1.922193in}{1.331243in}}%
\pgfpathlineto{\pgfqpoint{1.927018in}{1.279510in}}%
\pgfpathlineto{\pgfqpoint{1.933772in}{1.179314in}}%
\pgfpathlineto{\pgfqpoint{1.942455in}{1.013522in}}%
\pgfpathlineto{\pgfqpoint{1.974296in}{0.366265in}}%
\pgfpathlineto{\pgfqpoint{1.981050in}{0.285363in}}%
\pgfpathlineto{\pgfqpoint{1.985874in}{0.249416in}}%
\pgfpathlineto{\pgfqpoint{1.989734in}{0.234820in}}%
\pgfpathlineto{\pgfqpoint{1.991663in}{0.232375in}}%
\pgfpathlineto{\pgfqpoint{1.993593in}{0.233190in}}%
\pgfpathlineto{\pgfqpoint{1.995523in}{0.237261in}}%
\pgfpathlineto{\pgfqpoint{1.998417in}{0.249416in}}%
\pgfpathlineto{\pgfqpoint{2.002277in}{0.276648in}}%
\pgfpathlineto{\pgfqpoint{2.007101in}{0.327485in}}%
\pgfpathlineto{\pgfqpoint{2.013855in}{0.426614in}}%
\pgfpathlineto{\pgfqpoint{2.022539in}{0.591455in}}%
\pgfpathlineto{\pgfqpoint{2.054380in}{1.240242in}}%
\pgfpathlineto{\pgfqpoint{2.061134in}{1.322342in}}%
\pgfpathlineto{\pgfqpoint{2.065958in}{1.359247in}}%
\pgfpathlineto{\pgfqpoint{2.069817in}{1.374646in}}%
\pgfpathlineto{\pgfqpoint{2.071747in}{1.377498in}}%
\pgfpathlineto{\pgfqpoint{2.073677in}{1.377090in}}%
\pgfpathlineto{\pgfqpoint{2.075606in}{1.373425in}}%
\pgfpathlineto{\pgfqpoint{2.078501in}{1.361871in}}%
\pgfpathlineto{\pgfqpoint{2.082361in}{1.335414in}}%
\pgfpathlineto{\pgfqpoint{2.087185in}{1.285477in}}%
\pgfpathlineto{\pgfqpoint{2.093939in}{1.187426in}}%
\pgfpathlineto{\pgfqpoint{2.102623in}{1.023549in}}%
\pgfpathlineto{\pgfqpoint{2.135428in}{0.359398in}}%
\pgfpathlineto{\pgfqpoint{2.142182in}{0.280912in}}%
\pgfpathlineto{\pgfqpoint{2.147006in}{0.246891in}}%
\pgfpathlineto{\pgfqpoint{2.150866in}{0.233903in}}%
\pgfpathlineto{\pgfqpoint{2.152795in}{0.232273in}}%
\pgfpathlineto{\pgfqpoint{2.152795in}{0.232273in}}%
\pgfusepath{stroke}%
\end{pgfscope}%
\begin{pgfscope}%
\pgfpathrectangle{\pgfqpoint{0.225000in}{0.175000in}}{\pgfqpoint{4.050000in}{1.260000in}} %
\pgfusepath{clip}%
\pgfsetrectcap%
\pgfsetroundjoin%
\pgfsetlinewidth{1.003750pt}%
\definecolor{currentstroke}{rgb}{0.000000,0.000000,1.000000}%
\pgfsetstrokecolor{currentstroke}%
\pgfsetdash{}{0pt}%
\pgfpathmoveto{\pgfqpoint{2.152795in}{0.232273in}}%
\pgfpathlineto{\pgfqpoint{2.157620in}{0.233405in}}%
\pgfpathlineto{\pgfqpoint{2.162444in}{0.236798in}}%
\pgfpathlineto{\pgfqpoint{2.168233in}{0.243833in}}%
\pgfpathlineto{\pgfqpoint{2.174987in}{0.256076in}}%
\pgfpathlineto{\pgfqpoint{2.182706in}{0.275268in}}%
\pgfpathlineto{\pgfqpoint{2.191390in}{0.303254in}}%
\pgfpathlineto{\pgfqpoint{2.201039in}{0.341866in}}%
\pgfpathlineto{\pgfqpoint{2.212617in}{0.397798in}}%
\pgfpathlineto{\pgfqpoint{2.226125in}{0.474655in}}%
\pgfpathlineto{\pgfqpoint{2.242528in}{0.581466in}}%
\pgfpathlineto{\pgfqpoint{2.265684in}{0.748359in}}%
\pgfpathlineto{\pgfqpoint{2.309103in}{1.062844in}}%
\pgfpathlineto{\pgfqpoint{2.326471in}{1.171068in}}%
\pgfpathlineto{\pgfqpoint{2.339979in}{1.242575in}}%
\pgfpathlineto{\pgfqpoint{2.351557in}{1.293166in}}%
\pgfpathlineto{\pgfqpoint{2.362171in}{1.329812in}}%
\pgfpathlineto{\pgfqpoint{2.370854in}{1.352358in}}%
\pgfpathlineto{\pgfqpoint{2.378573in}{1.366524in}}%
\pgfpathlineto{\pgfqpoint{2.385327in}{1.374262in}}%
\pgfpathlineto{\pgfqpoint{2.391117in}{1.377385in}}%
\pgfpathlineto{\pgfqpoint{2.395941in}{1.377498in}}%
\pgfpathlineto{\pgfqpoint{2.400765in}{1.375347in}}%
\pgfpathlineto{\pgfqpoint{2.406554in}{1.369791in}}%
\pgfpathlineto{\pgfqpoint{2.412344in}{1.361018in}}%
\pgfpathlineto{\pgfqpoint{2.419098in}{1.346788in}}%
\pgfpathlineto{\pgfqpoint{2.426816in}{1.325391in}}%
\pgfpathlineto{\pgfqpoint{2.436465in}{1.291272in}}%
\pgfpathlineto{\pgfqpoint{2.447079in}{1.244890in}}%
\pgfpathlineto{\pgfqpoint{2.459622in}{1.179314in}}%
\pgfpathlineto{\pgfqpoint{2.474095in}{1.091364in}}%
\pgfpathlineto{\pgfqpoint{2.493392in}{0.958879in}}%
\pgfpathlineto{\pgfqpoint{2.562862in}{0.465887in}}%
\pgfpathlineto{\pgfqpoint{2.577335in}{0.385336in}}%
\pgfpathlineto{\pgfqpoint{2.589878in}{0.327485in}}%
\pgfpathlineto{\pgfqpoint{2.600492in}{0.288434in}}%
\pgfpathlineto{\pgfqpoint{2.609176in}{0.263799in}}%
\pgfpathlineto{\pgfqpoint{2.616894in}{0.247711in}}%
\pgfpathlineto{\pgfqpoint{2.623648in}{0.238255in}}%
\pgfpathlineto{\pgfqpoint{2.629438in}{0.233643in}}%
\pgfpathlineto{\pgfqpoint{2.634262in}{0.232284in}}%
\pgfpathlineto{\pgfqpoint{2.639086in}{0.233190in}}%
\pgfpathlineto{\pgfqpoint{2.643911in}{0.236357in}}%
\pgfpathlineto{\pgfqpoint{2.649700in}{0.243124in}}%
\pgfpathlineto{\pgfqpoint{2.656454in}{0.255059in}}%
\pgfpathlineto{\pgfqpoint{2.664173in}{0.273909in}}%
\pgfpathlineto{\pgfqpoint{2.672856in}{0.301527in}}%
\pgfpathlineto{\pgfqpoint{2.682505in}{0.339756in}}%
\pgfpathlineto{\pgfqpoint{2.694083in}{0.395273in}}%
\pgfpathlineto{\pgfqpoint{2.707591in}{0.471719in}}%
\pgfpathlineto{\pgfqpoint{2.723994in}{0.578154in}}%
\pgfpathlineto{\pgfqpoint{2.747151in}{0.744775in}}%
\pgfpathlineto{\pgfqpoint{2.790570in}{1.059622in}}%
\pgfpathlineto{\pgfqpoint{2.807937in}{1.168291in}}%
\pgfpathlineto{\pgfqpoint{2.821445in}{1.240242in}}%
\pgfpathlineto{\pgfqpoint{2.833024in}{1.291272in}}%
\pgfpathlineto{\pgfqpoint{2.843637in}{1.328359in}}%
\pgfpathlineto{\pgfqpoint{2.852321in}{1.351287in}}%
\pgfpathlineto{\pgfqpoint{2.860040in}{1.365804in}}%
\pgfpathlineto{\pgfqpoint{2.866794in}{1.373855in}}%
\pgfpathlineto{\pgfqpoint{2.872583in}{1.377249in}}%
\pgfpathlineto{\pgfqpoint{2.877407in}{1.377589in}}%
\pgfpathlineto{\pgfqpoint{2.882232in}{1.375664in}}%
\pgfpathlineto{\pgfqpoint{2.887056in}{1.371483in}}%
\pgfpathlineto{\pgfqpoint{2.892845in}{1.363511in}}%
\pgfpathlineto{\pgfqpoint{2.899599in}{1.350195in}}%
\pgfpathlineto{\pgfqpoint{2.907318in}{1.329812in}}%
\pgfpathlineto{\pgfqpoint{2.916002in}{1.300545in}}%
\pgfpathlineto{\pgfqpoint{2.926615in}{1.256204in}}%
\pgfpathlineto{\pgfqpoint{2.938194in}{1.198029in}}%
\pgfpathlineto{\pgfqpoint{2.952667in}{1.112916in}}%
\pgfpathlineto{\pgfqpoint{2.970034in}{0.996647in}}%
\pgfpathlineto{\pgfqpoint{2.998015in}{0.791493in}}%
\pgfpathlineto{\pgfqpoint{3.029856in}{0.561731in}}%
\pgfpathlineto{\pgfqpoint{3.047223in}{0.451543in}}%
\pgfpathlineto{\pgfqpoint{3.061696in}{0.373289in}}%
\pgfpathlineto{\pgfqpoint{3.073274in}{0.321604in}}%
\pgfpathlineto{\pgfqpoint{3.083888in}{0.283859in}}%
\pgfpathlineto{\pgfqpoint{3.092572in}{0.260360in}}%
\pgfpathlineto{\pgfqpoint{3.100291in}{0.245318in}}%
\pgfpathlineto{\pgfqpoint{3.107045in}{0.236798in}}%
\pgfpathlineto{\pgfqpoint{3.112834in}{0.232998in}}%
\pgfpathlineto{\pgfqpoint{3.116693in}{0.232273in}}%
\pgfpathlineto{\pgfqpoint{3.116693in}{0.232273in}}%
\pgfusepath{stroke}%
\end{pgfscope}%
\begin{pgfscope}%
\pgfpathrectangle{\pgfqpoint{0.225000in}{0.175000in}}{\pgfqpoint{4.050000in}{1.260000in}} %
\pgfusepath{clip}%
\pgfsetrectcap%
\pgfsetroundjoin%
\pgfsetlinewidth{1.003750pt}%
\definecolor{currentstroke}{rgb}{1.000000,0.000000,1.000000}%
\pgfsetstrokecolor{currentstroke}%
\pgfsetdash{}{0pt}%
\pgfpathmoveto{\pgfqpoint{3.116693in}{0.232273in}}%
\pgfpathlineto{\pgfqpoint{3.118623in}{0.233903in}}%
\pgfpathlineto{\pgfqpoint{3.121518in}{0.242438in}}%
\pgfpathlineto{\pgfqpoint{3.125377in}{0.264988in}}%
\pgfpathlineto{\pgfqpoint{3.130201in}{0.310360in}}%
\pgfpathlineto{\pgfqpoint{3.135990in}{0.387796in}}%
\pgfpathlineto{\pgfqpoint{3.143709in}{0.523328in}}%
\pgfpathlineto{\pgfqpoint{3.156253in}{0.791493in}}%
\pgfpathlineto{\pgfqpoint{3.172655in}{1.136814in}}%
\pgfpathlineto{\pgfqpoint{3.181339in}{1.273375in}}%
\pgfpathlineto{\pgfqpoint{3.187128in}{1.335414in}}%
\pgfpathlineto{\pgfqpoint{3.191953in}{1.366524in}}%
\pgfpathlineto{\pgfqpoint{3.194847in}{1.375664in}}%
\pgfpathlineto{\pgfqpoint{3.196777in}{1.377702in}}%
\pgfpathlineto{\pgfqpoint{3.198707in}{1.376479in}}%
\pgfpathlineto{\pgfqpoint{3.200636in}{1.372002in}}%
\pgfpathlineto{\pgfqpoint{3.203531in}{1.359247in}}%
\pgfpathlineto{\pgfqpoint{3.207390in}{1.331243in}}%
\pgfpathlineto{\pgfqpoint{3.212215in}{1.279510in}}%
\pgfpathlineto{\pgfqpoint{3.218969in}{1.179314in}}%
\pgfpathlineto{\pgfqpoint{3.227652in}{1.013522in}}%
\pgfpathlineto{\pgfqpoint{3.259493in}{0.366265in}}%
\pgfpathlineto{\pgfqpoint{3.266247in}{0.285363in}}%
\pgfpathlineto{\pgfqpoint{3.271071in}{0.249416in}}%
\pgfpathlineto{\pgfqpoint{3.274931in}{0.234820in}}%
\pgfpathlineto{\pgfqpoint{3.276860in}{0.232375in}}%
\pgfpathlineto{\pgfqpoint{3.278790in}{0.233190in}}%
\pgfpathlineto{\pgfqpoint{3.280720in}{0.237261in}}%
\pgfpathlineto{\pgfqpoint{3.283614in}{0.249416in}}%
\pgfpathlineto{\pgfqpoint{3.287474in}{0.276648in}}%
\pgfpathlineto{\pgfqpoint{3.292298in}{0.327485in}}%
\pgfpathlineto{\pgfqpoint{3.299052in}{0.426614in}}%
\pgfpathlineto{\pgfqpoint{3.307736in}{0.591455in}}%
\pgfpathlineto{\pgfqpoint{3.339577in}{1.240242in}}%
\pgfpathlineto{\pgfqpoint{3.346331in}{1.322342in}}%
\pgfpathlineto{\pgfqpoint{3.351155in}{1.359247in}}%
\pgfpathlineto{\pgfqpoint{3.355014in}{1.374646in}}%
\pgfpathlineto{\pgfqpoint{3.356944in}{1.377498in}}%
\pgfpathlineto{\pgfqpoint{3.358874in}{1.377090in}}%
\pgfpathlineto{\pgfqpoint{3.360803in}{1.373425in}}%
\pgfpathlineto{\pgfqpoint{3.363698in}{1.361871in}}%
\pgfpathlineto{\pgfqpoint{3.367558in}{1.335414in}}%
\pgfpathlineto{\pgfqpoint{3.372382in}{1.285477in}}%
\pgfpathlineto{\pgfqpoint{3.379136in}{1.187426in}}%
\pgfpathlineto{\pgfqpoint{3.387820in}{1.023549in}}%
\pgfpathlineto{\pgfqpoint{3.420625in}{0.359398in}}%
\pgfpathlineto{\pgfqpoint{3.427379in}{0.280912in}}%
\pgfpathlineto{\pgfqpoint{3.432203in}{0.246891in}}%
\pgfpathlineto{\pgfqpoint{3.436063in}{0.233903in}}%
\pgfpathlineto{\pgfqpoint{3.437992in}{0.232273in}}%
\pgfpathlineto{\pgfqpoint{3.439922in}{0.233903in}}%
\pgfpathlineto{\pgfqpoint{3.442817in}{0.242438in}}%
\pgfpathlineto{\pgfqpoint{3.446676in}{0.264988in}}%
\pgfpathlineto{\pgfqpoint{3.451501in}{0.310360in}}%
\pgfpathlineto{\pgfqpoint{3.457290in}{0.387796in}}%
\pgfpathlineto{\pgfqpoint{3.465009in}{0.523328in}}%
\pgfpathlineto{\pgfqpoint{3.477552in}{0.791493in}}%
\pgfpathlineto{\pgfqpoint{3.493955in}{1.136814in}}%
\pgfpathlineto{\pgfqpoint{3.502638in}{1.273375in}}%
\pgfpathlineto{\pgfqpoint{3.508427in}{1.335414in}}%
\pgfpathlineto{\pgfqpoint{3.513252in}{1.366524in}}%
\pgfpathlineto{\pgfqpoint{3.516146in}{1.375664in}}%
\pgfpathlineto{\pgfqpoint{3.518076in}{1.377702in}}%
\pgfpathlineto{\pgfqpoint{3.520006in}{1.376479in}}%
\pgfpathlineto{\pgfqpoint{3.521936in}{1.372002in}}%
\pgfpathlineto{\pgfqpoint{3.524830in}{1.359247in}}%
\pgfpathlineto{\pgfqpoint{3.528690in}{1.331243in}}%
\pgfpathlineto{\pgfqpoint{3.533514in}{1.279510in}}%
\pgfpathlineto{\pgfqpoint{3.540268in}{1.179314in}}%
\pgfpathlineto{\pgfqpoint{3.548952in}{1.013522in}}%
\pgfpathlineto{\pgfqpoint{3.580792in}{0.366265in}}%
\pgfpathlineto{\pgfqpoint{3.587546in}{0.285363in}}%
\pgfpathlineto{\pgfqpoint{3.592371in}{0.249416in}}%
\pgfpathlineto{\pgfqpoint{3.596230in}{0.234820in}}%
\pgfpathlineto{\pgfqpoint{3.598160in}{0.232375in}}%
\pgfpathlineto{\pgfqpoint{3.600089in}{0.233190in}}%
\pgfpathlineto{\pgfqpoint{3.602019in}{0.237261in}}%
\pgfpathlineto{\pgfqpoint{3.604914in}{0.249416in}}%
\pgfpathlineto{\pgfqpoint{3.608773in}{0.276648in}}%
\pgfpathlineto{\pgfqpoint{3.613597in}{0.327485in}}%
\pgfpathlineto{\pgfqpoint{3.620352in}{0.426614in}}%
\pgfpathlineto{\pgfqpoint{3.629035in}{0.591455in}}%
\pgfpathlineto{\pgfqpoint{3.660876in}{1.240242in}}%
\pgfpathlineto{\pgfqpoint{3.667630in}{1.322342in}}%
\pgfpathlineto{\pgfqpoint{3.672454in}{1.359247in}}%
\pgfpathlineto{\pgfqpoint{3.676314in}{1.374646in}}%
\pgfpathlineto{\pgfqpoint{3.678243in}{1.377498in}}%
\pgfpathlineto{\pgfqpoint{3.680173in}{1.377090in}}%
\pgfpathlineto{\pgfqpoint{3.682103in}{1.373425in}}%
\pgfpathlineto{\pgfqpoint{3.684997in}{1.361871in}}%
\pgfpathlineto{\pgfqpoint{3.688857in}{1.335414in}}%
\pgfpathlineto{\pgfqpoint{3.693681in}{1.285477in}}%
\pgfpathlineto{\pgfqpoint{3.700435in}{1.187426in}}%
\pgfpathlineto{\pgfqpoint{3.709119in}{1.023549in}}%
\pgfpathlineto{\pgfqpoint{3.741924in}{0.359398in}}%
\pgfpathlineto{\pgfqpoint{3.748678in}{0.280912in}}%
\pgfpathlineto{\pgfqpoint{3.753503in}{0.246891in}}%
\pgfpathlineto{\pgfqpoint{3.757362in}{0.233903in}}%
\pgfpathlineto{\pgfqpoint{3.759292in}{0.232273in}}%
\pgfpathlineto{\pgfqpoint{3.761221in}{0.233903in}}%
\pgfpathlineto{\pgfqpoint{3.764116in}{0.242438in}}%
\pgfpathlineto{\pgfqpoint{3.767975in}{0.264988in}}%
\pgfpathlineto{\pgfqpoint{3.772800in}{0.310360in}}%
\pgfpathlineto{\pgfqpoint{3.778589in}{0.387796in}}%
\pgfpathlineto{\pgfqpoint{3.786308in}{0.523328in}}%
\pgfpathlineto{\pgfqpoint{3.798851in}{0.791493in}}%
\pgfpathlineto{\pgfqpoint{3.815254in}{1.136814in}}%
\pgfpathlineto{\pgfqpoint{3.823938in}{1.273375in}}%
\pgfpathlineto{\pgfqpoint{3.829727in}{1.335414in}}%
\pgfpathlineto{\pgfqpoint{3.834551in}{1.366524in}}%
\pgfpathlineto{\pgfqpoint{3.837446in}{1.375664in}}%
\pgfpathlineto{\pgfqpoint{3.839375in}{1.377702in}}%
\pgfpathlineto{\pgfqpoint{3.841305in}{1.376479in}}%
\pgfpathlineto{\pgfqpoint{3.843235in}{1.372002in}}%
\pgfpathlineto{\pgfqpoint{3.846129in}{1.359247in}}%
\pgfpathlineto{\pgfqpoint{3.849989in}{1.331243in}}%
\pgfpathlineto{\pgfqpoint{3.854813in}{1.279510in}}%
\pgfpathlineto{\pgfqpoint{3.861567in}{1.179314in}}%
\pgfpathlineto{\pgfqpoint{3.870251in}{1.013522in}}%
\pgfpathlineto{\pgfqpoint{3.902091in}{0.366265in}}%
\pgfpathlineto{\pgfqpoint{3.908845in}{0.285363in}}%
\pgfpathlineto{\pgfqpoint{3.913670in}{0.249416in}}%
\pgfpathlineto{\pgfqpoint{3.917529in}{0.234820in}}%
\pgfpathlineto{\pgfqpoint{3.919459in}{0.232375in}}%
\pgfpathlineto{\pgfqpoint{3.921389in}{0.233190in}}%
\pgfpathlineto{\pgfqpoint{3.923318in}{0.237261in}}%
\pgfpathlineto{\pgfqpoint{3.926213in}{0.249416in}}%
\pgfpathlineto{\pgfqpoint{3.930072in}{0.276648in}}%
\pgfpathlineto{\pgfqpoint{3.934897in}{0.327485in}}%
\pgfpathlineto{\pgfqpoint{3.941651in}{0.426614in}}%
\pgfpathlineto{\pgfqpoint{3.950335in}{0.591455in}}%
\pgfpathlineto{\pgfqpoint{3.982175in}{1.240242in}}%
\pgfpathlineto{\pgfqpoint{3.988929in}{1.322342in}}%
\pgfpathlineto{\pgfqpoint{3.993753in}{1.359247in}}%
\pgfpathlineto{\pgfqpoint{3.997613in}{1.374646in}}%
\pgfpathlineto{\pgfqpoint{3.999543in}{1.377498in}}%
\pgfpathlineto{\pgfqpoint{4.001472in}{1.377090in}}%
\pgfpathlineto{\pgfqpoint{4.003402in}{1.373425in}}%
\pgfpathlineto{\pgfqpoint{4.006297in}{1.361871in}}%
\pgfpathlineto{\pgfqpoint{4.010156in}{1.335414in}}%
\pgfpathlineto{\pgfqpoint{4.014980in}{1.285477in}}%
\pgfpathlineto{\pgfqpoint{4.021734in}{1.187426in}}%
\pgfpathlineto{\pgfqpoint{4.030418in}{1.023549in}}%
\pgfpathlineto{\pgfqpoint{4.063223in}{0.359398in}}%
\pgfpathlineto{\pgfqpoint{4.069977in}{0.280912in}}%
\pgfpathlineto{\pgfqpoint{4.074802in}{0.246891in}}%
\pgfpathlineto{\pgfqpoint{4.078661in}{0.233903in}}%
\pgfpathlineto{\pgfqpoint{4.080591in}{0.232273in}}%
\pgfpathlineto{\pgfqpoint{4.080591in}{0.232273in}}%
\pgfusepath{stroke}%
\end{pgfscope}%
\begin{pgfscope}%
\pgfsetrectcap%
\pgfsetmiterjoin%
\pgfsetlinewidth{1.003750pt}%
\definecolor{currentstroke}{rgb}{0.000000,0.000000,0.000000}%
\pgfsetstrokecolor{currentstroke}%
\pgfsetdash{}{0pt}%
\pgfpathmoveto{\pgfqpoint{4.275000in}{0.175000in}}%
\pgfpathlineto{\pgfqpoint{4.275000in}{1.435000in}}%
\pgfusepath{stroke}%
\end{pgfscope}%
\begin{pgfscope}%
\pgfsetrectcap%
\pgfsetmiterjoin%
\pgfsetlinewidth{1.003750pt}%
\definecolor{currentstroke}{rgb}{0.000000,0.000000,0.000000}%
\pgfsetstrokecolor{currentstroke}%
\pgfsetdash{}{0pt}%
\pgfpathmoveto{\pgfqpoint{0.225000in}{0.175000in}}%
\pgfpathlineto{\pgfqpoint{4.275000in}{0.175000in}}%
\pgfusepath{stroke}%
\end{pgfscope}%
\begin{pgfscope}%
\pgfsetrectcap%
\pgfsetmiterjoin%
\pgfsetlinewidth{1.003750pt}%
\definecolor{currentstroke}{rgb}{0.000000,0.000000,0.000000}%
\pgfsetstrokecolor{currentstroke}%
\pgfsetdash{}{0pt}%
\pgfpathmoveto{\pgfqpoint{0.225000in}{1.435000in}}%
\pgfpathlineto{\pgfqpoint{4.275000in}{1.435000in}}%
\pgfusepath{stroke}%
\end{pgfscope}%
\begin{pgfscope}%
\pgfsetrectcap%
\pgfsetmiterjoin%
\pgfsetlinewidth{1.003750pt}%
\definecolor{currentstroke}{rgb}{0.000000,0.000000,0.000000}%
\pgfsetstrokecolor{currentstroke}%
\pgfsetdash{}{0pt}%
\pgfpathmoveto{\pgfqpoint{0.225000in}{0.175000in}}%
\pgfpathlineto{\pgfqpoint{0.225000in}{1.435000in}}%
\pgfusepath{stroke}%
\end{pgfscope}%
\begin{pgfscope}%
\pgfsetbuttcap%
\pgfsetroundjoin%
\definecolor{currentfill}{rgb}{0.000000,0.000000,0.000000}%
\pgfsetfillcolor{currentfill}%
\pgfsetlinewidth{0.501875pt}%
\definecolor{currentstroke}{rgb}{0.000000,0.000000,0.000000}%
\pgfsetstrokecolor{currentstroke}%
\pgfsetdash{}{0pt}%
\pgfsys@defobject{currentmarker}{\pgfqpoint{0.000000in}{0.000000in}}{\pgfqpoint{0.000000in}{0.055556in}}{%
\pgfpathmoveto{\pgfqpoint{0.000000in}{0.000000in}}%
\pgfpathlineto{\pgfqpoint{0.000000in}{0.055556in}}%
\pgfusepath{stroke,fill}%
}%
\begin{pgfscope}%
\pgfsys@transformshift{0.225000in}{0.175000in}%
\pgfsys@useobject{currentmarker}{}%
\end{pgfscope}%
\end{pgfscope}%
\begin{pgfscope}%
\pgfsetbuttcap%
\pgfsetroundjoin%
\definecolor{currentfill}{rgb}{0.000000,0.000000,0.000000}%
\pgfsetfillcolor{currentfill}%
\pgfsetlinewidth{0.501875pt}%
\definecolor{currentstroke}{rgb}{0.000000,0.000000,0.000000}%
\pgfsetstrokecolor{currentstroke}%
\pgfsetdash{}{0pt}%
\pgfsys@defobject{currentmarker}{\pgfqpoint{0.000000in}{-0.055556in}}{\pgfqpoint{0.000000in}{0.000000in}}{%
\pgfpathmoveto{\pgfqpoint{0.000000in}{0.000000in}}%
\pgfpathlineto{\pgfqpoint{0.000000in}{-0.055556in}}%
\pgfusepath{stroke,fill}%
}%
\begin{pgfscope}%
\pgfsys@transformshift{0.225000in}{1.435000in}%
\pgfsys@useobject{currentmarker}{}%
\end{pgfscope}%
\end{pgfscope}%
\begin{pgfscope}%
\pgftext[x=0.225000in,y=0.119444in,,top]{\rmfamily\fontsize{9.000000}{10.800000}\selectfont \(\displaystyle 0.0000\)}%
\end{pgfscope}%
\begin{pgfscope}%
\pgfsetbuttcap%
\pgfsetroundjoin%
\definecolor{currentfill}{rgb}{0.000000,0.000000,0.000000}%
\pgfsetfillcolor{currentfill}%
\pgfsetlinewidth{0.501875pt}%
\definecolor{currentstroke}{rgb}{0.000000,0.000000,0.000000}%
\pgfsetstrokecolor{currentstroke}%
\pgfsetdash{}{0pt}%
\pgfsys@defobject{currentmarker}{\pgfqpoint{0.000000in}{0.000000in}}{\pgfqpoint{0.000000in}{0.055556in}}{%
\pgfpathmoveto{\pgfqpoint{0.000000in}{0.000000in}}%
\pgfpathlineto{\pgfqpoint{0.000000in}{0.055556in}}%
\pgfusepath{stroke,fill}%
}%
\begin{pgfscope}%
\pgfsys@transformshift{0.731250in}{0.175000in}%
\pgfsys@useobject{currentmarker}{}%
\end{pgfscope}%
\end{pgfscope}%
\begin{pgfscope}%
\pgfsetbuttcap%
\pgfsetroundjoin%
\definecolor{currentfill}{rgb}{0.000000,0.000000,0.000000}%
\pgfsetfillcolor{currentfill}%
\pgfsetlinewidth{0.501875pt}%
\definecolor{currentstroke}{rgb}{0.000000,0.000000,0.000000}%
\pgfsetstrokecolor{currentstroke}%
\pgfsetdash{}{0pt}%
\pgfsys@defobject{currentmarker}{\pgfqpoint{0.000000in}{-0.055556in}}{\pgfqpoint{0.000000in}{0.000000in}}{%
\pgfpathmoveto{\pgfqpoint{0.000000in}{0.000000in}}%
\pgfpathlineto{\pgfqpoint{0.000000in}{-0.055556in}}%
\pgfusepath{stroke,fill}%
}%
\begin{pgfscope}%
\pgfsys@transformshift{0.731250in}{1.435000in}%
\pgfsys@useobject{currentmarker}{}%
\end{pgfscope}%
\end{pgfscope}%
\begin{pgfscope}%
\pgftext[x=0.731250in,y=0.119444in,,top]{\rmfamily\fontsize{9.000000}{10.800000}\selectfont \(\displaystyle 0.0002\)}%
\end{pgfscope}%
\begin{pgfscope}%
\pgfsetbuttcap%
\pgfsetroundjoin%
\definecolor{currentfill}{rgb}{0.000000,0.000000,0.000000}%
\pgfsetfillcolor{currentfill}%
\pgfsetlinewidth{0.501875pt}%
\definecolor{currentstroke}{rgb}{0.000000,0.000000,0.000000}%
\pgfsetstrokecolor{currentstroke}%
\pgfsetdash{}{0pt}%
\pgfsys@defobject{currentmarker}{\pgfqpoint{0.000000in}{0.000000in}}{\pgfqpoint{0.000000in}{0.055556in}}{%
\pgfpathmoveto{\pgfqpoint{0.000000in}{0.000000in}}%
\pgfpathlineto{\pgfqpoint{0.000000in}{0.055556in}}%
\pgfusepath{stroke,fill}%
}%
\begin{pgfscope}%
\pgfsys@transformshift{1.237500in}{0.175000in}%
\pgfsys@useobject{currentmarker}{}%
\end{pgfscope}%
\end{pgfscope}%
\begin{pgfscope}%
\pgfsetbuttcap%
\pgfsetroundjoin%
\definecolor{currentfill}{rgb}{0.000000,0.000000,0.000000}%
\pgfsetfillcolor{currentfill}%
\pgfsetlinewidth{0.501875pt}%
\definecolor{currentstroke}{rgb}{0.000000,0.000000,0.000000}%
\pgfsetstrokecolor{currentstroke}%
\pgfsetdash{}{0pt}%
\pgfsys@defobject{currentmarker}{\pgfqpoint{0.000000in}{-0.055556in}}{\pgfqpoint{0.000000in}{0.000000in}}{%
\pgfpathmoveto{\pgfqpoint{0.000000in}{0.000000in}}%
\pgfpathlineto{\pgfqpoint{0.000000in}{-0.055556in}}%
\pgfusepath{stroke,fill}%
}%
\begin{pgfscope}%
\pgfsys@transformshift{1.237500in}{1.435000in}%
\pgfsys@useobject{currentmarker}{}%
\end{pgfscope}%
\end{pgfscope}%
\begin{pgfscope}%
\pgftext[x=1.237500in,y=0.119444in,,top]{\rmfamily\fontsize{9.000000}{10.800000}\selectfont \(\displaystyle 0.0004\)}%
\end{pgfscope}%
\begin{pgfscope}%
\pgfsetbuttcap%
\pgfsetroundjoin%
\definecolor{currentfill}{rgb}{0.000000,0.000000,0.000000}%
\pgfsetfillcolor{currentfill}%
\pgfsetlinewidth{0.501875pt}%
\definecolor{currentstroke}{rgb}{0.000000,0.000000,0.000000}%
\pgfsetstrokecolor{currentstroke}%
\pgfsetdash{}{0pt}%
\pgfsys@defobject{currentmarker}{\pgfqpoint{0.000000in}{0.000000in}}{\pgfqpoint{0.000000in}{0.055556in}}{%
\pgfpathmoveto{\pgfqpoint{0.000000in}{0.000000in}}%
\pgfpathlineto{\pgfqpoint{0.000000in}{0.055556in}}%
\pgfusepath{stroke,fill}%
}%
\begin{pgfscope}%
\pgfsys@transformshift{1.743750in}{0.175000in}%
\pgfsys@useobject{currentmarker}{}%
\end{pgfscope}%
\end{pgfscope}%
\begin{pgfscope}%
\pgfsetbuttcap%
\pgfsetroundjoin%
\definecolor{currentfill}{rgb}{0.000000,0.000000,0.000000}%
\pgfsetfillcolor{currentfill}%
\pgfsetlinewidth{0.501875pt}%
\definecolor{currentstroke}{rgb}{0.000000,0.000000,0.000000}%
\pgfsetstrokecolor{currentstroke}%
\pgfsetdash{}{0pt}%
\pgfsys@defobject{currentmarker}{\pgfqpoint{0.000000in}{-0.055556in}}{\pgfqpoint{0.000000in}{0.000000in}}{%
\pgfpathmoveto{\pgfqpoint{0.000000in}{0.000000in}}%
\pgfpathlineto{\pgfqpoint{0.000000in}{-0.055556in}}%
\pgfusepath{stroke,fill}%
}%
\begin{pgfscope}%
\pgfsys@transformshift{1.743750in}{1.435000in}%
\pgfsys@useobject{currentmarker}{}%
\end{pgfscope}%
\end{pgfscope}%
\begin{pgfscope}%
\pgftext[x=1.743750in,y=0.119444in,,top]{\rmfamily\fontsize{9.000000}{10.800000}\selectfont \(\displaystyle 0.0006\)}%
\end{pgfscope}%
\begin{pgfscope}%
\pgfsetbuttcap%
\pgfsetroundjoin%
\definecolor{currentfill}{rgb}{0.000000,0.000000,0.000000}%
\pgfsetfillcolor{currentfill}%
\pgfsetlinewidth{0.501875pt}%
\definecolor{currentstroke}{rgb}{0.000000,0.000000,0.000000}%
\pgfsetstrokecolor{currentstroke}%
\pgfsetdash{}{0pt}%
\pgfsys@defobject{currentmarker}{\pgfqpoint{0.000000in}{0.000000in}}{\pgfqpoint{0.000000in}{0.055556in}}{%
\pgfpathmoveto{\pgfqpoint{0.000000in}{0.000000in}}%
\pgfpathlineto{\pgfqpoint{0.000000in}{0.055556in}}%
\pgfusepath{stroke,fill}%
}%
\begin{pgfscope}%
\pgfsys@transformshift{2.250000in}{0.175000in}%
\pgfsys@useobject{currentmarker}{}%
\end{pgfscope}%
\end{pgfscope}%
\begin{pgfscope}%
\pgfsetbuttcap%
\pgfsetroundjoin%
\definecolor{currentfill}{rgb}{0.000000,0.000000,0.000000}%
\pgfsetfillcolor{currentfill}%
\pgfsetlinewidth{0.501875pt}%
\definecolor{currentstroke}{rgb}{0.000000,0.000000,0.000000}%
\pgfsetstrokecolor{currentstroke}%
\pgfsetdash{}{0pt}%
\pgfsys@defobject{currentmarker}{\pgfqpoint{0.000000in}{-0.055556in}}{\pgfqpoint{0.000000in}{0.000000in}}{%
\pgfpathmoveto{\pgfqpoint{0.000000in}{0.000000in}}%
\pgfpathlineto{\pgfqpoint{0.000000in}{-0.055556in}}%
\pgfusepath{stroke,fill}%
}%
\begin{pgfscope}%
\pgfsys@transformshift{2.250000in}{1.435000in}%
\pgfsys@useobject{currentmarker}{}%
\end{pgfscope}%
\end{pgfscope}%
\begin{pgfscope}%
\pgftext[x=2.250000in,y=0.119444in,,top]{\rmfamily\fontsize{9.000000}{10.800000}\selectfont \(\displaystyle 0.0008\)}%
\end{pgfscope}%
\begin{pgfscope}%
\pgfsetbuttcap%
\pgfsetroundjoin%
\definecolor{currentfill}{rgb}{0.000000,0.000000,0.000000}%
\pgfsetfillcolor{currentfill}%
\pgfsetlinewidth{0.501875pt}%
\definecolor{currentstroke}{rgb}{0.000000,0.000000,0.000000}%
\pgfsetstrokecolor{currentstroke}%
\pgfsetdash{}{0pt}%
\pgfsys@defobject{currentmarker}{\pgfqpoint{0.000000in}{0.000000in}}{\pgfqpoint{0.000000in}{0.055556in}}{%
\pgfpathmoveto{\pgfqpoint{0.000000in}{0.000000in}}%
\pgfpathlineto{\pgfqpoint{0.000000in}{0.055556in}}%
\pgfusepath{stroke,fill}%
}%
\begin{pgfscope}%
\pgfsys@transformshift{2.756250in}{0.175000in}%
\pgfsys@useobject{currentmarker}{}%
\end{pgfscope}%
\end{pgfscope}%
\begin{pgfscope}%
\pgfsetbuttcap%
\pgfsetroundjoin%
\definecolor{currentfill}{rgb}{0.000000,0.000000,0.000000}%
\pgfsetfillcolor{currentfill}%
\pgfsetlinewidth{0.501875pt}%
\definecolor{currentstroke}{rgb}{0.000000,0.000000,0.000000}%
\pgfsetstrokecolor{currentstroke}%
\pgfsetdash{}{0pt}%
\pgfsys@defobject{currentmarker}{\pgfqpoint{0.000000in}{-0.055556in}}{\pgfqpoint{0.000000in}{0.000000in}}{%
\pgfpathmoveto{\pgfqpoint{0.000000in}{0.000000in}}%
\pgfpathlineto{\pgfqpoint{0.000000in}{-0.055556in}}%
\pgfusepath{stroke,fill}%
}%
\begin{pgfscope}%
\pgfsys@transformshift{2.756250in}{1.435000in}%
\pgfsys@useobject{currentmarker}{}%
\end{pgfscope}%
\end{pgfscope}%
\begin{pgfscope}%
\pgftext[x=2.756250in,y=0.119444in,,top]{\rmfamily\fontsize{9.000000}{10.800000}\selectfont \(\displaystyle 0.0010\)}%
\end{pgfscope}%
\begin{pgfscope}%
\pgfsetbuttcap%
\pgfsetroundjoin%
\definecolor{currentfill}{rgb}{0.000000,0.000000,0.000000}%
\pgfsetfillcolor{currentfill}%
\pgfsetlinewidth{0.501875pt}%
\definecolor{currentstroke}{rgb}{0.000000,0.000000,0.000000}%
\pgfsetstrokecolor{currentstroke}%
\pgfsetdash{}{0pt}%
\pgfsys@defobject{currentmarker}{\pgfqpoint{0.000000in}{0.000000in}}{\pgfqpoint{0.000000in}{0.055556in}}{%
\pgfpathmoveto{\pgfqpoint{0.000000in}{0.000000in}}%
\pgfpathlineto{\pgfqpoint{0.000000in}{0.055556in}}%
\pgfusepath{stroke,fill}%
}%
\begin{pgfscope}%
\pgfsys@transformshift{3.262500in}{0.175000in}%
\pgfsys@useobject{currentmarker}{}%
\end{pgfscope}%
\end{pgfscope}%
\begin{pgfscope}%
\pgfsetbuttcap%
\pgfsetroundjoin%
\definecolor{currentfill}{rgb}{0.000000,0.000000,0.000000}%
\pgfsetfillcolor{currentfill}%
\pgfsetlinewidth{0.501875pt}%
\definecolor{currentstroke}{rgb}{0.000000,0.000000,0.000000}%
\pgfsetstrokecolor{currentstroke}%
\pgfsetdash{}{0pt}%
\pgfsys@defobject{currentmarker}{\pgfqpoint{0.000000in}{-0.055556in}}{\pgfqpoint{0.000000in}{0.000000in}}{%
\pgfpathmoveto{\pgfqpoint{0.000000in}{0.000000in}}%
\pgfpathlineto{\pgfqpoint{0.000000in}{-0.055556in}}%
\pgfusepath{stroke,fill}%
}%
\begin{pgfscope}%
\pgfsys@transformshift{3.262500in}{1.435000in}%
\pgfsys@useobject{currentmarker}{}%
\end{pgfscope}%
\end{pgfscope}%
\begin{pgfscope}%
\pgftext[x=3.262500in,y=0.119444in,,top]{\rmfamily\fontsize{9.000000}{10.800000}\selectfont \(\displaystyle 0.0012\)}%
\end{pgfscope}%
\begin{pgfscope}%
\pgfsetbuttcap%
\pgfsetroundjoin%
\definecolor{currentfill}{rgb}{0.000000,0.000000,0.000000}%
\pgfsetfillcolor{currentfill}%
\pgfsetlinewidth{0.501875pt}%
\definecolor{currentstroke}{rgb}{0.000000,0.000000,0.000000}%
\pgfsetstrokecolor{currentstroke}%
\pgfsetdash{}{0pt}%
\pgfsys@defobject{currentmarker}{\pgfqpoint{0.000000in}{0.000000in}}{\pgfqpoint{0.000000in}{0.055556in}}{%
\pgfpathmoveto{\pgfqpoint{0.000000in}{0.000000in}}%
\pgfpathlineto{\pgfqpoint{0.000000in}{0.055556in}}%
\pgfusepath{stroke,fill}%
}%
\begin{pgfscope}%
\pgfsys@transformshift{3.768750in}{0.175000in}%
\pgfsys@useobject{currentmarker}{}%
\end{pgfscope}%
\end{pgfscope}%
\begin{pgfscope}%
\pgfsetbuttcap%
\pgfsetroundjoin%
\definecolor{currentfill}{rgb}{0.000000,0.000000,0.000000}%
\pgfsetfillcolor{currentfill}%
\pgfsetlinewidth{0.501875pt}%
\definecolor{currentstroke}{rgb}{0.000000,0.000000,0.000000}%
\pgfsetstrokecolor{currentstroke}%
\pgfsetdash{}{0pt}%
\pgfsys@defobject{currentmarker}{\pgfqpoint{0.000000in}{-0.055556in}}{\pgfqpoint{0.000000in}{0.000000in}}{%
\pgfpathmoveto{\pgfqpoint{0.000000in}{0.000000in}}%
\pgfpathlineto{\pgfqpoint{0.000000in}{-0.055556in}}%
\pgfusepath{stroke,fill}%
}%
\begin{pgfscope}%
\pgfsys@transformshift{3.768750in}{1.435000in}%
\pgfsys@useobject{currentmarker}{}%
\end{pgfscope}%
\end{pgfscope}%
\begin{pgfscope}%
\pgftext[x=3.768750in,y=0.119444in,,top]{\rmfamily\fontsize{9.000000}{10.800000}\selectfont \(\displaystyle 0.0014\)}%
\end{pgfscope}%
\begin{pgfscope}%
\pgfsetbuttcap%
\pgfsetroundjoin%
\definecolor{currentfill}{rgb}{0.000000,0.000000,0.000000}%
\pgfsetfillcolor{currentfill}%
\pgfsetlinewidth{0.501875pt}%
\definecolor{currentstroke}{rgb}{0.000000,0.000000,0.000000}%
\pgfsetstrokecolor{currentstroke}%
\pgfsetdash{}{0pt}%
\pgfsys@defobject{currentmarker}{\pgfqpoint{0.000000in}{0.000000in}}{\pgfqpoint{0.000000in}{0.055556in}}{%
\pgfpathmoveto{\pgfqpoint{0.000000in}{0.000000in}}%
\pgfpathlineto{\pgfqpoint{0.000000in}{0.055556in}}%
\pgfusepath{stroke,fill}%
}%
\begin{pgfscope}%
\pgfsys@transformshift{4.275000in}{0.175000in}%
\pgfsys@useobject{currentmarker}{}%
\end{pgfscope}%
\end{pgfscope}%
\begin{pgfscope}%
\pgfsetbuttcap%
\pgfsetroundjoin%
\definecolor{currentfill}{rgb}{0.000000,0.000000,0.000000}%
\pgfsetfillcolor{currentfill}%
\pgfsetlinewidth{0.501875pt}%
\definecolor{currentstroke}{rgb}{0.000000,0.000000,0.000000}%
\pgfsetstrokecolor{currentstroke}%
\pgfsetdash{}{0pt}%
\pgfsys@defobject{currentmarker}{\pgfqpoint{0.000000in}{-0.055556in}}{\pgfqpoint{0.000000in}{0.000000in}}{%
\pgfpathmoveto{\pgfqpoint{0.000000in}{0.000000in}}%
\pgfpathlineto{\pgfqpoint{0.000000in}{-0.055556in}}%
\pgfusepath{stroke,fill}%
}%
\begin{pgfscope}%
\pgfsys@transformshift{4.275000in}{1.435000in}%
\pgfsys@useobject{currentmarker}{}%
\end{pgfscope}%
\end{pgfscope}%
\begin{pgfscope}%
\pgftext[x=4.275000in,y=0.119444in,,top]{\rmfamily\fontsize{9.000000}{10.800000}\selectfont \(\displaystyle 0.0016\)}%
\end{pgfscope}%
\begin{pgfscope}%
\pgfsetbuttcap%
\pgfsetroundjoin%
\definecolor{currentfill}{rgb}{0.000000,0.000000,0.000000}%
\pgfsetfillcolor{currentfill}%
\pgfsetlinewidth{0.501875pt}%
\definecolor{currentstroke}{rgb}{0.000000,0.000000,0.000000}%
\pgfsetstrokecolor{currentstroke}%
\pgfsetdash{}{0pt}%
\pgfsys@defobject{currentmarker}{\pgfqpoint{0.000000in}{0.000000in}}{\pgfqpoint{0.055556in}{0.000000in}}{%
\pgfpathmoveto{\pgfqpoint{0.000000in}{0.000000in}}%
\pgfpathlineto{\pgfqpoint{0.055556in}{0.000000in}}%
\pgfusepath{stroke,fill}%
}%
\begin{pgfscope}%
\pgfsys@transformshift{0.225000in}{0.232273in}%
\pgfsys@useobject{currentmarker}{}%
\end{pgfscope}%
\end{pgfscope}%
\begin{pgfscope}%
\pgfsetbuttcap%
\pgfsetroundjoin%
\definecolor{currentfill}{rgb}{0.000000,0.000000,0.000000}%
\pgfsetfillcolor{currentfill}%
\pgfsetlinewidth{0.501875pt}%
\definecolor{currentstroke}{rgb}{0.000000,0.000000,0.000000}%
\pgfsetstrokecolor{currentstroke}%
\pgfsetdash{}{0pt}%
\pgfsys@defobject{currentmarker}{\pgfqpoint{-0.055556in}{0.000000in}}{\pgfqpoint{0.000000in}{0.000000in}}{%
\pgfpathmoveto{\pgfqpoint{0.000000in}{0.000000in}}%
\pgfpathlineto{\pgfqpoint{-0.055556in}{0.000000in}}%
\pgfusepath{stroke,fill}%
}%
\begin{pgfscope}%
\pgfsys@transformshift{4.275000in}{0.232273in}%
\pgfsys@useobject{currentmarker}{}%
\end{pgfscope}%
\end{pgfscope}%
\begin{pgfscope}%
\pgftext[x=0.169444in,y=0.232273in,right,]{\rmfamily\fontsize{9.000000}{10.800000}\selectfont \(\displaystyle -1.5\)}%
\end{pgfscope}%
\begin{pgfscope}%
\pgfsetbuttcap%
\pgfsetroundjoin%
\definecolor{currentfill}{rgb}{0.000000,0.000000,0.000000}%
\pgfsetfillcolor{currentfill}%
\pgfsetlinewidth{0.501875pt}%
\definecolor{currentstroke}{rgb}{0.000000,0.000000,0.000000}%
\pgfsetstrokecolor{currentstroke}%
\pgfsetdash{}{0pt}%
\pgfsys@defobject{currentmarker}{\pgfqpoint{0.000000in}{0.000000in}}{\pgfqpoint{0.055556in}{0.000000in}}{%
\pgfpathmoveto{\pgfqpoint{0.000000in}{0.000000in}}%
\pgfpathlineto{\pgfqpoint{0.055556in}{0.000000in}}%
\pgfusepath{stroke,fill}%
}%
\begin{pgfscope}%
\pgfsys@transformshift{0.225000in}{0.423182in}%
\pgfsys@useobject{currentmarker}{}%
\end{pgfscope}%
\end{pgfscope}%
\begin{pgfscope}%
\pgfsetbuttcap%
\pgfsetroundjoin%
\definecolor{currentfill}{rgb}{0.000000,0.000000,0.000000}%
\pgfsetfillcolor{currentfill}%
\pgfsetlinewidth{0.501875pt}%
\definecolor{currentstroke}{rgb}{0.000000,0.000000,0.000000}%
\pgfsetstrokecolor{currentstroke}%
\pgfsetdash{}{0pt}%
\pgfsys@defobject{currentmarker}{\pgfqpoint{-0.055556in}{0.000000in}}{\pgfqpoint{0.000000in}{0.000000in}}{%
\pgfpathmoveto{\pgfqpoint{0.000000in}{0.000000in}}%
\pgfpathlineto{\pgfqpoint{-0.055556in}{0.000000in}}%
\pgfusepath{stroke,fill}%
}%
\begin{pgfscope}%
\pgfsys@transformshift{4.275000in}{0.423182in}%
\pgfsys@useobject{currentmarker}{}%
\end{pgfscope}%
\end{pgfscope}%
\begin{pgfscope}%
\pgftext[x=0.169444in,y=0.423182in,right,]{\rmfamily\fontsize{9.000000}{10.800000}\selectfont \(\displaystyle -1.0\)}%
\end{pgfscope}%
\begin{pgfscope}%
\pgfsetbuttcap%
\pgfsetroundjoin%
\definecolor{currentfill}{rgb}{0.000000,0.000000,0.000000}%
\pgfsetfillcolor{currentfill}%
\pgfsetlinewidth{0.501875pt}%
\definecolor{currentstroke}{rgb}{0.000000,0.000000,0.000000}%
\pgfsetstrokecolor{currentstroke}%
\pgfsetdash{}{0pt}%
\pgfsys@defobject{currentmarker}{\pgfqpoint{0.000000in}{0.000000in}}{\pgfqpoint{0.055556in}{0.000000in}}{%
\pgfpathmoveto{\pgfqpoint{0.000000in}{0.000000in}}%
\pgfpathlineto{\pgfqpoint{0.055556in}{0.000000in}}%
\pgfusepath{stroke,fill}%
}%
\begin{pgfscope}%
\pgfsys@transformshift{0.225000in}{0.614091in}%
\pgfsys@useobject{currentmarker}{}%
\end{pgfscope}%
\end{pgfscope}%
\begin{pgfscope}%
\pgfsetbuttcap%
\pgfsetroundjoin%
\definecolor{currentfill}{rgb}{0.000000,0.000000,0.000000}%
\pgfsetfillcolor{currentfill}%
\pgfsetlinewidth{0.501875pt}%
\definecolor{currentstroke}{rgb}{0.000000,0.000000,0.000000}%
\pgfsetstrokecolor{currentstroke}%
\pgfsetdash{}{0pt}%
\pgfsys@defobject{currentmarker}{\pgfqpoint{-0.055556in}{0.000000in}}{\pgfqpoint{0.000000in}{0.000000in}}{%
\pgfpathmoveto{\pgfqpoint{0.000000in}{0.000000in}}%
\pgfpathlineto{\pgfqpoint{-0.055556in}{0.000000in}}%
\pgfusepath{stroke,fill}%
}%
\begin{pgfscope}%
\pgfsys@transformshift{4.275000in}{0.614091in}%
\pgfsys@useobject{currentmarker}{}%
\end{pgfscope}%
\end{pgfscope}%
\begin{pgfscope}%
\pgftext[x=0.169444in,y=0.614091in,right,]{\rmfamily\fontsize{9.000000}{10.800000}\selectfont \(\displaystyle -0.5\)}%
\end{pgfscope}%
\begin{pgfscope}%
\pgfsetbuttcap%
\pgfsetroundjoin%
\definecolor{currentfill}{rgb}{0.000000,0.000000,0.000000}%
\pgfsetfillcolor{currentfill}%
\pgfsetlinewidth{0.501875pt}%
\definecolor{currentstroke}{rgb}{0.000000,0.000000,0.000000}%
\pgfsetstrokecolor{currentstroke}%
\pgfsetdash{}{0pt}%
\pgfsys@defobject{currentmarker}{\pgfqpoint{0.000000in}{0.000000in}}{\pgfqpoint{0.055556in}{0.000000in}}{%
\pgfpathmoveto{\pgfqpoint{0.000000in}{0.000000in}}%
\pgfpathlineto{\pgfqpoint{0.055556in}{0.000000in}}%
\pgfusepath{stroke,fill}%
}%
\begin{pgfscope}%
\pgfsys@transformshift{0.225000in}{0.805000in}%
\pgfsys@useobject{currentmarker}{}%
\end{pgfscope}%
\end{pgfscope}%
\begin{pgfscope}%
\pgfsetbuttcap%
\pgfsetroundjoin%
\definecolor{currentfill}{rgb}{0.000000,0.000000,0.000000}%
\pgfsetfillcolor{currentfill}%
\pgfsetlinewidth{0.501875pt}%
\definecolor{currentstroke}{rgb}{0.000000,0.000000,0.000000}%
\pgfsetstrokecolor{currentstroke}%
\pgfsetdash{}{0pt}%
\pgfsys@defobject{currentmarker}{\pgfqpoint{-0.055556in}{0.000000in}}{\pgfqpoint{0.000000in}{0.000000in}}{%
\pgfpathmoveto{\pgfqpoint{0.000000in}{0.000000in}}%
\pgfpathlineto{\pgfqpoint{-0.055556in}{0.000000in}}%
\pgfusepath{stroke,fill}%
}%
\begin{pgfscope}%
\pgfsys@transformshift{4.275000in}{0.805000in}%
\pgfsys@useobject{currentmarker}{}%
\end{pgfscope}%
\end{pgfscope}%
\begin{pgfscope}%
\pgftext[x=0.169444in,y=0.805000in,right,]{\rmfamily\fontsize{9.000000}{10.800000}\selectfont \(\displaystyle 0.0\)}%
\end{pgfscope}%
\begin{pgfscope}%
\pgfsetbuttcap%
\pgfsetroundjoin%
\definecolor{currentfill}{rgb}{0.000000,0.000000,0.000000}%
\pgfsetfillcolor{currentfill}%
\pgfsetlinewidth{0.501875pt}%
\definecolor{currentstroke}{rgb}{0.000000,0.000000,0.000000}%
\pgfsetstrokecolor{currentstroke}%
\pgfsetdash{}{0pt}%
\pgfsys@defobject{currentmarker}{\pgfqpoint{0.000000in}{0.000000in}}{\pgfqpoint{0.055556in}{0.000000in}}{%
\pgfpathmoveto{\pgfqpoint{0.000000in}{0.000000in}}%
\pgfpathlineto{\pgfqpoint{0.055556in}{0.000000in}}%
\pgfusepath{stroke,fill}%
}%
\begin{pgfscope}%
\pgfsys@transformshift{0.225000in}{0.995909in}%
\pgfsys@useobject{currentmarker}{}%
\end{pgfscope}%
\end{pgfscope}%
\begin{pgfscope}%
\pgfsetbuttcap%
\pgfsetroundjoin%
\definecolor{currentfill}{rgb}{0.000000,0.000000,0.000000}%
\pgfsetfillcolor{currentfill}%
\pgfsetlinewidth{0.501875pt}%
\definecolor{currentstroke}{rgb}{0.000000,0.000000,0.000000}%
\pgfsetstrokecolor{currentstroke}%
\pgfsetdash{}{0pt}%
\pgfsys@defobject{currentmarker}{\pgfqpoint{-0.055556in}{0.000000in}}{\pgfqpoint{0.000000in}{0.000000in}}{%
\pgfpathmoveto{\pgfqpoint{0.000000in}{0.000000in}}%
\pgfpathlineto{\pgfqpoint{-0.055556in}{0.000000in}}%
\pgfusepath{stroke,fill}%
}%
\begin{pgfscope}%
\pgfsys@transformshift{4.275000in}{0.995909in}%
\pgfsys@useobject{currentmarker}{}%
\end{pgfscope}%
\end{pgfscope}%
\begin{pgfscope}%
\pgftext[x=0.169444in,y=0.995909in,right,]{\rmfamily\fontsize{9.000000}{10.800000}\selectfont \(\displaystyle 0.5\)}%
\end{pgfscope}%
\begin{pgfscope}%
\pgfsetbuttcap%
\pgfsetroundjoin%
\definecolor{currentfill}{rgb}{0.000000,0.000000,0.000000}%
\pgfsetfillcolor{currentfill}%
\pgfsetlinewidth{0.501875pt}%
\definecolor{currentstroke}{rgb}{0.000000,0.000000,0.000000}%
\pgfsetstrokecolor{currentstroke}%
\pgfsetdash{}{0pt}%
\pgfsys@defobject{currentmarker}{\pgfqpoint{0.000000in}{0.000000in}}{\pgfqpoint{0.055556in}{0.000000in}}{%
\pgfpathmoveto{\pgfqpoint{0.000000in}{0.000000in}}%
\pgfpathlineto{\pgfqpoint{0.055556in}{0.000000in}}%
\pgfusepath{stroke,fill}%
}%
\begin{pgfscope}%
\pgfsys@transformshift{0.225000in}{1.186818in}%
\pgfsys@useobject{currentmarker}{}%
\end{pgfscope}%
\end{pgfscope}%
\begin{pgfscope}%
\pgfsetbuttcap%
\pgfsetroundjoin%
\definecolor{currentfill}{rgb}{0.000000,0.000000,0.000000}%
\pgfsetfillcolor{currentfill}%
\pgfsetlinewidth{0.501875pt}%
\definecolor{currentstroke}{rgb}{0.000000,0.000000,0.000000}%
\pgfsetstrokecolor{currentstroke}%
\pgfsetdash{}{0pt}%
\pgfsys@defobject{currentmarker}{\pgfqpoint{-0.055556in}{0.000000in}}{\pgfqpoint{0.000000in}{0.000000in}}{%
\pgfpathmoveto{\pgfqpoint{0.000000in}{0.000000in}}%
\pgfpathlineto{\pgfqpoint{-0.055556in}{0.000000in}}%
\pgfusepath{stroke,fill}%
}%
\begin{pgfscope}%
\pgfsys@transformshift{4.275000in}{1.186818in}%
\pgfsys@useobject{currentmarker}{}%
\end{pgfscope}%
\end{pgfscope}%
\begin{pgfscope}%
\pgftext[x=0.169444in,y=1.186818in,right,]{\rmfamily\fontsize{9.000000}{10.800000}\selectfont \(\displaystyle 1.0\)}%
\end{pgfscope}%
\begin{pgfscope}%
\pgfsetbuttcap%
\pgfsetroundjoin%
\definecolor{currentfill}{rgb}{0.000000,0.000000,0.000000}%
\pgfsetfillcolor{currentfill}%
\pgfsetlinewidth{0.501875pt}%
\definecolor{currentstroke}{rgb}{0.000000,0.000000,0.000000}%
\pgfsetstrokecolor{currentstroke}%
\pgfsetdash{}{0pt}%
\pgfsys@defobject{currentmarker}{\pgfqpoint{0.000000in}{0.000000in}}{\pgfqpoint{0.055556in}{0.000000in}}{%
\pgfpathmoveto{\pgfqpoint{0.000000in}{0.000000in}}%
\pgfpathlineto{\pgfqpoint{0.055556in}{0.000000in}}%
\pgfusepath{stroke,fill}%
}%
\begin{pgfscope}%
\pgfsys@transformshift{0.225000in}{1.377727in}%
\pgfsys@useobject{currentmarker}{}%
\end{pgfscope}%
\end{pgfscope}%
\begin{pgfscope}%
\pgfsetbuttcap%
\pgfsetroundjoin%
\definecolor{currentfill}{rgb}{0.000000,0.000000,0.000000}%
\pgfsetfillcolor{currentfill}%
\pgfsetlinewidth{0.501875pt}%
\definecolor{currentstroke}{rgb}{0.000000,0.000000,0.000000}%
\pgfsetstrokecolor{currentstroke}%
\pgfsetdash{}{0pt}%
\pgfsys@defobject{currentmarker}{\pgfqpoint{-0.055556in}{0.000000in}}{\pgfqpoint{0.000000in}{0.000000in}}{%
\pgfpathmoveto{\pgfqpoint{0.000000in}{0.000000in}}%
\pgfpathlineto{\pgfqpoint{-0.055556in}{0.000000in}}%
\pgfusepath{stroke,fill}%
}%
\begin{pgfscope}%
\pgfsys@transformshift{4.275000in}{1.377727in}%
\pgfsys@useobject{currentmarker}{}%
\end{pgfscope}%
\end{pgfscope}%
\begin{pgfscope}%
\pgftext[x=0.169444in,y=1.377727in,right,]{\rmfamily\fontsize{9.000000}{10.800000}\selectfont \(\displaystyle 1.5\)}%
\end{pgfscope}%
\begin{pgfscope}%
\pgftext[x=2.250000in,y=1.504444in,,base]{\rmfamily\fontsize{11.000000}{13.200000}\selectfont Moduliertes Signal}%
\end{pgfscope}%
\end{pgfpicture}%
\makeatother%
\endgroup%

    \caption{%
        \emph{Frequency-shift  keying}: Oben   sind  die   zu  \"ubertragenden
        digitalen  Daten  als  \code{1}  und \code{0}  abgebildet,  unten  das
        zugeh\"orige Verhalten des  modulierten Signals. Dieses Signal w\"urde
        bei unserem System auf eine \SI{960}{\volt}-DC-Spannung aufmoduliert.%
    }
    \label{fig:fsk:concept}
\end{figure}

\myfancybreak

\emph{Amplitude-shift  keying}  (\emph{Amplitudenumtastung}): Statt  diskreten
Frequenzen werden in diesem  Fall diskrete Amplituden als Informationstr\"ager
benutzt. Die   Umsetzung    w\"urde   darauf   basieren,    dass   kurzfristig
(mit   Frequenzen   im  Bereich   von   einigen   Kilohertz)  ein   Solarmodul
\todo{Modul? Zelle? String?} in einem  String (Serieschaltung) kurzgeschlossen
w\"urde. Dies w\"urde die  auf der Leitung anliegende Spannung  in dieser Zeit
um  den Spannungsanteil  des kurzgeschlossenen  Moduls einbrechen  lassen. Die
Einbr\"uche  w\"urden  als  Informationstr\"ager   fungieren  und  im  \Master
ausgewertet.

Grunds\"atzlich handelt es  sich beim aufmodulierten Signal  um \emph{ASK} mit
zwei Amplituden: \SI{0}{\volt}  (wenn kein  Modul kurzgeschlossen ist  und ein
reiner Gleichstrom \"uber  die Leitung fliesst) und  einer negativen Amplitude
im Umfang der Spannung des kurzgeschlossenen Modules.

F\"ur  einen kurzen  Einblick ins  Thema  sei auch  an dieser  Stelle auf  den
entsprechenden Artikel auf Wikipedia verwiesen \cite{ref:ask:wikipedia}.

Abbildung~\ref{fig:ask:concept} stellt das  Konzept des Verfahrens schematisch
dar. \todo{Achsen: Einheiten}

\begin{figure}[h!tb]
    \centering
    %% Creator: Matplotlib, PGF backend
%%
%% To include the figure in your LaTeX document, write
%%   \input{<filename>.pgf}
%%
%% Make sure the required packages are loaded in your preamble
%%   \usepackage{pgf}
%%
%% Figures using additional raster images can only be included by \input if
%% they are in the same directory as the main LaTeX file. For loading figures
%% from other directories you can use the `import` package
%%   \usepackage{import}
%% and then include the figures with
%%   \import{<path to file>}{<filename>.pgf}
%%
%% Matplotlib used the following preamble
%%   \usepackage{fontspec}
%%   \setmainfont{Bitstream Vera Serif}
%%   \setsansfont{Bitstream Vera Sans}
%%   \setmonofont{Bitstream Vera Sans Mono}
%%
\begingroup%
\makeatletter%
\begin{pgfpicture}%
\pgfpathrectangle{\pgfpointorigin}{\pgfqpoint{4.500000in}{3.500000in}}%
\pgfusepath{use as bounding box, clip}%
\begin{pgfscope}%
\pgfsetbuttcap%
\pgfsetmiterjoin%
\pgfsetlinewidth{0.000000pt}%
\definecolor{currentstroke}{rgb}{0.000000,0.000000,0.000000}%
\pgfsetstrokecolor{currentstroke}%
\pgfsetstrokeopacity{0.000000}%
\pgfsetdash{}{0pt}%
\pgfpathmoveto{\pgfqpoint{0.000000in}{0.000000in}}%
\pgfpathlineto{\pgfqpoint{4.500000in}{0.000000in}}%
\pgfpathlineto{\pgfqpoint{4.500000in}{3.500000in}}%
\pgfpathlineto{\pgfqpoint{0.000000in}{3.500000in}}%
\pgfpathclose%
\pgfusepath{}%
\end{pgfscope}%
\begin{pgfscope}%
\pgfsetbuttcap%
\pgfsetmiterjoin%
\pgfsetlinewidth{0.000000pt}%
\definecolor{currentstroke}{rgb}{0.000000,0.000000,0.000000}%
\pgfsetstrokecolor{currentstroke}%
\pgfsetstrokeopacity{0.000000}%
\pgfsetdash{}{0pt}%
\pgfpathmoveto{\pgfqpoint{0.225000in}{2.065000in}}%
\pgfpathlineto{\pgfqpoint{4.275000in}{2.065000in}}%
\pgfpathlineto{\pgfqpoint{4.275000in}{3.325000in}}%
\pgfpathlineto{\pgfqpoint{0.225000in}{3.325000in}}%
\pgfpathclose%
\pgfusepath{}%
\end{pgfscope}%
\begin{pgfscope}%
\pgfpathrectangle{\pgfqpoint{0.225000in}{2.065000in}}{\pgfqpoint{4.050000in}{1.260000in}} %
\pgfusepath{clip}%
\pgfsetrectcap%
\pgfsetroundjoin%
\pgfsetlinewidth{1.003750pt}%
\definecolor{currentstroke}{rgb}{0.000000,0.000000,1.000000}%
\pgfsetstrokecolor{currentstroke}%
\pgfsetdash{}{0pt}%
\pgfpathmoveto{\pgfqpoint{0.225000in}{2.170000in}}%
\pgfpathlineto{\pgfqpoint{1.188898in}{2.170000in}}%
\pgfusepath{stroke}%
\end{pgfscope}%
\begin{pgfscope}%
\pgfpathrectangle{\pgfqpoint{0.225000in}{2.065000in}}{\pgfqpoint{4.050000in}{1.260000in}} %
\pgfusepath{clip}%
\pgfsetrectcap%
\pgfsetroundjoin%
\pgfsetlinewidth{1.003750pt}%
\definecolor{currentstroke}{rgb}{0.501961,0.501961,0.501961}%
\pgfsetstrokecolor{currentstroke}%
\pgfsetdash{}{0pt}%
\pgfpathmoveto{\pgfqpoint{1.188898in}{2.170000in}}%
\pgfpathlineto{\pgfqpoint{1.188898in}{3.220000in}}%
\pgfusepath{stroke}%
\end{pgfscope}%
\begin{pgfscope}%
\pgfpathrectangle{\pgfqpoint{0.225000in}{2.065000in}}{\pgfqpoint{4.050000in}{1.260000in}} %
\pgfusepath{clip}%
\pgfsetrectcap%
\pgfsetroundjoin%
\pgfsetlinewidth{1.003750pt}%
\definecolor{currentstroke}{rgb}{1.000000,0.000000,1.000000}%
\pgfsetstrokecolor{currentstroke}%
\pgfsetdash{}{0pt}%
\pgfpathmoveto{\pgfqpoint{1.188898in}{3.220000in}}%
\pgfpathlineto{\pgfqpoint{2.152795in}{3.220000in}}%
\pgfusepath{stroke}%
\end{pgfscope}%
\begin{pgfscope}%
\pgfpathrectangle{\pgfqpoint{0.225000in}{2.065000in}}{\pgfqpoint{4.050000in}{1.260000in}} %
\pgfusepath{clip}%
\pgfsetrectcap%
\pgfsetroundjoin%
\pgfsetlinewidth{1.003750pt}%
\definecolor{currentstroke}{rgb}{0.501961,0.501961,0.501961}%
\pgfsetstrokecolor{currentstroke}%
\pgfsetdash{}{0pt}%
\pgfpathmoveto{\pgfqpoint{2.152795in}{3.220000in}}%
\pgfpathlineto{\pgfqpoint{2.152795in}{2.170000in}}%
\pgfusepath{stroke}%
\end{pgfscope}%
\begin{pgfscope}%
\pgfpathrectangle{\pgfqpoint{0.225000in}{2.065000in}}{\pgfqpoint{4.050000in}{1.260000in}} %
\pgfusepath{clip}%
\pgfsetrectcap%
\pgfsetroundjoin%
\pgfsetlinewidth{1.003750pt}%
\definecolor{currentstroke}{rgb}{0.000000,0.000000,1.000000}%
\pgfsetstrokecolor{currentstroke}%
\pgfsetdash{}{0pt}%
\pgfpathmoveto{\pgfqpoint{2.152795in}{2.170000in}}%
\pgfpathlineto{\pgfqpoint{3.116693in}{2.170000in}}%
\pgfusepath{stroke}%
\end{pgfscope}%
\begin{pgfscope}%
\pgfpathrectangle{\pgfqpoint{0.225000in}{2.065000in}}{\pgfqpoint{4.050000in}{1.260000in}} %
\pgfusepath{clip}%
\pgfsetrectcap%
\pgfsetroundjoin%
\pgfsetlinewidth{1.003750pt}%
\definecolor{currentstroke}{rgb}{0.501961,0.501961,0.501961}%
\pgfsetstrokecolor{currentstroke}%
\pgfsetdash{}{0pt}%
\pgfpathmoveto{\pgfqpoint{3.116693in}{2.170000in}}%
\pgfpathlineto{\pgfqpoint{3.116693in}{3.220000in}}%
\pgfusepath{stroke}%
\end{pgfscope}%
\begin{pgfscope}%
\pgfpathrectangle{\pgfqpoint{0.225000in}{2.065000in}}{\pgfqpoint{4.050000in}{1.260000in}} %
\pgfusepath{clip}%
\pgfsetrectcap%
\pgfsetroundjoin%
\pgfsetlinewidth{1.003750pt}%
\definecolor{currentstroke}{rgb}{1.000000,0.000000,1.000000}%
\pgfsetstrokecolor{currentstroke}%
\pgfsetdash{}{0pt}%
\pgfpathmoveto{\pgfqpoint{3.116693in}{3.220000in}}%
\pgfpathlineto{\pgfqpoint{4.080591in}{3.220000in}}%
\pgfusepath{stroke}%
\end{pgfscope}%
\begin{pgfscope}%
\pgfsetrectcap%
\pgfsetmiterjoin%
\pgfsetlinewidth{1.003750pt}%
\definecolor{currentstroke}{rgb}{0.000000,0.000000,0.000000}%
\pgfsetstrokecolor{currentstroke}%
\pgfsetdash{}{0pt}%
\pgfpathmoveto{\pgfqpoint{0.225000in}{2.065000in}}%
\pgfpathlineto{\pgfqpoint{0.225000in}{3.325000in}}%
\pgfusepath{stroke}%
\end{pgfscope}%
\begin{pgfscope}%
\pgfsetrectcap%
\pgfsetmiterjoin%
\pgfsetlinewidth{1.003750pt}%
\definecolor{currentstroke}{rgb}{0.000000,0.000000,0.000000}%
\pgfsetstrokecolor{currentstroke}%
\pgfsetdash{}{0pt}%
\pgfpathmoveto{\pgfqpoint{0.225000in}{3.325000in}}%
\pgfpathlineto{\pgfqpoint{4.275000in}{3.325000in}}%
\pgfusepath{stroke}%
\end{pgfscope}%
\begin{pgfscope}%
\pgfsetrectcap%
\pgfsetmiterjoin%
\pgfsetlinewidth{1.003750pt}%
\definecolor{currentstroke}{rgb}{0.000000,0.000000,0.000000}%
\pgfsetstrokecolor{currentstroke}%
\pgfsetdash{}{0pt}%
\pgfpathmoveto{\pgfqpoint{4.275000in}{2.065000in}}%
\pgfpathlineto{\pgfqpoint{4.275000in}{3.325000in}}%
\pgfusepath{stroke}%
\end{pgfscope}%
\begin{pgfscope}%
\pgfsetrectcap%
\pgfsetmiterjoin%
\pgfsetlinewidth{1.003750pt}%
\definecolor{currentstroke}{rgb}{0.000000,0.000000,0.000000}%
\pgfsetstrokecolor{currentstroke}%
\pgfsetdash{}{0pt}%
\pgfpathmoveto{\pgfqpoint{0.225000in}{2.065000in}}%
\pgfpathlineto{\pgfqpoint{4.275000in}{2.065000in}}%
\pgfusepath{stroke}%
\end{pgfscope}%
\begin{pgfscope}%
\pgfsetbuttcap%
\pgfsetroundjoin%
\definecolor{currentfill}{rgb}{0.000000,0.000000,0.000000}%
\pgfsetfillcolor{currentfill}%
\pgfsetlinewidth{0.501875pt}%
\definecolor{currentstroke}{rgb}{0.000000,0.000000,0.000000}%
\pgfsetstrokecolor{currentstroke}%
\pgfsetdash{}{0pt}%
\pgfsys@defobject{currentmarker}{\pgfqpoint{0.000000in}{0.000000in}}{\pgfqpoint{0.000000in}{0.055556in}}{%
\pgfpathmoveto{\pgfqpoint{0.000000in}{0.000000in}}%
\pgfpathlineto{\pgfqpoint{0.000000in}{0.055556in}}%
\pgfusepath{stroke,fill}%
}%
\begin{pgfscope}%
\pgfsys@transformshift{0.225000in}{2.065000in}%
\pgfsys@useobject{currentmarker}{}%
\end{pgfscope}%
\end{pgfscope}%
\begin{pgfscope}%
\pgfsetbuttcap%
\pgfsetroundjoin%
\definecolor{currentfill}{rgb}{0.000000,0.000000,0.000000}%
\pgfsetfillcolor{currentfill}%
\pgfsetlinewidth{0.501875pt}%
\definecolor{currentstroke}{rgb}{0.000000,0.000000,0.000000}%
\pgfsetstrokecolor{currentstroke}%
\pgfsetdash{}{0pt}%
\pgfsys@defobject{currentmarker}{\pgfqpoint{0.000000in}{-0.055556in}}{\pgfqpoint{0.000000in}{0.000000in}}{%
\pgfpathmoveto{\pgfqpoint{0.000000in}{0.000000in}}%
\pgfpathlineto{\pgfqpoint{0.000000in}{-0.055556in}}%
\pgfusepath{stroke,fill}%
}%
\begin{pgfscope}%
\pgfsys@transformshift{0.225000in}{3.325000in}%
\pgfsys@useobject{currentmarker}{}%
\end{pgfscope}%
\end{pgfscope}%
\begin{pgfscope}%
\pgftext[x=0.225000in,y=2.009444in,,top]{\rmfamily\fontsize{9.000000}{10.800000}\selectfont \(\displaystyle 0.0000\)}%
\end{pgfscope}%
\begin{pgfscope}%
\pgfsetbuttcap%
\pgfsetroundjoin%
\definecolor{currentfill}{rgb}{0.000000,0.000000,0.000000}%
\pgfsetfillcolor{currentfill}%
\pgfsetlinewidth{0.501875pt}%
\definecolor{currentstroke}{rgb}{0.000000,0.000000,0.000000}%
\pgfsetstrokecolor{currentstroke}%
\pgfsetdash{}{0pt}%
\pgfsys@defobject{currentmarker}{\pgfqpoint{0.000000in}{0.000000in}}{\pgfqpoint{0.000000in}{0.055556in}}{%
\pgfpathmoveto{\pgfqpoint{0.000000in}{0.000000in}}%
\pgfpathlineto{\pgfqpoint{0.000000in}{0.055556in}}%
\pgfusepath{stroke,fill}%
}%
\begin{pgfscope}%
\pgfsys@transformshift{0.731250in}{2.065000in}%
\pgfsys@useobject{currentmarker}{}%
\end{pgfscope}%
\end{pgfscope}%
\begin{pgfscope}%
\pgfsetbuttcap%
\pgfsetroundjoin%
\definecolor{currentfill}{rgb}{0.000000,0.000000,0.000000}%
\pgfsetfillcolor{currentfill}%
\pgfsetlinewidth{0.501875pt}%
\definecolor{currentstroke}{rgb}{0.000000,0.000000,0.000000}%
\pgfsetstrokecolor{currentstroke}%
\pgfsetdash{}{0pt}%
\pgfsys@defobject{currentmarker}{\pgfqpoint{0.000000in}{-0.055556in}}{\pgfqpoint{0.000000in}{0.000000in}}{%
\pgfpathmoveto{\pgfqpoint{0.000000in}{0.000000in}}%
\pgfpathlineto{\pgfqpoint{0.000000in}{-0.055556in}}%
\pgfusepath{stroke,fill}%
}%
\begin{pgfscope}%
\pgfsys@transformshift{0.731250in}{3.325000in}%
\pgfsys@useobject{currentmarker}{}%
\end{pgfscope}%
\end{pgfscope}%
\begin{pgfscope}%
\pgftext[x=0.731250in,y=2.009444in,,top]{\rmfamily\fontsize{9.000000}{10.800000}\selectfont \(\displaystyle 0.0002\)}%
\end{pgfscope}%
\begin{pgfscope}%
\pgfsetbuttcap%
\pgfsetroundjoin%
\definecolor{currentfill}{rgb}{0.000000,0.000000,0.000000}%
\pgfsetfillcolor{currentfill}%
\pgfsetlinewidth{0.501875pt}%
\definecolor{currentstroke}{rgb}{0.000000,0.000000,0.000000}%
\pgfsetstrokecolor{currentstroke}%
\pgfsetdash{}{0pt}%
\pgfsys@defobject{currentmarker}{\pgfqpoint{0.000000in}{0.000000in}}{\pgfqpoint{0.000000in}{0.055556in}}{%
\pgfpathmoveto{\pgfqpoint{0.000000in}{0.000000in}}%
\pgfpathlineto{\pgfqpoint{0.000000in}{0.055556in}}%
\pgfusepath{stroke,fill}%
}%
\begin{pgfscope}%
\pgfsys@transformshift{1.237500in}{2.065000in}%
\pgfsys@useobject{currentmarker}{}%
\end{pgfscope}%
\end{pgfscope}%
\begin{pgfscope}%
\pgfsetbuttcap%
\pgfsetroundjoin%
\definecolor{currentfill}{rgb}{0.000000,0.000000,0.000000}%
\pgfsetfillcolor{currentfill}%
\pgfsetlinewidth{0.501875pt}%
\definecolor{currentstroke}{rgb}{0.000000,0.000000,0.000000}%
\pgfsetstrokecolor{currentstroke}%
\pgfsetdash{}{0pt}%
\pgfsys@defobject{currentmarker}{\pgfqpoint{0.000000in}{-0.055556in}}{\pgfqpoint{0.000000in}{0.000000in}}{%
\pgfpathmoveto{\pgfqpoint{0.000000in}{0.000000in}}%
\pgfpathlineto{\pgfqpoint{0.000000in}{-0.055556in}}%
\pgfusepath{stroke,fill}%
}%
\begin{pgfscope}%
\pgfsys@transformshift{1.237500in}{3.325000in}%
\pgfsys@useobject{currentmarker}{}%
\end{pgfscope}%
\end{pgfscope}%
\begin{pgfscope}%
\pgftext[x=1.237500in,y=2.009444in,,top]{\rmfamily\fontsize{9.000000}{10.800000}\selectfont \(\displaystyle 0.0004\)}%
\end{pgfscope}%
\begin{pgfscope}%
\pgfsetbuttcap%
\pgfsetroundjoin%
\definecolor{currentfill}{rgb}{0.000000,0.000000,0.000000}%
\pgfsetfillcolor{currentfill}%
\pgfsetlinewidth{0.501875pt}%
\definecolor{currentstroke}{rgb}{0.000000,0.000000,0.000000}%
\pgfsetstrokecolor{currentstroke}%
\pgfsetdash{}{0pt}%
\pgfsys@defobject{currentmarker}{\pgfqpoint{0.000000in}{0.000000in}}{\pgfqpoint{0.000000in}{0.055556in}}{%
\pgfpathmoveto{\pgfqpoint{0.000000in}{0.000000in}}%
\pgfpathlineto{\pgfqpoint{0.000000in}{0.055556in}}%
\pgfusepath{stroke,fill}%
}%
\begin{pgfscope}%
\pgfsys@transformshift{1.743750in}{2.065000in}%
\pgfsys@useobject{currentmarker}{}%
\end{pgfscope}%
\end{pgfscope}%
\begin{pgfscope}%
\pgfsetbuttcap%
\pgfsetroundjoin%
\definecolor{currentfill}{rgb}{0.000000,0.000000,0.000000}%
\pgfsetfillcolor{currentfill}%
\pgfsetlinewidth{0.501875pt}%
\definecolor{currentstroke}{rgb}{0.000000,0.000000,0.000000}%
\pgfsetstrokecolor{currentstroke}%
\pgfsetdash{}{0pt}%
\pgfsys@defobject{currentmarker}{\pgfqpoint{0.000000in}{-0.055556in}}{\pgfqpoint{0.000000in}{0.000000in}}{%
\pgfpathmoveto{\pgfqpoint{0.000000in}{0.000000in}}%
\pgfpathlineto{\pgfqpoint{0.000000in}{-0.055556in}}%
\pgfusepath{stroke,fill}%
}%
\begin{pgfscope}%
\pgfsys@transformshift{1.743750in}{3.325000in}%
\pgfsys@useobject{currentmarker}{}%
\end{pgfscope}%
\end{pgfscope}%
\begin{pgfscope}%
\pgftext[x=1.743750in,y=2.009444in,,top]{\rmfamily\fontsize{9.000000}{10.800000}\selectfont \(\displaystyle 0.0006\)}%
\end{pgfscope}%
\begin{pgfscope}%
\pgfsetbuttcap%
\pgfsetroundjoin%
\definecolor{currentfill}{rgb}{0.000000,0.000000,0.000000}%
\pgfsetfillcolor{currentfill}%
\pgfsetlinewidth{0.501875pt}%
\definecolor{currentstroke}{rgb}{0.000000,0.000000,0.000000}%
\pgfsetstrokecolor{currentstroke}%
\pgfsetdash{}{0pt}%
\pgfsys@defobject{currentmarker}{\pgfqpoint{0.000000in}{0.000000in}}{\pgfqpoint{0.000000in}{0.055556in}}{%
\pgfpathmoveto{\pgfqpoint{0.000000in}{0.000000in}}%
\pgfpathlineto{\pgfqpoint{0.000000in}{0.055556in}}%
\pgfusepath{stroke,fill}%
}%
\begin{pgfscope}%
\pgfsys@transformshift{2.250000in}{2.065000in}%
\pgfsys@useobject{currentmarker}{}%
\end{pgfscope}%
\end{pgfscope}%
\begin{pgfscope}%
\pgfsetbuttcap%
\pgfsetroundjoin%
\definecolor{currentfill}{rgb}{0.000000,0.000000,0.000000}%
\pgfsetfillcolor{currentfill}%
\pgfsetlinewidth{0.501875pt}%
\definecolor{currentstroke}{rgb}{0.000000,0.000000,0.000000}%
\pgfsetstrokecolor{currentstroke}%
\pgfsetdash{}{0pt}%
\pgfsys@defobject{currentmarker}{\pgfqpoint{0.000000in}{-0.055556in}}{\pgfqpoint{0.000000in}{0.000000in}}{%
\pgfpathmoveto{\pgfqpoint{0.000000in}{0.000000in}}%
\pgfpathlineto{\pgfqpoint{0.000000in}{-0.055556in}}%
\pgfusepath{stroke,fill}%
}%
\begin{pgfscope}%
\pgfsys@transformshift{2.250000in}{3.325000in}%
\pgfsys@useobject{currentmarker}{}%
\end{pgfscope}%
\end{pgfscope}%
\begin{pgfscope}%
\pgftext[x=2.250000in,y=2.009444in,,top]{\rmfamily\fontsize{9.000000}{10.800000}\selectfont \(\displaystyle 0.0008\)}%
\end{pgfscope}%
\begin{pgfscope}%
\pgfsetbuttcap%
\pgfsetroundjoin%
\definecolor{currentfill}{rgb}{0.000000,0.000000,0.000000}%
\pgfsetfillcolor{currentfill}%
\pgfsetlinewidth{0.501875pt}%
\definecolor{currentstroke}{rgb}{0.000000,0.000000,0.000000}%
\pgfsetstrokecolor{currentstroke}%
\pgfsetdash{}{0pt}%
\pgfsys@defobject{currentmarker}{\pgfqpoint{0.000000in}{0.000000in}}{\pgfqpoint{0.000000in}{0.055556in}}{%
\pgfpathmoveto{\pgfqpoint{0.000000in}{0.000000in}}%
\pgfpathlineto{\pgfqpoint{0.000000in}{0.055556in}}%
\pgfusepath{stroke,fill}%
}%
\begin{pgfscope}%
\pgfsys@transformshift{2.756250in}{2.065000in}%
\pgfsys@useobject{currentmarker}{}%
\end{pgfscope}%
\end{pgfscope}%
\begin{pgfscope}%
\pgfsetbuttcap%
\pgfsetroundjoin%
\definecolor{currentfill}{rgb}{0.000000,0.000000,0.000000}%
\pgfsetfillcolor{currentfill}%
\pgfsetlinewidth{0.501875pt}%
\definecolor{currentstroke}{rgb}{0.000000,0.000000,0.000000}%
\pgfsetstrokecolor{currentstroke}%
\pgfsetdash{}{0pt}%
\pgfsys@defobject{currentmarker}{\pgfqpoint{0.000000in}{-0.055556in}}{\pgfqpoint{0.000000in}{0.000000in}}{%
\pgfpathmoveto{\pgfqpoint{0.000000in}{0.000000in}}%
\pgfpathlineto{\pgfqpoint{0.000000in}{-0.055556in}}%
\pgfusepath{stroke,fill}%
}%
\begin{pgfscope}%
\pgfsys@transformshift{2.756250in}{3.325000in}%
\pgfsys@useobject{currentmarker}{}%
\end{pgfscope}%
\end{pgfscope}%
\begin{pgfscope}%
\pgftext[x=2.756250in,y=2.009444in,,top]{\rmfamily\fontsize{9.000000}{10.800000}\selectfont \(\displaystyle 0.0010\)}%
\end{pgfscope}%
\begin{pgfscope}%
\pgfsetbuttcap%
\pgfsetroundjoin%
\definecolor{currentfill}{rgb}{0.000000,0.000000,0.000000}%
\pgfsetfillcolor{currentfill}%
\pgfsetlinewidth{0.501875pt}%
\definecolor{currentstroke}{rgb}{0.000000,0.000000,0.000000}%
\pgfsetstrokecolor{currentstroke}%
\pgfsetdash{}{0pt}%
\pgfsys@defobject{currentmarker}{\pgfqpoint{0.000000in}{0.000000in}}{\pgfqpoint{0.000000in}{0.055556in}}{%
\pgfpathmoveto{\pgfqpoint{0.000000in}{0.000000in}}%
\pgfpathlineto{\pgfqpoint{0.000000in}{0.055556in}}%
\pgfusepath{stroke,fill}%
}%
\begin{pgfscope}%
\pgfsys@transformshift{3.262500in}{2.065000in}%
\pgfsys@useobject{currentmarker}{}%
\end{pgfscope}%
\end{pgfscope}%
\begin{pgfscope}%
\pgfsetbuttcap%
\pgfsetroundjoin%
\definecolor{currentfill}{rgb}{0.000000,0.000000,0.000000}%
\pgfsetfillcolor{currentfill}%
\pgfsetlinewidth{0.501875pt}%
\definecolor{currentstroke}{rgb}{0.000000,0.000000,0.000000}%
\pgfsetstrokecolor{currentstroke}%
\pgfsetdash{}{0pt}%
\pgfsys@defobject{currentmarker}{\pgfqpoint{0.000000in}{-0.055556in}}{\pgfqpoint{0.000000in}{0.000000in}}{%
\pgfpathmoveto{\pgfqpoint{0.000000in}{0.000000in}}%
\pgfpathlineto{\pgfqpoint{0.000000in}{-0.055556in}}%
\pgfusepath{stroke,fill}%
}%
\begin{pgfscope}%
\pgfsys@transformshift{3.262500in}{3.325000in}%
\pgfsys@useobject{currentmarker}{}%
\end{pgfscope}%
\end{pgfscope}%
\begin{pgfscope}%
\pgftext[x=3.262500in,y=2.009444in,,top]{\rmfamily\fontsize{9.000000}{10.800000}\selectfont \(\displaystyle 0.0012\)}%
\end{pgfscope}%
\begin{pgfscope}%
\pgfsetbuttcap%
\pgfsetroundjoin%
\definecolor{currentfill}{rgb}{0.000000,0.000000,0.000000}%
\pgfsetfillcolor{currentfill}%
\pgfsetlinewidth{0.501875pt}%
\definecolor{currentstroke}{rgb}{0.000000,0.000000,0.000000}%
\pgfsetstrokecolor{currentstroke}%
\pgfsetdash{}{0pt}%
\pgfsys@defobject{currentmarker}{\pgfqpoint{0.000000in}{0.000000in}}{\pgfqpoint{0.000000in}{0.055556in}}{%
\pgfpathmoveto{\pgfqpoint{0.000000in}{0.000000in}}%
\pgfpathlineto{\pgfqpoint{0.000000in}{0.055556in}}%
\pgfusepath{stroke,fill}%
}%
\begin{pgfscope}%
\pgfsys@transformshift{3.768750in}{2.065000in}%
\pgfsys@useobject{currentmarker}{}%
\end{pgfscope}%
\end{pgfscope}%
\begin{pgfscope}%
\pgfsetbuttcap%
\pgfsetroundjoin%
\definecolor{currentfill}{rgb}{0.000000,0.000000,0.000000}%
\pgfsetfillcolor{currentfill}%
\pgfsetlinewidth{0.501875pt}%
\definecolor{currentstroke}{rgb}{0.000000,0.000000,0.000000}%
\pgfsetstrokecolor{currentstroke}%
\pgfsetdash{}{0pt}%
\pgfsys@defobject{currentmarker}{\pgfqpoint{0.000000in}{-0.055556in}}{\pgfqpoint{0.000000in}{0.000000in}}{%
\pgfpathmoveto{\pgfqpoint{0.000000in}{0.000000in}}%
\pgfpathlineto{\pgfqpoint{0.000000in}{-0.055556in}}%
\pgfusepath{stroke,fill}%
}%
\begin{pgfscope}%
\pgfsys@transformshift{3.768750in}{3.325000in}%
\pgfsys@useobject{currentmarker}{}%
\end{pgfscope}%
\end{pgfscope}%
\begin{pgfscope}%
\pgftext[x=3.768750in,y=2.009444in,,top]{\rmfamily\fontsize{9.000000}{10.800000}\selectfont \(\displaystyle 0.0014\)}%
\end{pgfscope}%
\begin{pgfscope}%
\pgfsetbuttcap%
\pgfsetroundjoin%
\definecolor{currentfill}{rgb}{0.000000,0.000000,0.000000}%
\pgfsetfillcolor{currentfill}%
\pgfsetlinewidth{0.501875pt}%
\definecolor{currentstroke}{rgb}{0.000000,0.000000,0.000000}%
\pgfsetstrokecolor{currentstroke}%
\pgfsetdash{}{0pt}%
\pgfsys@defobject{currentmarker}{\pgfqpoint{0.000000in}{0.000000in}}{\pgfqpoint{0.000000in}{0.055556in}}{%
\pgfpathmoveto{\pgfqpoint{0.000000in}{0.000000in}}%
\pgfpathlineto{\pgfqpoint{0.000000in}{0.055556in}}%
\pgfusepath{stroke,fill}%
}%
\begin{pgfscope}%
\pgfsys@transformshift{4.275000in}{2.065000in}%
\pgfsys@useobject{currentmarker}{}%
\end{pgfscope}%
\end{pgfscope}%
\begin{pgfscope}%
\pgfsetbuttcap%
\pgfsetroundjoin%
\definecolor{currentfill}{rgb}{0.000000,0.000000,0.000000}%
\pgfsetfillcolor{currentfill}%
\pgfsetlinewidth{0.501875pt}%
\definecolor{currentstroke}{rgb}{0.000000,0.000000,0.000000}%
\pgfsetstrokecolor{currentstroke}%
\pgfsetdash{}{0pt}%
\pgfsys@defobject{currentmarker}{\pgfqpoint{0.000000in}{-0.055556in}}{\pgfqpoint{0.000000in}{0.000000in}}{%
\pgfpathmoveto{\pgfqpoint{0.000000in}{0.000000in}}%
\pgfpathlineto{\pgfqpoint{0.000000in}{-0.055556in}}%
\pgfusepath{stroke,fill}%
}%
\begin{pgfscope}%
\pgfsys@transformshift{4.275000in}{3.325000in}%
\pgfsys@useobject{currentmarker}{}%
\end{pgfscope}%
\end{pgfscope}%
\begin{pgfscope}%
\pgftext[x=4.275000in,y=2.009444in,,top]{\rmfamily\fontsize{9.000000}{10.800000}\selectfont \(\displaystyle 0.0016\)}%
\end{pgfscope}%
\begin{pgfscope}%
\pgfsetbuttcap%
\pgfsetroundjoin%
\definecolor{currentfill}{rgb}{0.000000,0.000000,0.000000}%
\pgfsetfillcolor{currentfill}%
\pgfsetlinewidth{0.501875pt}%
\definecolor{currentstroke}{rgb}{0.000000,0.000000,0.000000}%
\pgfsetstrokecolor{currentstroke}%
\pgfsetdash{}{0pt}%
\pgfsys@defobject{currentmarker}{\pgfqpoint{0.000000in}{0.000000in}}{\pgfqpoint{0.055556in}{0.000000in}}{%
\pgfpathmoveto{\pgfqpoint{0.000000in}{0.000000in}}%
\pgfpathlineto{\pgfqpoint{0.055556in}{0.000000in}}%
\pgfusepath{stroke,fill}%
}%
\begin{pgfscope}%
\pgfsys@transformshift{0.225000in}{2.170000in}%
\pgfsys@useobject{currentmarker}{}%
\end{pgfscope}%
\end{pgfscope}%
\begin{pgfscope}%
\pgfsetbuttcap%
\pgfsetroundjoin%
\definecolor{currentfill}{rgb}{0.000000,0.000000,0.000000}%
\pgfsetfillcolor{currentfill}%
\pgfsetlinewidth{0.501875pt}%
\definecolor{currentstroke}{rgb}{0.000000,0.000000,0.000000}%
\pgfsetstrokecolor{currentstroke}%
\pgfsetdash{}{0pt}%
\pgfsys@defobject{currentmarker}{\pgfqpoint{-0.055556in}{0.000000in}}{\pgfqpoint{0.000000in}{0.000000in}}{%
\pgfpathmoveto{\pgfqpoint{0.000000in}{0.000000in}}%
\pgfpathlineto{\pgfqpoint{-0.055556in}{0.000000in}}%
\pgfusepath{stroke,fill}%
}%
\begin{pgfscope}%
\pgfsys@transformshift{4.275000in}{2.170000in}%
\pgfsys@useobject{currentmarker}{}%
\end{pgfscope}%
\end{pgfscope}%
\begin{pgfscope}%
\pgftext[x=0.169444in,y=2.170000in,right,]{\rmfamily\fontsize{9.000000}{10.800000}\selectfont \(\displaystyle 0.0\)}%
\end{pgfscope}%
\begin{pgfscope}%
\pgfsetbuttcap%
\pgfsetroundjoin%
\definecolor{currentfill}{rgb}{0.000000,0.000000,0.000000}%
\pgfsetfillcolor{currentfill}%
\pgfsetlinewidth{0.501875pt}%
\definecolor{currentstroke}{rgb}{0.000000,0.000000,0.000000}%
\pgfsetstrokecolor{currentstroke}%
\pgfsetdash{}{0pt}%
\pgfsys@defobject{currentmarker}{\pgfqpoint{0.000000in}{0.000000in}}{\pgfqpoint{0.055556in}{0.000000in}}{%
\pgfpathmoveto{\pgfqpoint{0.000000in}{0.000000in}}%
\pgfpathlineto{\pgfqpoint{0.055556in}{0.000000in}}%
\pgfusepath{stroke,fill}%
}%
\begin{pgfscope}%
\pgfsys@transformshift{0.225000in}{2.380000in}%
\pgfsys@useobject{currentmarker}{}%
\end{pgfscope}%
\end{pgfscope}%
\begin{pgfscope}%
\pgfsetbuttcap%
\pgfsetroundjoin%
\definecolor{currentfill}{rgb}{0.000000,0.000000,0.000000}%
\pgfsetfillcolor{currentfill}%
\pgfsetlinewidth{0.501875pt}%
\definecolor{currentstroke}{rgb}{0.000000,0.000000,0.000000}%
\pgfsetstrokecolor{currentstroke}%
\pgfsetdash{}{0pt}%
\pgfsys@defobject{currentmarker}{\pgfqpoint{-0.055556in}{0.000000in}}{\pgfqpoint{0.000000in}{0.000000in}}{%
\pgfpathmoveto{\pgfqpoint{0.000000in}{0.000000in}}%
\pgfpathlineto{\pgfqpoint{-0.055556in}{0.000000in}}%
\pgfusepath{stroke,fill}%
}%
\begin{pgfscope}%
\pgfsys@transformshift{4.275000in}{2.380000in}%
\pgfsys@useobject{currentmarker}{}%
\end{pgfscope}%
\end{pgfscope}%
\begin{pgfscope}%
\pgftext[x=0.169444in,y=2.380000in,right,]{\rmfamily\fontsize{9.000000}{10.800000}\selectfont \(\displaystyle 0.2\)}%
\end{pgfscope}%
\begin{pgfscope}%
\pgfsetbuttcap%
\pgfsetroundjoin%
\definecolor{currentfill}{rgb}{0.000000,0.000000,0.000000}%
\pgfsetfillcolor{currentfill}%
\pgfsetlinewidth{0.501875pt}%
\definecolor{currentstroke}{rgb}{0.000000,0.000000,0.000000}%
\pgfsetstrokecolor{currentstroke}%
\pgfsetdash{}{0pt}%
\pgfsys@defobject{currentmarker}{\pgfqpoint{0.000000in}{0.000000in}}{\pgfqpoint{0.055556in}{0.000000in}}{%
\pgfpathmoveto{\pgfqpoint{0.000000in}{0.000000in}}%
\pgfpathlineto{\pgfqpoint{0.055556in}{0.000000in}}%
\pgfusepath{stroke,fill}%
}%
\begin{pgfscope}%
\pgfsys@transformshift{0.225000in}{2.590000in}%
\pgfsys@useobject{currentmarker}{}%
\end{pgfscope}%
\end{pgfscope}%
\begin{pgfscope}%
\pgfsetbuttcap%
\pgfsetroundjoin%
\definecolor{currentfill}{rgb}{0.000000,0.000000,0.000000}%
\pgfsetfillcolor{currentfill}%
\pgfsetlinewidth{0.501875pt}%
\definecolor{currentstroke}{rgb}{0.000000,0.000000,0.000000}%
\pgfsetstrokecolor{currentstroke}%
\pgfsetdash{}{0pt}%
\pgfsys@defobject{currentmarker}{\pgfqpoint{-0.055556in}{0.000000in}}{\pgfqpoint{0.000000in}{0.000000in}}{%
\pgfpathmoveto{\pgfqpoint{0.000000in}{0.000000in}}%
\pgfpathlineto{\pgfqpoint{-0.055556in}{0.000000in}}%
\pgfusepath{stroke,fill}%
}%
\begin{pgfscope}%
\pgfsys@transformshift{4.275000in}{2.590000in}%
\pgfsys@useobject{currentmarker}{}%
\end{pgfscope}%
\end{pgfscope}%
\begin{pgfscope}%
\pgftext[x=0.169444in,y=2.590000in,right,]{\rmfamily\fontsize{9.000000}{10.800000}\selectfont \(\displaystyle 0.4\)}%
\end{pgfscope}%
\begin{pgfscope}%
\pgfsetbuttcap%
\pgfsetroundjoin%
\definecolor{currentfill}{rgb}{0.000000,0.000000,0.000000}%
\pgfsetfillcolor{currentfill}%
\pgfsetlinewidth{0.501875pt}%
\definecolor{currentstroke}{rgb}{0.000000,0.000000,0.000000}%
\pgfsetstrokecolor{currentstroke}%
\pgfsetdash{}{0pt}%
\pgfsys@defobject{currentmarker}{\pgfqpoint{0.000000in}{0.000000in}}{\pgfqpoint{0.055556in}{0.000000in}}{%
\pgfpathmoveto{\pgfqpoint{0.000000in}{0.000000in}}%
\pgfpathlineto{\pgfqpoint{0.055556in}{0.000000in}}%
\pgfusepath{stroke,fill}%
}%
\begin{pgfscope}%
\pgfsys@transformshift{0.225000in}{2.800000in}%
\pgfsys@useobject{currentmarker}{}%
\end{pgfscope}%
\end{pgfscope}%
\begin{pgfscope}%
\pgfsetbuttcap%
\pgfsetroundjoin%
\definecolor{currentfill}{rgb}{0.000000,0.000000,0.000000}%
\pgfsetfillcolor{currentfill}%
\pgfsetlinewidth{0.501875pt}%
\definecolor{currentstroke}{rgb}{0.000000,0.000000,0.000000}%
\pgfsetstrokecolor{currentstroke}%
\pgfsetdash{}{0pt}%
\pgfsys@defobject{currentmarker}{\pgfqpoint{-0.055556in}{0.000000in}}{\pgfqpoint{0.000000in}{0.000000in}}{%
\pgfpathmoveto{\pgfqpoint{0.000000in}{0.000000in}}%
\pgfpathlineto{\pgfqpoint{-0.055556in}{0.000000in}}%
\pgfusepath{stroke,fill}%
}%
\begin{pgfscope}%
\pgfsys@transformshift{4.275000in}{2.800000in}%
\pgfsys@useobject{currentmarker}{}%
\end{pgfscope}%
\end{pgfscope}%
\begin{pgfscope}%
\pgftext[x=0.169444in,y=2.800000in,right,]{\rmfamily\fontsize{9.000000}{10.800000}\selectfont \(\displaystyle 0.6\)}%
\end{pgfscope}%
\begin{pgfscope}%
\pgfsetbuttcap%
\pgfsetroundjoin%
\definecolor{currentfill}{rgb}{0.000000,0.000000,0.000000}%
\pgfsetfillcolor{currentfill}%
\pgfsetlinewidth{0.501875pt}%
\definecolor{currentstroke}{rgb}{0.000000,0.000000,0.000000}%
\pgfsetstrokecolor{currentstroke}%
\pgfsetdash{}{0pt}%
\pgfsys@defobject{currentmarker}{\pgfqpoint{0.000000in}{0.000000in}}{\pgfqpoint{0.055556in}{0.000000in}}{%
\pgfpathmoveto{\pgfqpoint{0.000000in}{0.000000in}}%
\pgfpathlineto{\pgfqpoint{0.055556in}{0.000000in}}%
\pgfusepath{stroke,fill}%
}%
\begin{pgfscope}%
\pgfsys@transformshift{0.225000in}{3.010000in}%
\pgfsys@useobject{currentmarker}{}%
\end{pgfscope}%
\end{pgfscope}%
\begin{pgfscope}%
\pgfsetbuttcap%
\pgfsetroundjoin%
\definecolor{currentfill}{rgb}{0.000000,0.000000,0.000000}%
\pgfsetfillcolor{currentfill}%
\pgfsetlinewidth{0.501875pt}%
\definecolor{currentstroke}{rgb}{0.000000,0.000000,0.000000}%
\pgfsetstrokecolor{currentstroke}%
\pgfsetdash{}{0pt}%
\pgfsys@defobject{currentmarker}{\pgfqpoint{-0.055556in}{0.000000in}}{\pgfqpoint{0.000000in}{0.000000in}}{%
\pgfpathmoveto{\pgfqpoint{0.000000in}{0.000000in}}%
\pgfpathlineto{\pgfqpoint{-0.055556in}{0.000000in}}%
\pgfusepath{stroke,fill}%
}%
\begin{pgfscope}%
\pgfsys@transformshift{4.275000in}{3.010000in}%
\pgfsys@useobject{currentmarker}{}%
\end{pgfscope}%
\end{pgfscope}%
\begin{pgfscope}%
\pgftext[x=0.169444in,y=3.010000in,right,]{\rmfamily\fontsize{9.000000}{10.800000}\selectfont \(\displaystyle 0.8\)}%
\end{pgfscope}%
\begin{pgfscope}%
\pgfsetbuttcap%
\pgfsetroundjoin%
\definecolor{currentfill}{rgb}{0.000000,0.000000,0.000000}%
\pgfsetfillcolor{currentfill}%
\pgfsetlinewidth{0.501875pt}%
\definecolor{currentstroke}{rgb}{0.000000,0.000000,0.000000}%
\pgfsetstrokecolor{currentstroke}%
\pgfsetdash{}{0pt}%
\pgfsys@defobject{currentmarker}{\pgfqpoint{0.000000in}{0.000000in}}{\pgfqpoint{0.055556in}{0.000000in}}{%
\pgfpathmoveto{\pgfqpoint{0.000000in}{0.000000in}}%
\pgfpathlineto{\pgfqpoint{0.055556in}{0.000000in}}%
\pgfusepath{stroke,fill}%
}%
\begin{pgfscope}%
\pgfsys@transformshift{0.225000in}{3.220000in}%
\pgfsys@useobject{currentmarker}{}%
\end{pgfscope}%
\end{pgfscope}%
\begin{pgfscope}%
\pgfsetbuttcap%
\pgfsetroundjoin%
\definecolor{currentfill}{rgb}{0.000000,0.000000,0.000000}%
\pgfsetfillcolor{currentfill}%
\pgfsetlinewidth{0.501875pt}%
\definecolor{currentstroke}{rgb}{0.000000,0.000000,0.000000}%
\pgfsetstrokecolor{currentstroke}%
\pgfsetdash{}{0pt}%
\pgfsys@defobject{currentmarker}{\pgfqpoint{-0.055556in}{0.000000in}}{\pgfqpoint{0.000000in}{0.000000in}}{%
\pgfpathmoveto{\pgfqpoint{0.000000in}{0.000000in}}%
\pgfpathlineto{\pgfqpoint{-0.055556in}{0.000000in}}%
\pgfusepath{stroke,fill}%
}%
\begin{pgfscope}%
\pgfsys@transformshift{4.275000in}{3.220000in}%
\pgfsys@useobject{currentmarker}{}%
\end{pgfscope}%
\end{pgfscope}%
\begin{pgfscope}%
\pgftext[x=0.169444in,y=3.220000in,right,]{\rmfamily\fontsize{9.000000}{10.800000}\selectfont \(\displaystyle 1.0\)}%
\end{pgfscope}%
\begin{pgfscope}%
\pgftext[x=2.250000in,y=3.394444in,,base]{\rmfamily\fontsize{11.000000}{13.200000}\selectfont Daten}%
\end{pgfscope}%
\begin{pgfscope}%
\pgfsetbuttcap%
\pgfsetmiterjoin%
\pgfsetlinewidth{0.000000pt}%
\definecolor{currentstroke}{rgb}{0.000000,0.000000,0.000000}%
\pgfsetstrokecolor{currentstroke}%
\pgfsetstrokeopacity{0.000000}%
\pgfsetdash{}{0pt}%
\pgfpathmoveto{\pgfqpoint{0.225000in}{0.175000in}}%
\pgfpathlineto{\pgfqpoint{4.275000in}{0.175000in}}%
\pgfpathlineto{\pgfqpoint{4.275000in}{1.435000in}}%
\pgfpathlineto{\pgfqpoint{0.225000in}{1.435000in}}%
\pgfpathclose%
\pgfusepath{}%
\end{pgfscope}%
\begin{pgfscope}%
\pgfpathrectangle{\pgfqpoint{0.225000in}{0.175000in}}{\pgfqpoint{4.050000in}{1.260000in}} %
\pgfusepath{clip}%
\pgfsetrectcap%
\pgfsetroundjoin%
\pgfsetlinewidth{1.003750pt}%
\definecolor{currentstroke}{rgb}{0.000000,0.000000,1.000000}%
\pgfsetstrokecolor{currentstroke}%
\pgfsetdash{}{0pt}%
\pgfpathmoveto{\pgfqpoint{0.225000in}{0.805000in}}%
\pgfpathlineto{\pgfqpoint{0.246227in}{0.855175in}}%
\pgfpathlineto{\pgfqpoint{0.257805in}{0.877050in}}%
\pgfpathlineto{\pgfqpoint{0.266489in}{0.889267in}}%
\pgfpathlineto{\pgfqpoint{0.274208in}{0.896530in}}%
\pgfpathlineto{\pgfqpoint{0.280962in}{0.899860in}}%
\pgfpathlineto{\pgfqpoint{0.286751in}{0.900381in}}%
\pgfpathlineto{\pgfqpoint{0.292540in}{0.898732in}}%
\pgfpathlineto{\pgfqpoint{0.299294in}{0.894120in}}%
\pgfpathlineto{\pgfqpoint{0.306048in}{0.886751in}}%
\pgfpathlineto{\pgfqpoint{0.314732in}{0.873602in}}%
\pgfpathlineto{\pgfqpoint{0.325346in}{0.852857in}}%
\pgfpathlineto{\pgfqpoint{0.339819in}{0.819057in}}%
\pgfpathlineto{\pgfqpoint{0.368765in}{0.750557in}}%
\pgfpathlineto{\pgfqpoint{0.379378in}{0.731207in}}%
\pgfpathlineto{\pgfqpoint{0.388062in}{0.719498in}}%
\pgfpathlineto{\pgfqpoint{0.395781in}{0.712740in}}%
\pgfpathlineto{\pgfqpoint{0.402535in}{0.709877in}}%
\pgfpathlineto{\pgfqpoint{0.408324in}{0.709764in}}%
\pgfpathlineto{\pgfqpoint{0.414113in}{0.711816in}}%
\pgfpathlineto{\pgfqpoint{0.420867in}{0.716883in}}%
\pgfpathlineto{\pgfqpoint{0.428586in}{0.725999in}}%
\pgfpathlineto{\pgfqpoint{0.437270in}{0.740043in}}%
\pgfpathlineto{\pgfqpoint{0.447883in}{0.761624in}}%
\pgfpathlineto{\pgfqpoint{0.464286in}{0.800799in}}%
\pgfpathlineto{\pgfqpoint{0.487443in}{0.855685in}}%
\pgfpathlineto{\pgfqpoint{0.499021in}{0.877443in}}%
\pgfpathlineto{\pgfqpoint{0.507705in}{0.889548in}}%
\pgfpathlineto{\pgfqpoint{0.515424in}{0.896699in}}%
\pgfpathlineto{\pgfqpoint{0.522178in}{0.899925in}}%
\pgfpathlineto{\pgfqpoint{0.527967in}{0.900355in}}%
\pgfpathlineto{\pgfqpoint{0.533756in}{0.898617in}}%
\pgfpathlineto{\pgfqpoint{0.540510in}{0.893903in}}%
\pgfpathlineto{\pgfqpoint{0.547264in}{0.886439in}}%
\pgfpathlineto{\pgfqpoint{0.555948in}{0.873183in}}%
\pgfpathlineto{\pgfqpoint{0.566561in}{0.852337in}}%
\pgfpathlineto{\pgfqpoint{0.581999in}{0.816082in}}%
\pgfpathlineto{\pgfqpoint{0.609015in}{0.752046in}}%
\pgfpathlineto{\pgfqpoint{0.619629in}{0.732362in}}%
\pgfpathlineto{\pgfqpoint{0.628313in}{0.720313in}}%
\pgfpathlineto{\pgfqpoint{0.636031in}{0.713218in}}%
\pgfpathlineto{\pgfqpoint{0.642786in}{0.710044in}}%
\pgfpathlineto{\pgfqpoint{0.648575in}{0.709659in}}%
\pgfpathlineto{\pgfqpoint{0.654364in}{0.711442in}}%
\pgfpathlineto{\pgfqpoint{0.661118in}{0.716207in}}%
\pgfpathlineto{\pgfqpoint{0.668837in}{0.725002in}}%
\pgfpathlineto{\pgfqpoint{0.677521in}{0.738735in}}%
\pgfpathlineto{\pgfqpoint{0.688134in}{0.760028in}}%
\pgfpathlineto{\pgfqpoint{0.703572in}{0.796606in}}%
\pgfpathlineto{\pgfqpoint{0.728658in}{0.856193in}}%
\pgfpathlineto{\pgfqpoint{0.740237in}{0.877832in}}%
\pgfpathlineto{\pgfqpoint{0.748920in}{0.889825in}}%
\pgfpathlineto{\pgfqpoint{0.756639in}{0.896864in}}%
\pgfpathlineto{\pgfqpoint{0.763393in}{0.899987in}}%
\pgfpathlineto{\pgfqpoint{0.769183in}{0.900326in}}%
\pgfpathlineto{\pgfqpoint{0.774972in}{0.898498in}}%
\pgfpathlineto{\pgfqpoint{0.781726in}{0.893683in}}%
\pgfpathlineto{\pgfqpoint{0.789445in}{0.884833in}}%
\pgfpathlineto{\pgfqpoint{0.798128in}{0.871049in}}%
\pgfpathlineto{\pgfqpoint{0.808742in}{0.849707in}}%
\pgfpathlineto{\pgfqpoint{0.824180in}{0.813095in}}%
\pgfpathlineto{\pgfqpoint{0.849266in}{0.753554in}}%
\pgfpathlineto{\pgfqpoint{0.860844in}{0.731974in}}%
\pgfpathlineto{\pgfqpoint{0.869528in}{0.720038in}}%
\pgfpathlineto{\pgfqpoint{0.877247in}{0.713055in}}%
\pgfpathlineto{\pgfqpoint{0.884001in}{0.709984in}}%
\pgfpathlineto{\pgfqpoint{0.889790in}{0.709690in}}%
\pgfpathlineto{\pgfqpoint{0.895580in}{0.711563in}}%
\pgfpathlineto{\pgfqpoint{0.902334in}{0.716429in}}%
\pgfpathlineto{\pgfqpoint{0.910052in}{0.725331in}}%
\pgfpathlineto{\pgfqpoint{0.918736in}{0.739168in}}%
\pgfpathlineto{\pgfqpoint{0.929350in}{0.760558in}}%
\pgfpathlineto{\pgfqpoint{0.945752in}{0.799600in}}%
\pgfpathlineto{\pgfqpoint{0.969874in}{0.856698in}}%
\pgfpathlineto{\pgfqpoint{0.981452in}{0.878219in}}%
\pgfpathlineto{\pgfqpoint{0.990136in}{0.890098in}}%
\pgfpathlineto{\pgfqpoint{0.997855in}{0.897025in}}%
\pgfpathlineto{\pgfqpoint{1.004609in}{0.900044in}}%
\pgfpathlineto{\pgfqpoint{1.010398in}{0.900293in}}%
\pgfpathlineto{\pgfqpoint{1.016187in}{0.898375in}}%
\pgfpathlineto{\pgfqpoint{1.022941in}{0.893459in}}%
\pgfpathlineto{\pgfqpoint{1.030660in}{0.884503in}}%
\pgfpathlineto{\pgfqpoint{1.039344in}{0.870614in}}%
\pgfpathlineto{\pgfqpoint{1.049958in}{0.849176in}}%
\pgfpathlineto{\pgfqpoint{1.066360in}{0.810101in}}%
\pgfpathlineto{\pgfqpoint{1.090482in}{0.753049in}}%
\pgfpathlineto{\pgfqpoint{1.102060in}{0.731589in}}%
\pgfpathlineto{\pgfqpoint{1.110744in}{0.719766in}}%
\pgfpathlineto{\pgfqpoint{1.118463in}{0.712896in}}%
\pgfpathlineto{\pgfqpoint{1.125217in}{0.709929in}}%
\pgfpathlineto{\pgfqpoint{1.131006in}{0.709725in}}%
\pgfpathlineto{\pgfqpoint{1.136795in}{0.711688in}}%
\pgfpathlineto{\pgfqpoint{1.143549in}{0.716654in}}%
\pgfpathlineto{\pgfqpoint{1.151268in}{0.725664in}}%
\pgfpathlineto{\pgfqpoint{1.159952in}{0.739604in}}%
\pgfpathlineto{\pgfqpoint{1.170565in}{0.761090in}}%
\pgfpathlineto{\pgfqpoint{1.186968in}{0.800199in}}%
\pgfpathlineto{\pgfqpoint{1.188898in}{0.805000in}}%
\pgfpathlineto{\pgfqpoint{1.188898in}{0.805000in}}%
\pgfusepath{stroke}%
\end{pgfscope}%
\begin{pgfscope}%
\pgfpathrectangle{\pgfqpoint{0.225000in}{0.175000in}}{\pgfqpoint{4.050000in}{1.260000in}} %
\pgfusepath{clip}%
\pgfsetrectcap%
\pgfsetroundjoin%
\pgfsetlinewidth{1.003750pt}%
\definecolor{currentstroke}{rgb}{1.000000,0.000000,1.000000}%
\pgfsetstrokecolor{currentstroke}%
\pgfsetdash{}{0pt}%
\pgfpathmoveto{\pgfqpoint{1.188898in}{0.805000in}}%
\pgfpathlineto{\pgfqpoint{1.210125in}{1.106052in}}%
\pgfpathlineto{\pgfqpoint{1.220738in}{1.227715in}}%
\pgfpathlineto{\pgfqpoint{1.229422in}{1.303676in}}%
\pgfpathlineto{\pgfqpoint{1.236176in}{1.345311in}}%
\pgfpathlineto{\pgfqpoint{1.241965in}{1.367731in}}%
\pgfpathlineto{\pgfqpoint{1.245825in}{1.375587in}}%
\pgfpathlineto{\pgfqpoint{1.248719in}{1.377693in}}%
\pgfpathlineto{\pgfqpoint{1.251614in}{1.376538in}}%
\pgfpathlineto{\pgfqpoint{1.254508in}{1.372128in}}%
\pgfpathlineto{\pgfqpoint{1.258368in}{1.361234in}}%
\pgfpathlineto{\pgfqpoint{1.263192in}{1.339719in}}%
\pgfpathlineto{\pgfqpoint{1.268981in}{1.302788in}}%
\pgfpathlineto{\pgfqpoint{1.276700in}{1.236118in}}%
\pgfpathlineto{\pgfqpoint{1.285384in}{1.140474in}}%
\pgfpathlineto{\pgfqpoint{1.296962in}{0.987284in}}%
\pgfpathlineto{\pgfqpoint{1.338451in}{0.411316in}}%
\pgfpathlineto{\pgfqpoint{1.347135in}{0.327983in}}%
\pgfpathlineto{\pgfqpoint{1.354854in}{0.274247in}}%
\pgfpathlineto{\pgfqpoint{1.360643in}{0.247919in}}%
\pgfpathlineto{\pgfqpoint{1.365468in}{0.235640in}}%
\pgfpathlineto{\pgfqpoint{1.368362in}{0.232585in}}%
\pgfpathlineto{\pgfqpoint{1.371257in}{0.232789in}}%
\pgfpathlineto{\pgfqpoint{1.374151in}{0.236251in}}%
\pgfpathlineto{\pgfqpoint{1.378011in}{0.245897in}}%
\pgfpathlineto{\pgfqpoint{1.382835in}{0.265894in}}%
\pgfpathlineto{\pgfqpoint{1.388624in}{0.301099in}}%
\pgfpathlineto{\pgfqpoint{1.395378in}{0.356583in}}%
\pgfpathlineto{\pgfqpoint{1.404062in}{0.448011in}}%
\pgfpathlineto{\pgfqpoint{1.415640in}{0.597317in}}%
\pgfpathlineto{\pgfqpoint{1.437832in}{0.923019in}}%
\pgfpathlineto{\pgfqpoint{1.453270in}{1.133134in}}%
\pgfpathlineto{\pgfqpoint{1.463884in}{1.248900in}}%
\pgfpathlineto{\pgfqpoint{1.471602in}{1.312286in}}%
\pgfpathlineto{\pgfqpoint{1.478357in}{1.351016in}}%
\pgfpathlineto{\pgfqpoint{1.483181in}{1.368390in}}%
\pgfpathlineto{\pgfqpoint{1.487040in}{1.375887in}}%
\pgfpathlineto{\pgfqpoint{1.489935in}{1.377721in}}%
\pgfpathlineto{\pgfqpoint{1.492829in}{1.376294in}}%
\pgfpathlineto{\pgfqpoint{1.495724in}{1.371615in}}%
\pgfpathlineto{\pgfqpoint{1.499584in}{1.360364in}}%
\pgfpathlineto{\pgfqpoint{1.504408in}{1.338418in}}%
\pgfpathlineto{\pgfqpoint{1.510197in}{1.300996in}}%
\pgfpathlineto{\pgfqpoint{1.517916in}{1.233738in}}%
\pgfpathlineto{\pgfqpoint{1.527565in}{1.125713in}}%
\pgfpathlineto{\pgfqpoint{1.540108in}{0.956275in}}%
\pgfpathlineto{\pgfqpoint{1.576773in}{0.441013in}}%
\pgfpathlineto{\pgfqpoint{1.586421in}{0.342396in}}%
\pgfpathlineto{\pgfqpoint{1.594140in}{0.284233in}}%
\pgfpathlineto{\pgfqpoint{1.599929in}{0.254311in}}%
\pgfpathlineto{\pgfqpoint{1.604754in}{0.238921in}}%
\pgfpathlineto{\pgfqpoint{1.608613in}{0.233044in}}%
\pgfpathlineto{\pgfqpoint{1.611508in}{0.232432in}}%
\pgfpathlineto{\pgfqpoint{1.614402in}{0.235080in}}%
\pgfpathlineto{\pgfqpoint{1.618262in}{0.243654in}}%
\pgfpathlineto{\pgfqpoint{1.623086in}{0.262342in}}%
\pgfpathlineto{\pgfqpoint{1.628875in}{0.296052in}}%
\pgfpathlineto{\pgfqpoint{1.635629in}{0.349941in}}%
\pgfpathlineto{\pgfqpoint{1.644313in}{0.439625in}}%
\pgfpathlineto{\pgfqpoint{1.654926in}{0.574027in}}%
\pgfpathlineto{\pgfqpoint{1.673259in}{0.840998in}}%
\pgfpathlineto{\pgfqpoint{1.692556in}{1.112156in}}%
\pgfpathlineto{\pgfqpoint{1.703170in}{1.232542in}}%
\pgfpathlineto{\pgfqpoint{1.711853in}{1.307179in}}%
\pgfpathlineto{\pgfqpoint{1.718607in}{1.347658in}}%
\pgfpathlineto{\pgfqpoint{1.723432in}{1.366346in}}%
\pgfpathlineto{\pgfqpoint{1.727291in}{1.374920in}}%
\pgfpathlineto{\pgfqpoint{1.730186in}{1.377568in}}%
\pgfpathlineto{\pgfqpoint{1.733080in}{1.376956in}}%
\pgfpathlineto{\pgfqpoint{1.735975in}{1.373088in}}%
\pgfpathlineto{\pgfqpoint{1.739834in}{1.362906in}}%
\pgfpathlineto{\pgfqpoint{1.744659in}{1.342258in}}%
\pgfpathlineto{\pgfqpoint{1.750448in}{1.306311in}}%
\pgfpathlineto{\pgfqpoint{1.757202in}{1.250036in}}%
\pgfpathlineto{\pgfqpoint{1.765886in}{1.157748in}}%
\pgfpathlineto{\pgfqpoint{1.777464in}{1.007638in}}%
\pgfpathlineto{\pgfqpoint{1.801586in}{0.653725in}}%
\pgfpathlineto{\pgfqpoint{1.816058in}{0.460828in}}%
\pgfpathlineto{\pgfqpoint{1.825707in}{0.357705in}}%
\pgfpathlineto{\pgfqpoint{1.833426in}{0.295229in}}%
\pgfpathlineto{\pgfqpoint{1.840180in}{0.257377in}}%
\pgfpathlineto{\pgfqpoint{1.845004in}{0.240664in}}%
\pgfpathlineto{\pgfqpoint{1.848864in}{0.233706in}}%
\pgfpathlineto{\pgfqpoint{1.851758in}{0.232279in}}%
\pgfpathlineto{\pgfqpoint{1.854653in}{0.234113in}}%
\pgfpathlineto{\pgfqpoint{1.857548in}{0.239198in}}%
\pgfpathlineto{\pgfqpoint{1.861407in}{0.250981in}}%
\pgfpathlineto{\pgfqpoint{1.866231in}{0.273573in}}%
\pgfpathlineto{\pgfqpoint{1.872020in}{0.311727in}}%
\pgfpathlineto{\pgfqpoint{1.879739in}{0.379863in}}%
\pgfpathlineto{\pgfqpoint{1.889388in}{0.488778in}}%
\pgfpathlineto{\pgfqpoint{1.901931in}{0.658943in}}%
\pgfpathlineto{\pgfqpoint{1.937631in}{1.161989in}}%
\pgfpathlineto{\pgfqpoint{1.947280in}{1.262238in}}%
\pgfpathlineto{\pgfqpoint{1.954999in}{1.321955in}}%
\pgfpathlineto{\pgfqpoint{1.960788in}{1.353147in}}%
\pgfpathlineto{\pgfqpoint{1.965612in}{1.369641in}}%
\pgfpathlineto{\pgfqpoint{1.969472in}{1.376419in}}%
\pgfpathlineto{\pgfqpoint{1.972366in}{1.377710in}}%
\pgfpathlineto{\pgfqpoint{1.975261in}{1.375740in}}%
\pgfpathlineto{\pgfqpoint{1.978155in}{1.370520in}}%
\pgfpathlineto{\pgfqpoint{1.982015in}{1.358560in}}%
\pgfpathlineto{\pgfqpoint{1.986839in}{1.335753in}}%
\pgfpathlineto{\pgfqpoint{1.992628in}{1.297355in}}%
\pgfpathlineto{\pgfqpoint{2.000347in}{1.228928in}}%
\pgfpathlineto{\pgfqpoint{2.009996in}{1.119719in}}%
\pgfpathlineto{\pgfqpoint{2.022539in}{0.949315in}}%
\pgfpathlineto{\pgfqpoint{2.058239in}{0.446604in}}%
\pgfpathlineto{\pgfqpoint{2.067888in}{0.346680in}}%
\pgfpathlineto{\pgfqpoint{2.075606in}{0.287273in}}%
\pgfpathlineto{\pgfqpoint{2.081396in}{0.256333in}}%
\pgfpathlineto{\pgfqpoint{2.086220in}{0.240061in}}%
\pgfpathlineto{\pgfqpoint{2.090079in}{0.233462in}}%
\pgfpathlineto{\pgfqpoint{2.092974in}{0.232307in}}%
\pgfpathlineto{\pgfqpoint{2.095869in}{0.234413in}}%
\pgfpathlineto{\pgfqpoint{2.098763in}{0.239767in}}%
\pgfpathlineto{\pgfqpoint{2.102623in}{0.251905in}}%
\pgfpathlineto{\pgfqpoint{2.107447in}{0.274926in}}%
\pgfpathlineto{\pgfqpoint{2.113236in}{0.313567in}}%
\pgfpathlineto{\pgfqpoint{2.120955in}{0.382285in}}%
\pgfpathlineto{\pgfqpoint{2.130604in}{0.491788in}}%
\pgfpathlineto{\pgfqpoint{2.143147in}{0.662429in}}%
\pgfpathlineto{\pgfqpoint{2.152795in}{0.805000in}}%
\pgfpathlineto{\pgfqpoint{2.152795in}{0.805000in}}%
\pgfusepath{stroke}%
\end{pgfscope}%
\begin{pgfscope}%
\pgfpathrectangle{\pgfqpoint{0.225000in}{0.175000in}}{\pgfqpoint{4.050000in}{1.260000in}} %
\pgfusepath{clip}%
\pgfsetrectcap%
\pgfsetroundjoin%
\pgfsetlinewidth{1.003750pt}%
\definecolor{currentstroke}{rgb}{0.000000,0.000000,1.000000}%
\pgfsetstrokecolor{currentstroke}%
\pgfsetdash{}{0pt}%
\pgfpathmoveto{\pgfqpoint{2.152795in}{0.805000in}}%
\pgfpathlineto{\pgfqpoint{2.174022in}{0.855175in}}%
\pgfpathlineto{\pgfqpoint{2.185601in}{0.877050in}}%
\pgfpathlineto{\pgfqpoint{2.194285in}{0.889267in}}%
\pgfpathlineto{\pgfqpoint{2.202003in}{0.896530in}}%
\pgfpathlineto{\pgfqpoint{2.208758in}{0.899860in}}%
\pgfpathlineto{\pgfqpoint{2.214547in}{0.900381in}}%
\pgfpathlineto{\pgfqpoint{2.220336in}{0.898732in}}%
\pgfpathlineto{\pgfqpoint{2.227090in}{0.894120in}}%
\pgfpathlineto{\pgfqpoint{2.233844in}{0.886751in}}%
\pgfpathlineto{\pgfqpoint{2.242528in}{0.873602in}}%
\pgfpathlineto{\pgfqpoint{2.253141in}{0.852857in}}%
\pgfpathlineto{\pgfqpoint{2.267614in}{0.819057in}}%
\pgfpathlineto{\pgfqpoint{2.296560in}{0.750557in}}%
\pgfpathlineto{\pgfqpoint{2.307174in}{0.731207in}}%
\pgfpathlineto{\pgfqpoint{2.315857in}{0.719498in}}%
\pgfpathlineto{\pgfqpoint{2.323576in}{0.712740in}}%
\pgfpathlineto{\pgfqpoint{2.330330in}{0.709877in}}%
\pgfpathlineto{\pgfqpoint{2.336119in}{0.709764in}}%
\pgfpathlineto{\pgfqpoint{2.341909in}{0.711816in}}%
\pgfpathlineto{\pgfqpoint{2.348663in}{0.716883in}}%
\pgfpathlineto{\pgfqpoint{2.356382in}{0.725999in}}%
\pgfpathlineto{\pgfqpoint{2.365065in}{0.740043in}}%
\pgfpathlineto{\pgfqpoint{2.375679in}{0.761624in}}%
\pgfpathlineto{\pgfqpoint{2.392081in}{0.800799in}}%
\pgfpathlineto{\pgfqpoint{2.415238in}{0.855685in}}%
\pgfpathlineto{\pgfqpoint{2.426816in}{0.877443in}}%
\pgfpathlineto{\pgfqpoint{2.435500in}{0.889548in}}%
\pgfpathlineto{\pgfqpoint{2.443219in}{0.896699in}}%
\pgfpathlineto{\pgfqpoint{2.449973in}{0.899925in}}%
\pgfpathlineto{\pgfqpoint{2.455762in}{0.900355in}}%
\pgfpathlineto{\pgfqpoint{2.461552in}{0.898617in}}%
\pgfpathlineto{\pgfqpoint{2.468306in}{0.893903in}}%
\pgfpathlineto{\pgfqpoint{2.475060in}{0.886439in}}%
\pgfpathlineto{\pgfqpoint{2.483743in}{0.873183in}}%
\pgfpathlineto{\pgfqpoint{2.494357in}{0.852337in}}%
\pgfpathlineto{\pgfqpoint{2.509795in}{0.816082in}}%
\pgfpathlineto{\pgfqpoint{2.536811in}{0.752046in}}%
\pgfpathlineto{\pgfqpoint{2.547424in}{0.732362in}}%
\pgfpathlineto{\pgfqpoint{2.556108in}{0.720313in}}%
\pgfpathlineto{\pgfqpoint{2.563827in}{0.713218in}}%
\pgfpathlineto{\pgfqpoint{2.570581in}{0.710044in}}%
\pgfpathlineto{\pgfqpoint{2.576370in}{0.709659in}}%
\pgfpathlineto{\pgfqpoint{2.582159in}{0.711442in}}%
\pgfpathlineto{\pgfqpoint{2.588913in}{0.716207in}}%
\pgfpathlineto{\pgfqpoint{2.596632in}{0.725002in}}%
\pgfpathlineto{\pgfqpoint{2.605316in}{0.738735in}}%
\pgfpathlineto{\pgfqpoint{2.615930in}{0.760028in}}%
\pgfpathlineto{\pgfqpoint{2.631367in}{0.796606in}}%
\pgfpathlineto{\pgfqpoint{2.656454in}{0.856193in}}%
\pgfpathlineto{\pgfqpoint{2.668032in}{0.877832in}}%
\pgfpathlineto{\pgfqpoint{2.676716in}{0.889825in}}%
\pgfpathlineto{\pgfqpoint{2.684435in}{0.896864in}}%
\pgfpathlineto{\pgfqpoint{2.691189in}{0.899987in}}%
\pgfpathlineto{\pgfqpoint{2.696978in}{0.900326in}}%
\pgfpathlineto{\pgfqpoint{2.702767in}{0.898498in}}%
\pgfpathlineto{\pgfqpoint{2.709521in}{0.893683in}}%
\pgfpathlineto{\pgfqpoint{2.717240in}{0.884833in}}%
\pgfpathlineto{\pgfqpoint{2.725924in}{0.871049in}}%
\pgfpathlineto{\pgfqpoint{2.736537in}{0.849707in}}%
\pgfpathlineto{\pgfqpoint{2.751975in}{0.813095in}}%
\pgfpathlineto{\pgfqpoint{2.777062in}{0.753554in}}%
\pgfpathlineto{\pgfqpoint{2.788640in}{0.731974in}}%
\pgfpathlineto{\pgfqpoint{2.797324in}{0.720038in}}%
\pgfpathlineto{\pgfqpoint{2.805043in}{0.713055in}}%
\pgfpathlineto{\pgfqpoint{2.811797in}{0.709984in}}%
\pgfpathlineto{\pgfqpoint{2.817586in}{0.709690in}}%
\pgfpathlineto{\pgfqpoint{2.823375in}{0.711563in}}%
\pgfpathlineto{\pgfqpoint{2.830129in}{0.716429in}}%
\pgfpathlineto{\pgfqpoint{2.837848in}{0.725331in}}%
\pgfpathlineto{\pgfqpoint{2.846532in}{0.739168in}}%
\pgfpathlineto{\pgfqpoint{2.857145in}{0.760558in}}%
\pgfpathlineto{\pgfqpoint{2.873548in}{0.799600in}}%
\pgfpathlineto{\pgfqpoint{2.897669in}{0.856698in}}%
\pgfpathlineto{\pgfqpoint{2.909248in}{0.878219in}}%
\pgfpathlineto{\pgfqpoint{2.917932in}{0.890098in}}%
\pgfpathlineto{\pgfqpoint{2.925650in}{0.897025in}}%
\pgfpathlineto{\pgfqpoint{2.932404in}{0.900044in}}%
\pgfpathlineto{\pgfqpoint{2.938194in}{0.900293in}}%
\pgfpathlineto{\pgfqpoint{2.943983in}{0.898375in}}%
\pgfpathlineto{\pgfqpoint{2.950737in}{0.893459in}}%
\pgfpathlineto{\pgfqpoint{2.958456in}{0.884503in}}%
\pgfpathlineto{\pgfqpoint{2.967140in}{0.870614in}}%
\pgfpathlineto{\pgfqpoint{2.977753in}{0.849176in}}%
\pgfpathlineto{\pgfqpoint{2.994156in}{0.810101in}}%
\pgfpathlineto{\pgfqpoint{3.018277in}{0.753049in}}%
\pgfpathlineto{\pgfqpoint{3.029856in}{0.731589in}}%
\pgfpathlineto{\pgfqpoint{3.038539in}{0.719766in}}%
\pgfpathlineto{\pgfqpoint{3.046258in}{0.712896in}}%
\pgfpathlineto{\pgfqpoint{3.053012in}{0.709929in}}%
\pgfpathlineto{\pgfqpoint{3.058801in}{0.709725in}}%
\pgfpathlineto{\pgfqpoint{3.064591in}{0.711688in}}%
\pgfpathlineto{\pgfqpoint{3.071345in}{0.716654in}}%
\pgfpathlineto{\pgfqpoint{3.079064in}{0.725664in}}%
\pgfpathlineto{\pgfqpoint{3.087747in}{0.739604in}}%
\pgfpathlineto{\pgfqpoint{3.098361in}{0.761090in}}%
\pgfpathlineto{\pgfqpoint{3.114764in}{0.800199in}}%
\pgfpathlineto{\pgfqpoint{3.116693in}{0.805000in}}%
\pgfpathlineto{\pgfqpoint{3.116693in}{0.805000in}}%
\pgfusepath{stroke}%
\end{pgfscope}%
\begin{pgfscope}%
\pgfpathrectangle{\pgfqpoint{0.225000in}{0.175000in}}{\pgfqpoint{4.050000in}{1.260000in}} %
\pgfusepath{clip}%
\pgfsetrectcap%
\pgfsetroundjoin%
\pgfsetlinewidth{1.003750pt}%
\definecolor{currentstroke}{rgb}{1.000000,0.000000,1.000000}%
\pgfsetstrokecolor{currentstroke}%
\pgfsetdash{}{0pt}%
\pgfpathmoveto{\pgfqpoint{3.116693in}{0.805000in}}%
\pgfpathlineto{\pgfqpoint{3.137920in}{1.106052in}}%
\pgfpathlineto{\pgfqpoint{3.148534in}{1.227715in}}%
\pgfpathlineto{\pgfqpoint{3.157217in}{1.303676in}}%
\pgfpathlineto{\pgfqpoint{3.163972in}{1.345311in}}%
\pgfpathlineto{\pgfqpoint{3.169761in}{1.367731in}}%
\pgfpathlineto{\pgfqpoint{3.173620in}{1.375587in}}%
\pgfpathlineto{\pgfqpoint{3.176515in}{1.377693in}}%
\pgfpathlineto{\pgfqpoint{3.179409in}{1.376538in}}%
\pgfpathlineto{\pgfqpoint{3.182304in}{1.372128in}}%
\pgfpathlineto{\pgfqpoint{3.186163in}{1.361234in}}%
\pgfpathlineto{\pgfqpoint{3.190988in}{1.339719in}}%
\pgfpathlineto{\pgfqpoint{3.196777in}{1.302788in}}%
\pgfpathlineto{\pgfqpoint{3.204496in}{1.236118in}}%
\pgfpathlineto{\pgfqpoint{3.213179in}{1.140474in}}%
\pgfpathlineto{\pgfqpoint{3.224758in}{0.987284in}}%
\pgfpathlineto{\pgfqpoint{3.266247in}{0.411316in}}%
\pgfpathlineto{\pgfqpoint{3.274931in}{0.327983in}}%
\pgfpathlineto{\pgfqpoint{3.282650in}{0.274247in}}%
\pgfpathlineto{\pgfqpoint{3.288439in}{0.247919in}}%
\pgfpathlineto{\pgfqpoint{3.293263in}{0.235640in}}%
\pgfpathlineto{\pgfqpoint{3.296158in}{0.232585in}}%
\pgfpathlineto{\pgfqpoint{3.299052in}{0.232789in}}%
\pgfpathlineto{\pgfqpoint{3.301947in}{0.236251in}}%
\pgfpathlineto{\pgfqpoint{3.305806in}{0.245897in}}%
\pgfpathlineto{\pgfqpoint{3.310631in}{0.265894in}}%
\pgfpathlineto{\pgfqpoint{3.316420in}{0.301099in}}%
\pgfpathlineto{\pgfqpoint{3.323174in}{0.356583in}}%
\pgfpathlineto{\pgfqpoint{3.331858in}{0.448011in}}%
\pgfpathlineto{\pgfqpoint{3.343436in}{0.597317in}}%
\pgfpathlineto{\pgfqpoint{3.365628in}{0.923019in}}%
\pgfpathlineto{\pgfqpoint{3.381066in}{1.133134in}}%
\pgfpathlineto{\pgfqpoint{3.391679in}{1.248900in}}%
\pgfpathlineto{\pgfqpoint{3.399398in}{1.312286in}}%
\pgfpathlineto{\pgfqpoint{3.406152in}{1.351016in}}%
\pgfpathlineto{\pgfqpoint{3.410976in}{1.368390in}}%
\pgfpathlineto{\pgfqpoint{3.414836in}{1.375887in}}%
\pgfpathlineto{\pgfqpoint{3.417730in}{1.377721in}}%
\pgfpathlineto{\pgfqpoint{3.420625in}{1.376294in}}%
\pgfpathlineto{\pgfqpoint{3.423520in}{1.371615in}}%
\pgfpathlineto{\pgfqpoint{3.427379in}{1.360364in}}%
\pgfpathlineto{\pgfqpoint{3.432203in}{1.338418in}}%
\pgfpathlineto{\pgfqpoint{3.437992in}{1.300996in}}%
\pgfpathlineto{\pgfqpoint{3.445711in}{1.233738in}}%
\pgfpathlineto{\pgfqpoint{3.455360in}{1.125713in}}%
\pgfpathlineto{\pgfqpoint{3.467903in}{0.956275in}}%
\pgfpathlineto{\pgfqpoint{3.504568in}{0.441013in}}%
\pgfpathlineto{\pgfqpoint{3.514217in}{0.342396in}}%
\pgfpathlineto{\pgfqpoint{3.521936in}{0.284233in}}%
\pgfpathlineto{\pgfqpoint{3.527725in}{0.254311in}}%
\pgfpathlineto{\pgfqpoint{3.532549in}{0.238921in}}%
\pgfpathlineto{\pgfqpoint{3.536408in}{0.233044in}}%
\pgfpathlineto{\pgfqpoint{3.539303in}{0.232432in}}%
\pgfpathlineto{\pgfqpoint{3.542198in}{0.235080in}}%
\pgfpathlineto{\pgfqpoint{3.546057in}{0.243654in}}%
\pgfpathlineto{\pgfqpoint{3.550881in}{0.262342in}}%
\pgfpathlineto{\pgfqpoint{3.556671in}{0.296052in}}%
\pgfpathlineto{\pgfqpoint{3.563425in}{0.349941in}}%
\pgfpathlineto{\pgfqpoint{3.572108in}{0.439625in}}%
\pgfpathlineto{\pgfqpoint{3.582722in}{0.574027in}}%
\pgfpathlineto{\pgfqpoint{3.601054in}{0.840998in}}%
\pgfpathlineto{\pgfqpoint{3.620352in}{1.112156in}}%
\pgfpathlineto{\pgfqpoint{3.630965in}{1.232542in}}%
\pgfpathlineto{\pgfqpoint{3.639649in}{1.307179in}}%
\pgfpathlineto{\pgfqpoint{3.646403in}{1.347658in}}%
\pgfpathlineto{\pgfqpoint{3.651227in}{1.366346in}}%
\pgfpathlineto{\pgfqpoint{3.655087in}{1.374920in}}%
\pgfpathlineto{\pgfqpoint{3.657981in}{1.377568in}}%
\pgfpathlineto{\pgfqpoint{3.660876in}{1.376956in}}%
\pgfpathlineto{\pgfqpoint{3.663770in}{1.373088in}}%
\pgfpathlineto{\pgfqpoint{3.667630in}{1.362906in}}%
\pgfpathlineto{\pgfqpoint{3.672454in}{1.342258in}}%
\pgfpathlineto{\pgfqpoint{3.678243in}{1.306311in}}%
\pgfpathlineto{\pgfqpoint{3.684997in}{1.250036in}}%
\pgfpathlineto{\pgfqpoint{3.693681in}{1.157748in}}%
\pgfpathlineto{\pgfqpoint{3.705259in}{1.007638in}}%
\pgfpathlineto{\pgfqpoint{3.729381in}{0.653725in}}%
\pgfpathlineto{\pgfqpoint{3.743854in}{0.460828in}}%
\pgfpathlineto{\pgfqpoint{3.753503in}{0.357705in}}%
\pgfpathlineto{\pgfqpoint{3.761221in}{0.295229in}}%
\pgfpathlineto{\pgfqpoint{3.767975in}{0.257377in}}%
\pgfpathlineto{\pgfqpoint{3.772800in}{0.240664in}}%
\pgfpathlineto{\pgfqpoint{3.776659in}{0.233706in}}%
\pgfpathlineto{\pgfqpoint{3.779554in}{0.232279in}}%
\pgfpathlineto{\pgfqpoint{3.782448in}{0.234113in}}%
\pgfpathlineto{\pgfqpoint{3.785343in}{0.239198in}}%
\pgfpathlineto{\pgfqpoint{3.789202in}{0.250981in}}%
\pgfpathlineto{\pgfqpoint{3.794027in}{0.273573in}}%
\pgfpathlineto{\pgfqpoint{3.799816in}{0.311727in}}%
\pgfpathlineto{\pgfqpoint{3.807535in}{0.379863in}}%
\pgfpathlineto{\pgfqpoint{3.817183in}{0.488778in}}%
\pgfpathlineto{\pgfqpoint{3.829727in}{0.658943in}}%
\pgfpathlineto{\pgfqpoint{3.865427in}{1.161989in}}%
\pgfpathlineto{\pgfqpoint{3.875075in}{1.262238in}}%
\pgfpathlineto{\pgfqpoint{3.882794in}{1.321955in}}%
\pgfpathlineto{\pgfqpoint{3.888583in}{1.353147in}}%
\pgfpathlineto{\pgfqpoint{3.893408in}{1.369641in}}%
\pgfpathlineto{\pgfqpoint{3.897267in}{1.376419in}}%
\pgfpathlineto{\pgfqpoint{3.900162in}{1.377710in}}%
\pgfpathlineto{\pgfqpoint{3.903056in}{1.375740in}}%
\pgfpathlineto{\pgfqpoint{3.905951in}{1.370520in}}%
\pgfpathlineto{\pgfqpoint{3.909810in}{1.358560in}}%
\pgfpathlineto{\pgfqpoint{3.914635in}{1.335753in}}%
\pgfpathlineto{\pgfqpoint{3.920424in}{1.297355in}}%
\pgfpathlineto{\pgfqpoint{3.928143in}{1.228928in}}%
\pgfpathlineto{\pgfqpoint{3.937791in}{1.119719in}}%
\pgfpathlineto{\pgfqpoint{3.950335in}{0.949315in}}%
\pgfpathlineto{\pgfqpoint{3.986034in}{0.446604in}}%
\pgfpathlineto{\pgfqpoint{3.995683in}{0.346680in}}%
\pgfpathlineto{\pgfqpoint{4.003402in}{0.287273in}}%
\pgfpathlineto{\pgfqpoint{4.009191in}{0.256333in}}%
\pgfpathlineto{\pgfqpoint{4.014015in}{0.240061in}}%
\pgfpathlineto{\pgfqpoint{4.017875in}{0.233462in}}%
\pgfpathlineto{\pgfqpoint{4.020770in}{0.232307in}}%
\pgfpathlineto{\pgfqpoint{4.023664in}{0.234413in}}%
\pgfpathlineto{\pgfqpoint{4.026559in}{0.239767in}}%
\pgfpathlineto{\pgfqpoint{4.030418in}{0.251905in}}%
\pgfpathlineto{\pgfqpoint{4.035242in}{0.274926in}}%
\pgfpathlineto{\pgfqpoint{4.041032in}{0.313567in}}%
\pgfpathlineto{\pgfqpoint{4.048751in}{0.382285in}}%
\pgfpathlineto{\pgfqpoint{4.058399in}{0.491788in}}%
\pgfpathlineto{\pgfqpoint{4.070942in}{0.662429in}}%
\pgfpathlineto{\pgfqpoint{4.080591in}{0.805000in}}%
\pgfpathlineto{\pgfqpoint{4.080591in}{0.805000in}}%
\pgfusepath{stroke}%
\end{pgfscope}%
\begin{pgfscope}%
\pgfsetrectcap%
\pgfsetmiterjoin%
\pgfsetlinewidth{1.003750pt}%
\definecolor{currentstroke}{rgb}{0.000000,0.000000,0.000000}%
\pgfsetstrokecolor{currentstroke}%
\pgfsetdash{}{0pt}%
\pgfpathmoveto{\pgfqpoint{0.225000in}{0.175000in}}%
\pgfpathlineto{\pgfqpoint{0.225000in}{1.435000in}}%
\pgfusepath{stroke}%
\end{pgfscope}%
\begin{pgfscope}%
\pgfsetrectcap%
\pgfsetmiterjoin%
\pgfsetlinewidth{1.003750pt}%
\definecolor{currentstroke}{rgb}{0.000000,0.000000,0.000000}%
\pgfsetstrokecolor{currentstroke}%
\pgfsetdash{}{0pt}%
\pgfpathmoveto{\pgfqpoint{0.225000in}{1.435000in}}%
\pgfpathlineto{\pgfqpoint{4.275000in}{1.435000in}}%
\pgfusepath{stroke}%
\end{pgfscope}%
\begin{pgfscope}%
\pgfsetrectcap%
\pgfsetmiterjoin%
\pgfsetlinewidth{1.003750pt}%
\definecolor{currentstroke}{rgb}{0.000000,0.000000,0.000000}%
\pgfsetstrokecolor{currentstroke}%
\pgfsetdash{}{0pt}%
\pgfpathmoveto{\pgfqpoint{4.275000in}{0.175000in}}%
\pgfpathlineto{\pgfqpoint{4.275000in}{1.435000in}}%
\pgfusepath{stroke}%
\end{pgfscope}%
\begin{pgfscope}%
\pgfsetrectcap%
\pgfsetmiterjoin%
\pgfsetlinewidth{1.003750pt}%
\definecolor{currentstroke}{rgb}{0.000000,0.000000,0.000000}%
\pgfsetstrokecolor{currentstroke}%
\pgfsetdash{}{0pt}%
\pgfpathmoveto{\pgfqpoint{0.225000in}{0.175000in}}%
\pgfpathlineto{\pgfqpoint{4.275000in}{0.175000in}}%
\pgfusepath{stroke}%
\end{pgfscope}%
\begin{pgfscope}%
\pgfsetbuttcap%
\pgfsetroundjoin%
\definecolor{currentfill}{rgb}{0.000000,0.000000,0.000000}%
\pgfsetfillcolor{currentfill}%
\pgfsetlinewidth{0.501875pt}%
\definecolor{currentstroke}{rgb}{0.000000,0.000000,0.000000}%
\pgfsetstrokecolor{currentstroke}%
\pgfsetdash{}{0pt}%
\pgfsys@defobject{currentmarker}{\pgfqpoint{0.000000in}{0.000000in}}{\pgfqpoint{0.000000in}{0.055556in}}{%
\pgfpathmoveto{\pgfqpoint{0.000000in}{0.000000in}}%
\pgfpathlineto{\pgfqpoint{0.000000in}{0.055556in}}%
\pgfusepath{stroke,fill}%
}%
\begin{pgfscope}%
\pgfsys@transformshift{0.225000in}{0.175000in}%
\pgfsys@useobject{currentmarker}{}%
\end{pgfscope}%
\end{pgfscope}%
\begin{pgfscope}%
\pgfsetbuttcap%
\pgfsetroundjoin%
\definecolor{currentfill}{rgb}{0.000000,0.000000,0.000000}%
\pgfsetfillcolor{currentfill}%
\pgfsetlinewidth{0.501875pt}%
\definecolor{currentstroke}{rgb}{0.000000,0.000000,0.000000}%
\pgfsetstrokecolor{currentstroke}%
\pgfsetdash{}{0pt}%
\pgfsys@defobject{currentmarker}{\pgfqpoint{0.000000in}{-0.055556in}}{\pgfqpoint{0.000000in}{0.000000in}}{%
\pgfpathmoveto{\pgfqpoint{0.000000in}{0.000000in}}%
\pgfpathlineto{\pgfqpoint{0.000000in}{-0.055556in}}%
\pgfusepath{stroke,fill}%
}%
\begin{pgfscope}%
\pgfsys@transformshift{0.225000in}{1.435000in}%
\pgfsys@useobject{currentmarker}{}%
\end{pgfscope}%
\end{pgfscope}%
\begin{pgfscope}%
\pgftext[x=0.225000in,y=0.119444in,,top]{\rmfamily\fontsize{9.000000}{10.800000}\selectfont \(\displaystyle 0.0000\)}%
\end{pgfscope}%
\begin{pgfscope}%
\pgfsetbuttcap%
\pgfsetroundjoin%
\definecolor{currentfill}{rgb}{0.000000,0.000000,0.000000}%
\pgfsetfillcolor{currentfill}%
\pgfsetlinewidth{0.501875pt}%
\definecolor{currentstroke}{rgb}{0.000000,0.000000,0.000000}%
\pgfsetstrokecolor{currentstroke}%
\pgfsetdash{}{0pt}%
\pgfsys@defobject{currentmarker}{\pgfqpoint{0.000000in}{0.000000in}}{\pgfqpoint{0.000000in}{0.055556in}}{%
\pgfpathmoveto{\pgfqpoint{0.000000in}{0.000000in}}%
\pgfpathlineto{\pgfqpoint{0.000000in}{0.055556in}}%
\pgfusepath{stroke,fill}%
}%
\begin{pgfscope}%
\pgfsys@transformshift{0.731250in}{0.175000in}%
\pgfsys@useobject{currentmarker}{}%
\end{pgfscope}%
\end{pgfscope}%
\begin{pgfscope}%
\pgfsetbuttcap%
\pgfsetroundjoin%
\definecolor{currentfill}{rgb}{0.000000,0.000000,0.000000}%
\pgfsetfillcolor{currentfill}%
\pgfsetlinewidth{0.501875pt}%
\definecolor{currentstroke}{rgb}{0.000000,0.000000,0.000000}%
\pgfsetstrokecolor{currentstroke}%
\pgfsetdash{}{0pt}%
\pgfsys@defobject{currentmarker}{\pgfqpoint{0.000000in}{-0.055556in}}{\pgfqpoint{0.000000in}{0.000000in}}{%
\pgfpathmoveto{\pgfqpoint{0.000000in}{0.000000in}}%
\pgfpathlineto{\pgfqpoint{0.000000in}{-0.055556in}}%
\pgfusepath{stroke,fill}%
}%
\begin{pgfscope}%
\pgfsys@transformshift{0.731250in}{1.435000in}%
\pgfsys@useobject{currentmarker}{}%
\end{pgfscope}%
\end{pgfscope}%
\begin{pgfscope}%
\pgftext[x=0.731250in,y=0.119444in,,top]{\rmfamily\fontsize{9.000000}{10.800000}\selectfont \(\displaystyle 0.0002\)}%
\end{pgfscope}%
\begin{pgfscope}%
\pgfsetbuttcap%
\pgfsetroundjoin%
\definecolor{currentfill}{rgb}{0.000000,0.000000,0.000000}%
\pgfsetfillcolor{currentfill}%
\pgfsetlinewidth{0.501875pt}%
\definecolor{currentstroke}{rgb}{0.000000,0.000000,0.000000}%
\pgfsetstrokecolor{currentstroke}%
\pgfsetdash{}{0pt}%
\pgfsys@defobject{currentmarker}{\pgfqpoint{0.000000in}{0.000000in}}{\pgfqpoint{0.000000in}{0.055556in}}{%
\pgfpathmoveto{\pgfqpoint{0.000000in}{0.000000in}}%
\pgfpathlineto{\pgfqpoint{0.000000in}{0.055556in}}%
\pgfusepath{stroke,fill}%
}%
\begin{pgfscope}%
\pgfsys@transformshift{1.237500in}{0.175000in}%
\pgfsys@useobject{currentmarker}{}%
\end{pgfscope}%
\end{pgfscope}%
\begin{pgfscope}%
\pgfsetbuttcap%
\pgfsetroundjoin%
\definecolor{currentfill}{rgb}{0.000000,0.000000,0.000000}%
\pgfsetfillcolor{currentfill}%
\pgfsetlinewidth{0.501875pt}%
\definecolor{currentstroke}{rgb}{0.000000,0.000000,0.000000}%
\pgfsetstrokecolor{currentstroke}%
\pgfsetdash{}{0pt}%
\pgfsys@defobject{currentmarker}{\pgfqpoint{0.000000in}{-0.055556in}}{\pgfqpoint{0.000000in}{0.000000in}}{%
\pgfpathmoveto{\pgfqpoint{0.000000in}{0.000000in}}%
\pgfpathlineto{\pgfqpoint{0.000000in}{-0.055556in}}%
\pgfusepath{stroke,fill}%
}%
\begin{pgfscope}%
\pgfsys@transformshift{1.237500in}{1.435000in}%
\pgfsys@useobject{currentmarker}{}%
\end{pgfscope}%
\end{pgfscope}%
\begin{pgfscope}%
\pgftext[x=1.237500in,y=0.119444in,,top]{\rmfamily\fontsize{9.000000}{10.800000}\selectfont \(\displaystyle 0.0004\)}%
\end{pgfscope}%
\begin{pgfscope}%
\pgfsetbuttcap%
\pgfsetroundjoin%
\definecolor{currentfill}{rgb}{0.000000,0.000000,0.000000}%
\pgfsetfillcolor{currentfill}%
\pgfsetlinewidth{0.501875pt}%
\definecolor{currentstroke}{rgb}{0.000000,0.000000,0.000000}%
\pgfsetstrokecolor{currentstroke}%
\pgfsetdash{}{0pt}%
\pgfsys@defobject{currentmarker}{\pgfqpoint{0.000000in}{0.000000in}}{\pgfqpoint{0.000000in}{0.055556in}}{%
\pgfpathmoveto{\pgfqpoint{0.000000in}{0.000000in}}%
\pgfpathlineto{\pgfqpoint{0.000000in}{0.055556in}}%
\pgfusepath{stroke,fill}%
}%
\begin{pgfscope}%
\pgfsys@transformshift{1.743750in}{0.175000in}%
\pgfsys@useobject{currentmarker}{}%
\end{pgfscope}%
\end{pgfscope}%
\begin{pgfscope}%
\pgfsetbuttcap%
\pgfsetroundjoin%
\definecolor{currentfill}{rgb}{0.000000,0.000000,0.000000}%
\pgfsetfillcolor{currentfill}%
\pgfsetlinewidth{0.501875pt}%
\definecolor{currentstroke}{rgb}{0.000000,0.000000,0.000000}%
\pgfsetstrokecolor{currentstroke}%
\pgfsetdash{}{0pt}%
\pgfsys@defobject{currentmarker}{\pgfqpoint{0.000000in}{-0.055556in}}{\pgfqpoint{0.000000in}{0.000000in}}{%
\pgfpathmoveto{\pgfqpoint{0.000000in}{0.000000in}}%
\pgfpathlineto{\pgfqpoint{0.000000in}{-0.055556in}}%
\pgfusepath{stroke,fill}%
}%
\begin{pgfscope}%
\pgfsys@transformshift{1.743750in}{1.435000in}%
\pgfsys@useobject{currentmarker}{}%
\end{pgfscope}%
\end{pgfscope}%
\begin{pgfscope}%
\pgftext[x=1.743750in,y=0.119444in,,top]{\rmfamily\fontsize{9.000000}{10.800000}\selectfont \(\displaystyle 0.0006\)}%
\end{pgfscope}%
\begin{pgfscope}%
\pgfsetbuttcap%
\pgfsetroundjoin%
\definecolor{currentfill}{rgb}{0.000000,0.000000,0.000000}%
\pgfsetfillcolor{currentfill}%
\pgfsetlinewidth{0.501875pt}%
\definecolor{currentstroke}{rgb}{0.000000,0.000000,0.000000}%
\pgfsetstrokecolor{currentstroke}%
\pgfsetdash{}{0pt}%
\pgfsys@defobject{currentmarker}{\pgfqpoint{0.000000in}{0.000000in}}{\pgfqpoint{0.000000in}{0.055556in}}{%
\pgfpathmoveto{\pgfqpoint{0.000000in}{0.000000in}}%
\pgfpathlineto{\pgfqpoint{0.000000in}{0.055556in}}%
\pgfusepath{stroke,fill}%
}%
\begin{pgfscope}%
\pgfsys@transformshift{2.250000in}{0.175000in}%
\pgfsys@useobject{currentmarker}{}%
\end{pgfscope}%
\end{pgfscope}%
\begin{pgfscope}%
\pgfsetbuttcap%
\pgfsetroundjoin%
\definecolor{currentfill}{rgb}{0.000000,0.000000,0.000000}%
\pgfsetfillcolor{currentfill}%
\pgfsetlinewidth{0.501875pt}%
\definecolor{currentstroke}{rgb}{0.000000,0.000000,0.000000}%
\pgfsetstrokecolor{currentstroke}%
\pgfsetdash{}{0pt}%
\pgfsys@defobject{currentmarker}{\pgfqpoint{0.000000in}{-0.055556in}}{\pgfqpoint{0.000000in}{0.000000in}}{%
\pgfpathmoveto{\pgfqpoint{0.000000in}{0.000000in}}%
\pgfpathlineto{\pgfqpoint{0.000000in}{-0.055556in}}%
\pgfusepath{stroke,fill}%
}%
\begin{pgfscope}%
\pgfsys@transformshift{2.250000in}{1.435000in}%
\pgfsys@useobject{currentmarker}{}%
\end{pgfscope}%
\end{pgfscope}%
\begin{pgfscope}%
\pgftext[x=2.250000in,y=0.119444in,,top]{\rmfamily\fontsize{9.000000}{10.800000}\selectfont \(\displaystyle 0.0008\)}%
\end{pgfscope}%
\begin{pgfscope}%
\pgfsetbuttcap%
\pgfsetroundjoin%
\definecolor{currentfill}{rgb}{0.000000,0.000000,0.000000}%
\pgfsetfillcolor{currentfill}%
\pgfsetlinewidth{0.501875pt}%
\definecolor{currentstroke}{rgb}{0.000000,0.000000,0.000000}%
\pgfsetstrokecolor{currentstroke}%
\pgfsetdash{}{0pt}%
\pgfsys@defobject{currentmarker}{\pgfqpoint{0.000000in}{0.000000in}}{\pgfqpoint{0.000000in}{0.055556in}}{%
\pgfpathmoveto{\pgfqpoint{0.000000in}{0.000000in}}%
\pgfpathlineto{\pgfqpoint{0.000000in}{0.055556in}}%
\pgfusepath{stroke,fill}%
}%
\begin{pgfscope}%
\pgfsys@transformshift{2.756250in}{0.175000in}%
\pgfsys@useobject{currentmarker}{}%
\end{pgfscope}%
\end{pgfscope}%
\begin{pgfscope}%
\pgfsetbuttcap%
\pgfsetroundjoin%
\definecolor{currentfill}{rgb}{0.000000,0.000000,0.000000}%
\pgfsetfillcolor{currentfill}%
\pgfsetlinewidth{0.501875pt}%
\definecolor{currentstroke}{rgb}{0.000000,0.000000,0.000000}%
\pgfsetstrokecolor{currentstroke}%
\pgfsetdash{}{0pt}%
\pgfsys@defobject{currentmarker}{\pgfqpoint{0.000000in}{-0.055556in}}{\pgfqpoint{0.000000in}{0.000000in}}{%
\pgfpathmoveto{\pgfqpoint{0.000000in}{0.000000in}}%
\pgfpathlineto{\pgfqpoint{0.000000in}{-0.055556in}}%
\pgfusepath{stroke,fill}%
}%
\begin{pgfscope}%
\pgfsys@transformshift{2.756250in}{1.435000in}%
\pgfsys@useobject{currentmarker}{}%
\end{pgfscope}%
\end{pgfscope}%
\begin{pgfscope}%
\pgftext[x=2.756250in,y=0.119444in,,top]{\rmfamily\fontsize{9.000000}{10.800000}\selectfont \(\displaystyle 0.0010\)}%
\end{pgfscope}%
\begin{pgfscope}%
\pgfsetbuttcap%
\pgfsetroundjoin%
\definecolor{currentfill}{rgb}{0.000000,0.000000,0.000000}%
\pgfsetfillcolor{currentfill}%
\pgfsetlinewidth{0.501875pt}%
\definecolor{currentstroke}{rgb}{0.000000,0.000000,0.000000}%
\pgfsetstrokecolor{currentstroke}%
\pgfsetdash{}{0pt}%
\pgfsys@defobject{currentmarker}{\pgfqpoint{0.000000in}{0.000000in}}{\pgfqpoint{0.000000in}{0.055556in}}{%
\pgfpathmoveto{\pgfqpoint{0.000000in}{0.000000in}}%
\pgfpathlineto{\pgfqpoint{0.000000in}{0.055556in}}%
\pgfusepath{stroke,fill}%
}%
\begin{pgfscope}%
\pgfsys@transformshift{3.262500in}{0.175000in}%
\pgfsys@useobject{currentmarker}{}%
\end{pgfscope}%
\end{pgfscope}%
\begin{pgfscope}%
\pgfsetbuttcap%
\pgfsetroundjoin%
\definecolor{currentfill}{rgb}{0.000000,0.000000,0.000000}%
\pgfsetfillcolor{currentfill}%
\pgfsetlinewidth{0.501875pt}%
\definecolor{currentstroke}{rgb}{0.000000,0.000000,0.000000}%
\pgfsetstrokecolor{currentstroke}%
\pgfsetdash{}{0pt}%
\pgfsys@defobject{currentmarker}{\pgfqpoint{0.000000in}{-0.055556in}}{\pgfqpoint{0.000000in}{0.000000in}}{%
\pgfpathmoveto{\pgfqpoint{0.000000in}{0.000000in}}%
\pgfpathlineto{\pgfqpoint{0.000000in}{-0.055556in}}%
\pgfusepath{stroke,fill}%
}%
\begin{pgfscope}%
\pgfsys@transformshift{3.262500in}{1.435000in}%
\pgfsys@useobject{currentmarker}{}%
\end{pgfscope}%
\end{pgfscope}%
\begin{pgfscope}%
\pgftext[x=3.262500in,y=0.119444in,,top]{\rmfamily\fontsize{9.000000}{10.800000}\selectfont \(\displaystyle 0.0012\)}%
\end{pgfscope}%
\begin{pgfscope}%
\pgfsetbuttcap%
\pgfsetroundjoin%
\definecolor{currentfill}{rgb}{0.000000,0.000000,0.000000}%
\pgfsetfillcolor{currentfill}%
\pgfsetlinewidth{0.501875pt}%
\definecolor{currentstroke}{rgb}{0.000000,0.000000,0.000000}%
\pgfsetstrokecolor{currentstroke}%
\pgfsetdash{}{0pt}%
\pgfsys@defobject{currentmarker}{\pgfqpoint{0.000000in}{0.000000in}}{\pgfqpoint{0.000000in}{0.055556in}}{%
\pgfpathmoveto{\pgfqpoint{0.000000in}{0.000000in}}%
\pgfpathlineto{\pgfqpoint{0.000000in}{0.055556in}}%
\pgfusepath{stroke,fill}%
}%
\begin{pgfscope}%
\pgfsys@transformshift{3.768750in}{0.175000in}%
\pgfsys@useobject{currentmarker}{}%
\end{pgfscope}%
\end{pgfscope}%
\begin{pgfscope}%
\pgfsetbuttcap%
\pgfsetroundjoin%
\definecolor{currentfill}{rgb}{0.000000,0.000000,0.000000}%
\pgfsetfillcolor{currentfill}%
\pgfsetlinewidth{0.501875pt}%
\definecolor{currentstroke}{rgb}{0.000000,0.000000,0.000000}%
\pgfsetstrokecolor{currentstroke}%
\pgfsetdash{}{0pt}%
\pgfsys@defobject{currentmarker}{\pgfqpoint{0.000000in}{-0.055556in}}{\pgfqpoint{0.000000in}{0.000000in}}{%
\pgfpathmoveto{\pgfqpoint{0.000000in}{0.000000in}}%
\pgfpathlineto{\pgfqpoint{0.000000in}{-0.055556in}}%
\pgfusepath{stroke,fill}%
}%
\begin{pgfscope}%
\pgfsys@transformshift{3.768750in}{1.435000in}%
\pgfsys@useobject{currentmarker}{}%
\end{pgfscope}%
\end{pgfscope}%
\begin{pgfscope}%
\pgftext[x=3.768750in,y=0.119444in,,top]{\rmfamily\fontsize{9.000000}{10.800000}\selectfont \(\displaystyle 0.0014\)}%
\end{pgfscope}%
\begin{pgfscope}%
\pgfsetbuttcap%
\pgfsetroundjoin%
\definecolor{currentfill}{rgb}{0.000000,0.000000,0.000000}%
\pgfsetfillcolor{currentfill}%
\pgfsetlinewidth{0.501875pt}%
\definecolor{currentstroke}{rgb}{0.000000,0.000000,0.000000}%
\pgfsetstrokecolor{currentstroke}%
\pgfsetdash{}{0pt}%
\pgfsys@defobject{currentmarker}{\pgfqpoint{0.000000in}{0.000000in}}{\pgfqpoint{0.000000in}{0.055556in}}{%
\pgfpathmoveto{\pgfqpoint{0.000000in}{0.000000in}}%
\pgfpathlineto{\pgfqpoint{0.000000in}{0.055556in}}%
\pgfusepath{stroke,fill}%
}%
\begin{pgfscope}%
\pgfsys@transformshift{4.275000in}{0.175000in}%
\pgfsys@useobject{currentmarker}{}%
\end{pgfscope}%
\end{pgfscope}%
\begin{pgfscope}%
\pgfsetbuttcap%
\pgfsetroundjoin%
\definecolor{currentfill}{rgb}{0.000000,0.000000,0.000000}%
\pgfsetfillcolor{currentfill}%
\pgfsetlinewidth{0.501875pt}%
\definecolor{currentstroke}{rgb}{0.000000,0.000000,0.000000}%
\pgfsetstrokecolor{currentstroke}%
\pgfsetdash{}{0pt}%
\pgfsys@defobject{currentmarker}{\pgfqpoint{0.000000in}{-0.055556in}}{\pgfqpoint{0.000000in}{0.000000in}}{%
\pgfpathmoveto{\pgfqpoint{0.000000in}{0.000000in}}%
\pgfpathlineto{\pgfqpoint{0.000000in}{-0.055556in}}%
\pgfusepath{stroke,fill}%
}%
\begin{pgfscope}%
\pgfsys@transformshift{4.275000in}{1.435000in}%
\pgfsys@useobject{currentmarker}{}%
\end{pgfscope}%
\end{pgfscope}%
\begin{pgfscope}%
\pgftext[x=4.275000in,y=0.119444in,,top]{\rmfamily\fontsize{9.000000}{10.800000}\selectfont \(\displaystyle 0.0016\)}%
\end{pgfscope}%
\begin{pgfscope}%
\pgfsetbuttcap%
\pgfsetroundjoin%
\definecolor{currentfill}{rgb}{0.000000,0.000000,0.000000}%
\pgfsetfillcolor{currentfill}%
\pgfsetlinewidth{0.501875pt}%
\definecolor{currentstroke}{rgb}{0.000000,0.000000,0.000000}%
\pgfsetstrokecolor{currentstroke}%
\pgfsetdash{}{0pt}%
\pgfsys@defobject{currentmarker}{\pgfqpoint{0.000000in}{0.000000in}}{\pgfqpoint{0.055556in}{0.000000in}}{%
\pgfpathmoveto{\pgfqpoint{0.000000in}{0.000000in}}%
\pgfpathlineto{\pgfqpoint{0.055556in}{0.000000in}}%
\pgfusepath{stroke,fill}%
}%
\begin{pgfscope}%
\pgfsys@transformshift{0.225000in}{0.232273in}%
\pgfsys@useobject{currentmarker}{}%
\end{pgfscope}%
\end{pgfscope}%
\begin{pgfscope}%
\pgfsetbuttcap%
\pgfsetroundjoin%
\definecolor{currentfill}{rgb}{0.000000,0.000000,0.000000}%
\pgfsetfillcolor{currentfill}%
\pgfsetlinewidth{0.501875pt}%
\definecolor{currentstroke}{rgb}{0.000000,0.000000,0.000000}%
\pgfsetstrokecolor{currentstroke}%
\pgfsetdash{}{0pt}%
\pgfsys@defobject{currentmarker}{\pgfqpoint{-0.055556in}{0.000000in}}{\pgfqpoint{0.000000in}{0.000000in}}{%
\pgfpathmoveto{\pgfqpoint{0.000000in}{0.000000in}}%
\pgfpathlineto{\pgfqpoint{-0.055556in}{0.000000in}}%
\pgfusepath{stroke,fill}%
}%
\begin{pgfscope}%
\pgfsys@transformshift{4.275000in}{0.232273in}%
\pgfsys@useobject{currentmarker}{}%
\end{pgfscope}%
\end{pgfscope}%
\begin{pgfscope}%
\pgftext[x=0.169444in,y=0.232273in,right,]{\rmfamily\fontsize{9.000000}{10.800000}\selectfont \(\displaystyle -30\)}%
\end{pgfscope}%
\begin{pgfscope}%
\pgfsetbuttcap%
\pgfsetroundjoin%
\definecolor{currentfill}{rgb}{0.000000,0.000000,0.000000}%
\pgfsetfillcolor{currentfill}%
\pgfsetlinewidth{0.501875pt}%
\definecolor{currentstroke}{rgb}{0.000000,0.000000,0.000000}%
\pgfsetstrokecolor{currentstroke}%
\pgfsetdash{}{0pt}%
\pgfsys@defobject{currentmarker}{\pgfqpoint{0.000000in}{0.000000in}}{\pgfqpoint{0.055556in}{0.000000in}}{%
\pgfpathmoveto{\pgfqpoint{0.000000in}{0.000000in}}%
\pgfpathlineto{\pgfqpoint{0.055556in}{0.000000in}}%
\pgfusepath{stroke,fill}%
}%
\begin{pgfscope}%
\pgfsys@transformshift{0.225000in}{0.423182in}%
\pgfsys@useobject{currentmarker}{}%
\end{pgfscope}%
\end{pgfscope}%
\begin{pgfscope}%
\pgfsetbuttcap%
\pgfsetroundjoin%
\definecolor{currentfill}{rgb}{0.000000,0.000000,0.000000}%
\pgfsetfillcolor{currentfill}%
\pgfsetlinewidth{0.501875pt}%
\definecolor{currentstroke}{rgb}{0.000000,0.000000,0.000000}%
\pgfsetstrokecolor{currentstroke}%
\pgfsetdash{}{0pt}%
\pgfsys@defobject{currentmarker}{\pgfqpoint{-0.055556in}{0.000000in}}{\pgfqpoint{0.000000in}{0.000000in}}{%
\pgfpathmoveto{\pgfqpoint{0.000000in}{0.000000in}}%
\pgfpathlineto{\pgfqpoint{-0.055556in}{0.000000in}}%
\pgfusepath{stroke,fill}%
}%
\begin{pgfscope}%
\pgfsys@transformshift{4.275000in}{0.423182in}%
\pgfsys@useobject{currentmarker}{}%
\end{pgfscope}%
\end{pgfscope}%
\begin{pgfscope}%
\pgftext[x=0.169444in,y=0.423182in,right,]{\rmfamily\fontsize{9.000000}{10.800000}\selectfont \(\displaystyle -20\)}%
\end{pgfscope}%
\begin{pgfscope}%
\pgfsetbuttcap%
\pgfsetroundjoin%
\definecolor{currentfill}{rgb}{0.000000,0.000000,0.000000}%
\pgfsetfillcolor{currentfill}%
\pgfsetlinewidth{0.501875pt}%
\definecolor{currentstroke}{rgb}{0.000000,0.000000,0.000000}%
\pgfsetstrokecolor{currentstroke}%
\pgfsetdash{}{0pt}%
\pgfsys@defobject{currentmarker}{\pgfqpoint{0.000000in}{0.000000in}}{\pgfqpoint{0.055556in}{0.000000in}}{%
\pgfpathmoveto{\pgfqpoint{0.000000in}{0.000000in}}%
\pgfpathlineto{\pgfqpoint{0.055556in}{0.000000in}}%
\pgfusepath{stroke,fill}%
}%
\begin{pgfscope}%
\pgfsys@transformshift{0.225000in}{0.614091in}%
\pgfsys@useobject{currentmarker}{}%
\end{pgfscope}%
\end{pgfscope}%
\begin{pgfscope}%
\pgfsetbuttcap%
\pgfsetroundjoin%
\definecolor{currentfill}{rgb}{0.000000,0.000000,0.000000}%
\pgfsetfillcolor{currentfill}%
\pgfsetlinewidth{0.501875pt}%
\definecolor{currentstroke}{rgb}{0.000000,0.000000,0.000000}%
\pgfsetstrokecolor{currentstroke}%
\pgfsetdash{}{0pt}%
\pgfsys@defobject{currentmarker}{\pgfqpoint{-0.055556in}{0.000000in}}{\pgfqpoint{0.000000in}{0.000000in}}{%
\pgfpathmoveto{\pgfqpoint{0.000000in}{0.000000in}}%
\pgfpathlineto{\pgfqpoint{-0.055556in}{0.000000in}}%
\pgfusepath{stroke,fill}%
}%
\begin{pgfscope}%
\pgfsys@transformshift{4.275000in}{0.614091in}%
\pgfsys@useobject{currentmarker}{}%
\end{pgfscope}%
\end{pgfscope}%
\begin{pgfscope}%
\pgftext[x=0.169444in,y=0.614091in,right,]{\rmfamily\fontsize{9.000000}{10.800000}\selectfont \(\displaystyle -10\)}%
\end{pgfscope}%
\begin{pgfscope}%
\pgfsetbuttcap%
\pgfsetroundjoin%
\definecolor{currentfill}{rgb}{0.000000,0.000000,0.000000}%
\pgfsetfillcolor{currentfill}%
\pgfsetlinewidth{0.501875pt}%
\definecolor{currentstroke}{rgb}{0.000000,0.000000,0.000000}%
\pgfsetstrokecolor{currentstroke}%
\pgfsetdash{}{0pt}%
\pgfsys@defobject{currentmarker}{\pgfqpoint{0.000000in}{0.000000in}}{\pgfqpoint{0.055556in}{0.000000in}}{%
\pgfpathmoveto{\pgfqpoint{0.000000in}{0.000000in}}%
\pgfpathlineto{\pgfqpoint{0.055556in}{0.000000in}}%
\pgfusepath{stroke,fill}%
}%
\begin{pgfscope}%
\pgfsys@transformshift{0.225000in}{0.805000in}%
\pgfsys@useobject{currentmarker}{}%
\end{pgfscope}%
\end{pgfscope}%
\begin{pgfscope}%
\pgfsetbuttcap%
\pgfsetroundjoin%
\definecolor{currentfill}{rgb}{0.000000,0.000000,0.000000}%
\pgfsetfillcolor{currentfill}%
\pgfsetlinewidth{0.501875pt}%
\definecolor{currentstroke}{rgb}{0.000000,0.000000,0.000000}%
\pgfsetstrokecolor{currentstroke}%
\pgfsetdash{}{0pt}%
\pgfsys@defobject{currentmarker}{\pgfqpoint{-0.055556in}{0.000000in}}{\pgfqpoint{0.000000in}{0.000000in}}{%
\pgfpathmoveto{\pgfqpoint{0.000000in}{0.000000in}}%
\pgfpathlineto{\pgfqpoint{-0.055556in}{0.000000in}}%
\pgfusepath{stroke,fill}%
}%
\begin{pgfscope}%
\pgfsys@transformshift{4.275000in}{0.805000in}%
\pgfsys@useobject{currentmarker}{}%
\end{pgfscope}%
\end{pgfscope}%
\begin{pgfscope}%
\pgftext[x=0.169444in,y=0.805000in,right,]{\rmfamily\fontsize{9.000000}{10.800000}\selectfont \(\displaystyle 0\)}%
\end{pgfscope}%
\begin{pgfscope}%
\pgfsetbuttcap%
\pgfsetroundjoin%
\definecolor{currentfill}{rgb}{0.000000,0.000000,0.000000}%
\pgfsetfillcolor{currentfill}%
\pgfsetlinewidth{0.501875pt}%
\definecolor{currentstroke}{rgb}{0.000000,0.000000,0.000000}%
\pgfsetstrokecolor{currentstroke}%
\pgfsetdash{}{0pt}%
\pgfsys@defobject{currentmarker}{\pgfqpoint{0.000000in}{0.000000in}}{\pgfqpoint{0.055556in}{0.000000in}}{%
\pgfpathmoveto{\pgfqpoint{0.000000in}{0.000000in}}%
\pgfpathlineto{\pgfqpoint{0.055556in}{0.000000in}}%
\pgfusepath{stroke,fill}%
}%
\begin{pgfscope}%
\pgfsys@transformshift{0.225000in}{0.995909in}%
\pgfsys@useobject{currentmarker}{}%
\end{pgfscope}%
\end{pgfscope}%
\begin{pgfscope}%
\pgfsetbuttcap%
\pgfsetroundjoin%
\definecolor{currentfill}{rgb}{0.000000,0.000000,0.000000}%
\pgfsetfillcolor{currentfill}%
\pgfsetlinewidth{0.501875pt}%
\definecolor{currentstroke}{rgb}{0.000000,0.000000,0.000000}%
\pgfsetstrokecolor{currentstroke}%
\pgfsetdash{}{0pt}%
\pgfsys@defobject{currentmarker}{\pgfqpoint{-0.055556in}{0.000000in}}{\pgfqpoint{0.000000in}{0.000000in}}{%
\pgfpathmoveto{\pgfqpoint{0.000000in}{0.000000in}}%
\pgfpathlineto{\pgfqpoint{-0.055556in}{0.000000in}}%
\pgfusepath{stroke,fill}%
}%
\begin{pgfscope}%
\pgfsys@transformshift{4.275000in}{0.995909in}%
\pgfsys@useobject{currentmarker}{}%
\end{pgfscope}%
\end{pgfscope}%
\begin{pgfscope}%
\pgftext[x=0.169444in,y=0.995909in,right,]{\rmfamily\fontsize{9.000000}{10.800000}\selectfont \(\displaystyle 10\)}%
\end{pgfscope}%
\begin{pgfscope}%
\pgfsetbuttcap%
\pgfsetroundjoin%
\definecolor{currentfill}{rgb}{0.000000,0.000000,0.000000}%
\pgfsetfillcolor{currentfill}%
\pgfsetlinewidth{0.501875pt}%
\definecolor{currentstroke}{rgb}{0.000000,0.000000,0.000000}%
\pgfsetstrokecolor{currentstroke}%
\pgfsetdash{}{0pt}%
\pgfsys@defobject{currentmarker}{\pgfqpoint{0.000000in}{0.000000in}}{\pgfqpoint{0.055556in}{0.000000in}}{%
\pgfpathmoveto{\pgfqpoint{0.000000in}{0.000000in}}%
\pgfpathlineto{\pgfqpoint{0.055556in}{0.000000in}}%
\pgfusepath{stroke,fill}%
}%
\begin{pgfscope}%
\pgfsys@transformshift{0.225000in}{1.186818in}%
\pgfsys@useobject{currentmarker}{}%
\end{pgfscope}%
\end{pgfscope}%
\begin{pgfscope}%
\pgfsetbuttcap%
\pgfsetroundjoin%
\definecolor{currentfill}{rgb}{0.000000,0.000000,0.000000}%
\pgfsetfillcolor{currentfill}%
\pgfsetlinewidth{0.501875pt}%
\definecolor{currentstroke}{rgb}{0.000000,0.000000,0.000000}%
\pgfsetstrokecolor{currentstroke}%
\pgfsetdash{}{0pt}%
\pgfsys@defobject{currentmarker}{\pgfqpoint{-0.055556in}{0.000000in}}{\pgfqpoint{0.000000in}{0.000000in}}{%
\pgfpathmoveto{\pgfqpoint{0.000000in}{0.000000in}}%
\pgfpathlineto{\pgfqpoint{-0.055556in}{0.000000in}}%
\pgfusepath{stroke,fill}%
}%
\begin{pgfscope}%
\pgfsys@transformshift{4.275000in}{1.186818in}%
\pgfsys@useobject{currentmarker}{}%
\end{pgfscope}%
\end{pgfscope}%
\begin{pgfscope}%
\pgftext[x=0.169444in,y=1.186818in,right,]{\rmfamily\fontsize{9.000000}{10.800000}\selectfont \(\displaystyle 20\)}%
\end{pgfscope}%
\begin{pgfscope}%
\pgfsetbuttcap%
\pgfsetroundjoin%
\definecolor{currentfill}{rgb}{0.000000,0.000000,0.000000}%
\pgfsetfillcolor{currentfill}%
\pgfsetlinewidth{0.501875pt}%
\definecolor{currentstroke}{rgb}{0.000000,0.000000,0.000000}%
\pgfsetstrokecolor{currentstroke}%
\pgfsetdash{}{0pt}%
\pgfsys@defobject{currentmarker}{\pgfqpoint{0.000000in}{0.000000in}}{\pgfqpoint{0.055556in}{0.000000in}}{%
\pgfpathmoveto{\pgfqpoint{0.000000in}{0.000000in}}%
\pgfpathlineto{\pgfqpoint{0.055556in}{0.000000in}}%
\pgfusepath{stroke,fill}%
}%
\begin{pgfscope}%
\pgfsys@transformshift{0.225000in}{1.377727in}%
\pgfsys@useobject{currentmarker}{}%
\end{pgfscope}%
\end{pgfscope}%
\begin{pgfscope}%
\pgfsetbuttcap%
\pgfsetroundjoin%
\definecolor{currentfill}{rgb}{0.000000,0.000000,0.000000}%
\pgfsetfillcolor{currentfill}%
\pgfsetlinewidth{0.501875pt}%
\definecolor{currentstroke}{rgb}{0.000000,0.000000,0.000000}%
\pgfsetstrokecolor{currentstroke}%
\pgfsetdash{}{0pt}%
\pgfsys@defobject{currentmarker}{\pgfqpoint{-0.055556in}{0.000000in}}{\pgfqpoint{0.000000in}{0.000000in}}{%
\pgfpathmoveto{\pgfqpoint{0.000000in}{0.000000in}}%
\pgfpathlineto{\pgfqpoint{-0.055556in}{0.000000in}}%
\pgfusepath{stroke,fill}%
}%
\begin{pgfscope}%
\pgfsys@transformshift{4.275000in}{1.377727in}%
\pgfsys@useobject{currentmarker}{}%
\end{pgfscope}%
\end{pgfscope}%
\begin{pgfscope}%
\pgftext[x=0.169444in,y=1.377727in,right,]{\rmfamily\fontsize{9.000000}{10.800000}\selectfont \(\displaystyle 30\)}%
\end{pgfscope}%
\begin{pgfscope}%
\pgftext[x=2.250000in,y=1.504444in,,base]{\rmfamily\fontsize{11.000000}{13.200000}\selectfont Moduliertes Signal}%
\end{pgfscope}%
\end{pgfpicture}%
\makeatother%
\endgroup%

    \caption{%
        \emph{Amplitude-shift  keying}: Oben   sind  die   zu  \"ubertragenden
        digitalen  Daten  als  \code{1}  und \code{0}  abgebildet,  unten  das
        zugeh\"orige  Verhalten  des   \"ubertragenen  Signals. Die  Form  des
        Signals w\"urde  in unserem System  etwas anders aussehen, da  wir ein
        \SI{960}{\volt}-Signal  kurzzeitig um  jeweils  bis zu  \SI{60}{\volt}
        einbrechen lassen  w\"urden, anstatt mit  sch\"onen Sinus-Schwingungen
        zu arbeiten. Das  Konzept einer diskreten Anzahl  Frequenzen und ihrer
        Zuordnung zu entweder \code{0} oder \code{1} bleibt aber identisch.%
    }
    \label{fig:ask:concept}
\end{figure}
\todo{Verweis auf Simulationen}

Die  Implementierung   einer  FSK-L\"osung  ist  prinzipiell   dich  sicherere
Variante.  Fertige  Microchips, auf denen  man eine solche  Schaltung aufbauen
kann,  existieren  bereits  am  Markt. Die prim\"are  Schwierigkeit  liegt  in
der  Frage,  wie man  das  Signal  in  die DC-Leitung  einkoppelt. Wichtig  an
der  Einkopplung  ist,  dass  sie potentialgetrennt  von  der  DC-Leitung  ist
\todo{Korrekt? Begr\"undung?};  man kann  also  nicht  einfach die  DC-Leitung
mit   dem  Ausgang   des   Signalgebers  verbinden   \todo{Strom  geht   durch
Signalgeber? Spannung?} Daf\"ur bietet sich entweder eine kapazitive oder eine
induktive Einkopplung an \todo{Schaltungen}.

Ein  Modul kurzzuschliessen  und damit  eine  Form von  ASK zu  implementieren
w\"are  eine sehr  sparsame L\"osung  mit  minimalem Aufwand  (und somit  auch
Kosten) f\"ur die Schaltung. Allerdings  bringt sie einige Schwierigkeiten mit
sich. Es  ist  ein  Konzept,  das  nicht  verbreitet  und  gut  getestet  ist;
die  Machbarkeit  ist  nicht  garantiert. Es ist  daher  sinnvoll,  vor  einer
allf\"alligen Umsetzung so viele Informationen wie m\"oglich \"uber m\"ogliche
Problemzonen einzuholen, anhand  derer beurteilt werden kann,  ob die L\"osung
im Rahmen  des Projektes  realisiertbar ist,  oder ob  ein Erfolg  zu unsicher
w\"are.

% ---------------------------------------------------------------------------- %
\subsection{FSK: Kapazitive oder induktive Einkopplung}
\label{subsec:hw:fsk:kapaVsInduk}
% ---------------------------------------------------------------------------- %

Eine kapazitive Einkopplung hat die Vor- und Nachteile, dass \ldots, w\"ahrend
\ldots Nachteile sind.\todo{Vor- Nachteile}
\todo{Circuit}


Eine induktive Einkopplung legt eine  Spule um die DC-Leitung. Auf diese Spule
wird von der FSK-Schaltung das zu  \"ubertragende Signal gegeben und die Spule
induziert in  der DC-Leitung  entsprechende Spannungs-Rippel, die  vom \Master
ausgewertet werden k\"onnen. Der entsprechende  Schaltkreis ist schematisch in
Abbildung \ref{fig:circ:coupling:inductive} dargestellt.

Verglichen mit  Kondensatoren sind Spulen relativ  gross und teuer. Allerdings
ist das Prinzip  der induktiven Einkopplung gut dokumentiert  und die Aussicht
auf Erfolg (bei vern\"unftigem Aufwand) somit gut.

\begin{figure}[h!tb]
    \centering
    \begin{circuitikz}
    \draw
    (-1,0) to[empty photodiode,o-,l_=PV-Zelle] (1,0) to[short] (6,0)
    %(2,-2) to[short,o-] (2,-0.05) to[L=L] (4,-0.05) to[short,-o] (4,-2)
    (2,-2) -- (2,-0.05) to[L,l^=Kopplung] (4,-0.05) -- (4,-2) to[vco,l^=$U_{\mathrm{Signal}}$] (2,-2)
    ;
\end{circuitikz}

    \caption{Induktive Einkopplung}
    \label{fig:circ:coupling:inductive}
\end{figure}


% ---------------------------------------------------------------------------- %
\subsection{ASK: Machbarkeitsanalyse}
\label{subsec:hw:ask:machbarkeit}
% ---------------------------------------------------------------------------- %

Um die Machbarkeit dieser L\"osungsvariante zu beurteilen, wurden Simulationen
durchgef\"uhrt. Dazu  wurde  zuerst  eine  Modellschaltung  entwickelt,  deren
Parameter  definiert  und  anschliessend   Simulationen  mit  der  definierten
Konfiguration durchgef\"uhrt.

Als   Simulationstool  wird   \code{LTspice   IV}   von  Linear   Technologies
eingesetzt~\cite{ref:ltspice}.


% ---------------------------------------------------------------------------- %
\subsubsection{Modellierung eines Solarmoduls}
\label{subsubsec:hw:ask:modell}
% ---------------------------------------------------------------------------- %

Als  Ausganspunkt  des  Modells  f\"ur ein  Modul  dient  das  Eindiodenmodell
\todo{Bindestrich  oder   ein  Wort?}   gem\"ass  Abbildung. Dabei   wird  das
klassische   Eindiodenmodell  um   einen  Kondensator   der  Kapazit\"at   $C$
parallel zur  Stromquelle erg\"anzt  (basierend auf Informationen  aus Quellen
\cite{ref:solar:scofield} und \cite{ref:solar:friesen}).

\begin{figure}[h!tb]
    \centering
    \begin{circuitikz}
    \draw
    % Source to top terminal
    (0,0) to[current source,i=$I_{\mathrm{Zelle}}$] (0,4) -- (5,4) to[R,l^=$R_{\mathrm{S}}$] (10,4) node[ocirc] {}

    % Open terminal at bottom
    (10,0) node[ocirc] {} -- (0,0)

    % Parallel elements
    (1.5,4) to[empty diode,*-*,l_=$D$] (1.5,0)
    (3,4) to[C,*-*,l_=$C$] (3,0)
    (4.5,4) to[R,*-*,l_=$R_{\mathrm{P}}$] (4.5,0)
    ;
\end{circuitikz}

    \caption{%
        Schaltschema    zur    Modellierung    einer    Solarzelle    gem\"ass
        Eindiodenmodell mit zus\"atzlicher Kapazit\"at%
    }
    \label{fig:circuit:solarCell}
\end{figure}

Zur   Ermittlung  der   Parameter   dieses  Modells   wurde  pr\"imar   Quelle
\cite{ref:solar:scofield}  herangezogen. Es wurden  $C$,  $R_{S}$ und  $R_{P}$
einer   Solarzelle   der   Gr\"osse   \SI{0.43}{\centi\meter\squared}   \"uber
einen   Frequenzbereicht  von   \SI{1}{\kilo\hertz}  bis   \SI{1}{\mega\hertz}
ausgemessen. Die Resultate waren:

\begin{align}
    C_{\mathrm{Messung}}    &= \SI{8}{\nano\farad} \text{bis} \SI{20}{\nano\farad}  = \SI{14 \pm 6}{\nano\farad} \\
    R_{\mathrm{S, Messung}} &= \SI{0.2}{\ohm}      \text{bis} \SI{20}{\ohm}         = \SI{10.1 \pm 9.9}{\ohm}     \\
    R_{\mathrm{P, Messung}} &= \SI{0.5}{\kilo\ohm} \text{bis} \SI{500}{\kilo\ohm}   = \SI{250.25 \pm 249.75}{\kilo\ohm}
\end{align}

Um  die obigen  Werte  f\"ur  unsere Zwecke  verwenden  zu k\"onnen,  m\"ussen
sie   von  einer   Solarzelle  der   Gr\"osse  \SI{0.43}{\centi\meter\squared}
auf   die  Gr\"osse   einer   Zelle,   wie  sie   in   der  Praxis   verwendet
wird,  heraufskaliert   werden. Wir  werden   hierbei  von  einer   Zelle  der
Gr\"osse $\SI{100}{\milli\meter} \times  \SI{100}{\milli\meter}$ ausgehen, was
ungef\"ahr  der $230$-fachen  Fl\"ache  der Solarzellen  aus obigen  Messungen
entspricht.
\todo{Referenzen auf spezifische Textstellen?}

$R_{\mathrm{P}}$   und  $R_{\mathrm{S}}$   skalieren  umgekehrt   proportional
zur  Zellfl\"ache,  wogegen  $C$   bei  gr\"osser  werdender  Zelloberfl\"ache
ansteigt~\cite{ref:solar:scofield}. Wir  nehmen  die  Mittelwerte  aus  obigen
Gleichungen und bestimmen unsere Parameter:

\begin{align}
    C               &= C_{\mathrm{Messung}}    \cdot 230 = \SI{3.2}{\micro\farad} \\
    R_{\mathrm{S,}} &= R_{\mathrm{S, Messung}} \div  230 = \SI{44}{\milli\ohm}    \\
    R_{\mathrm{P,}} &= R_{\mathrm{P, Messung}} \div  230 = \SI{1.1}{\kilo\ohm}
\end{align}

Zur Bestimmung  werden die Spezifikationen eines  \emph{Solarex MSX-60}-Moduls
herangezogen  (verf\"ugbar in  Quelle \cite{ref:solar:bonkoungou}). Das  Modul
besteht  aus  36  in  Serie  verbundenen   Zellen  und  hat  einen  Strom  von
\todo{Kurzschlusstrom   oder   Peak-Power   Strom?}    \SI{3.8}{\ampere}   bei
maximaler  Leistungsabgabe. Die Stromquelle  im Modell  wird daher  auf diesen
Wert  eingestellt.
%Kurzzschlussstrom von \SI{3.8}{\ampere} bei
Die  Spannung  bei   maximaler  Leistungsabgabe  liegt  bei
\SI{17.1}{\volt}, die
Leerlaufspannung gem\"ass IV-Kurve bei etwa \SI{21}{\volt}.

Heruntergerechnet  auf  eine einzelne  Zelle  ergibt  dies eine  Spannung  bei
maximaler Leistungsabgabe von  \SI{475}{\milli\volt} und eine Leerlaufspannung
pro Zelle von \SI{580}{\milli\volt}.

Als  letzte Komponente  ist  die Diode  zu  bestimmen. Dies erfolgt  iterative
in  \code{LTspice}.
\todo{Erkl\"arung von IS und N}
\todo{Schema von LTspice}

\begin{itemize}
    \item
        Definition eines simplen  Diodenmodells mit gesch\"atzten Startwerten:
        \code{.model diode1 D(IS=1e-6 N=2)}
    \item
        Transientensimulation der Zelle: \code{.tran 1m}
    \item
        Messung    der   Zellenspannnung,    Vergleich   mit    Zielwert   von
        ca. \SI{580}{\milli\volt}
    \item
        \"Andern von  \code{IS} und  \code{N} im Diodenmodell  und Wiederholen
        des Vorgangs, bis ein zufriedenstellendes Ergebnis erzielt ist.
\end{itemize}

Nach einigen Iterationen liefert dieser Prozess:\todo{Einheiten}

\begin{align}
    \label{eq:cell:diode:IS:N:result}
    I_s &= 3e-6 \\
    N   &= 1.5
\end{align}


%\begin{figure}[h!tb]
%    \centering
%    \includegraphics[width=\textwidth]{images/ltspice/solarmodul-zellenbasiert.eps}
%    \caption{Solarmodul, modelliert durch 2 parallele Strings mit je 36 Zellen in Serie}
%    \label{fig:ltspice:solarmodul:cellBased}
%\end{figure}

% ---------------------------------------------------------------------------- %
\clearpage
\section{Sensorplatine}
\label{sec:hw:sensorplatine}
% ---------------------------------------------------------------------------- %

Was  ist der  Zweck  der Sensorplatine? Welche  Aufgaben  muss sie  erf\"ullen
k\"onnen? Wie ist sie aufgebaut? PCB-Layout? Energieversorgung?

\anweisung Dimensionierung der Bauteile: Berechnungen

% ---------------------------------------------------------------------------- %
\subsection{Energieversorgung}
\label{subsec:sensor:pcb}
% ---------------------------------------------------------------------------- %

Energiebezug,  Leistungsanforderungen, Standby,  Verhalten bei  ungen\"ugender
Leistungszufuhr, ...


% ---------------------------------------------------------------------------- %
\subsection{PCB}
\label{subsec:sensor:pcb}
% ---------------------------------------------------------------------------- %

Wie sieht das PCB aus, und weshalb?

% ---------------------------------------------------------------------------- %
\section{Master-Ger\"at}
\label{sec:hw:mastergerat}
% ---------------------------------------------------------------------------- %

Was ist der Zweck des Master-Ger\"ats? Aufgaben? Aufbau?

\anweisung Dimensionierung der Bauteile: Berechnungen


% ---------------------------------------------------------------------------- %
\subsection{Speisung}
\label{subsec:mastergerat:speisung}
% ---------------------------------------------------------------------------- %

Woher bezieht das Master-Ger\"at  seine Energie, weshalb? Wieviel Energie wird
ben\"otigt?


% ---------------------------------------------------------------------------- %
\subsection{Benutzer-Interface}
\label{subsec:mastergerat:interface}
% ---------------------------------------------------------------------------- %

\anweisung Das Benutzerinterface ist schon in der Anleitung im vorigen Kapitel
beschrieben worden. Es soll  hier um die Komponentenwahl gehen,  nicht um eine
Rekapitulation bereits gemachter Informationen.

% ---------------------------------------------------------------------------- %
\subsection{PCB}
\label{subsec:mastergerat:pcb}
% ---------------------------------------------------------------------------- %

Wie sieht das PCB aus, und weshalb?

% ---------------------------------------------------------------------------- %
\subsection{Montage/Geh\"ause}
\label{subsec:mastergerat:pcb}
% ---------------------------------------------------------------------------- %

Einbindung in die Umgebung, Wahl des Geh\"auses.


% ---------------------------------------------------------------------------- %
\section{Kommunikation}
\label{sec:kommunikation}
% ---------------------------------------------------------------------------- %

Die   Kommunikation  zwischen   Sensor  und   Master-Ger\"at  ist   eines  der
Herzst\"ucke des ganzen Systems und betrifft beide Sub-Systeme. Daher in einem
eigenen Abschnitt.

\begin{itemize}
    \item
        Grunds\"atzliches Problem
    \item
        Verfolgte L\"osungsans\"atze
    \item
        Ausgew\"ahlter L\"osungsansatz (mit Begr\"undung(en))
    \item
        Genauere Beschreibung dieses L\"osungsansatzes
\end{itemize}
