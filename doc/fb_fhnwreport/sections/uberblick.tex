Dieses Kapitel  beschreibt zuerst die grobe  Idee unsers L\"osungskonzepts. Es
wird  dargelegt, wie  unser System  in  eine Solaranlage  (bestehend oder  neu
aufgebaut) integriert wird, wie das System mit seiner Umgebung interagiert und
wie es zu bedienen ist.

Es  wird sowohl  die  Inbetriebnahme  wie auch  die  Benutzung im  regul\"aren
Betrieb beschrieben.


% ---------------------------------------------------------------------------- %
\subsection{Einbettung in Umwelt}
\label{subsec:einbettung}
% ---------------------------------------------------------------------------- %

Hier  wird  beschrieben,  wie  unser System  physisch  mit  einer  Solaranlage
integriert wird und welche Schnittstellen es zu welchem Zweck zur Anlage hat.


% ---------------------------------------------------------------------------- %
\subsection{Installation und Inbetriebnahme}
\label{subsec:installation}
% ---------------------------------------------------------------------------- %

Wie wird das Ger\"at installiert, was  gibt es dabei zu beachten? Wie wird das
\"Uberwachungssystem nach  der Installation in Betrieb  genommen? Diese Fragen
sollen an dieser Stelle beantwortet werden.


% ---------------------------------------------------------------------------- %
\subsection{Regul\"are Benutzung}
\label{subsec:regular}
% ---------------------------------------------------------------------------- %

Hier wird beschrieben, was der Benutzer  von der Anlage im regul\"aren Betrieb
erwarten kann.
