Die    Risikoanalyse   ist    aufgeteilt   nach    organisatorischen   Risiken
(Tabelle   \ref{tab:risikoanalyse:org})  und   technischen  Risiken   (Tabelle
\ref{tab:risikoanalyse:tech}, Seite \pageref{tab:risikoanalyse:tech}). Tabelle
\ref{tab:risikoanalyseLegende}   enth\"alt  eine   Legende  zu   den  Tabellen
\ref{tab:risikoanalyse:org} und \ref{tab:risikoanalyse:tech} mit Erkl\"arungen
zu den einzelnen Risikostufen.

\begin{center}
\setcounter{table}{2}
\captionof{table}{Risikoanalyse, organisatorische Risiken. \\ \textbf{p}: Eintrittswahrscheinlichkeit, \textbf{S}: Schadensausmass bei Eintreten, \textbf{R}: Risiko. $R = p \cdot S$}
\label{tab:risikoanalyse:org}
%\begin{table}[h!]
{%
    \small
\begin{longtable}{
        >{\raggedright}p{47.5mm} % Schadensereignis
        >{\raggedright}p{30mm} % Konsequenz
        >{\raggedright}r % Eintrittswahrscheinlichkeit
        >{\raggedright}r % Schadensausmass
        >{\raggedright}r % Riskio
        >{\raggedright}p{50mm} % Massnahmen
        >{\raggedright}p{30mm}   % Auswirkung
        >{\raggedright}r   % p
        >{\raggedright}r   % S
        >{\raggedright\arraybackslash}r   % R
}
\toprule

% 1
& % 2
& \multicolumn{3}{p{18mm}}{ohne Massnahmen} % 4,5,6
& % 6
& % 7
& \multicolumn{3}{p{18mm}}{mit Massnahmen} \\ %9,10,11

\textsc{Schadensereignis}
& \textsc{Konsequenz}
& $p$
& $S$
& $R$
& \textsc{Vorbeugende Massnahmen}
& \textsc{Auswirkung}
& $p$
& $S$
& $R$ \\

%\large{\textsc{Aufgabe}}
%& \large{\textsc{Zweck}}
%& \large{\textsc{Empf\"anger}}
%& \large{\textsc{Mittel}}
%& \large{\textsc{H\"aufigkeit}}
%& \large{\textsc{Verantwortlicher}} \\

\midrule
\endhead
\midrule
\endfoot
\bottomrule
\endlastfoot

%\rowcolor[gray]{0.9}

Meinungsverschiedenheiten im Team
& Verz\"ogerung des Projektplans
& $3$
& $3$
& $9$
& Kompromissbereitschaft \& Probleme ansprechen
& Vermeidung von Konflikten
& $2$
& $3$
& $6$ \\
[2mm]

\rowcolor{black!10}
Nicht gen\"ugend Fachwissen
& Aufgaben werden falsch erledigt
& $4$
& $3$
& $12$
& Fachwissen aneignen \& externe Hilfe einholen
& Minimierung von Wissensl\"ucken
& $3$
& $3$
& $9$ \\
[2mm]

Termine nicht eingehalten
& Nichterf\"ullen der Projektziele
& $3$
& $5$
& $15$
& genaue Zeitplanung
& Bessere Termin\"ubersicht
& $2$
& $5$
& $10$ \\
[2mm]

\rowcolor{black!10}
Projektmitglied bricht Studium ab
& Mehraufwand f\"ur das restliche Team
& $2$
& $3$
& $6$
& keine
& keine
& $2$
& $3$
& $6$ \\
[2mm]

Datenverlust
& Doppelte Arbeit
& $2$
& $4$
& $8$
& Gen\"ugend Backups erstellen
& Minimierung der Verluste
& $2$
& $3$
& $6$ \\
[2mm]

\rowcolor{black!10}
Krankheit eines Teammitglieds
& Mehraufwand f\"ur das restliche Team
& $4$
& $3$
& $12$
& Ausf\"uhrliche Protokolle
& Stand der Dinge nachvollziehbar
& $4$
& $2$
& $8$ \\
\end{longtable}
}

\setcounter{table}{3}
\captionof{table}{Legende zu Tabelle \ref{tab:risikoanalyse:org} und \ref{tab:risikoanalyse:tech}}
\label{tab:risikoanalyseLegende}
\begin{tabular}{cll}
    \toprule
    \textsc{Gewichtung} & Eintrittswahrscheinlichkeit & Schadensausmass \\
    \midrule
    1 & so gut wie ausgeschlossen & vernachl\"assigbar \\
    2 & unwahrscheinlich          & gering             \\
    3 & m\"oglich                 & mittel             \\
    4 & wahrscheinlich            & hoch               \\
    5 & so gut wie sicher         & gravierend         \\
    \bottomrule
\end{tabular}

\clearpage
\setcounter{table}{4}
\captionof{table}{Risikoanalyse, technische Risiken. \\ \textbf{p}: Eintrittswahrscheinlichkeit, \textbf{S}: Schadensausmass bei Eintreten, \textbf{R}: Risiko. $R = p \cdot S$}
\label{tab:risikoanalyse:tech}
{%
    \small
\begin{longtable}{
        >{\raggedright}p{47.5mm} % Schadensereignis
        >{\raggedright}p{30mm} % Konsequenz
        >{\raggedright}r % Eintrittswahrscheinlichkeit
        >{\raggedright}r % Schadensausmass
        >{\raggedright}r % Riskio
        >{\raggedright}p{50mm} % Massnahmen
        >{\raggedright}p{30mm}   % Auswirkung
        >{\raggedright}r   % p
        >{\raggedright}r   % S
        >{\raggedright\arraybackslash}r   % R
}
\toprule
% 1
& % 2
& \multicolumn{3}{p{18mm}}{ohne Massnahmen} %3,4,5
& % 6
& % 7
& \multicolumn{3}{p{18mm}}{mit Massnahmen} \\ %9,10,11

% 1
\textsc{Schadensereignis}
& \textsc{Konsequenz}
& $p$
& $S$
& $R$
& \textsc{Vorbeugende Massnahmen}
& \textsc{Auswirkung}
& $p$
& $S$
& $R$ \\

%\large{\textsc{Aufgabe}}
%& \large{\textsc{Zweck}}
%& \large{\textsc{Empf\"anger}}
%& \large{\textsc{Mittel}}
%& \large{\textsc{H\"aufigkeit}}
%& \large{\textsc{Verantwortlicher}} \\

\midrule
\endhead
\midrule
\endfoot
\bottomrule
\endlastfoot
Meinungsverschiedenheiten im Team
& Verz\"ogerung des Projektplans
& $3$
& $3$
& $9$
& Kompromissbereitschaft \& Probleme ansprechen
& Vermeidung von Konflikten
& $2$
& $3$
& $6$ \\
[2mm]

\rowcolor{black!10}
Softwarefehler
& Programmabsturz
& $3$
& $2$
& $6$
& Verifizierung durch mehrere Teammitglieder
& Fehlerquote reduzieren
& $2$
& $2$
& $4$ \\
[2mm]

Hardwarefehler
& Tests nicht erfolgreich
& $3$
& $3$
& $9$
& Genaue Planungsarbeiten
& Fehlerquote reduzieren
& $2$
& $4$
& $6$ \\
[2mm]

\rowcolor{black!10}
Hardware fehlerhaft geliefert
& Komponenten sind nicht rechtzeitig bereit
& $3$
& $4$
& $12$
& Lieferzeiten doppelt einplanen
& Fr\"uhzeitig die Hardwarekomponenten bestimmen
& $3$
& $3$
& $9$ \\
[2mm]

Hardware zu sp\"at geliefert
& Komponenten sind nicht rechtzeitig bereit
& $3$
& $4$
& $12$
& Lieferzeiten doppelt einplanen
& Fr\"uhzeitig die Hardwarekomponenten bestimmen
& $2$
& $3$
& $6$ \\
[2mm]

\rowcolor{black!10}
Signal\"ubertragung funktioniert nicht korrekt
& Daten werden fehlerhaft oder nicht \"ubertragen
& $4$
& $5$
& $20$
& Tests \"uber grosse Distanzen durchf\"uhren
& Ausgrenzen einniger Signal\"ubertragungsverfahren
& $2$
& $5$
& $10$ \\
[2mm]

Baugr\"osse nicht realisierbar
& Komponenten passen nicht in den vorgesehenen Anschlusskasten
& $3$
& $5$
& $15$
& Maximale Baugr\"osse definieren
& Eingrenzen der Komponentenauswahl
& $2$
& $5$
& $10$ \\
\end{longtable}
}

%\thetable
%\setcounter{table}{3}
\end{center}
