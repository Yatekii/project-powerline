F\"ur   die  Regler-Dimensionierung   sowie  das   grafische  Darstellen   der
Schrittantwort sind einige Berechnungen notwendig, welche im folgenden Kapitel
erl\"autert werden.   Grunds\"atzlich k\"onnen diese Berechnungen  in folgende
Teilgebiete aufgeteilt werden:

\begin{itemize}
    \item
        Frequenzgangberechnung der Regelstrecke aus $T_u$, $T_g$ und $K_s$
    \item
        Regler-Dimensionierung mit den Faustformeln
    \item
        Regler-Dimensionierung anhand der Phasengang-Methode
    \item
        Umrechnung der Regler-Darstellung zwischen Bode- und Reglerkonform
    \item
        Berechnung der Schrittantwort des geschlossenen Regelkreises
\end{itemize}


\subsection{Frequenzgangberechnung der Regelstrecke}

F\"ur  die  Identifikation  des  Freqeunzgangs steht  die  Matlab-Funktion
\code{p2\_sani.m} zur  Verf\"ugung, welche in der Lage  ist, PTn  Strecken mit
einem Grad zwischen 1 und 8 zu identifizieren. Deshalb wird hier nicht n\"aher
auf  die  Berechnung  eingegangen,  da die  Funktion  lediglich  in  Java-Code
``\"ubersetzt'' werden muss.


\subsection{Regler Dimensionierung mit den Faustformeln}

Im   Praxiseinsatz  stehen   f\"ur   die  Dimensieung   der  Regler   einfache
Berechnungsformeln f\"ur die Einstellwerte der  Regler anhand von $T_u$, $T_g$
und $K_s$ zur Verf\"ugung.

Einige   dieser  Faustformeln   werden  in   der  Applikation   zum  Vergleich
mitberechnet. Die    dazugeh\"origen    Berechnungen    sind    der    Tabelle
\ref{tab:faustformeln} zu entnehmen.


%\begin{longtable}{p{43mm}p{72.5mm}p{54mm}}
\begin{longtable}{p{50mm}rrrrr}
    \toprule

    %\multicolumn{3}{l}{\large{\textsc{Auftragsanalyse und Hintergrundinformationen}}} \\

    Faustformel
    &
    \multicolumn{2}{l}{PI-Regler}
    &
    \multicolumn{2}{l}{PID-T1-Regler}
    \\

    &
    $T_n$
    &
    $K_p$
    &
    $T_n$
    &
    $T_v$
    &
    $K_p$
    \\

    \midrule

    \endhead
    \endfoot
    \endlastfoot

    % CONTENT HERE ---------------------------------------------------------- %

    \pbox{45mm}{Chiens, Hrones, Reswick \\ \small{\textbf{(0\% \"Uberschwingen)}} \\ \cite{ref:chiens_tsn}, \cite{ref:chiens_wiki}}
    &
    $1.2\cdot T_g$
    &
    $\frac{0.35}{K_s} \cdot \frac{T_g}{T_u}$
    &
    $T_g$
    &
    $0.5\cdot T_u$
    &
    $ \frac{0.6}{K_s} \cdot \frac{T_g}{T_u} $
    \\

    \addlinespace[1em]

    \pbox{45mm}{Chiens, Hrones, Reswick \small{\textbf{(20\% \"Uberschwingen)}} \\ \cite{ref:chiens_tsn}, \cite{ref:chiens_wiki}}
    &
    $T_g$
    &
    $\frac{0.6}{K_s} \cdot \frac{T_g}{T_u}$
    &
    $1.35\cdot T_g$
    &
    $0.47 \cdot T_u$
    &
    $ \frac{0.95}{K_s} \cdot \frac{T_g}{T_u} $
    \\

    \addlinespace[1em]

    Oppelt \cite{ref:op_ros_zieg}
    &
    $3 \cdot T_u$
    &
    $\frac{0.8}{K_s} \cdot \frac{T_g}{T_u}$
    &
    $2 \cdot T_u$
    &
    $ 0.42 \cdot T_u $
    &
    $ \frac{1.2}{K_s} \cdot \frac{T_g}{T_u} $
    \\

    \addlinespace[1em]

    Rosenberg \cite{ref:op_ros_zieg}
    &
    $3.3 \cdot T_u $
    &
    $ \frac{0.91}{K_s} \cdot \frac{T_g}{T_u} $
    &
    $ 2 \cdot T_u $
    &
    $ 0.45 \cdot T_u $
    &
    $ \frac{1.2}{T_s} \cdot \frac{T_g}{T_u}$
    \\

    \addlinespace[1em]

    Ziegler/Nichols \cite{ref:op_ros_zieg}
    &
    $ 3.33 \cdot T_u $
    &
    $ \frac{0.9}{K_s} \cdot \frac{T_g}{T_u} $
    &
    $ 2 \cdot T_u $
    &
    $ 0.5 \cdot T_u $
    &
    $ \frac{1.2}{K_s} \cdot \frac{T_g}{t_u} $
    \\

    \bottomrule
\caption{Faustformeln zur Reglerdimensionierung}
\label{tab:faustformeln}
\end{longtable}


\clearpage
\subsection{Regler-Dimensionierung anhand der Phasengangmethode}

Als     Hauptberechnung    wird     die    sogenannte     ``Phasengang-Methode
zur      Reglerdimensionierung''      von     Jakob      Zellweger      (FHNW)
\cite{regelungstechnik:zellweger_short}   zu   Hilfe   genommen. Diese   wurde
urspr\"unglich  als  vereinfachte  grafische  Methode  zur  Approximation  der
-20dB/Dek Methode erarbeitet und soll im Rahmen dieses Projektes automatisiert
werden. Schlussendlich   soll   sie   zur   numerischen   Berechnung   mittels
N\"aherungen in das Tool implementiert sein.

Um  die   Berechnung  mit   der  Phasengang-Methode  zu   erm\"oglichen,  muss
zuerst   die  vermessene   Regelstrecke  in   eine  Funktion   im  Bildbereich
gewandelt   werden. Dazu   wird   die   bereits   erw\"ahnte   Matlab-Funktion
\code{p2\_sani.m}  verwendet. Die  Berechnung  anhand  der  Phasengang-Methode
wird  nachfolgend  als  ``Rezept'' aufgef\"uhrt. Genauere  Informationen  sind
dem   Skript   \cite{regelungstechnik:zellweger_short}  zu   entnehmen.    Das
\"Uberschwingverhalten  des Regelkreises  soll  f\"ur dieses  Projekt in  drei
Stufen berechnet werden. Verwendet wird dazu folgende Abstufung:


\begin{itemize}
    \item
        wenig \"Uberschwingen (ca. 0\%)
    \item
        mittleres \"Uberschwingen (ca. 16\%)
    \item
        starkes \"Uberschwingen (ca. 23\%)
\end{itemize}

\subsubsection{Rezept}
Als Erstes sollten folgende Begriffe definiert werden:
\begin{longtable}{lp{60mm}}
    \toprule
    \endhead
    \endfoot
    \endlastfoot

    % CONTENT HERE ---------------------------------------------------------- %

    $H_s(j\omega)                                 $ &  \"Ubertragungsfunktion der Regelstrecke \\
    $A_s(j\omega)=|H_s(j\omega)|                  $ &  Amplitudengang der Regelstrecke \\
    $\varphi_s(j\omega)=arg(H_s(j\omega))         $ &  Phasengang der Regelstrecke \\
    $H_r(j\omega)                                 $ &  \"Ubertragungsfunktion des Reglers \\
    $A_r(j\omega)=|H_r(j\omega)|                  $ &  Amplitudengang des Reglers \\
    $\varphi_r(j\omega)=arg(H_r(j\omega))         $ &  Phasengang des Reglers \\
    $H_o(j\omega)=H_s \cdot H_r(j\omega)          $ &  \"Ubertragungsfunktion des nicht geschlossenen Regelkreises \\
    $A_o(j\omega)=|H_o(j\omega)|                  $ &  Amplitudengang des nicht geschlossenen Regelkreises \\
    $\varphi_o(j\omega)=arg(H_o(j\omega))=\varphi_s(j\omega)+\varphi_r(j\omega) $ &   Phasengang des nicht geschlossenen Regelkreises \\
    $H_{rpid}= K_{rk}\Big[ \frac{(1+sT_{nk})(1+sT_{vk})}{sT_{nk}}\Big]              $ & \"Ubertragungsfunktion des PID-Reglers \\
    $H_{rpi} = K_{rk}\Big[ 1 + \frac{1}{sT_{nk}} \Big] $ & \"Ubertragungsfunktion des PI-Reglers \\


    \bottomrule
%\caption{Formeln f\"ur Faustregeln}
\end{longtable}

\subsubsection{Rezept PID-Regler}

\begin{enumerate}
    \item
        Im  Phasengang muss  die  Frequenz  $\omega_{pid}$ gem\"ass  Gleichung
        \ref{eq:phi_s} bestimmt werden \footnotemark[1].
        \begin{equation} \label{eq:phi_s}
            \varphi_s(\omega_{pid}) = -135 \degree.
        \end{equation}
    \item
        Mittels  Ableitung  wird  die  Steigung des  Phasengangs  im  Punkt  $
        \omega_{pid} $ bestimmt.
    \item
        $\beta$ ist so w\"ahlen, dass Gleichung \ref{eq:dphi_o} erf\"ullt ist

        \begin{equation} \label{eq:dphi_o}
            \frac{d\varphi_o}{d\omega_{pid}}= -\frac{1}{2}.
        \end{equation}

        wobei:
        \vspace*{1em}

        \begin{center}
            $\frac{\omega_{pid}}{\beta}=\frac{1}{T_{vk}}$,
            $\omega_{pid} \cdot \beta=\frac{1}{T_{nk}}$,
            $T_p = 0$,
            $K_{rk} = 1$
        \end{center}

        \vspace*{1em}
        {\em{Mann Beachte: Falls $\beta$ komplex werden sollte, muss $\beta=1$ gesetzt werden.}}
        \vspace*{1em}
    \item
        Nun   werden    $T_{vk}$,   $T_{nk}$   sowie   $K_{rk}=1$    in   die
        \"Ubertragungsfunktion  $H_r$  eingesetzt. Daraus folgen  $\varphi_o$
        sowie $A_o$.   Je nach  gew\"unschtem \"Uberschwingverhalten  wird der
        entsprechende Wert  f\"ur $\varphi_s$ aus der  Tabelle \ref{tab:phi_s}
        herausgelesen.

        Durch Suchen des Punktes  $\omega_d$ gem\"ass Gleichung \ref{eq:phi_o}
        wird berechnet, an welchem Punkt  f\"ur $A_o$ eine Verst\"arkung von 1
        herrschen muss.

        \begin{equation} \label{eq:phi_o}
            \varphi_o(\omega_d)=\varphi_s.
        \end{equation}

        \begin{longtable}{llll}
            \toprule
            \endhead
            \endfoot
            \endlastfoot

            % CONTENT HERE ---------------------------------------------------------- %

            \"Uberschwingen & 0\%              & 16.3\%           & 23.3\% \\
            $\varphi_s$        & $-103.7 \degree$ & $-128.5 \degree$ & $-135 \degree$ \\

            \bottomrule
            \caption{Werte f\"ur $\varphi_s$}
            \label{tab:phi_s}
        \end{longtable}

    \item
        Durch geeignete Wahl von  $K_{rk}$ mithilfe von Gleichung \ref{eq:A_o}
        eine Verst\"arkung von 1 erzwingen:

        \begin{equation} \label{eq:A_o}
            A_o(\omega_d) \cdot K_{rk} = 1
        \end{equation}

    \item
        Alle  Freiheitsgrade des  PID-Reglers  sind hiermit  bestimmt und  der
        Regler nach der Phasengang-Methode vollst\"andig dimensioniert.
\end{enumerate}

\footnotetext[1]{Der Winkel stellt keinen endg\"ultigen Wert dar. Dieser wurde von Jakob Zellweger fixiert, um eine grafische Evaluation \"uberhaupt zu erm\"oglichen. Durch \"andern dieses Wertes kann je nach Regelstrecke das
Regelverhalten weiter optimiert werden.}


\subsubsection{Rezept PI-Regler}

\begin{enumerate}
    \item
        Im  Phasengang  muss  die Frequenz  $\omega_{pi}$  gem\"ass  Gleichung
        \ref{eq:phi_s_pi} bestimmt werden \footnotemark[1].

        \begin{equation} \label{eq:phi_s_pi}
            \varphi_s(\omega_{pi})=-90 \degree.
        \end{equation}

    \item
        $T_{nk}$ kann dadurch direkt bestimmt werden.
        \begin{equation} \label{eq:Tnk_pi}
            T_{nk}=\frac{1}{\omega_{pi}}.
        \end{equation}

    \item
        Anschliessend    werden    $T_{nk}$     und    $K_{rk}=1$    in    die
        \"Ubertragungsfunktion $H_r$  eingesetzt, was $\varphi_o$  sowie $A_o$
        liefert. Jetzt  muss  je  nach gew\"ahltem  \"Uberschwingverhalten  der
        entsprechende Wert  f\"ur $\varphi_s$ aus der  Tabelle \ref{tab:phi_s}
        herausgelesen werden.

        Durch Suchen des Punktes  $\omega_d$ gem\"ass Gleichung \ref{eq:phi_o}
        wird festgelegt an  welchem Punkt f\"ur $A_o$ eine  Verst\"arkung von 1
        definiert werden muss.

    \item
        $K_{rk}$ wird  so gew\"ahlt, dass mithilfe  von Gleichung \ref{eq:A_o}
        eine Verst\"arkung von 1 erzwungen wird.

    \item
        Somit sind alle Freiheitsgrade des  PI-Reglers bestimmt und der Regler
        nach der Phasengang-Methode komplett dimensioniert.
\end{enumerate}

\newpage
\subsection{Umrechnung der Regler-Darstellung zwischen Bode- und Reglerkonform}

Die    Formeln   in    Tabelle   \ref{tab:bode_regler_konform}    dienen   zur
Umrechnung  zwischen der  bodekonformen  Darstellung  und der  reglerkonformen
Darstellung. N\"ahere  Informationen  zu den  verschiedenen  Darstellungsarten
k\"onnen der Quelle \cite{regelungstechnik:zellweger} entnommen werden.

\begin{longtable}{l|ll}
    \toprule

    %\multicolumn{3}{l}{\large{\textsc{auftragsanalyse und hintergrundinformationen}}} \\
    %\multicolumn{2}{l}{pi}

    &
    bodekonform $\rightarrow$ reglerkonform
    &
    reglerkonform $\rightarrow$ bodekonform
    \\

    \midrule

    \endhead
    \endfoot
    \endlastfoot

    % content here ---------------------------------------------------------- %

    PI
    &
    $T_n = T_{nk} $ %reglerkonform
    &
    $K_{rk} = K_r $ %bodekonform
    \\

    \midrule

    PID
    &
    $T_n = T_{nk}+T_{vk}-T_p$
    &
    $T_{nk}=0.5 \cdot (T_n+T_p) \cdot (1+\epsilon)$
    \\

    &
    $T_v=\frac{T_{nk} \cdot T_{vk}}{T_{nk}+T_{vk}-T_p}-T_p$
    &
    $T_{vk}=0.5 \cdot (T_n+t_p) \cdot (1-\epsilon)$
    \\

    &
    %$k_r=k_{rk} \cdot \frac{1+t_{vk}}{t_{nk}}$
    $K_r=K_{rk} \cdot (1 + \frac{T_{vk}-T_p}{T_{nk}})$
    &
    $K_{rk} = 0.5 \cdot K_r \cdot (1 + \frac{T_p}{T_{nk}}) \cdot (1+\epsilon )$
    \\
    \\

    &
    \multicolumn{2}{l}{wobei $\epsilon^2 = 1-(4 \cdot T_n \cdot \frac{T_v-T_p}{(T_n+T_p)^2})$}
    \\
    \bottomrule
    \caption{Formeln zur Umrechung zwischen bode- zu reglerkonformer Darstellung \cite{regelungstechnik:zellweger}, \cite{regelungstechnik:schumleon}}
    \label{tab:bode_regler_konform}
\end{longtable}


F\"ur die Berechnungen in diesem Projekt wird, wenn nicht anders angegeben, mit $t_p=\frac{1}{10} \cdot t_v$ gerechnet.


\subsection{Berechnung der Schrittantwort des geschlossenen Regelkreises}

Zum   aktuellen  Zeitpunkt   fehlen  die   Kenntnisse,  um   genauere  Angaben
zur  Berechnung  des  geschlossenen  Regelkreises  machen  zu  k\"onnen. Diese
Teilproblematik wird w\"ahrend des Projekts weiter vertieft und ein geeigneter
L\"osungsweg definiert.
