% **************************************************************************** %
\chapter*{Abstract}
\label{chap:abstract}
% **************************************************************************** %

\emph{Note: I have  added headings  for the five  sections of  the ``abstract
hand'' to  highlight my thinking process;  obviously those will be  removed in
the final version.}

\vspace{2em}
\textbf{Research Question: What does the client expect?}\\
This  project's aim  was  to  develop a  system  for  real-time monitoring  of
photovoltaic  facilities on  the level  of single  panels. The system  must be
cost-effective and should scale from small single-household solutions to large
industrial-scale solar farms. Panels which are  not operating at full capacity
for whatever reason  (dirt, shade, defects) must be detected  and announced to
the user so that appropriate measures (cleaning, replacement) can be taken.

\textbf{Relevance}\\
Monitoring individual photovoltaic modules  will allow the facility's operator
to run  their solar farm at  optimal conditions, thus reducing  losses in both
power output and money.

\textbf{Background}\\
Current solutions for per-module monitoring  of solar facilities are expensive
and are  therefore often  foregone in  order to save  costs when  building new
facilities. This approach is short-sighted and more expensive in the long term
because not being able  to properly monitor a solar facility  means that it is
often not operating at optimal  conditions. As non-renewable sources of energy
such  as fossil  fuels  and nuclear  power are  losing  ground to  alternative
sources of  energy such  as wind and  solar power, the  overall losses  in the
energy industry of power and money  incurred due to insufficient monitoring of
solar facilities will become unsustainable.

\textbf{Method}\\
Our  system consists  of  two primary  components: A  controller is  installed
centrally  near  the  inverter. Additionally,  a sensor  is  mounted  on  each
photovoltaic  panel.  Communication  between  the sensors  and  the master  is
routed  through the  direct  current power  transmission  line; no  additional
wiring is needed. A  coil is used to  couple the signal to  the power line. In
case  of  an  error (e.g.  a  dirty  panel),  a  text  message is  sent  to  a
user-configurable phone  number. Additionally, local alarms such  as sirens or
warning lights  can be connected  to our system  and are triggered  whenever a
text message is sent.

\textbf{Results}\\
Simulations for  various coupling methods  for a string  of 20 PV  panels have
been performed. Inductive  coupling at  non-resonance conditions results  in a
signal  level  of roughly  \SI{6}{\milli\volt}  peak-to-peak  at the  receiver
without  amplification. Operating  the  circuit  at resonance  yields  a  much
improved peak-to-peak  voltage of \SI{250}{\milli\volt} at  the receiver (also
without amplification), which is sufficient for our purposes.

A  frequency sweep  for  the coupling  coil has  been  measured and  inductive
behavior up to  \SI{20}{\mega\hertz} verified, thus ensuring that  the coil we
have  picked is  suitable for  our uses  and  allows the  of a  vast range  of
frequencies as  needed. The system can  therefore adapt to conditions  in each
specific setup.

\vspace{1em}
\emph{Question: How negative  should we  go in  the paragraph  below? We don't
exactly  want to  sound like  depressed failures,  but then  again, facts  are
facts, and things  just aren't working properly. Should we  also add something
about how  we would proceed if  we had the time  or is this not  the place for
that?}

\vspace{1em}
The system  in its  entirety is  at this  point not  operational. The sensor's
components  do not  yet  work perfectly  in unison;  error  analysis is  still
ongoing. The printed circuit board for the  master device is not available due
to an issue with the  manufacturer when ordering. However, development for the
master's software is mostly completed.  A functioning graphical user interface
has been implemented and a database to which it connects is functional.

\vspace{2em}
\textbf{Key  words:}  photovoltaic  technology,  photovoltaic  module,  remote
monitoring, solar technology, PV cell, power efficiency, alternative energy,
powerline, communication, inductive coupling, capacitive coupling
