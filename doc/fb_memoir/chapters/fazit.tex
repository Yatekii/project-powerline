% **************************************************************************** %
\chapter{Fazit}
\label{chap:fazit}
% **************************************************************************** %

Was  l\"auft? Was  l\"auft  nicht? Weshalb? Weiteres Vorgehen  bei  gen\"ugend
Zeit? Schwierigkeiten w\"ahren des Projekts?

\lipsum[1]
\begin{equation}
    \label{eq:errorTurbulent:simple}
    s_{\overline{\dot{V}}} = \sqrt{ A^2 + B^2 + C^2 } = \SI{0.304}{\liter\per\minute}
\end{equation}

\lipsum[2]

\begin{align}
    \label{eq:errorTurbulent:complicated}
    A
    &=
    \frac{%
        2
        \cdot
        \pi
        \cdot
        \overline{R}^2
        \cdot
        \overline{k}^2
        \cdot
        s_{\overline{v}}
    }{%
        (
            \overline{k}
            +
            1
        )
        (
            2
            \cdot
            \overline{k}
            +
            1
        )
    }
    \\
    B
    &=
    \frac{%
        4
        \cdot
        \pi
        \cdot
        \overline{R}
        \cdot
        \overline{k}^2
        \cdot
        s_{\overline{R}}
        \cdot
        \overline{v}_{\mathrm{max}}
    }{%
        (
            \overline{k}
            +
            1
        )
        (
            2
            \cdot
            \overline{k}
            +
            1
        )
    }
    \\
    C
    &=
    s_{\overline{k}}
    \cdot
    \left(
        -
        \frac{%
            4
            \cdot
            \pi
            \overline{R}^2
            \cdot
            \overline{k}^2
            \cdot
            \overline{v}_{\mathrm{max}}
        }{%
            (
                \overline{k}
                +
                1
            )
            \cdot
            (
                2
                \cdot
                \overline{k}
                +
                1
            )^2
        }
        -
        \frac{%
            2
            \cdot
            \pi
            \cdot
            \overline{R}^2
            \cdot
            \overline{k}^2
            \cdot
            \overline{v}_{\mathrm{max}}
        }{%
            (
                \overline{k}
                +
                1
            )^2
            \cdot
            (
                2
                \cdot
                \overline{k}
                +
                1
            )
        }
        +
        \frac{%
            4
            \cdot
            \pi
            \cdot
            \overline{R}^2
            \cdot
            \overline{k}
            \cdot
            \overline{v}_{\mathrm{max}}
        }{%
            (
                \overline{k}
                +
                1
            )
            \cdot
            (
                2
                \cdot
                \overline{k}
                +
                1
            )
        }
    \right)
\end{align}

\begin{conditions}
    \overline{R}                 & \SI{20}{\milli\meter}, Durchmesser Messleitung                   \\
    \overline{k}                 & \num{7.8876}, aus Regression                                     \\
    \overline{v}_{\mathrm{max}}  & \SI{10.793}{\centi\meter\per\second}, aus Regression             \\
    s_{\overline{v}}             & \SI{0.15720}{\centi\meter\per\second}, aus Regression            \\
    s_{\overline{R}}             & \SI{0.25}{\milli\meter}, Unsicherheit Durchmesser Messleitung    \\
    s_{\overline{k}}             & \num{1.572}, aus Regression                                      \\
\end{conditions}

\lipsum[3]
