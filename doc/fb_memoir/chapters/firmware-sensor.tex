\subsection{SAMD09 \"Ubersicht}

Als CPU auf dem Sensor agiert wie im Abschnitt~\ref{sec:hw:sensorplatine} erw\"ahnt ein Atmel Smart ARM Cortex M0+ gew\"ahlt.
Die Firmware ist in C geschrieben.

\subsection{Benutze Bibliotheken}

Es wird komplett freie Software verwendet, weswegen zum bauen der Binaries GNU Makefiles benutzt werden.

Die Firmware ist auf dem Atmel Software Framework (ASF) aufgebaut. Das ASF f\"uhrt einen Hardware Abstraction Layer (HAL) ein, welcher die Hardware-Bl\"ocke der einzelnen Atmel CPUs in einfache Interfaces in Form von C-Funtionen abstrahiert.
Um die g\"angigen ARM Schnittstellen zu nutzen, wird CMSIS im ASF verwendet.
Das ASF darf f\"ur Atmel Chips ohne weiteres verwendet werden solange die Copyright Bemerkungen nicht entfernt werden.

\subsection{Die Firmware}

Die Firmware besteht im Kern aus einer simplen main() Funktion welche in einer Endlosschleife l\"auft.

\subsubsection{UART}
\label{subs:UART}

Die Endlosschleife \"uberpr\"uft zuerst ob Daten \"uber die UART empfangen wurden, sprich Anweisungen vom Master oder Antworten von anderen Sensoren. Ist das der Fall, so wird darauf reagiert. Dies heisst dass die UART die gemittelten Daten der letzten x Samples an den Master adressiert und verschickt.
Hierzu wird ein Datenpakt wie es in Abbildung \ref{fig:sensor:firmware:datenpaket} zu sehen ist verschickt:

\begin{figure*}[ht!]
  \centering
  \begin{bytefield}{32}
    \bitheader{0,7,8,15,16,23,24,31} \\
    \begin{rightwordgroup}{Adresse}
      \bitbox{32}{Adressbyte 0} \\
      \bitbox{32}{Adressbyte 1} \\
      \bitbox{32}{Adressbyte 2} \\
      \bitbox{32}{Adressbyte 3} \\
    \end{rightwordgroup} \\
    \bitbox{16}{Spannung}\bitbox{16}{Nicht verwendet}\\
    \bitbox{32}{CRC}\\
  \end{bytefield}
  \caption{\label{fig:sensor:firmware:datenpaket}Sample packet}
\end{figure*}

Damit der Empf\"anger eines Paketes erkennen kann dass es keine Technische St\"orung beim \"ubertragen gab und er falsche Daten liest, wird eine Pr\"ufsumme (CRC) der zu verschickenden Daten mitgeschickt.
Um die CRC zu berechnen wird zuerst das Datenpaket zusammengestellt. Davon wird dann mithilfe des ASF eine Pr\"ufsumme erstellt und an das bestehende Datenpaket angeh\"angt. Dann wird das Datenpaket per UART verschickt.

\subsubsection{Sensor}
\label{subs:Sensor}

Der Spannungssensor selbst wird mittels Analog Digital Konverters (ADC) verwirklicht.
Daf\"ur wird mithilfe des ASF ein 12 Bit ADC ausgelesen und von diesen Werten ein Moving Average erstellt.
Deshalb kommt an zweiter Stelle in der Endlosschleife ein Lesevorgang auf dem ADC.
Dann wird mithilfe des sogenannten Cascaded Integrator–Comb Filters
%(TODO: ref to https://en.wikipedia.org/wiki/Cascaded_integrator%E2%80%93comb_filter)
der aktuelle Moving Average berechnet. Die Formel~\ref{eq:hogenauer} zeigt gut wie dies berechnet wird.

\begin{equation}\label{eq:hogenauer}
  \begin{align}
    y[n] &= \sum_{k=0}^{RM-1} x[n-k] \\
         &= y[n-1] + x[n] - x[n-RM].
  \end{align}
\end{equation}

Dieser Moving Average wird nun per UART verschickt.

\subsubsection{Statusanzeige}
\label{subs:Statusanzeige}

Damit man gut sieht ob der Sensor noch korrekt leuchtet kommt an letzter Stelle der Endlosschleife noch ein simples Toggeln der LED, so dass man sieht dass etwas nicht stimmt falls es einmal nicht mehr toggeln sollte.

\subsection{Open On Chip Debugger}

Zum Programmieren der CPUs haben wir OpenOCD gew\"ahlt. Auch hier ist wieder anzumerken, dass es ein St\"uck freie Software ist.
OpenOCD extrem leicht zu erweitern und unterst\"utzt eine grosse Breite an Programmierschnittstellen wie den STLinkv2 oder den Segger J-Link. Es werden Protokolle wie JTag und SWD ohne weiteres unterst\"utzt.
Ebenfalls gibt es einen Reichtum an Chips die unterst\"utzt sind.

Da der SAMD09 erst im ersten Quartal 2016 auf den Markt kam, waren sehr wenige Projekte vorhanden, welche ihn bereits benutzt haben. Dies hatte zur Ursache, dass OpenOCD den Chip nicht unterst\"utzte. Also haben wir einen Treiber daf\"ur geschrieben beziehungsweise den bestehenden f\"ur Atmel Chips erweitert.

Um die Kosten Tief zu halten haben wir einen StLinkv2 als Programmierschnittstelle verwendet. Auch daf\"ur brauchte es einen Patch des Treibers, da der STLinkv2 nur 32 Bit Schreibbefehle unterstutzt, der SAMD09 aber 16 Bit Schreibbefehle erwartet. Wir haben einen unsch\"onen Patch der nun 32 Bit statt 16 Bit Schreibbefehle schickt geschrieben um dieses Problem zu l\"osen.
