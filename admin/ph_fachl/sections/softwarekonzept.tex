Das Softwarekonzept  h\"alt die grundlegende Struktur  des Programms fest. Das
Design der  grafischen Benutzeroberfl\"ache  (GUI) wird in  einem Erstentwurf
dargestellt  und wird  zu   einem  sp\"ateren   Zeitpunkt  mit   dem  Kunden
weiterentwickelt.


\subsection{GUI und Beschreibung des Programmablaufs}

%\begin{center}
%    \begin{figure}[h! width=\pagewidth]
%		%\hspace{-1in}
%		\includegraphics[width=0.95\textwidth]{images/GUI_Entwurf_2_150304.pdf}
%		\caption{Entwurf des GUI Hauptbildschirms}
%    \end{figure}
%\end{center}


\subsubsection*{Start}
Die  Software wird  beim  Start im  Default-Modus geladen. Alle  Eingabefelder
sind  mit Nullwerten  versehen. Die Oberfl\"ache  (Abblidung 1)  gliedert sich
in  eine  linke,  numerische  und  eine  rechte,  grafische  Bedienseite. Die
linke  Seite  zeigt   Eingabe  der  Basis-Parameter  sowie   zur  Ausgabe  der
Reglerwerte. Die   Rechte   beinhaltet   die   Plots   der   Schrittantworten,
der    PTn-Strecke\footnotemark[1],   des    Amplituden-\footnotemark[1]   und
Phasengangs\footnotemark[1]. Sofern   die   Berechnungszeit  gen\"ugend   kurz
gehalten werden kann,  wird zus\"atzlich noch die  M\"oglichkeit zur manuellen
Anpassung der Reglerwerte\footnotemark[2]  hinzugef\"ugt (grafisches Feedback
muss sofort erfolgen).


\subsubsection*{Dateneingabe}
Im  Panel ``Schrittantwort  vermessen''  werden  die gemessenen,  numerischen
Daten eingegeben. Alle Felder m\"ussen zur Berechnung ausgef\"ullt sein.

Beim  Klick  in ein  Textfeld  wird  der Text  markiert. Ung\"ultige  Eingaben
werden nicht  akzeptiert.  \"Uber den  Button „Berechnen“ wird  die Berechnung
gestartet sowie die  Plots gezeichnet.  \"Uber die  Schaltfl\"achen PI, PID-T1
im  entsprechenden Panel   kann  zwischen den  zwei Regler-Typen  umgeschaltet
werden. Ein  Umschalten  l\"ost  automatisch   eine  Neuberechnung  sowie  ein
Neuzeichnen aus. Als  zus\"atzliches Wunschziel wird das  Ein- und Ausblenden
des rechten (Plot-)Bereichs\footnotemark[1] definiert.


\subsubsection*{Ausgabe}
Das Panel "Reglerwerte" gibt die  Werte f\"ur die Phasengang-Methode sowie f\"ur
eine Auswahl  an Faustformeln aus. Bei  der Phasengang-Methode sind  die Werte
f\"ur wenig, mittel oder  starkes \"Uberschwingen angegeben. Ist der Reglertyp
``PID-T1''  gew\"ahlt, so  ist  der Wert  f\"ur  $T_P$ verstellbar  (Standard:
$\frac{1}{10} \cdot T_v$ ).

Im    oberen    rechten    Bereich     kann    zwischen    den    Plots    der
PTn-Strecke,    des    Amplituden-    und Phasengangs (alles Wunschziele
sowie   der    M\"oglichkeit   die   Reglerwerte
anzupassen\footnotemark[2] mittels Tabs umgeschaltet werden.

Im Panel  ``Schrittantworten'' erfolgt die grafische  Ausgabe der berechneten
Werte. Die verschiedenen Methoden  (Phasengang-Methode, Faustformeln) k\"onnen
einzeln \"uber Check-Boxen dazu- oder weggeschaltet werden. Das Zoomen mittels
Maus im Plot ist ein Wunschziel.


\footnotetext[1]{Wunschziel}
\footnotetext[2]{Wird nur bei gen\"ugend schneller Programmgeschwindigkeit umgesetzt.}


{\em{\"Anderungen am GUI  sind noch m\"oglich und werden  in Zusammenarbeit mit dem Kunden erarbeitet.}}


\subsection{Softwarestruktur}
Die  Software  baut  auf  dem  Prinzip  des  MVC-Pattern  (Model-View-Control)
auf. Dies  gew\"ahrleistet gr\"osstm\"ogliche  Flexibilit\"at  in der  Wartung
sowie der Wiederverwendbarkeit der Packages.


\subsubsection*{Klassendiagramm}
Das  Klassendiagramm (siehe  Abbildung \ref{fig:klassendiagramm})  zeigt einen
Grobentwurf  des Aufbaus  sowie  die Abh\"angigkeiten  zwischen den  einzelnen
Klassen.

%\begin{center}
%\begin{figure}[h!, width=\pagewidth]
%        \includegraphics[angle=90,width=\textwidth]{images/klassendiagramm.pdf}
%        \caption{Klassendiagramm mit MVC-Pattern sowie m\"oglichen Klassen}
%        \label{fig:klassendiagramm}
%\end{figure}
%\end{center}


\subsubsection*{Model}
Das  Model \"ubernimmt  die gesamten  Berechnungen der  Applikation. So werden
in   den  dazugeh\"origen   Klassen  die   Resultate  der   Reglerwerte  sowie
die  Daten  f\"ur  die  Plots  berechnet  und  zwischengespeichert. Model  ist
observable: \"Anderungen der Werte k\"onnen somit  von anderen Klassen aus der
View \"uberwacht werden.


\subsubsection*{View}
Die View baut  aus den Daten des  Models das gesamte GUI  auf. \"Uber die View
findet  die  Daten-Eingabe sowie  -Ausgabe  statt. Der  genaue Aufbau  ist  im
Kapitel  „GUI-Beschreibung“ zu  finden. Die View  implementiert das  Interface
\code{Observer} sowie die Methode  \code{update()}. Die View und die Subpanels
delegieren \code{update()}  an die darunterliegenden  Panels. Folgende Klassen
sind voraussichtlich Bestandteil der View:

\paragraph{\code{View}}: Hauptpanel, das  alle darunterliegenden  Panels aufnimmt, f\"ullt
den gesamten Frame aus.

\begin{itemize}
    \item{\code{ViewLeftPanel}}:
        Stellt Panel des linken Bereichs dar.
        \begin{itemize}
            \item{\code{ViewInputPanel}}:
                Die   Werte   der   vermessenen  Impulsantwort   werden   hier
                eingegeben. Der Button  zum Start der Berechung  befindet sich
                ebenfalls hier.
            \item{\code{ViewRegulatorChooserPanel}}:
                Wahl zwischen PI- und PID-T1-Regler, erfolgt mittels Buttons.
            \item{\code{ViewRegulatorPanel}}:
                Beinhaltet   die  Reglerwerte   der  Faustformeln   sowie  der
                Phasengang-Methode.
            \item{\code{ViewRulesOfThumb}}:
                Gibt die Reglerwerte der Faustformeln an.
            \item{\code{ViewPhaseResponseMethod}}:
                Gibt die Reglerwerte der Phasengang-Methode aus.
        \end{itemize}
    \item{\code{ViewRightPanel}}:
        Stellt Panel des rechten Bereichs dar.
        \begin{itemize}
            \item{\code{ViewStepResponsePanel}}:
                Stellt  die   Schrittantworten  der  Faustformeln   sowie  der
                Phasengang-Methode dar.
        \end{itemize}
\end{itemize}


\subsubsection*{Controller}
Der  Controller ist  die Schnittstelle  zwischen der  View und  dem Model. Die
Eingabe-Werte der View werden gepr\"uft,  dann an das Model weitergeleitet
oder als  fehlerhaft zur\"uckgewiesen. Er  steuert somit das Model in
 Abh\"angigkeit der Eingaben \"uber das GUI.
