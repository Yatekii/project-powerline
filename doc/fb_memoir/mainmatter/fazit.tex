% **************************************************************************** %
\chapter{Fazit}
\label{chap:fazit}
% **************************************************************************** %

Nach  Abschluss  des   Projektes  steht  dem  Auftraggeber   ein  Produkt  zur
Verf\"ugung, dass als  Gesamtsystem die Spannungen an  PV-Modulen erfassen und
auswerten  kann. Die einzelnen  Teilsysteme sind  mehrheitlich als  Prototypen
vorhanden, wurden im Labor getestet und funktionieren einwandfrei.

Der Sensor kann Spannungen messen  und Werte ausgeben. Mit einer Abmessung von
knapp 5cm*5cm bieten  wir dem Auftraggeber eine  Sensorplatine, die garantiert
in jedem  Anschlussgeh\"ause Platz findet. Die einwandfreie  Firmware zusammen
mit den gew\"ahlten Hardwarekomponenten erm\"oglichen eine saubere Messung der
Modulspannung. Die Hauptproblematik  im Projekt  bestand darin,  eine optimale
L\"osung  f\"ur  die  Kommunikation  zwischen  Sensor  und  Master-Ger\"at  zu
finden. Mit dem  FSK (Frequency-shift  keying) wurde eine  Variante gew\"ahlt,
die es uns erm\"oglicht, die  gemessenen Daten der PV-Module dem Masterger\"at
zu   \"ubermitteln. Das  Grundprinzip   dieser  \"Ubertragungsmethode   konnte
validiert werden.

Anhand  eines Testaufbaus  konnten  Signale vom  Master-Ger\"at empfangen  und
an  den  Raspberry  Pi   weitergeleitet  werden. Mithilfe  der  Mastersoftware
ist  der Raspberry  Pi  im  Stande, die  empfangenen  Messwerte zu  berechnen,
auszuwerten,  mit   anderen  Messdaten   zu  vergleichen   und  n\"otigenfalls
eine  Fehlermeldung  auf  dem Display  auszugeben. Das  Master-Ger\"at  bietet
einerseits  eine benutzerfreundliche  Oberfl\"ache, andererseits  l\"asst sich
das schlichte  Geh\"ause bequem in einen  bestehenden Generatoranschlusskasten
installieren,  auch   wenn  nur  wenig  Platz   vorhanden  ist. Die  Bedienung
und   Inbetriebnahme   \"uber   das   2.5``   Touch-Display   ist   \"ausserst
komfortabel und selbsterkl\"arend. Das GSM-Modul  der Masterplatine, f\"ur die
Fehlerbenachrichtigung  per  SMS,  konnte als  einzelne  Komponente  ebenfalls
\"uberpr\"uft werden. Das Master-Ger\"at  ist somit im Stande,  im Falle eines
Modulfehlers,  neben der  Bet\"atigung  der beiden  Relaiskontakte, auch  eine
Fehlermeldung per  SMS zu  versenden. Die geplante Zusatzfunktion  zur Messung
der  Strangstr\"ome  am  Master-Ger\"at  konnte wegen  Zeitmangel  nicht  mehr
getestet werden.

Als  n\"achster   Schritt  sollte  die  Daten\"ubermittlung   vom  Sensor  zum
Master-Ger\"at mittels  Powerline getestet werden, wie  auch das Zusammenspiel
von  Tochterboard  und Raspberry  Pi. Folglich  sollten  alle Teilsysteme  des
gesamten Systems als Ganzes \"uberpr\"uft werden.
