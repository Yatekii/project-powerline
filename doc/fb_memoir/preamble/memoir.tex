% -------------------------------------------------------- %
% This file contains various options for the memoir class  %
% itself.                                                  %
% -------------------------------------------------------- %


% -------------------------------------------------------- %
% Rename Appendices to whatever we want.                   %
% See p78 of memman.pdf                                    %
% -------------------------------------------------------- %
\def\appendixpagename{\sffamily\HUGE Anh\"ange}
\def\appendixtocname{Anh\"ange}

% -------------------------------------------------------- %
% Choose a page layout                                     %
% -------------------------------------------------------- %
%                                                          %
% The  layout  is based  on isopage,  with  the  following %
% changes:                                                 %
% - The spine margin is 30 mm. This is so that binding the %
% document with  a ringbinder does  not come too  close to %
% the text.                                                %
% - The edge  margin is  1.5 times the spine  margin, i.e. %
% 45 mm.                                                   %
% - The top margin is 1/9  of the paper height, same as in %
% \isopage                                                 %
% - The bottom margin is decreased from isopage's value to %
% 1.5 times the top margin, i.e. about 50 mm.              %
% -------------------------------------------------------- %

%    +----------^--------------------+
%    |          | 33mm               |
%    |     +----v------------+       |
%    |30mm |                 | 45mm  |
%    |<--->|                 |<----->|
%    |     |                 |       |
%    |     |                 |       |
%    |     |                 |       |
%    |     |                 |       |
%    |     |                 |       |
%    |     |                 |       |
%    |     |                 |       |
%    |     |                 |       |
%    |     |                 |       |
%    |     |                 |       |
%    |     |                 |       |
%    |     |                 |       |
%    |     +----^------------+       |
%    |          | 49.5mm             |
%    |          |                    |
%    +----------v--------------------+
\isopage%
\setlrmarginsandblock{0.142857111\paperwidth}{0.190476190\paperwidth}{*}
\setulmarginsandblock{0.111111111\paperheight}{*}{1.5}%
\checkandfixthelayout%


% -------------------------------------------------------- %
% Choose a chapter style                                   %
% We're  going  with  veelo, but  removing  the  "Chapter" %
% designation in front of the chapter number.              %
% -------------------------------------------------------- %
% -------------------------------------------------------- %
% This  chapterstyle is  baesd on  veelo, but  removes the %
% "Chapter" designation in front of the chapter number.    %
% -------------------------------------------------------- %
%
% TODO: Check if this is allowed by the  LPPL, under which
% 'memoir.cls' is distributed, which contains the original
% code for the veelo chapter style.
%
% http://www.latex-project.org/lppl.txt
%
% If this is not permitted, implement alternative via this:
% http://tex.stackexchange.com/questions/51527/chapter-heading-with-the-word-chapter-replaced-by-chapter-name

\makeatletter
\makechapterstyle{fhnw}{%
   \setlength{\afterchapskip}{40pt}
  \renewcommand*{\chapterheadstart}{\vspace*{40pt}}
  \renewcommand*{\afterchapternum}{\par\nobreak\vskip 25pt}
   \renewcommand*{\chapnamefont}{\normalfont\LARGE\flushright}
   \renewcommand*{\chapnumfont}{\normalfont\HUGE}
   \renewcommand*{\chaptitlefont}{\normalfont\HUGE\bfseries\flushright}
   \renewcommand*{\printchaptername}{%
       \chapnamefont\MakeTextUppercase{}}
   \renewcommand*{\chapternamenum}{}
  \setlength{\beforechapskip}{18mm}%  \numberheight
  \setlength{\midchapskip}{\paperwidth}% \barlength
  \addtolength{\midchapskip}{-\textwidth}
  \addtolength{\midchapskip}{-\spinemargin}
   \renewcommand*{\printchapternum}{%
     \makebox[0pt][l]{%
       \hspace{.8em}%
       \resizebox{!}{\beforechapskip}{\chapnumfont \thechapter}%
       \hspace{.8em}%
       \rule{\midchapskip}{\beforechapskip}%
     }%
   }%
   \makeoddfoot{plain}{}{}{\thepage}}
\makeatother

\chapterstyle{fhnw} % requires graphicx package

\setsecheadstyle{\Large\bfseries\sffamily}
\setsubsecheadstyle{\large\bfseries\sffamily}
\setsubsubsecheadstyle{\bfseries\sffamily}

% -------------------------------------------------------- %
% Pagestyle (Headers, Footers)                             %
% -------------------------------------------------------- %
\pagestyle{headings}

% -------------------------------------------------------- %
% Cross Referencing Stuff                                  %
% -------------------------------------------------------- %

% Apparently these are not automatically translated into
% German by babel:
\def\figurerefname{Abbildung}
\def\tablerefname{Tabelle}
\def\pagerefname{Seite}

% -------------------------------------------------------- %
% Caption Configurations                                   %
% -------------------------------------------------------- %
\captionnamefont{\bfseries\small}
\captiontitlefont{\small}
\captiondelim{: }

% Captions for use outside of floats
\newfixedcaption{\figcaption}{figure}
\newfixedcaption{\tabcaption}{table}

% -------------------------------------------------------- %
% Define  what  sort  of  sectional  divisions  should  be %
% numbered.  See also "Document Divisions -- Numbering" in %
% the memoir manual.                                       %
%                                                          %
%            Division            |  Level                  %
%          ----------------------+---------                %
%            \book               |  -2                     %
%            \part               |  -1                     %
%            \chapter            |   0                     %
%            \section            |   1                     %
%            \subsection         |   2                     %
%            \subsubsection      |   3                     %
%            \paragraph          |   4                     %
%            \subparagraph       |   5                     %
%                                                          %
% \setsecnumdepth{<division>} sets  division numberings so %
% that <division>  and above will be  numbered.  When used %
% in  the preamble  (such as  here), \setsecnumdepth  also %
% calls \maxsecnumdepth, which is the numbering level used %
% in the \mainmatter part of the document. \setsecnumdepth %
% can be used anywhere  in the mainmatter to (temporarily) %
% change the numbering level.                              %
%                                                          %
% This template  uses chapters, sections,  subsections and %
% subsubsections  by default,  so  we  will set  numbering %
% level to subsubsection per default. Adjust as needed. We %
% will set this to subsubsection per default.              %
% -------------------------------------------------------- %
\setsecnumdepth{subsubsection}
\maxsecnumdepth{subsubsection}
\settocdepth{subsubsection}
\maxtocdepth{subsubsection}
