% **************************************************************************** %
\chapter{\code{LTspice}-Schaltungen}
\label{app:ltspice}
% **************************************************************************** %


\begin{figure}[h!tb]
    \centering
    \includegraphics[width=\textwidth]{images/ltspice/singlecell.eps}
    \caption{%
        Schaltung        zur         Simulation        einer        Solarzelle
        gem\"ass        Abbildung       \ref{fig:circuit:solarCell}        von
        Seite     \pageref{fig:circuit:solarCell}. Die     Solarmodule     aus
        den     Abbildungen    \ref{fig:ltspice:module:cellBased:36x2}     und
        \ref{fig:ltspice:module:cellBased:72x1}  basieren auf  Zellen gem\"ass
        diesem Schaltschema.%
    }
    \label{fig:ltspice:solarCell}
\end{figure}

\begin{figure}[h!tb]
    \centering
    \includegraphics[width=\textwidth]{images/ltspice/module-72cells.eps}
    \caption{%
        Solarmodul     \"ahnlich     zu    Sunset     Solargenerator     AS150
        \cite{ref:solar:as150}, modelliert durch 2  parallele Str\"ange mit je
        36 Zellen gem\"ass Abbildung \ref{fig:ltspice:solarCell} in Serie.%
    }
    \label{fig:ltspice:module:cellBased:36x2}
\end{figure}

\begin{figure}[h!tb]
    \centering
    \includegraphics[width=\textwidth]{images/ltspice/module-72cells-series.eps}
    \caption{%
        Solarmodul   \"ahnlich   zu   Sunmodule  Pro-Series   XL   SW320   aus
        \cite{ref:solar:sunmodulePro},  modelliert durch  einen Strang  mit 72
        Zellen gem\"ass Abbildung \ref{fig:ltspice:solarCell} in Serie.%
    }
    \label{fig:ltspice:module:cellBased:72x1}
\end{figure}
\todo{Transistoren aus Schaltbildern entfernen, IN/OUT benennen, Abbildung sollte nur Modul sein}
