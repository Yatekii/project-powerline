% ---------------------------------------------------------------------------- %
\section{Software \Master}
\label{sec:software:master}
% ---------------------------------------------------------------------------- %

Es wird  im Folgenden auf  die Software des \Master~  genauer eingegangen. Die
verwendeten  Komponenten  werden beschrieben  und  es  wird auf  die  Lizenzen
ebendieser  Komponenten eingegangen. Anschliessend  werden der  Aufbau unseres
Software-Stacks und die Funktionsprinzipien dokumentiert.


% ---------------------------------------------------------------------------- %
\subsection{Komponenten}
\label{subsec:software:master:components}
% ---------------------------------------------------------------------------- %

Als Betriebssystem kommt Raspbian zum  Einsatz. Raspbian ist eine Variante von
Debian-Linux mit  einigen Erweiterungen,  welche das  System auf  einem \Raspi
lauff\"ahig machen. Als  graphische Oberfl\"ache wird LXDE  benutzt, aufbauend
auf  X11. Die wichtigsten  Komponenten  des gesamten  Software-Stackn sind  in
Abbildung \ref{fig:softwarestack} dargestellt.

Die  Funktionalit\"at  unserer  Software  wird  mit  einigen  Python-Libraries
implementiert. PyQt  wird  benutzt,  um  die  graphische  Benutzeroberfl\"ache
zu   programmieren,  SQLAlchemy   ist  daf\"ur   dient  als   Datenbanktreiber
verantwortlich  und \emph{WiringPi  for  Python}  ist daf\"ur  verantwortlich,
die  Hardware-Schnittstellen  des  \Raspi~   zu  abstrahieren  und  in  Python
bereitzustellen. Tabelle  \ref{tab:pythonLibs} listet  die Libraries  und ihre
Aufgaben in einer \"Ubersicht auf.

\begin{figure}[h!tb]
    \centering
    \begin{bytefield}{32}
        %\begin{rightwordgroup}{Adresse}
        \colorbitbox{solarized-base2}  {solarized-base02}{32}{Unsere Software} \\
        \colorbitbox{solarized-blue}   {solarized-base3}  {8}{PyQt}
        \colorbitbox{solarized-blue}   {solarized-base3}  {8}{SQLAlchemy}
        \colorbitbox{solarized-blue}   {solarized-base3}  {8}{WiringPi}
        \colorbitbox{solarized-magenta}{solarized-base3}  {8}{LXDE} \\
        \colorbitbox{solarized-blue}   {solarized-base3} {24}{Python 3}
        \colorbitbox{solarized-magenta}{solarized-base3}  {8}{X11} \\
        \colorbitbox{solarized-base01} {solarized-base3} {32}{Linux} \\
        %\end{rightwordgroup} \\
  \end{bytefield}
  \caption{Software-Stack f\"ur unser Projekt}
  \label{fig:softwarestack}
\end{figure}

\begin{table}[h!tb]
    \centering
    \caption{Liste der verwendeten Python-Libraries}
    \label{tab:pythonLibs}
    \small
    \begin{tabular}{lrp{50mm}r}
        \toprule
        \textsc{Library} & \textsc{Version} & \textsc{Zweck} & \textsc{Website} \\
        \midrule
        PyQt & 5 & Erstellen und verwalten der graphischen Bedienelemente & \cite{ref:pyqt} \\
        [2mm]
        \rowcolor{solarized-base2}
        SQLAlchemy & 1.0 & Datenbankabstraktion                           & \cite{ref:sqlalchemy} \\
        [2mm]
        WiringPi for Python & 2 & Abstraktion der Hardware-Schnittstellen & \cite{ref:wiringpi} \\
        \bottomrule
    \end{tabular}
\end{table}

% ---------------------------------------------------------------------------- %
\clearpage
\subsection{Lizenzen}
\label{subsec:software:master:licenses}
% ---------------------------------------------------------------------------- %

Bei    der   Auswahl    von    Drittsoftware   wird    auf   die    jeweiligen
Lizenzbedingungen   geachtet,   um   keine   Konflikte   zu   verursachen. Die
wichtigsten  drei Lizenz-Bereiche  und ihre  Charakteristiken sind  in Tabelle
\ref{tab:lincenseAreas} aufgef\"uhrt.

\begin{table}[h!tb]
    \centering
    \caption{Lizenzbereiche}
    \label{tab:licenseAreas}
    \small
    \begin{tabular}{>{\raggedright}p{30mm}>{\raggedright}p{30mm}p{50mm}}
        \toprule
        \textsc{Bereich} &
        \textsc{Lizenz} &
        \textsc{Bedingungen} \\
        \midrule
        Linux-Kernel &
        GPL &
        Quellcode und \"Anderungen m\"ussen \"offentlich sein. \\
        [2mm]

        \rowcolor{solarized-base2}
        Treiber f\"ur Raspberry Pi und Display &
        Restricted &
        Quellcode wird vom Hersteller geheim gehalten \\
        [2mm]

        Raspbian &
        DFSG (Sammlung diverser Lizenzen) &
        Darf frei verwendet, aber nicht unbedingt verkauft, werden. \\
        \bottomrule
    \end{tabular}
\end{table}

Grundlage  f\"ur die  Software  bildet  das angepasste  Betriebssystem. Dieses
wird  von  der Raspberry  Pi  Foundation  frei  zur Verf\"ugung  gestellt  und
unterliegt den  Bedingungen den  DFSG (\emph{Debian Free  Software Guidelines}
\cite{ref:socialContract}). Die  darauf  aufbauenden Programmbibliotheken  zur
Abstraktion  von Betriebsystemfunktionen  und weiteren  Hardware-Aufrufen sind
alle aus den Raspbian-Repositories verf\"ugbar und unterliegen daher ebenfalls
den  DFSG.   Da  die  Mastersoftware zwar  auf  diesen  Komponenten  aufsetzt,
sie  aber  nicht ver\"andert  oder  statisch  verlinkt wird,  entstehen  keine
Lizenzkonflikte. Zu beachten ist hier, dass diese Drittsoftware im Allgemeinen
nicht als Eigenwerk  verkauft werden darf. Das heisst, dass  sie zwar beliebig
verbreitet werden darf, nicht aber zum Produkt hinzugez\"ahlt werden kann.

Die DFSG stellen  insbesondere folgende Anforderung an  alle Programme, welche
Teil von Raspbian sind:

\begin{itemize}
    \tightlist
\item
    Die Software darf frei verbreitet werden (Regel 1)
\item
    Die Software darf f\"ur beliebige Zwecke eingesetzt werden (Regel 6)
\item
    Die Software beschr\"ankt unzusammenh\"angende Software nicht (Regel 9)
\end{itemize}

Der eigentliche  Mastersoftware-Quellcode dagegen  ist nicht  \"offentlich und
kann als Bestandteil des Produkts verkauft werden.



% ---------------------------------------------------------------------------- %
\subsection{Threads}
\label{subsec:software:master:threads}
% ---------------------------------------------------------------------------- %


Die  Mastersoftware  ist in  mehrere  Threads  gegliedert, unter  welchen  die
Funktionen aufgeteild  sind. Sie werden alle von  einem Hauptthread gestartet,
welcher die Koordination mittels Semaphoren \"ubernimmt. Dazu initialisiert er
alle  Resourcen,  auf  welche  aus  mehreren  Threads  zugegriffen  sind. Dies
sind  Datenbank und  Logging-System, welche  beide Multithreading  beherrschen
und  threadsafe  sind. Die zentrale  Koordination  bedeutet  zudem, dass  alle
Arbeitsschritte nur sofern n\"otig und nicht mit veralteten Daten ausgef\"uhrt
werden.

Um  die Threadsicherheit  zu  gew\"ahrleisten verwenden  die beiden  geteilten
Resourcen spezielle Mechanismen:
\begin{itemize}
    \tightlist
    \item
        Die   Loggingfunktion   setzt  eine   Queue   eim   um  Meldungen   zu
        zwischenzuspeichern und asynchron hintereinander abzuarbeiten.
     \item
        SQLAlchemy  beinhaltet   mit  der  "ScopedSession"  eine   Methode  um
        aus  mehreren  Threads  parallel  aufgerufen  zu  werden. Dazu  werden
        die  Anfragen  aus jeweils  einem  Thread  gruppiert und  dann  atomar
        ausgef\"uhrt.
 \end{itemize}

Alle anderen  externen Ressourcen, dies  sind \ISC-bus, UART und  GPIO, werden
jeweils  von  nur einem  Thread  genutzt,  wodurch keine  Konflikte  auftreten
k\"onnen.

Im Detail sieht die Aufgabenteilung folgendermassen aus:

\begin{itemize}
    \tightlist
    \item
        \textbf{Prozesssteuerung:} Der  Hauptthread  verwaltet  Resourcen  und
        alle weiteren Threads.
    \item
        \textbf{Datensammlung:} Ein  Thread  empf\"angt   die  Spannungs-  und
        Strommesswerte von den Sensoren und speichert sie in der Datenbank ab.
    \item
        \textbf{Datenauswertung:} Ein   separater    Thread   untersucht   die
        gespeicherten  Messwerte   auf  defekte   Panels  und   speichert  die
        Ergebnisse ebenfalls in der Datenbank ab.
    \item
        \textbf{Graphisches  Benutzerinterface:} Der   GUI-Thread  stellt  ein
        Fenster  dar,   mit  welchem   der  Benutzer  die   Einstellungen  von
        Alarmierung  und   Telefonnummer  konfigurieren  kann   und  speichert
        \"Anderungen in der Datenbank.
    \item
        \textbf{Ausgabe:} Die  Umsetzung  der definierten  Massnahmen  obliegt
        einem Thread, welcher das Modem verwaltet und bei Bedarf die digitalen
        Ausg\"ange zur Steuerung der Relais bet\"atigt
\end{itemize}


% ---------------------------------------------------------------------------- %
\subsection{Benutzeroberfl\"ache}
\label{subsec:software:master:GUI}
% ---------------------------------------------------------------------------- %

\todo{unterliegender aufbau statt benutzerf\"uhrung?}
Die Benutzeroberfl\"ache wurde bewusst sehr schlicht gehalten. Sie besteht aus
folgenden  Ansichten: Hauptmen\"u,  Einstellungen, Eingabe  der  Telefonnummer
und   dem   Fehlerverlauf. Im   regul\"aren  Betrieb   ist   das   Hauptmen\"u
ersichtlich. Es  informiert  \"uber  den  Anlagestatus  sowie  den  gemessenen
Strom  der Str\"ange. Vom  Hauptmen\"u  aus kann  durch  dr\"ucken des  Feldes
\emph{Einstellungen} zu den Einstellungen gelangt werden. In den Einstellungen
k\"onnen die  Relaiskontakte 1  und 2  aktiviert oder  deaktiviert werden. Sie
dienen  zur Bet\"atigung  einer externen  Signalisation im  Fehlerfall. Weiter
kann   die  SMS   Benachrichtigung   ebenfalls   aktiviert  oder   deaktiviert
werden. Durch  bet\"atigen  des  Feldes  \emph{Speichern}  wird  die  aktuelle
Auswahl  gespeichert. Falls  die  SMS Benachrichtigung  gew\"ahlt  wird,  muss
eine  g\"ultige Telefonnummer  hinterlegt werden.   Die Ansicht  Eingabe einer
Telefonnummer  wird vom  Feld \emph{Telefonnr.  hinzuf\"ugen} aufgerufen. Nach
der  Eingabe  einer  g\"ultigen  Nummer,  wird  mittels  \emph{Speichern}  die
Telefonnummer  gespeichert und  es kann  mit dem  Feld \emph{Zur\"uck}  zu den
Einstellungen  zur\"uckgekehrt  werden. Diese  Ansichten sind  im  regul\"aren
Betrieb ersichtlich und k\"onnen nach Belieben eingestellt werden.

Folgend werden  die St\"orbetrieb Ansichten erl\"autert. Wird  ein Modulfehler
erkannt,  signalisiert   dies  die  Ansicht  \emph{Modulfeher}. Es   wird  die
Modulseriennummer und die Zeit mit Datum angezeigt. Damit ist es m\"oglich ein
fehlerhaftes Modul zu erkennen  und entsprechend zu handeln. Durch bet\"atigen
des  Feldes  \emph{OK} wird  der  externe  Alarm der  Relais  1  und 2  (falls
aktiviert)  quittiert  und das  Hauptmen\"u  wird  wieder angezeigt. Das  Feld
\emph{Fehlerverlauf}  f\"uhrt  zu  einer   \"Ubersicht  aller  bis  zu  diesem
Zeitpunkt aufgetretenen  Modulfehler. Jeweils wird  die Modulnummer,  die Zeit
und das Datum in einer Zeile angezeigt.


% ---------------------------------------------------------------------------- %
\subsection{Datenbank}
\label{subsec:software:master:database}
% ---------------------------------------------------------------------------- %

\begin{figure}[h!tb]
    \centering
    \begin{tikzpicture}
        \sffamily
        \small
        %\graph [left anchor=east,grow right sep,right anchor=west,grow right,branch down,nodes={rounded rectangle,draw}]
        \begin{scope}[
                every node/.style = {
                    draw = solarized-base02,
                    rounded rectangle,
                    fill=solarized-base2,
                    text=solarized-base02,
                    inner sep=1mm,
                    text height=0.7em,
                },
                node distance=3mm,
                table/.style={
                    fill=solarized-cyan,
                    text=solarized-base2,
                    inner sep=2mm,
                },
                column/.style={
                    minimum width=25mm,
                }
            ]
            \node (database)[
                fill=solarized-blue,
                text=solarized-base2,
                inner sep=2mm
            ] at (0,0) {Datenbank};

            \node (strings)   [table,below left=15mm of database] {strings};
            \node (sID)       [column,below right=12mm of strings] {ID};
            \node (sStringNo) [column,below=of sID]            {stringnumber};
            \node (sCurr)     [column,below=of sStringNo]      {stringcurrent};
            \node (sTs)       [column,below=of sCurr]          {timestamp};
            \node (sDev)      [column,below=of sTs]            {deviation};
            \node (sReported) [column,below=of sDev]           {flag\_reported};

            \node (panels)    [table,left=of strings]  {panels};
            \node (pID)       [column,below left=12mm of panels]    {ID};
            \node (pSerial)   [column,below=of pID]       {serialnumber};
            \node (pVoltage)  [column,below=of pSerial]   {voltage};
            \node (pStringNo) [column,below=of pVoltage]  {stringnumber};
            \node (pTs)       [column,below=of pStringNo] {timestamp};
            \node (pDev)      [column,below=of pTs]       {deviation};
            \node (pReported) [column,below=of pDev]      {flag\_reported};

            \node (string1)  [table,below right=15mm of database] {string\{1,2,3\}};
            \node (s1ID)     [column,below right=12mm of string1]        {ID};
            \node (s1Serial) [column,below=of s1ID]           {serialnumber};
        \end{scope}
        \begin{scope}[on background layer]
        \end{scope}

        \begin{scope}[
                rounded corners,
                draw=solarized-base02
            ] % connections
            \draw[-latex] (database) -| (panels);
            \draw[-latex] (database) -| (strings);
            \draw[-latex] (database) -| (string1);

            \draw[-latex] (panels) |- (pID);
            \draw[-latex] (panels) |- (pSerial);
            \draw[-latex] (panels) |-  (pVoltage);
            \draw[-latex] (panels) |-  (pStringNo);
            \draw[-latex] (panels) |-  (pTs);
            \draw[-latex] (panels) |-  (pDev);
            \draw[-latex] (panels) |-  (pReported);

            \draw[-latex] (strings) |- (sID);
            \draw[-latex] (strings) |- (sStringNo);
            \draw[-latex] (strings) |- (sCurr);
            \draw[-latex] (strings) |- (sTs);
            \draw[-latex] (strings) |- (sDev);
            \draw[-latex] (strings) |- (sReported);

            \draw[-latex] (string1) |- (s1ID);
            \draw[-latex] (string1) |- (s1Serial);
        \end{scope}
    \end{tikzpicture}
    \caption{Datenbank-Layout}
  \label{fig:database:layout}
\end{figure}

% ---------------------------------------------------------------------------- %
\subsection{Funktionen}
\label{subsec:software:master:functions}
% ---------------------------------------------------------------------------- %

