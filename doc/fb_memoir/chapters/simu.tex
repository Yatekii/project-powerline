% **************************************************************************** %
\chapter{L\"osungsans\"atze und Simulationen}
\label{chap:simu}
% **************************************************************************** %

Bevor  simuliert  werden  kann,  m\"ussen  die  daf\"ur  ben\"otigten  Modelle
vorhanden sein. Es  wird zuerst ein  Modell f\"ur eine  Solarzelle entwickelt.
Anschliessend wird dieses Zellenmodell benutzt, um ein Modul aufzubauen.

In  diesem Abschnitt  werden Schaltungen  f\"ur drei  L\"osungsans\"atze (zwei
zur  FSK, eine  zur  ASK)  vorgestellt, mit  \code{LTspice  IV} simuliert  und
beurteilt.\todo{LTspice: source}

Es werden  jeweils die  Schaltung f\"ur  den Sender,  den Empf\"anger  und das
Gesamtsystem untersucht.


% ---------------------------------------------------------------------------- %
\section{Modellierung einer PV-Zelle}
\label{sec:simu:model:cell}
% ---------------------------------------------------------------------------- %


Ziel ist, eine Zelle zu modellieren, welche folgende Kriterien erf\"ullt:

\begin{align}
    \label{eq:diode:ULeerlauf}
    U_{\mathrm{Leerlauf}}                    &= \SI{600}{\milli\volt} \\
    \label{eq:diode:IKurzschluss:polyk}
    I_{\mathrm{Kurzschluss, polykristallin}} &=  \SI{5}{\ampere}      \\
    \label{eq:diode:IKurzschluss:monok}
    I_{\mathrm{Kurzschluss, monokristallin}} &= \SI{10}{\ampere}
\end{align}

Die Herleitung  dieser Werte ist  in Anhang \ref{app:subsec:cell:UI}  ab Seite
\pageref{app:subsec:cell:UI} zu  finden und basiert auf  den PV-Modulen, deren
Daten in Tabelle \ref{tab:moduleData:IU} auf Seite \pageref{tab:moduleData:IU}
aufgelistet  sind.   Es  werden  zwei  Zielwerte  f\"ur  den  Kurzschlussstrom
angegeben;  einer f\"ur  monokristalline Zellen,  einer f\"ur  polykristalline
Zellen.

Als  Grundlage f\"ur  das  Modell dient  das  Eindiodenmodell einer  PV-Zelle,
erweitert  um  eine parallele  Kapazit\"at  $C$  (basierend auf  Informationen
aus \cite{ref:solar:scofield} und \cite{ref:solar:friesen}). Die resultierende
Schaltung ist dargestellt in \fref{fig:circuit:solarCell}.

\clearpage
\begin{figure}[h!tb]
    \centering
    % Parametrized version, can be put at coordinates x and y
%
%\newcommand{\CTSolarCell}[2]{%
%    % Source to top terminal
%    (0+#1,0+#2) to[current source,i=$I_{\mathrm{Zelle}}$] (0+#1,4+#2) -- (5+#1,4+#2) to[R,l^=$R_{\mathrm{S}}$] (10+#1,4+#2) node[circ] {}
%
%    % Open terminal at bottom
%    (10+#1,0+#2) node[circ] {} -- (0+#1,0+#2)
%
%    % Voltage arrow over open terminals
%    %(10+#1,3.7+#2) -- (10+#1,0.3+#2) node[currarrow,rotate=-90] {}
%    %(10+#1,2+#2) node[anchor=west] {$V_{\mathrm{offen}}$}
%
%    % Parallel elements
%    (1.5+#1,4+#2) to[empty diode,*-*,l_=$D$] (1.5+#1,0+#2)
%    (3+#1,4+#2) to[C,*-*,l_=$C$] (3+#1,0+#2)
%    (4.5+#1,4+#2) to[R,*-*,l_=$R_{\mathrm{P}}$] (4.5+#1,0+#2)%
%}


% Single Diode
%\begin{circuitikz}
%    \draw
%    % Source to top terminal
%    (0,0) to[current source,i=$I_{\mathrm{Zelle}}$] (0,3) -- (5,3) to[R,l^=$R_{\mathrm{S}}$] (8,3) node[ocirc] {}
%
%    % Open terminal at bottom
%    (8,0) node[ocirc] {} -- (0,0)
%
%    % Voltage arrow over open terminals
%    (8,2.7) -- (8,0.3) node[currarrow,rotate=-90] {}
%    (8,1.5) node[anchor=west] {$V_{\mathrm{offen}}$}
%
%    % Parallel elements
%    (1.5,3) to[empty diode,*-*,l_=$D$] (1.5,0)
%    (3,3) to[C,*-*,l_=$C$] (3,0)
%    (4.5,3) to[R,*-*,l_=$R_{\mathrm{P}}$] (4.5,0)
%    ;
%\end{circuitikz}

% Two Diodes
\begin{circuitikz}
    \draw
    % Source to top terminal
    (0,0) to[current source,i=$I_{\mathrm{Ph}}$] (0,3) -- (7.5,3) to[R,l^=$R_{\mathrm{S}}$] (11,3) node[ocirc] {}

    % Open terminal at bottom
    (11,0) node[ocirc] {} -- (0,0)

    % Voltage arrow over open terminals
    (11,2.7) -- (11,0.3) node[currarrow,rotate=-90] {}
    (11,1.5) node[anchor=west] {$V_{\mathrm{offen}}$}

    % Parallel elements
    (2,3) to[empty diode,*-*,l_=$D_{\mathrm{Diff}}$] (2,0)
    (4.5,3) to[empty diode,*-*,l_=$D_{\mathrm{Rekomb}}$] (4.5,0)
    (6,3) to[C,*-*,l_=$C$] (6,0)
    (7.5,3) to[R,*-*,l_=$R_{\mathrm{Sh}}$] (7.5,0)
    ;
\end{circuitikz}

    \caption{%
        Schaltschema    zur    Modellierung    einer    Solarzelle    gem\"ass
        Eindiodenmodell mit zus\"atzlicher Kapazit\"at%
    }
    \label{fig:circuit:solarCell}
\end{figure}

Bei diesem Modell sind insgesamt f\"unf Parameter zu bestimmen:

\begin{itemize}
    \firmlist
    \item
        Seriewiderstand $R_{\mathrm{S}}$
    \item
        Shunt-Widerstand $R_{\mathrm{P}}$
    \item
        Parasit\"are Kapazit\"at $C$
    \item
        Diode   $D$: Reverse    Saturation   Current    $I_{\mathrm{S}}$   und
        Idealit\"atsfaktor $n$
\end{itemize}

Es werden zuerst die Kapazit\"at und die Widerst\"ande bestimmt.

\myfancybreak

In \cite{ref:solar:scofield} wurden $C$, $R_{\mathrm{S}}$ und $R_{\mathrm{P}}$
einer  Solarzelle der  Gr\"osse  \SI{0.43}{\centi\meter\squared} \"uber  einen
Frequenzbereich  von \SI{1}{\kilo\hertz} bis \SI{1}{\mega\hertz} gemessen. Die
Resultate waren:

\begin{alignat}{3}
    \label{eq:scofield:C}
    C_{\mathrm{Messung}}    &= \SI{8}{\nano\farad} \; & \text{bis} \; & \SI{20}{\nano\farad} \;  & = & \; \SI{14 \pm 6}{\nano\farad} \\
    \label{eq:scofield:Rs}
    R_{\mathrm{S, Messung}} &= \SI{0.2}{\ohm}      \; & \text{bis} \; & \SI{20}{\ohm}        \;  & = & \; \SI{10.1 \pm 9.9}{\ohm}     \\
    \label{eq:scofield:Rp}
    R_{\mathrm{P, Messung}} &= \SI{0.5}{\kilo\ohm} \; & \text{bis} \; & \SI{500}{\kilo\ohm}  \;  & = & \; \SI{250.25 \pm 249.75}{\kilo\ohm}
\end{alignat}

Um die  obigen Werte f\"ur unsere  Zwecke verwenden zu k\"onnen,  m\"ussen sie
auf eine Zelle der Fl\"ache \SI{240}{\centi\meter\squared} um ungef\"ahr einen
Faktor \num{600} hochskaliert werden (zur Herleitung dieses Werts siehe Anhang
\ref{app:subsec:cell:surface} ab Seite \pageref{app:subsec:cell:surface})

$R_{\mathrm{P}}$   und  $R_{\mathrm{S}}$   skalieren  umgekehrt   proportional
zur   Zellfl\"ache,    wogegen   $C$   bei   gr\"osser    werdender   Fl\"ache
ansteigt~\cite{ref:solar:scofield}. Bei    der   ASK-Implementation    mittels
gesteuertem   Kurzschluss    (siehe   Abbildung    \ref{fig:ask:concept}   auf
Seite    \pageref{fig:ask:concept}     und    Simulationen     in    Abschnitt
\ref{subsec:simu:ask:sensor})  nehmen   die  Schwierigkeiten   mit  steigendem
Ausgangsstrom   der  Zelle   zu. Das   Modell  wird   deshalb  auf   maximalen
Ausgangsstrom getrimmt:

\begin{symbols}
    \firmlist
    \item[$C$]
        Beim Kurzschliessen der  Zelle treten Stromspitzen auf,  wenn sich die
        Kapazit\"at $C$ entl\"adt. Je gr\"osser  die Kapazit\"at, umso h\"oher
        werden diese  Stromspitzen.  Wir nehmen also  \SI{20}{\nano\farad} aus
        Gleichung \ref{eq:scofield:C} als Ausgangswert.
    \item[$R_{\mathrm{S}}$]
        Der aus der Solarzelle fliessende Strom wird gr\"osser, je kleiner der
        Seriewiderstand vor dem Ausgang ist. Es wird daher der niedrigere Wert
        von \SI{0.2}{\ohm} aus Gleichung \ref{eq:scofield:Rs} gew\"ahlt.
    \item[$R_{\mathrm{P}}$]
        Je    gr\"osser   der    Parallelwiderstand   im    Verh\"altnis   zum
        Seriewiderstand  ist, um  so mehr  Strom fliesst  aus dem  Ausgang der
        Zelle.  Es wird deshalb der h\"ochste Wert von \SI{500}{\kilo\ohm} aus
        Gleichung \ref{eq:scofield:Rp} als Ausgangswert verwendet.
\end{symbols}

Die skalierten Werte sind somit:
\begin{alignat}{2}
    C               &= C_{\mathrm{Messung}}    &\cdot 600 &= \SI{12}{\micro\farad} \\
    R_{\mathrm{S,}} &= R_{\mathrm{S, Messung}} &\div  600 &= \SI{333}{\micro\ohm}    \\
    R_{\mathrm{P,}} &= R_{\mathrm{P, Messung}} &\div  600 &= \SI{833}{\ohm}
\end{alignat}

\myfancybreak

Es   bleiben  noch   die  Parameter   $I_{\mathrm{S}}$  und   $n$  der   Diode
zu   bestimmen. Erkl\"arungen   zu   diesen    Parametern   sind   in   Anhang
\ref{app:sec:cell:diodeInfo}  ab   Seite  \pageref{app:sec:cell:diodeInfo}  zu
finden.

Zur   Simulation  in   \code{LTspice}  wird   ein  simples   Diodenmodell  mit
$I_{\mathrm{S}}$   und   $n$  benutzt\footnotemark. Es   werden   Anfangswerte
gesch\"atzt, um  die Parameter  anschliessend mittels Iteration  und Vergleich
mit den Sollwerten (siehe Beginn dieses Abschnitts) zu optimieren.

\footnotetext{\code{LTspice}-Direktive: \code{.model diode1 D(IS=1e-6 N=2)}}

Mit   einer    Transientensimulation   wird   die   Zelle    simuliert,   ihre
Leerlaufspannung  $V_{\mathrm{offen}}$  gemessen  und  mit  dem  Zielwert  aus
Gleichung   \ref{eq:diode:ULeerlauf}    (Seite   \pageref{eq:diode:ULeerlauf})
verglichen \todo{``wird'' nur einmal oder wiederholen?}.  Anschliessend werden
die  Werte  f\"ur  \code{IS}  und \code{N}  angepasst,  bis  die  gew\"unschte
Leerlaufspannung  erreicht  ist.   Nach  einigen  Iterationen  liefert  dieser
Prozess\footnotemark:

\footnotetext{%
    Dies ist eine m\"ogliche L\"osung. Es gibt nat\"urlich noch beliebig viele
    weitere  Kombinationen von  \code{IS} und  \code{N}, welche  die gegebenen
    Bedingungen  erf\"ullen. Wir   sind  hier  lediglich  an   einer  L\"osung
    interessiert, von  der wir zuversichtlich  sind, dass sie  ein hinreichend
    gutes Modell liefert.%
}

\begin{align}
    \label{eq:cell:diode:IS:N:result}
    I_{\mathrm{S}} &= \SI{4}{\micro\ampere} \\
    N              &= 1.65
\end{align}

Somit ist  das Modell  einer einzelnen  Zelle bestimmt:

\begin{figure}[h!tb]
    \centering
    % Parametrized version, can be put at coordinates x and y
%
%\newcommand{\CTSolarCell}[2]{%
%    % Source to top terminal
%    (0+#1,0+#2) to[current source,i=$I_{\mathrm{Zelle}}$] (0+#1,4+#2) -- (5+#1,4+#2) to[R,l^=$R_{\mathrm{S}}$] (10+#1,4+#2) node[circ] {}
%
%    % Open terminal at bottom
%    (10+#1,0+#2) node[circ] {} -- (0+#1,0+#2)
%
%    % Voltage arrow over open terminals
%    %(10+#1,3.7+#2) -- (10+#1,0.3+#2) node[currarrow,rotate=-90] {}
%    %(10+#1,2+#2) node[anchor=west] {$V_{\mathrm{offen}}$}
%
%    % Parallel elements
%    (1.5+#1,4+#2) to[empty diode,*-*,l_=$D$] (1.5+#1,0+#2)
%    (3+#1,4+#2) to[C,*-*,l_=$C$] (3+#1,0+#2)
%    (4.5+#1,4+#2) to[R,*-*,l_=$R_{\mathrm{P}}$] (4.5+#1,0+#2)%
%}


\begin{circuitikz}
    \ctikzset{label/align = rotate}
    \draw
    % Source to top terminal
    (0,0) to[current source,i=$\SI{10}{\ampere}{,}~\SI{5}{\ampere}$] (0,3) -- (6,3) to[R,l^=\SI{333}{\micro\ohm}] (10,3) node[ocirc] {}

    % Open terminal at bottom
    (10,0) node[ocirc] {} -- (0,0)

    % Voltage arrow over open terminals
    (10,2.7) -- (10,0.3) node[currarrow,rotate=-90] {}
    (10,1.5) node[anchor=west] {$V_{\mathrm{offen}}$}

    % Parallel elements
    %(1.5,3) to[empty diode,*-*,l_=$I_{\mathrm{S}} = \SI{4}{\micro\ampere}, n = 1.65$] (1.5,0)
    (2,0) to[empty diode,*-*,l_={$\SI{4}{\micro\ohm}{,} 1.65$}] (2,3)
    (4,0) to[C,*-*,l_=\SI{12}{\micro\farad}] (4,3)
    (6,0) to[R,*-*,l_=$\SI{833}{\ohm}$] (6,3)
    ;
\end{circuitikz}

    \caption{%
        Schaltschema   zur  Modellierung   einer   Solarzelle  mit   numerisch
        bestimmten Elementen
    }
    \label{fig:circuit:solarCell}
\end{figure}

%\begin{align}
%    R_{\mathrm{S}} &= \SI{333}{\micro\ohm}   \\
%    R_{\mathrm{P}} &= \SI{833}{\ohm}         \\
%    C              &= \SI{12}{\micro\farad}  \\
%    I_{\mathrm{S}} &= \SI{4}{\micro\ampere}            \\
%    N              &= 1.65
%\end{align}

Im folgenden Abschnitt wird dieses Zellenmodell nun dazu benutzt, ein Modul
aufzubauen.


% ---------------------------------------------------------------------------- %
\clearpage
\section{Modellierung eines PV-Moduls}
\label{sec:simu:model:module}
% ---------------------------------------------------------------------------- %

Es  sollen   zwei  verschiedene   Module  simuliert  werden: Beide   sind  aus
72   Zellen   aufgebaut,   jedoch   sind   die   Zellen   beim   einen   Modul
in   zwei    parallelen   Str\"angen    zu   je   36    seriell   geschalteten
Zellen   (\fref{fig:circuit:36x2:simplified})    und   beim    anderen   Modul
in   einem   einzigen   Strang   aus   72   in   Serie   geschalteten   Zellen
aufgebaut(\fref{fig:circuit:72x1:simplified}).

Mit  diesen  beiden  Modellen  kann  jeweils  ein  typisches  monokristallines
($72  \times 1$)  und ein  polykristallines  Modul ($36  \times 2$)  simuliert
werden. Tabelle \ref{tab:moduleData:IU}  auf Seite \pageref{tab:moduleData:IU}
in Anhang \ref{app:models:develop:cell}  enth\"alt als exemplarische Beispiele
die Daten einiger  Module, die auf dem Markt  erh\"altlich sind. Basierend auf
diesen Informationen wird  der Kurzschlussstrom f\"ur ein  $36 \times 2$-Modul
auf \SI{5}{\ampere} festgelegt; der  Kurzschlusstrom des $72 \times 1$-Modells
auf \SI{10}{\ampere}.

Abgesehen  vom  Strom liegt  ein  weiterer  f\"ur unsere  Szenarien  wichtiger
Unterschied in der Gesamtkapazit\"at eines  Moduls: Das Modul mit 72 Zellen in
Serie hat die  halbe Kapazit\"at wie das Modul mit  zwei parallelen Str\"angen
von je 36 Modulen.

F\"ur     das    $72     \times     1$-Modul     wird    zus\"atzlich     noch
ein     vereinfachtes    Modell     entwickelt,    um     den    Rechenaufwand
bei      aus     mehreren      Modulen     aufgebauten      Schaltungen     zu
reduzierten. Anhang   \ref{app:models:develop:module:simple}    enth\"alt   ab
Seite     \pageref{app:models:develop:module:simple}     die     zugeh\"origen
Hintergrundinformationen,  Abbildung  \ref{fig:circuit:moduleSimplifiedParams}
ist eine schematische Darstellung des vereinfachten Modells mit den bestimmten
Parametern.

Die zugeh\"origen \code{LTspice}-Schaltungen  sind in Anhang \ref{app:ltspice}
ab Seite \pageref{app:ltspice} zu finden.

\todo{kleines Bild der gesamten Schaltung?}

\begin{figure}[h!tb]
    \centering
    \adjustbox{valign=t}{%
    \begin{minipage}{0.475\textwidth}
        \centering
        % x: 0, 1.5, 2.5, 4
\def\POSxUp{0,4}
\def\POSxDown{1.5,2.5}
\def\POSy{0,1,3}
\begin{circuitikz}
    \foreach \x in \POSxUp{
        \foreach \y in \POSy {
            \draw
            (\x,\y) to[empty photodiode] (\x,\y+1)
            ;
        }
        \draw (\x,2.5) node[rotate=90] {\ldots};
    }

    \draw(2,2.5) node {$\times 15$};

    \foreach \x in \POSxDown{
        \foreach \y in \POSy {
            \draw
            (\x,\y+1) to[empty photodiode] (\x,\y)
            ;
        }
        \draw (\x,2.5) node[rotate=90] {\ldots};
    }

    % Connecting Lines between Strings
    \draw (0,4) -- (1.5,4);
    \draw (1.5,0) -- (2.5,0);
    \draw (2.5,4) -- (4,4);

    \draw (2,0) node[circ] {} -- (2,-0.5) node[ocirc] {~~OUT};

    \draw (0,0) -- (0,-1) -- (4,-1) -- (4,0);
    \draw (2,-1) node[circ] {} -- (2,-1.5) node[ocirc] {~~IN};
\end{circuitikz}

        \figcaption{%
            Vereinfachtes   Schema  f\"ur   das   $36  \times   2$-Modul: Zwei
            parallele    Str\"ange    mit    je    36    Zellen    in    Serie
            geschaltet. Die  vollst\"andige  \code{LTspice}-Schaltung  ist  in
            Abbildung   \ref{fig:ltspice:module:cellBased:36x2}    auf   Seite
            \pageref{fig:ltspice:module:cellBased:36x2} dargestellt.%
        }
        \label{fig:circuit:36x2:simplified}
    \end{minipage}}\hspace*{0.05\textwidth}\adjustbox{valign=t}{%
    \begin{minipage}{0.475\textwidth}
        \centering
        % x: 0, 1, 3, 4
\def\POSxUp{0,3}
\def\POSxDown{1,4}
\def\POSy{0,1,3,4}
\begin{circuitikz}
    \foreach \x in \POSxUp{
        \foreach \y in \POSy {
            \draw
            (\x,\y) to[empty photodiode] (\x,\y+1)
            ;
        }
        \draw (\x,2.5) node[rotate=90] {\ldots};
    }
    \foreach \x in \POSxDown{
        \foreach \y in \POSy {
            \draw
            (\x,\y+1) to[empty photodiode] (\x,\y)
            ;
        }
        \draw (\x,2.5) node[rotate=90] {\ldots};
    }

    % Connecting Lines between Strings
    \draw (0,5) -- (1,5);
    \draw (1,0) -- (3,0);
    \draw (3,5) -- (4,5);

    \draw (0,0) -- (0,-0.5) node[ocirc] {~~IN};
    \draw (4,0) -- (4,-0.5) node[ocirc] {~~OUT};
\end{circuitikz}

        \figcaption{%
            Vereinfachtes      Schema      f\"ur      das      $72      \times
            1$-Modul: Ein     Strang    mit     72     Zellen     in     Serie
            geschaltet. Die  vollst\"andige  \code{LTspice}-Schaltung  ist  in
            Abbildung   \ref{fig:ltspice:module:cellBased:72x1}    auf   Seite
            \pageref{fig:ltspice:module:cellBased:72x1} dargestellt.%
        }
        \label{fig:circuit:72x1:simplified}
    \end{minipage}}
\end{figure}

\begin{figure}[h!tb]
    % Parametrized version, can be put at coordinates x and y
%
%\newcommand{\CTSolarCell}[2]{%
%    % Source to top terminal
%    (0+#1,0+#2) to[current source,i=$I_{\mathrm{Zelle}}$] (0+#1,4+#2) -- (5+#1,4+#2) to[R,l^=$R_{\mathrm{S}}$] (10+#1,4+#2) node[circ] {}
%
%    % Open terminal at bottom
%    (10+#1,0+#2) node[circ] {} -- (0+#1,0+#2)
%
%    % Voltage arrow over open terminals
%    %(10+#1,3.7+#2) -- (10+#1,0.3+#2) node[currarrow,rotate=-90] {}
%    %(10+#1,2+#2) node[anchor=west] {$V_{\mathrm{offen}}$}
%
%    % Parallel elements
%    (1.5+#1,4+#2) to[empty diode,*-*,l_=$D$] (1.5+#1,0+#2)
%    (3+#1,4+#2) to[C,*-*,l_=$C$] (3+#1,0+#2)
%    (4.5+#1,4+#2) to[R,*-*,l_=$R_{\mathrm{P}}$] (4.5+#1,0+#2)%
%}


\begin{circuitikz}
    \ctikzset{label/align = rotate}
    \draw
    % Source to top terminal
    (0,0) to[current source,i=$\SI{10}{\ampere}$] (0,3) -- (6,3) to[R,l^=\SI{12}{\milli\ohm}] (10,3) node[ocirc] {}

    % Open terminal at bottom
    (10,0) node[ocirc] {} -- (0,0)

    % Voltage arrow over open terminals
    (10,2.7) -- (10,0.3) node[currarrow,rotate=-90] {}
    (10,1.5) node[anchor=west] {$V_{\mathrm{offen}}$}

    % Parallel elements
    %(1.5,3) to[empty diode,*-*,l_=$I_{\mathrm{S}} = \SI{4}{\micro\ampere}, n = 1.65$] (1.5,0)
    (2,0) to[empty diode,*-*,l_={$\SI{2.88}{\micro\ampere}{,}~~118$}] (2,3)
    (4,0) to[C,*-*,l_=\SI{167}{\nano\farad}] (4,3)
    (6,0) to[R,*-*,l_=\SI{60}{\kilo\ohm}] (6,3)
    ;
\end{circuitikz}

    \caption{%
        Vereinfachtes  Modell eines  PV-Moduls mit  Parametern. Die Herleitung
        und Verifikation sind in Anhang \ref{app:models:develop:module:simple}
        ab Seite \pageref{app:models:develop:module:simple} dokumentiert.%
    }
    \label{fig:circuit:moduleSimplifiedParams}
\end{figure}

% ---------------------------------------------------------------------------- %
\section{Frequenzumtastung: Kapazitive Einkopplung}
\label{sec:simu:fsk:capacitive}
% ---------------------------------------------------------------------------- %

% ---------------------------------------------------------------------------- %
\subsection{Sender}
\label{sec:simu:fsk:capacitive:transmitter}
% ---------------------------------------------------------------------------- %

% ---------------------------------------------------------------------------- %
\subsection{Empf\"anger}
\label{sec:simu:fsk:capacitive:receiver}
% ---------------------------------------------------------------------------- %

% ---------------------------------------------------------------------------- %
\subsection{Gesamtsystem}
\label{sec:simu:fsk:capacitive:system}
% ---------------------------------------------------------------------------- %

% ---------------------------------------------------------------------------- %
\section{Frequenzumtastung: Induktive Einkopplung}
\label{sec:simu:fsk:inductive}
% ---------------------------------------------------------------------------- %

Eine induktive Einkopplung legt eine  Spule um die DC-Leitung. Auf diese Spule
wird von der FSK-Schaltung das zu  \"ubertragende Signal gegeben und die Spule
induziert in  der DC-Leitung  entsprechende Spannungs-Rippel, die  vom \Master
ausgewertet werden k\"onnen. Der entsprechende  Schaltkreis ist schematisch in
Abbildung \ref{fig:circ:coupling:inductive} dargestellt.

Verglichen mit  Kondensatoren sind Spulen relativ  gross und teuer. Allerdings
ist das Prinzip  der induktiven Einkopplung gut dokumentiert  und die Aussicht
auf Erfolg (bei vern\"unftigem Aufwand) somit gut.

\begin{figure}[h!tb]
    \centering
    \begin{circuitikz}
    \draw
    (-1,0) to[empty photodiode,o-,l_=PV-Modul] (1,0) to[short] (6,0)
    %(2,-2) to[short,o-] (2,-0.05) to[L=L] (4,-0.05) to[short,-o] (4,-2)
    (2,-2) -- (2,-0.05) to[L,l^=Kopplung] (4,-0.05) -- (4,-2) to[sinusoidal voltage source,l^=$U_{\mathrm{Signal}}$] (2,-2)
    ;
\end{circuitikz}

    \caption{Induktive Einkopplung}
    \label{fig:circ:coupling:inductive}
\end{figure}

% ---------------------------------------------------------------------------- %
\subsection{Sender}
\label{sec:simu:fsk:inductive:transmitter}
% ---------------------------------------------------------------------------- %

% ---------------------------------------------------------------------------- %
\subsection{Empf\"anger}
\label{sec:simu:fsk:inductive:receiver}
% ---------------------------------------------------------------------------- %

% ---------------------------------------------------------------------------- %
\subsection{Gesamtsystem}
\label{sec:simu:fsk:inductive:inductive}
% ---------------------------------------------------------------------------- %


% ---------------------------------------------------------------------------- %
\section{Amplitudenumtastung}
\label{sec:simu:ask}
% ---------------------------------------------------------------------------- %

Wie in Abschnitt \todo{reference} erw\"ahnt, wird bei dieser L\"osungsvariante
jeweils   ein   Modul   gesteuert   kurzgeschlossen. Dies   verursacht   kurze
Spannungseinbr\"uche auf  der DC-Leitung,  welche vom  Empf\"anger ausgewertet
werden k\"onnen, wie in Abbildung \todo{reference}vereinfact dargestellt.

Potentielle Probleme sind bei  dieser L\"osungsvariante in folgenden Bereichen
zu suchen:

\begin{symbols}
    \firmlist
    \item[\textbf{Induktivit\"at der Leitung:}]
        Der  Spannungsabfall   auf  der  DC-Leitung  wird   bei  geschlossenem
        Stromkreis zu Strom\"anderungen auf der DC-Leitung f\"uhren. Dies wird
        eine Spannungs\"anderung auf der DC-Leitung bewirken\footnotemark.
        \footnotetext{%
            Spannung in Abh\"angigkeit der Strom\"anderung:
            $v = L \cdot \frac{\mathrm{d}i}{\mathrm{d}t}$%
        }

        Gem\"ass    Lenz'scher     Regel    \todo{reference}     wird    diese
        Spannungs\"anderung  so gerichtet  sein,  dass  sich der  zugeh\"orige
        Strom der aufgezwungenen \"Anderung widersetzt.

        Es  kann also  sein, dass  die  Induktivit\"at der  DC-Leitung das  zu
        \"ubertragende  Signal  so  stark  kompensiert,  dass  es  nicht  mehr
        detektierbar  ist. Je   h\"oher  die  Frequenz,  mit   der  das  Modul
        kurzgeschlossen  und  wieder  ge\"offnet   wird,  um  so  h\"oher  die
        zugeh\"origen Strom\"anderungen
        $\frac{\mathrm{d}i}{\mathrm{d}{t}}$
        und somit Impedanz der Induktivit\"at (gem\"ass
        $Z_L = j \omega L$).
    \item[\textbf{Kapazit\"at des Solarmoduls:}]
        Dem  Solarmodul   wird  bei   diesem  Vorgehen  das   Verhalten  einer
        Wechselstromquelle   aufgezwungen. Besitzt  es   interne  parasit\"are
        Kapazit\"aten,  k\"onnen  diese  bei  den  abrupten  \"Anderungen  der
        Spannung hohe Str\"ome im  kurzgeschlossenen Pfad und seinen Bauteilen
        verursachen.\footnotemark
        \footnotetext{%
            Strom\"anderung in Abh\"angigkeit der Spannungs\"anderung:
            $I(t) = C \cdot \frac{\mathrm{V(t)}}{\mathrm{d}t}$%
        }

        Besitzen    diese   Bauteile    Ohm'sche   Widerst\"ande,    entstehen
        entsprechende thermische  Verluste, welche die  Bauteile besch\"adigen
        k\"onnen.
\end{symbols}

Die    Tr\"agerfrequenz    in    den    folgenden    Simulationen    betr\"agt
\SI{10}{\kilo\hertz}.


% ---------------------------------------------------------------------------- %
\subsection{Sender}
\label{subsec:simu:ask:sensor}
% ---------------------------------------------------------------------------- %

Das  gesteuerte Kurzschliessen  des Moduls  wird mit  einem MOSFET  umgesetzt,
welcher zwischen  Eingang und  Ausgang des Moduls  durchschalten kann  und vom
Microcontroller auf dem Sensor gesteuert wird. \fref{fig:module:mosfet:simple}
zeigt diesen Aufbau schematisch.

Die  Ansteuerung des  Transistors  erfolgt mit  \SI{3.3}{\volt},  da dies  die
maximale Spannung  ist, welche der  auf dem Sensor  platzierte Microcontroller
ausgeben kann.

\begin{figure}[h!tb]
    \centering
    % Pro memoriam:
%
%            |  D
%      | |---+
%      |
%      | |<--+  B
%      |     |
% G ---+ |---+
%            |  S
%(mos.B) node[anchor=west] {B}
%(mos.G) node[anchor=east] {G}
%(mos.D) node[anchor=north] {D}
%(mos.S) node[anchor=south] {S}

\begin{circuitikz}
    \small
    \draw
    (4.5,1) node[nigfete] (mos) {MOSFET}

    (0,0) to[empty photodiode,l_=PV-Modul] (0,2) -- (4.5,2) -- (mos.D)
    %(0,0) to[dcisource,l_=PV-Modul] (0,4) -- (8,4) -- (mos.D)
    (mos.S) -- (4.5,0) -- (0,0)

    (2.25,0.725) to[sinusoidal voltage source,l^=Controller] (mos.G)
    (2.25,0.725) -- (2,0.725) node[sground] {}
    ;
\end{circuitikz}

    \caption{%
        Gesteuerter     Kurzschluss     eines    Solarmoduls     mit     einem
        microcontroller-gesteuerten       Transistor. Die      vollst\"andigen
        \code{LTspice}-Schaltungen  sind  in Anhang  \ref{app:simu:module}  ab
        Seite \pageref{app:simu:module} dokumentiert.%
    }
    \label{fig:module:mosfet:simple}
\end{figure}

%TODO: line width
\begin{figure}
    \begin{tikzpicture}
       \begin{scope}[x={(0mm,0mm)},y={(120mm,\textwidth)}]
           \begin{axis}[%
                   height=40mm,
                   width=\textwidth,
                   at={(0,70mm)},
                   grid=both,
                   xlabel=Zeit (\si{\micro\second}),
                   ylabel=Strom (\si{\ampere}),
                   %x unit=u,
                   change x base=true,
                   %line width = 1pt,
                   thick,
                   x SI prefix=micro,
               ]
               \addplot[-,color=blue] table {data/module-72cells--I-MOSFET--0005u.dat};
           \end{axis}
           \begin{axis}[%
                   height=40mm,
                   width=\textwidth,
                   at={(0,35mm)},
                   grid=both,
                   xlabel=Zeit (\si{\micro\second}),
                   ylabel=Spannung (\si{\volt}),
                   %x unit=u,
                   change x base=true,
                   x SI prefix=micro,
               ]
               \addplot[-,color=magenta] table {data/module-72cells--U-MOSFET--0005u.dat};
           \end{axis}
           \begin{axis}[%
                   height=40mm,
                   width=\textwidth,
                   at={(0,0)},
                   grid=both,
                   xlabel=Zeit (\si{\micro\second}),
                   ylabel=Leistung (\si{\watt}),
                   %x unit=u,
                   change x base=true,
                   x SI prefix=micro,
               ]
               \addplot[-,color=teal] table {data/module-72cells--P-MOSFET--0005u.dat};
           \end{axis}
       \end{scope}
   \end{tikzpicture}
   \caption{%
       Verlauf    von    Strom,    Spannung    und    Leistung    am    MOSFET
       bei   einer   Schaltfrequenz   von   \SI{10}{\kilo\hertz}   bei   einer
       Modulkonfiguration    von    $36     \times    2$    Zellen    gem\"ass
       Schema    in    \fref{fig:ltspice:module:cellBased:36x2}   auf    Seite
       \ref{fig:ltspice:module:cellBased:36x2}.\protect\\
       Anhang       \ref{app:sec:simu:complementary:36x2}      auf       Seite
       \pageref{app:sec:simu:complementary:36x2} beinhaltet zum Vergleich noch
       Simulationen f\"ur den Zeitraum von einer Millisekunde.%
   }
   \label{fig:simu:results:36x2:3u}
\end{figure}

\begin{figure}
    \begin{tikzpicture}
       \begin{scope}[x={(0mm,0mm)},y={(120mm,\textwidth)}]
           \begin{axis}[%
                   height=40mm,
                   width=\textwidth,
                   at={(0,70mm)},
                   grid=both,
                   xlabel=Zeit (\si{\micro\second}),
                   ylabel=Strom (\si{\ampere}),
                   %x unit=u,
                   change x base=true,
                   x SI prefix=micro,
               ]
               \addplot[-,color=blue] table {data/module-72cells-series--I-MOSFET--0003u.dat};
           \end{axis}
           \begin{axis}[%
                   height=40mm,
                   width=\textwidth,
                   at={(0,35mm)},
                   grid=both,
                   xlabel=Zeit (\si{\micro\second}),
                   ylabel=Spannung (\si{\volt}),
                   %x unit=u,
                   change x base=true,
                   x SI prefix=micro,
               ]
               \addplot[-,color=magenta] table {data/module-72cells-series--U-MOSFET--0003u.dat};
           \end{axis}
           \begin{axis}[%
                   height=40mm,
                   width=\textwidth,
                   at={(0,0)},
                   grid=both,
                   xlabel=Zeit (\si{\micro\second}),
                   ylabel=Leistung (\si{\watt}),
                   %x unit=u,
                   change x base=true,
                   x SI prefix=micro,
               ]
               \addplot[-,color=teal] table {data/module-72cells-series--P-MOSFET--0003u.dat};
           \end{axis}
       \end{scope}
   \end{tikzpicture}
   \caption{%
       Verlauf    von    Strom,    Spannung    und    Leistung    am    MOSFET
       bei   einer   Schaltfrequenz   von   \SI{10}{\kilo\hertz}   bei   einer
       Modulkonfiguration    von    $72     \times    1$    Zellen    gem\"ass
       Schema    in    \fref{fig:ltspice:module:cellBased:36x2}   auf    Seite
       \ref{fig:ltspice:module:cellBased:72x1}. \protect\\
       Anhang     \ref{app:sec:simu:complementary:72x1}     auf     Seite
       \pageref{app:sec:simu:complementary:72x1} beinhaltet zum Vergleich noch
       Simulationen f\"ur den Zeitraum von einer Millisekunde.%
   }
   \label{fig:simu:results:72x1:3u}
\end{figure}

Die    Resultate   der    Simulation   f\"ur    ein   $36    \times   2$-Modul
sind   in   \fref{fig:simu:results:36x2:3u}   f\"ur  einen   Zeitbereich   von
\SI{3}{\micro\second}  dargestellt,  die  Ergebnisse   f\"ur  das  $72  \times
1$-Modul in \fref{fig:simu:results:72x1:3u}.  Tabelle \ref{tab:36x2:72x1:heat}
fasst   die  wichtigsten   Eckdaten  der   Simulationen  zusammen,   inklusive
Durchschnittswerte f\"ur Strom und Leistung.

\begin{table}[h!tb]
    \centering
    \caption{%
        Eckdaten zur Simulation des Kurzschlussverfahrens f\"ur ein Solarpanel
        mit  $36   \times  2$  Zellen  und   ein  Panel  mit  $72   \times  1$
        Zellen.\protect\\
        Bei     den     Durchschnittswerten     sind     sowohl     Ergebnisse
        f\"ur      \SI{1}{\milli\second}     wie      auch     \SI{1}{\second}
        angegeben,    um   zu    zeigen,   dass    sich   die    Konfiguration
        bereits    bei    einer    Millisekunde   stabilisiert    hat    (auch
        zu     sehen    in     Anhang    \ref{app:sec:simu:complementary:36x2}
        auf      Seite      \pageref{app:sec:simu:complementary:36x2}      und
        Anhang      \ref{app:sec:simu:complementary:72x1}       auf      Seite
        \pageref{app:sec:simu:complementary:72x1}).\protect\\
        Die     Dauer      der     Spitze     ist     beim      $36     \times
        2$-Modul     \SI{2}{\micro\second}    und     \SI{1.78}{\micro\second}
        beim    $72    \times    1$-Modul    (siehe    unterste    Plots    in
        den     Abbildungen      \ref{fig:simu:results:36x2:3u}     respektive
        \ref{fig:simu:results:72x1:3u}).%
    }
    \label{tab:36x2:72x1:heat}
    \begin{tabular}{lrr}

    \toprule
    Kriterium                                      & $36 \times 2$-Modul & $72 \times 1$-Modul \\
    \midrule
     Spitzenwert Strom:                            & \SI{32}{\ampere}    & \SI{24}{\ampere}    \\
     Spitzenwert Leistung:                         & \SI{310}{\watt}     & \SI{535}{\watt}     \\
     $\overline{P}$ f\"ur Dauer der Spitze         & \SI{158.77}{\watt}  & \SI{267.46}{\watt}  \\
     $\overline{P}$ f\"ur \SI{1}{\milli\second}    & \SI{4.312}{\watt}   & \SI{6.0577}{\watt}  \\
     $\overline{P}$ f\"ur \SI{1}{\second}          & \SI{4.3047}{\watt}  & \SI{6.0433}{\watt}  \\
    \bottomrule
    \end{tabular}
\end{table}

Aus \fref{fig:simu:results:36x2:5u}  und \tref{tab:36x2:72x1:heat}  ziehen wir
folgende Schlussfolgerungen:

\begin{enumerate}
    \firmlist
    \item
        Der  Spitzenwert f\"ur  den  Strom ist  hoch,  aber Transistoren,  die
        solche Str\"ome  verkraften k\"onnen,  sind zu  vern\"unftigen Preisen
        erh\"altlich.
    \item
        Die    Spitzenwerte    der    Leistungen   zwar    nur    sehr    kurz
        (etwa   \SI{2}{\micro\second}),   aber   sehr  hoch. Auch   wenn   der
        Durchschnittliche Leistungswert  weit unter  der Grenze  von g\"unstig
        erh\"altlichen  MOSFETs   liegt,  k\"onnte  die   Leistungsspitze  den
        Transistor irreversibel sch\"adigen.
    \item
        Der  durchschnittliche  Leistungsverbrauch  ist  weit  \"uber  dem  im
        Pflichtenheft angestrebten Wert von \SI{100}{\milli\watt}.
\end{enumerate}
\todo{korrekte Schlussfolgerungen?}


% ---------------------------------------------------------------------------- %
\clearpage
\subsection{\Master (Empf\"anger)}
\label{subsec:simu:ask:recv}
% ---------------------------------------------------------------------------- %


% ---------------------------------------------------------------------------- %
\subsection{Gesamtsystem}
\label{subsec:simu:ask:total}
% ---------------------------------------------------------------------------- %
