% **************************************************************************** %
\chapter{Firmware}
\label{chap:firmware}
% **************************************************************************** %

An  dieser  Stelle wird  genauer  erl\"autert,  welche \"Uberlegungen  in  die
Firmware  unseres  Systems  geflossen  sind. Weil  Sensor  und  Master-Ger\"at
jeweils  eine   eigene  Firmware   ben\"otigen,  ist  auch   dieser  Abschnitt
entsprechend aufgeteilt.

\anweisung   Diagramme    zur   Erkl\"arung    des   Aufbaus    der   Firmware
verwenden. Trockener    Text    \"uber    Software   ist    nicht    besonders
leserfreundlich.

\anweisung Bei der Verwendung von  Libraries sollen diese kurz beschrieben und
der Grund ihrer Wahl erl\"autert werden.

\anweisung   Unterkapitel  entsprechend   Gliederung  der   Firmware  sinnvoll
erg\"anzen.

% ---------------------------------------------------------------------------- %
\section{Sensorplatine}
\label{sec:fw:sensorplatine}
% ---------------------------------------------------------------------------- %

Sinnvolle Beschreibung der Firmware des Sensors.


% ---------------------------------------------------------------------------- %
\section{Master-Ger\"at}
\label{sec:fw:mastergerat}
% ---------------------------------------------------------------------------- %

Unterbau, GSM-Modul, Benutzeroberfl\"ache. Wichtigster Punkt: Datenauswertung.


% ---------------------------------------------------------------------------- %
\section{Kommunikation/Protokoll}
\label{sec:fw:sensorplatine}
% ---------------------------------------------------------------------------- %

Wie im Abschnitt  zur Hardware wird in einem  separaten Abschnitt beschrieben,
welche \"Uberlegungen in die Entwicklung  des Protokolls eingeflossen sind und
wie das Ergebnis aussieht.

%\KOMAoptions{paper=A4,pagesize}
%\recalctypearea

{\begin{a3pages}
    \lipsum
    \lipsum
    \lipsum
    \lipsum
    \lipsum
    \begin{minipage}{.5\textwidth}
        \lipsum
    \end{minipage}
    \begin{minipage}{.5\textwidth}
            %\pgfplotsset{
            %    tick label style={font=\normalsize},
            %    label style={font=\normalsize},
            %    legend style={font=\footnotesize}
            %}
            \centering
            \begin{tikzpicture}
                \begin{scope}[x={(0mm,0mm)},y={(170mm,\textwidth)}]
                    \begin{axis}[%
                            height=80mm,
                            width=\textwidth,
                            at={(0,80mm)},
                            %grid=both,
                        ]
                        \addplot[-,color=blue] table {data/fetsim.dat};
                    \end{axis}
                    \begin{axis}[%
                            height=80mm,
                            width=\textwidth,
                            at={(0,0mm)},
                            %grid=both,
                        ]
                        \addplot[-,color=red] table {data/fetsim.dat};
                    \end{axis}
                \end{scope}
            \end{tikzpicture}
        %\resizebox{0.5\textwidth}{!}{\begin{tikzpicture}[scale=1.5]
        %    \begin{axis}
        %        \addplot[-,color=blue] table {data/fetsim.dat};
        %    \end{axis}
        %\end{tikzpicture}}
        \captionof{figure}{test}
        \label{fig:test}
    \end{minipage}
\end{a3pages}}
%{
%    \clearpage
%    \setlength{\pdfpagewidth}{2\pdfpagewidth}
%    \setlength{\hsize}{\pdfpagewidth-\spinemargin-\foremargin} % for text paragraphs
%    \setlength{\textwidth}{\hsize}                             % headers, footers
%    \setlength{\stockwidth}{2\stockwidth}
%    \setlength{\paperwidth}{2\paperwidth}
%    \checkandfixthelayout
%
%    \lipsum
%    \lipsum
%    \lipsum
%    \clearpage
%}
\lipsum
