% **************************************************************************** %
\chapter{Software}
\label{chap:software}
% **************************************************************************** %

Die  Software   unseres  Systems   gliedert  sich   analog  zur   Hardware  in
zwei  prim\"are  Teile: Die   Firmware  des  Sensors  und   die  Software  des
Masters.

In  Abschnitt   \ref{sec:fw:dataFlow}  wird   im  Folgenden   kurz  dargelegt,
wie  diese  beiden  Teilsysteme  konzeptuell  verkn\"upft  sind. Anschliessend
erkl\"art Abschnitt \ref{sec:firmware:sensor}  die Funktionsweise der Firmware
des  Sensors. Die   Software  des   \Raspi  ist  abschliessend   in  Abschnitt
\ref{sec:software:master} dokumentiert.


{\begin{a3pages}
\setlength{\parindentbak}{\parindent}
    \noindent\adjustbox{valign=t}{\begin{minipage}{115mm}
    % ---------------------------------------------------------------------------- %
    \section{Datenfluss}
    \label{sec:fw:dataFlow}
    % ---------------------------------------------------------------------------- %
    Abbilung   \ref{fig:dataFlow}  zeigt   den  Ablauf   der  Spannungs-   und
    Strommessung von den Messsonden bis zur Speicherung der Messergebnisse und
    der zugeh\"origen Metadaten in der Datenbank.

    \setlength{\parindent}{\parindentbak} % restore paragraph indentation
    Die  Spannung  wird in  jedem  PV-Modul  von  einem Sensor  mittels  eines
    analog/digital-Konverters   gemessen. Anschliessend  wird   ein  laufender
    Durchschnittswert  (siehe  dazu  auch Abschnitt  \ref{subs:Sensor},  Seite
    \pageref{subs:Sensor})  zusammen mit  der Serienummer  des Microchips  und
    einer  Checksumme in  ein  Datenpaket gepackt  und  \"uber die  DC-Leitung
    verschickt.

    Die    ON/OFF-Shift   Keying-Schaltung    benutzt    als   Referenz    den
    Clock   eines   separaten,  spannungsgesteuerten   Oszillators   (englisch
    \emph{voltage-controlled oscillator}, VCO).

    Im  \Master~  wird das  Datenpaket  entpackt  und auf  seine  Integrit\"at
    gepr\"uft.   Passt die  Pr\"ufsumme nicht  zu  den Daten,  wird das  Paket
    verworfen. Sind  die Daten  intakt  (bzw. wird  keine Diskrepanz  zwischen
    Pr\"ufsumme und  Daten festgestellt),  werden Spannung und  Serienummer im
    zugeh\"origen Table der Datenbank gespeichert. Zur chronologischen Ordnung
    der Daten wird  noch ein Timestamp abgelegt; damit man  weiss, aus welchem
    Strang das Paket gekommen ist, wird  die Strang-Nummer noch in den Eintrag
    eingef\"ugt.

    Die  Messung der  Strang-Str\"ome  erfolgt direkt  vom  \Master~ aus;  die
    zugeh\"origen Messwerte werden in einem separaten Table abgelegt.

    Zur  Verwaltung  der  installierten  Sensoren  und  PV-Module  wird  f\"ur
    jeden  Strang  ein  Table  gef\"uhrt,  in  dem  die  Serienummern  der  in
    diesem Strang  installierten Sensoren (bzw. deren  Microchips) gespeichert
    sind. Die  Eintr\"age  in  diesen  Tables erfolgen  automatisch,  wenn  in
    einem ankommenden  Datenpaket eine bisher noch  nicht bekannte Serienummer
    detektiert wird.

    \vspace*{40mm}

    \hfill\adjustbox{valign=t}{\begin{minipage}{50mm}
        \figcaption
        [Datenfluss von Messungen bis Speicherung]
        {%
            Datenfluss von den Messungen der Modulspannungen und Strang-Str\"ome bis
            zur Speicherung in der Datenbank%
        }
        \label{fig:dataFlow}
    \end{minipage}}
\end{minipage}}
\hspace*{15mm}
\adjustbox{valign=t}{\begin{minipage}{215mm}
    \begin{tikzpicture}[%
        align=center,
        text=solarized-base02,
        draw=solarized-base02,
    ]
    \small
    \sffamily
    %\begin{scope}[x={(0mm,0.95\textwidth)},y={(0mm,50mm)}]

    %\draw [help lines] (0,0) grid (10,-15);

    \begin{scope}[
        every node/.style = draw,
        terminal/.append style={
            rounded rectangle,
            fill=solarized-violet,
            text=solarized-base3,
            inner sep=2mm,
        }, % data packages
        sign/.style={
            inner sep=2mm,
            rounded corners=1mm,
            fill=solarized-magenta,
            text=solarized-base3,
        },         % custom signal style
        circ/.style={
            inner sep=2mm,
            rounded corners=1mm,
            double,
            fill=solarized-base02,
            draw=solarized-base02,
            text=solarized-base2,
        }, % circuitry
        proc/.style={
            inner sep=2mm,
            rounded corners=1mm,
            fill=solarized-cyan,
            text=solarized-base3,
        },       % process/activity
        stor/.style={
            fill=cyan!30
        },         % storage
        dbtable/.style={
            text=solarized-base3,
            draw=solarized-base3,
            rounded corners=1mm,
            inner sep=2mm,
        } % database tables
    ]
        \node (voltage) [
            sign,
            signal,
            signal to=east
        ] at (0,0) {Spannung};

        % uC on Sensor
        \node (ADCVolt) [
            circ,
            right=30mm of voltage
        ] {ADC};
        \node (avg) [
            proc,
            right=of ADCVolt,
            align=center
        ] {Durchschnitt\\berechnen};
        \node (data) [
            terminal,
            right=of avg,
            align=center
        ] {Messdaten\\Modulspannung,\\Sensor-ID};
        \node (id) [
            sign,
            signal,
            signal to=east,
            above left=of data
        ] {Seriennummer};
        \node (crc) [
            proc,
            below right=of data
        ] {CRC berechnen};
        \node (package) [
            terminal,
            below=of data
        ] {Datenpaket};
        \node (uart) [
            circ,
            left=of package
        ] {UART-Modul};

        \node (oosk1)[
            circ,
            below=of uart,
            align=center
        ] {ON/OFF-Shift \\Keying-Schaltung};
        \node (clk1) [
            sign,
            signal,
            signal to=east,
            left=2cm of oosk1
        ] {CLK VCO};
        \node (oosk2) [
            circ,
            below=20mm of oosk1,
            align=center
        ] {ON/OFF-Shift \\Keying-Schaltung};
        \node (clk2) [
            sign,
            signal,
            signal to=east,
            left=2cm of oosk2
        ] {CLK VCO};
        \node (bridge) [
            circ,
            right=of oosk2
        ] {UART/I\textsuperscript{2}C-Br\"ucke};
        \node (package2) [
            terminal,
            below=of bridge
        ] {Datenpaket};

        % Raspi
        \node (crc2) [
            proc,
            align=center,
            diamond,
            below=of package2
        ] {CRC\\korrekt?};
        \node (discard) [
            draw=solarized-red,
            cross out,
            right=of crc2
        ] {Paket verwerfen};
        \node (data2) [
            terminal,
            left=of crc2,
            align=center
        ] {Messdaten\\Modulspannung,\\Sensor-ID};
        \node (dbVolts) [
            dbtable,
            left=of data2,
            text width=7em,
            align=left,
            minimum width=7.5em
        ] {Table:\\ Spannungen\\};
        \node (dbCurr) [
            dbtable,
            below=1mm of dbVolts,
            text width=7em,
            align=left,
            minimum width=7.5em
        ] {Table:\\ Strang-Str\"ome\\};
        \node (dbIDs) [
            dbtable,
            below=1mm of dbCurr,
            text width=7em,
            align=left,
            minimum width=7.5em
        ] {Tables:\\ Sensor-IDs \\ pro Strang};

        \node (newID) [
            proc,
            align=center,
            diamond,
            below=of crc2] {Sensor-ID\\bekannt?};
        \node (sensorID)  [
            terminal,
            left=of newID] {Sensor-ID};
        \node (stringNr2) [
            sign,
            signal,
            signal to=west,
            below=5mm of sensorID
        ] {String-Nr.};

        \node (stringNr) [
            sign,
            signal,
            signal to=east,
            left=15mm of dbVolts
        ] {String-Nr.};
        \node (timestamp) [
            sign,
            signal,
            signal to=east,
            above=1mm of stringNr
        ] {Timestamp};

        \node (dataCurr) [
            terminal,
            below=35mm of stringNr,
            align=center
        ] {Messdaten\\Strom};

        \node (bridge2) [
            circ,
            below=55mm of stringNr
        ] {UART/I\textsuperscript{2}C-Br\"ucke};
        \node (hall) [
            circ,
            right=of bridge2
        ] {Hall-Sensoren};
        \node (current) [
            sign,
            signal,
            signal to=west,
            right=of hall
        ] {Strang-Strom};
    \end{scope}

    \begin{scope}
        % Need to have this inside  a separate scope so that
        % its border does not get drawn.
        \node (db)        [
            text=solarized-base3,
            above=1mm of dbVolts,
            align=left
        ] {Datenbank};
        \node (dbSignals) [
            circle,
            inner sep=1pt,
            fill=black,
            right=3mm of stringNr
        ] { };
        \node (dbSignals2) [
            circle,
            inner sep=1pt,
            fill=black,
            left=4mm of dbCurr] { };
        \node (dbSignals3) [
            circle,
            inner sep=1pt,
            fill=black,
            left=of sensorID
        ] { };
    \end{scope}

    \begin{scope}[on background layer]
        \node (SENSOR) [%
            draw,
            double,
            rounded corners=5mm,
            fill=solarized-base3,
            inner sep=2em,
            fit=(ADCVolt) (avg) (data) (id) (crc) (package) (uart) (oosk1) (clk1),
            align=left,
            text height=21em,
            align=right,
        ] {\large{\textbf{Sensor}}};
        \node (uCSensor) [%
            draw,
            double,
            rounded corners=5mm,
            fill=solarized-base2,
            inner sep=1em,
            fit=(ADCVolt) (avg) (data) (id) (crc) (package) (uart),
            align=right,
            text height=1em,
        ] {\textbf{Atmel SAM D09}};
        \node (rasPi) [%
            draw,
            double,
            rounded corners=5mm,
            fill=solarized-base2,
            inner sep=1em,
            fit=(stringNr) (timestamp) (crc2) (newID) (discard) (stringNr2),
            align=right,
            text height=1em,
        ] {\textbf{Raspberry Pi}};
        \node (database) [%
            draw=solarized-base03,
            rounded corners=3mm,
            inner sep=2mm,
            fill=solarized-blue,
            fit=(dbVolts) (dbCurr) (dbIDs) (db),
        ] { };
    \end{scope}


    \begin{scope}[
        %every node/.style={ },
            rounded corners,
            every path/.append style={draw=solarized-base02,},
    ]
        \draw[-latex] (voltage) -- (ADCVolt);
        \draw[-latex] (ADCVolt) -- (avg);
        \draw[-latex] (id) -| (data);
        \draw[-latex] (avg) -- (data);
        \draw[-latex] (data) -| (crc);
        \draw[-latex] (crc) -- (package);
        \draw[-latex] (data) -- (package);
        \draw[-latex] (package) -- (uart);
        \draw[-latex] (uart) -- (oosk1);
        \draw[-latex] (clk1) -- (oosk1);
        \draw[-latex] (clk2) -- (oosk2);
        \draw[-latex] (oosk1) --  node[midway, anchor = west] {DC-Leitung} (oosk2);
        \draw[-latex] (oosk2) -- (bridge);
        \draw[-latex] (bridge) -- (package2);
        \draw[-latex] (package2) -- (crc2);
        \draw[-latex] (crc2) -- node[anchor=north] {nein} (discard);
        \draw[-latex] (crc2) -- node[anchor=north] {ja} (data2);
        \draw[-latex] (data2) -- (dbVolts);
        \draw[-] (stringNr) -- (dbSignals);
        \draw[-] (timestamp) -| (dbSignals);
        \draw[-latex] (dbSignals) -- (dbVolts);
        \draw[-latex] (dbSignals) -| (dbSignals2);
        \draw[-latex] (dbSignals2) -- (dbCurr);
        \draw[-latex] (crc2) -- node[anchor=west,align=right] {ja} (newID);
        \draw[-latex] (newID) -- node[anchor=north] {nein} (sensorID);
        \draw[-latex] (sensorID) -| (dbIDs);
        \draw[-latex] (stringNr2) -| (dbSignals3);
        \draw[-latex] (current) -- (hall);
        \draw[-latex] (hall) -- (bridge2);
        \draw[-latex] (bridge2) -- (dataCurr);
        \draw[-latex] (dataCurr) |- (dbSignals2);
    \end{scope}

\end{tikzpicture}

\end{minipage}}
\end{a3pages}}


% ---------------------------------------------------------------------------- %
\section{Firmware \Sensor}
\label{sec:firmware:sensor}
% ---------------------------------------------------------------------------- %
\subsection{SAMD09 \"Ubersicht}

Als CPU auf dem Sensor agiert wie im Abschnitt~\ref{sec:hw:sensorplatine} erw\"ahnt ein Atmel Smart ARM Cortex M0+ gew\"ahlt.
Die Firmware ist in C geschrieben.

\subsection{Benutze Bibliotheken}

Es wird komplett freie Software verwendet, weswegen zum bauen der Binaries GNU Makefiles benutzt werden.

Die Firmware ist auf dem Atmel Software Framework (ASF) aufgebaut. Das ASF f\"uhrt einen Hardware Abstraction Layer (HAL) ein, welcher die Hardware-Bl\"ocke der einzelnen Atmel CPUs in einfache Interfaces in Form von C-Funtionen abstrahiert.
Um die g\"angigen ARM Schnittstellen zu nutzen, wird CMSIS im ASF verwendet.
Das ASF darf f\"ur Atmel Chips ohne weiteres verwendet werden solange die Copyright Bemerkungen nicht entfernt werden.

\subsection{Die Firmware}

Die Firmware besteht im Kern aus einer simplen main() Funktion welche in einer Endlosschleife l\"auft.

\subsubsection{UART}
\label{subs:UART}

Die Endlosschleife \"uberpr\"uft zuerst ob Daten \"uber die UART empfangen wurden, sprich Anweisungen vom Master oder Antworten von anderen Sensoren. Ist das der Fall, so wird darauf reagiert. Dies heisst dass die UART die gemittelten Daten der letzten x Samples an den Master adressiert und verschickt.
Hierzu wird ein Datenpakt wie es in Abbildung \ref{fig:sensor:firmware:datenpaket} zu sehen ist verschickt:

\begin{figure*}[ht!]
  \centering
  \begin{bytefield}{32}
    \bitheader{0,7,8,15,16,23,24,31} \\
    \begin{rightwordgroup}{Adresse}
      \bitbox{32}{Adressbyte 0} \\
      \bitbox{32}{Adressbyte 1} \\
      \bitbox{32}{Adressbyte 2} \\
      \bitbox{32}{Adressbyte 3} \\
    \end{rightwordgroup} \\
    \bitbox{16}{Spannung}\bitbox{16}{Nicht verwendet}\\
    \bitbox{32}{CRC}\\
  \end{bytefield}
  \caption{\label{fig:sensor:firmware:datenpaket}Sample packet}
\end{figure*}

Damit der Empf\"anger eines Paketes erkennen kann dass es keine Technische St\"orung beim \"ubertragen gab und er falsche Daten liest, wird eine Pr\"ufsumme (CRC) der zu verschickenden Daten mitgeschickt.
Um die CRC zu berechnen wird zuerst das Datenpaket zusammengestellt. Davon wird dann mithilfe des ASF eine Pr\"ufsumme erstellt und an das bestehende Datenpaket angeh\"angt. Dann wird das Datenpaket per UART verschickt.

\subsubsection{Sensor}
\label{subs:Sensor}

Der Spannungssensor selbst wird mittels Analog Digital Konverters (ADC) verwirklicht.
Daf\"ur wird mithilfe des ASF ein 12 Bit ADC ausgelesen und von diesen Werten ein Moving Average erstellt.
Deshalb kommt an zweiter Stelle in der Endlosschleife ein Lesevorgang auf dem ADC.
Dann wird mithilfe des sogenannten Cascaded Integrator–Comb Filters
%(TODO: ref to https://en.wikipedia.org/wiki/Cascaded_integrator%E2%80%93comb_filter)
der aktuelle Moving Average berechnet. Die Formel~\ref{eq:hogenauer} zeigt gut wie dies berechnet wird.

\begin{equation}\label{eq:hogenauer}
  \begin{align}
    y[n] &= \sum_{k=0}^{RM-1} x[n-k] \\
         &= y[n-1] + x[n] - x[n-RM].
  \end{align}
\end{equation}

Dieser Moving Average wird nun per UART verschickt.

\subsubsection{Statusanzeige}
\label{subs:Statusanzeige}

Damit man gut sieht ob der Sensor noch korrekt leuchtet kommt an letzter Stelle der Endlosschleife noch ein simples Toggeln der LED, so dass man sieht dass etwas nicht stimmt falls es einmal nicht mehr toggeln sollte.

\subsection{Open On Chip Debugger}

Zum Programmieren der CPUs haben wir OpenOCD gew\"ahlt. Auch hier ist wieder anzumerken, dass es ein St\"uck freie Software ist.
OpenOCD extrem leicht zu erweitern und unterst\"utzt eine grosse Breite an Programmierschnittstellen wie den STLinkv2 oder den Segger J-Link. Es werden Protokolle wie JTag und SWD ohne weiteres unterst\"utzt.
Ebenfalls gibt es einen Reichtum an Chips die unterst\"utzt sind.

Da der SAMD09 erst im ersten Quartal 2016 auf den Markt kam, waren sehr wenige Projekte vorhanden, welche ihn bereits benutzt haben. Dies hatte zur Ursache, dass OpenOCD den Chip nicht unterst\"utzte. Also haben wir einen Treiber daf\"ur geschrieben beziehungsweise den bestehenden f\"ur Atmel Chips erweitert.

Um die Kosten Tief zu halten haben wir einen StLinkv2 als Programmierschnittstelle verwendet. Auch daf\"ur brauchte es einen Patch des Treibers, da der STLinkv2 nur 32 Bit Schreibbefehle unterstutzt, der SAMD09 aber 16 Bit Schreibbefehle erwartet. Wir haben einen unsch\"onen Patch der nun 32 Bit statt 16 Bit Schreibbefehle schickt geschrieben um dieses Problem zu l\"osen.



% ---------------------------------------------------------------------------- %
\clearpage
\section{Software \Master}
\label{sec:software:master}
% ---------------------------------------------------------------------------- %

Es wird  im Folgenden auf  die Software des \Master~  genauer eingegangen. Die
verwendeten  Komponenten  werden beschrieben  und  es  wird auf  die  Lizenzen
ebendieser  Komponenten eingegangen. Anschliessend  werden der  Aufbau unseres
Software-Stacks und die Funktionsprinzipien dokumentiert.


% ---------------------------------------------------------------------------- %
\subsection{Komponenten}
\label{subsec:software:master:components}
% ---------------------------------------------------------------------------- %

Als Betriebssystem kommt Raspbian zum  Einsatz. Raspbian ist eine Variante von
Debian-Linux mit  einigen Erweiterungen,  welche das  System auf  einem \Raspi
lauff\"ahig machen. Als  graphische Oberfl\"ache wird LXDE  benutzt, aufbauend
auf  X11. Die wichtigsten  Komponenten  des gesamten  Software-Stackn sind  in
Abbildung \ref{fig:softwarestack} dargestellt.

Die  Funktionalit\"at  unserer  Software  wird  mit  einigen  Python-Libraries
implementiert. PyQt  wird  benutzt,  um  die  graphische  Benutzeroberfl\"ache
zu   programmieren,  SQLAlchemy   ist  daf\"ur   dient  als   Datenbanktreiber
verantwortlich  und \emph{WiringPi  for  Python}  ist daf\"ur  verantwortlich,
die  Hardware-Schnittstellen  des  \Raspi~   zu  abstrahieren  und  in  Python
bereitzustellen. Tabelle  \ref{tab:pythonLibs} listet  die Libraries  und ihre
Aufgaben in einer \"Ubersicht auf.

\begin{figure}[h!tb]
    \centering
    \begin{bytefield}{32}
        %\begin{rightwordgroup}{Adresse}
        \colorbitbox{solarized-base2}  {solarized-base02}{32}{Unsere Software} \\
        \colorbitbox{solarized-blue}   {solarized-base3}  {8}{PyQt}
        \colorbitbox{solarized-blue}   {solarized-base3}  {8}{SQLAlchemy}
        \colorbitbox{solarized-blue}   {solarized-base3}  {8}{WiringPi}
        \colorbitbox{solarized-magenta}{solarized-base3}  {8}{LXDE} \\
        \colorbitbox{solarized-blue}   {solarized-base3} {24}{Python 3}
        \colorbitbox{solarized-magenta}{solarized-base3}  {8}{X11} \\
        \colorbitbox{solarized-base01} {solarized-base3} {32}{Linux} \\
        %\end{rightwordgroup} \\
  \end{bytefield}
  \caption{Software-Stack f\"ur unser Projekt}
  \label{fig:softwarestack}
\end{figure}

\begin{table}[h!tb]
    \centering
    \caption{Liste der verwendeten Python-Libraries}
    \label{tab:pythonLibs}
    \small
    \begin{tabular}{lrp{50mm}r}
        \toprule
        \textsc{Library} & \textsc{Version} & \textsc{Zweck} & \textsc{Website} \\
        \midrule
        PyQt & 5 & Erstellen und verwalten der graphischen Bedienelemente & \cite{ref:pyqt} \\
        [2mm]
        \rowcolor{solarized-base2}
        SQLAlchemy & 1.0 & Datenbankabstraktion                           & \cite{ref:sqlalchemy} \\
        [2mm]
        WiringPi for Python & 2 & Abstraktion der Hardware-Schnittstellen & \cite{ref:wiringpi} \\
        \bottomrule
    \end{tabular}
\end{table}

% ---------------------------------------------------------------------------- %
\clearpage
\subsection{Lizenzen}
\label{subsec:software:master:licenses}
% ---------------------------------------------------------------------------- %

Bei    der   Auswahl    von    Drittsoftware   wird    auf   die    jeweiligen
Lizenzbedingungen   geachtet,   um   keine   Konflikte   zu   verursachen. Die
wichtigsten  drei Lizenz-Bereiche  und ihre  Charakteristiken sind  in Tabelle
\ref{tab:lincenseAreas} aufgef\"uhrt.

\begin{table}[h!tb]
    \centering
    \caption{Lizenzbereiche}
    \label{tab:licenseAreas}
    \small
    \begin{tabular}{>{\raggedright}p{30mm}>{\raggedright}p{30mm}p{50mm}}
        \toprule
        \textsc{Bereich} &
        \textsc{Lizenz} &
        \textsc{Bedingungen} \\
        \midrule
        Linux-Kernel &
        GPL &
        Quellcode und \"Anderungen m\"ussen \"offentlich sein. \\
        [2mm]

        \rowcolor{solarized-base2}
        Treiber f\"ur Raspberry Pi und Display &
        Restricted &
        Quellcode wird vom Hersteller geheim gehalten \\
        [2mm]

        Raspbian &
        DFSG (Sammlung diverser Lizenzen) &
        Darf frei verwendet, aber nicht unbedingt verkauft, werden. \\
        \bottomrule
    \end{tabular}
\end{table}

Grundlage  f\"ur die  Software  bildet  das angepasste  Betriebssystem. Dieses
wird  von  der Raspberry  Pi  Foundation  frei  zur Verf\"ugung  gestellt  und
unterliegt den  Bedingungen den  DFSG (\emph{Debian Free  Software Guidelines}
\cite{ref:socialContract}). Die  darauf  aufbauenden Programmbibliotheken  zur
Abstraktion  von Betriebsystemfunktionen  und weiteren  Hardware-Aufrufen sind
alle aus den Raspbian-Repositories verf\"ugbar und unterliegen daher ebenfalls
den  DFSG.   Da  die  Mastersoftware zwar  auf  diesen  Komponenten  aufsetzt,
sie  aber  nicht ver\"andert  oder  statisch  verlinkt wird,  entstehen  keine
Lizenzkonflikte. Zu beachten ist hier, dass diese Drittsoftware im Allgemeinen
nicht als Eigenwerk  verkauft werden darf. Das heisst, dass  sie zwar beliebig
verbreitet werden darf, nicht aber zum Produkt hinzugez\"ahlt werden kann.

Die DFSG stellen  insbesondere folgende Anforderung an  alle Programme, welche
Teil von Raspbian sind:

\begin{itemize}
    \tightlist
\item
    Die Software darf frei verbreitet werden (Regel 1)
\item
    Die Software darf f\"ur beliebige Zwecke eingesetzt werden (Regel 6)
\item
    Die Software beschr\"ankt unzusammenh\"angende Software nicht (Regel 9)
\end{itemize}

Der eigentliche  Mastersoftware-Quellcode dagegen  ist nicht  \"offentlich und
kann als Bestandteil des Produkts verkauft werden.



% ---------------------------------------------------------------------------- %
\subsection{Threads}
\label{subsec:software:master:threads}
% ---------------------------------------------------------------------------- %


Die  Mastersoftware  ist in  mehrere  Threads  gegliedert, unter  welchen  die
Funktionen aufgeteild  sind. Sie werden alle von  einem Hauptthread gestartet,
welcher die Koordination mittels Semaphoren \"ubernimmt. Dazu initialisiert er
alle  Resourcen,  auf  welche  aus  mehreren  Threads  zugegriffen  sind. Dies
sind  Datenbank und  Logging-System, welche  beide Multithreading  beherrschen
und  threadsafe  sind. Die zentrale  Koordination  bedeutet  zudem, dass  alle
Arbeitsschritte nur sofern n\"otig und nicht mit veralteten Daten ausgef\"uhrt
werden.

Um  die Threadsicherheit  zu  gew\"ahrleisten verwenden  die beiden  geteilten
Resourcen spezielle Mechanismen:
\begin{itemize}
    \tightlist
    \item
        Die   Loggingfunktion   setzt  eine   Queue   eim   um  Meldungen   zu
        zwischenzuspeichern und asynchron hintereinander abzuarbeiten.
     \item
        SQLAlchemy  beinhaltet   mit  der  "ScopedSession"  eine   Methode  um
        aus  mehreren  Threads  parallel  aufgerufen  zu  werden. Dazu  werden
        die  Anfragen  aus jeweils  einem  Thread  gruppiert und  dann  atomar
        ausgef\"uhrt.
 \end{itemize}

Alle anderen  externen Ressourcen, dies  sind \ISC-bus, UART und  GPIO, werden
jeweils  von  nur einem  Thread  genutzt,  wodurch keine  Konflikte  auftreten
k\"onnen.

Im Detail sieht die Aufgabenteilung folgendermassen aus:

\begin{itemize}
    \tightlist
    \item
        \textbf{Prozesssteuerung:} Der  Hauptthread  verwaltet  Resourcen  und
        alle weiteren Threads.
    \item
        \textbf{Datensammlung:} Ein  Thread  empf\"angt   die  Spannungs-  und
        Strommesswerte von den Sensoren und speichert sie in der Datenbank ab.
    \item
        \textbf{Datenauswertung:} Ein   separater    Thread   untersucht   die
        gespeicherten  Messwerte   auf  defekte   Panels  und   speichert  die
        Ergebnisse ebenfalls in der Datenbank ab.
    \item
        \textbf{Graphisches  Benutzerinterface:} Der   GUI-Thread  stellt  ein
        Fenster  dar,   mit  welchem   der  Benutzer  die   Einstellungen  von
        Alarmierung  und   Telefonnummer  konfigurieren  kann   und  speichert
        \"Anderungen in der Datenbank.
    \item
        \textbf{Ausgabe:} Die  Umsetzung  der definierten  Massnahmen  obliegt
        einem Thread, welcher das Modem verwaltet und bei Bedarf die digitalen
        Ausg\"ange zur Steuerung der Relais bet\"atigt
\end{itemize}


% ---------------------------------------------------------------------------- %
\subsection{Benutzeroberfl\"ache}
\label{subsec:software:master:GUI}
% ---------------------------------------------------------------------------- %

\todo{unterliegender aufbau statt benutzerf\"uhrung?}
Die Benutzeroberfl\"ache wurde bewusst sehr schlicht gehalten. Sie besteht aus
folgenden  Ansichten: Hauptmen\"u,  Einstellungen, Eingabe  der  Telefonnummer
und   dem   Fehlerverlauf. Im   regul\"aren  Betrieb   ist   das   Hauptmen\"u
ersichtlich. Es  informiert  \"uber  den  Anlagestatus  sowie  den  gemessenen
Strom  der Str\"ange. Vom  Hauptmen\"u  aus kann  durch  dr\"ucken des  Feldes
\emph{Einstellungen} zu den Einstellungen gelangt werden. In den Einstellungen
k\"onnen die  Relaiskontakte 1  und 2  aktiviert oder  deaktiviert werden. Sie
dienen  zur Bet\"atigung  einer externen  Signalisation im  Fehlerfall. Weiter
kann   die  SMS   Benachrichtigung   ebenfalls   aktiviert  oder   deaktiviert
werden. Durch  bet\"atigen  des  Feldes  \emph{Speichern}  wird  die  aktuelle
Auswahl  gespeichert. Falls  die  SMS Benachrichtigung  gew\"ahlt  wird,  muss
eine  g\"ultige Telefonnummer  hinterlegt werden.   Die Ansicht  Eingabe einer
Telefonnummer  wird vom  Feld \emph{Telefonnr.  hinzuf\"ugen} aufgerufen. Nach
der  Eingabe  einer  g\"ultigen  Nummer,  wird  mittels  \emph{Speichern}  die
Telefonnummer  gespeichert und  es kann  mit dem  Feld \emph{Zur\"uck}  zu den
Einstellungen  zur\"uckgekehrt  werden. Diese  Ansichten sind  im  regul\"aren
Betrieb ersichtlich und k\"onnen nach Belieben eingestellt werden.

Folgend werden  die St\"orbetrieb Ansichten erl\"autert. Wird  ein Modulfehler
erkannt,  signalisiert   dies  die  Ansicht  \emph{Modulfeher}. Es   wird  die
Modulseriennummer und die Zeit mit Datum angezeigt. Damit ist es m\"oglich ein
fehlerhaftes Modul zu erkennen  und entsprechend zu handeln. Durch bet\"atigen
des  Feldes  \emph{OK} wird  der  externe  Alarm der  Relais  1  und 2  (falls
aktiviert)  quittiert  und das  Hauptmen\"u  wird  wieder angezeigt. Das  Feld
\emph{Fehlerverlauf}  f\"uhrt  zu  einer   \"Ubersicht  aller  bis  zu  diesem
Zeitpunkt aufgetretenen  Modulfehler. Jeweils wird  die Modulnummer,  die Zeit
und das Datum in einer Zeile angezeigt.


% ---------------------------------------------------------------------------- %
\subsection{Datenbank}
\label{subsec:software:master:database}
% ---------------------------------------------------------------------------- %

% ---------------------------------------------------------------------------- %
\subsection{Funktionen}
\label{subsec:software:master:functions}
% ---------------------------------------------------------------------------- %
