{\begin{a3pages}
% ---------------------------------------------------------------------------- %
\clearpage
\section{Sensorplatine}
\label{sec:hw:sensorplatine}
% ---------------------------------------------------------------------------- %
    \setlength{\parindentbak}{\parindent}

    \noindent\adjustbox{valign=t}{\begin{minipage}{135mm}
        Das Sensorboard besteht aus einer CPU welche alle Komponenten koordiniert, einem Modulator und einem Demodulator zur Kommunikation, einem Buck-Konverter der die Netzspannung f\"ur das Board transformiert.

        Als CPU dient ein Atmel SAM D09. Dieser wurde gew\"ahlt da er die g\"unstigste CPU in seiner Klasse ist und alle notwendigen Features mitbringt. (3)
        Der Mikrochip hat einen 12 Bit ADC, ein CRC Modul, hat eine 32 Bit ARM Architektur und braucht extrem wenig Leistung.

        Zum Speisen des Boardes wird ein LMR16006 verwendet. Dieser kann die volle Spanne von 12 Volt bis 60 Volt ohne Probleme auf 3.3 Volt regeln, welche das Board versorgen. (1)

        Als Modulator dient ein Voltage Controlled Oscillator (VCO) auf dem 74HC4640 Chip von Texas Instruments.
        Dieser kann mit der richtigen Beschaltung Frequenzen zwischen wenigen Kilohertz bis mehrere Megahertz erzeugen.(2)

        Zum Demodulieren wird ein einfaches Tiefpassfilter mit einer Diode und einem Verst\"arker benutzt.(6)

        Und nat\"urlich hat es einen Spannungsteiler zum Messen der Versorgungsspannung. (4)

        Im Folgenden sind die einzelnen Komponenten dokumentiert und ihre Wahl begr\"undet.

        \adjustbox{valign=t}{\begin{minipage}{0.475\textwidth}
            \centering
            \includegraphics[width=0.9\textwidth]{images/sensor-pcb/sensor-3d-front.png}
            \figcaption[Sensor: Vorderseite PCB]{PCB, Vorderseite}
            \label{fig:sensor:pcb:front}
        \end{minipage}}
        \adjustbox{valign=t}{\begin{minipage}{0.475\textwidth}
            \centering
            \includegraphics[width=0.9\textwidth]{images/sensor-pcb/sensor-3d-back.png}
            \figcaption[Sensor: R\"uckseite PCB]{PCB, R\"uckseite}
            \label{fig:sensor:pcb:back}
        \end{minipage}}
    \end{minipage}}
    \hspace*{15mm}
    \adjustbox{valign=t}{\begin{minipage}{195mm}
        \centering
        \includegraphics[width=\textwidth]{images/sensor-sch/sensor--sch--highlights.eps}%
        \figcaption[Schema Sensor, \"Ubersicht]{%
            Schema   des   \Sensor   s. Eine  Grossversion   ist   in   Anhang
            \label{app:chap:schemas}  zu  finden,   die  einzelnen  Baugruppen
            sind  in  den  folgenden  Abschnitten  beschrieben  und  gr\"osser
            abgebildet.%
        }
        \label{fig:sensor:schema:highlights}
    \end{minipage}}
\end{a3pages}}


% ---------------------------------------------------------------------------- %
\subsection{Speisung}
\label{subsec:hw:sensor:supply}
% ---------------------------------------------------------------------------- %

Die Speisung der Schaltung übernimmt ein LMR16006. Er ist bis 63 Volt input rated, das heisst es ist noch ein wenig Spiel zu den 60 Volt welche am Einganz anliegen können. Er kommt auch mit 4 Volt am Eingang noch klar. Da die Versorgungsspannung zwischen 12 Volt und 60 Volt schwankt durch den Tag, ist der LMR16006 soweit bestens geeignet.
Er hat einen sehr konstanten Wirkungsgrad von 70 bis 80 Prozent, je nach Speisespannung.
Der LMR16006 kann maximal 600 Miliampere liefern was ausreicht wenn man einen genug grossen Puffercap für Powerspikes einberechnet.
Zudem hat dieser Spannungsregler einen typischen Quiescent Current von nur 28 Mikroampere. Damit ist er auch extrem Stromsparend.
Die Beschaltung wurde nach Empehlung des Datenblatts gewählt, welche garantiert dass Powerspikes gut gefangen werden. $R_{FB1}$ und $R_{FB2}$ wurden explizit im Verhältnis $1:3.3$ gewählt, da 3.3 Volt die Spannung ist, die wir am Ausgang anstreben.

Der Active Low Shutdown Pin wird dauernd auf High gezogen, damit der Regler immer an ist sobald am Eingang Spannung anliegt.

Damit die Montage einfacher ist, ist ein Verpolungsschutz eingebaut. Dieser besteht aus einem P-Fet, einer Zenerdiode und einem Gate-Vorwiderstand.

\begin{figure}[h!t]
    \centering
    \includegraphics[width=1\textwidth]{images/sensor-sch/sensor--sch--supply.eps}
    \caption[Sensor: Schema Speisung]{Speisung Sensor}
\end{figure}

% ---------------------------------------------------------------------------- %
\subsection{Transmitter}
\label{subsec:hw:sensor:transmitter}
% ---------------------------------------------------------------------------- %

Der Transmitter besteht aus einem einfachen VCO der das Signal auf die Spannungsversorgung aufmoduliert.
Dafür wird am $VCO_{in}$ mithilfe eines Spannungsteilers eine fixe Spannung angelegt. Mit der korrekten Wahl von R7, R8 und C10 kann die Resonanazfrequenz für die Leitung erreicht werden, mit der dann moduliert wird. Diese Werte sind so gewählt wie es die Graphen im Datasheet zeigten. Es wurde eine Frequenz von 20 Megahertz ausgewählt da diese bei Leitungen unserer Länge optimal sind.
Das Signal wird dann moduliert indem der UART TX Pin direkt am INH Pin den VCO ein- und ausschaltet und somit eine OOSK zustande bringt. Hier ist wichtig dass der Pin Active High ist.

\begin{figure}[h!t]
    \centering
    \includegraphics[width=1\textwidth]{images/sensor-sch/sensor--sch--transmitter.eps}
    \caption[Sensor: Schema Transmitter]{Transmitter Sensor}
\end{figure}

\todo{R1, R21, RV1, R7, R8, C10, C15}


% ---------------------------------------------------------------------------- %
\subsection{Microcontroller}
\label{subsec:hw:sensor:mcu}
% ---------------------------------------------------------------------------- %

Der Mikrokontroller braucht bis auf einen Pullup am Active Low Reset und einen Stabilisierungskondensator an der Spannungsversorgung keine spezielle Beschaltung. Der Mikrochip ist so geroutet dass die beiden UARTs verfügbar sind. Eine zum Senden über die Leitung und eine zum Debuggen. Ebenfalls zu einem Stecker verbunden ist die Programmierschnittstelle (SWD).

\begin{figure}[h!t]
    \centering
    \includegraphics[width=0.5\textwidth]{images/sensor-sch/sensor--sch--mcu.eps}
    \caption[Sensor: Schema Microcontroller]{Microcontroller Sensor}
\end{figure}

\todo{R15, C14, Atmel Chip}

% ---------------------------------------------------------------------------- %
\subsection{Spannungsmessung}
\label{subsec:hw:sensor:voltageSense}
% ---------------------------------------------------------------------------- %

Der Spannungsteiler ist so eingestellt dass am ADC Pin 3.3 Volt anliegen wenn die Spannungsversorgung 60 Volt hat.
Ein Kondensator zur Glättung des Signales wurde ebenfalls hinzugefügt.

\begin{figure}[h!t]
    \centering
    \includegraphics[width=0.5\textwidth]{images/sensor-sch/sensor--sch--sensor.eps}
    \caption[Sensor: Schema Spannungsmessung]{Spannungsmessung Sensor}
\end{figure}

\todo{R16, R17, C18}

% ---------------------------------------------------------------------------- %
\subsection{Interface}
\label{subsec:hw:sensor:interface}
% ---------------------------------------------------------------------------- %

\begin{figure}[h!t]
    \centering
    \includegraphics[width=0.5\textwidth]{images/sensor-sch/sensor--sch--interface.eps}
    \caption[Sensor: Schema Interface]{Interface Sensor}
\end{figure}

\todo{C17, R18, R19, R20}

% ---------------------------------------------------------------------------- %
\subsection{Empf\"anger}
\label{subsec:hw:sensor:receiver}
% ---------------------------------------------------------------------------- %

Der Empfänger ist ein einfaches Tiefpassfilter.
Zuerst wird das Eingangssignal welches noch Moduliert ist durch die Diode D3 gleichgerichtet. Nun sorgt ein Tiefpass dahinter dafür, dass das Signal geglättet wird. Da diese Signal durch Berluste auf der Leitung und über der Diode eine viel zu kleine Amplitude hat, wird es zusätzlich durch einen nicht invertierenden Opamp verstärkt.

\begin{figure}[h!t]
    \centering
    \includegraphics[width=0.5\textwidth]{images/sensor-sch/sensor--sch--receiver.eps}
    \caption[Sensor: Schema Empf\"anger]{Empf\"anger Sensor}
\end{figure}

\todo{R5, D3, C5, R3, R2, R10, C4, U2}
