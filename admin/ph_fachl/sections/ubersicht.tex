% ---------------------------------------------------------------------------- %
\subsection{Ausgangslage}
% ---------------------------------------------------------------------------- %

Bei  den  heutzutage  verwendeten Photovoltaikanlagen  werden  \"ublicherweise
mehrere einzelne PV-Module zu einem String zusammengefasst, indem sie in Serie
geschaltet  werden. Dabei  kann  ein abgeschattetes,  verschmutztes  oder  gar
defektes Modul den Strom dieser Serieschaltung und somit auch die Leistung des
gesamten Strings und der Anlage stark beeintr\"achtigen.

Um  diese Einbussen  zu vermeiden,  braucht es  ein \"Uberwachungssystem,
welches ein  fehlerhaftes  Modul erkennen und  melden kann,  damit die
notwendigen  Massnahmen  getroffen  werden  k\"onnen. Die  bisher  vorhandenen
Systeme  f\"ur  diese \"Uberwachung  werden  meist  aus Kostengr\"unden  nicht
verwendet.

In diesem  Projekt soll  nun ein Modul\"uberwachungssystem  entwickelt werden,
welches genug kosteng\"unstig ist, um den Einbau dieses Systems wirtschaftlich
attraktiv zu gestalten. Dies wird realisiert,  indem bei den einzelnen Modulen
jeweils nur eine  sehr g\"unstige Sensorplatine angebracht wird  und die Daten
dieser Platinen  zentral in einem Meldeger\"at  ausgewertet werden. Somit kann
durch den Einbau der Sensoren  in die bereits vorhandenen Klemmengeh\"ause und
durch  das Verzichten  auf zus\"atzliche  Kabel der  finanzielle Aufwand  sehr
gering gehalten werden.


% ---------------------------------------------------------------------------- %
\subsection{Projektziele}
% ---------------------------------------------------------------------------- %

% ---------------------------------------------------------------------------- %
\subsubsection{Sollziele}
% ---------------------------------------------------------------------------- %

\textbf{Sensorplatine}:
\begin{enumerate}
    \item
        Die Sensorplatine  soll klein  genug sein, um im  Klemmengeh\"ause des
        PV-Moduls Platz zu haben.
    \item
        Die Spannungsversorgung erfolgt direkt vom Modul.
    \item
        Die   Platine  nimmt   durchschnittlich   nicht   mehr  Leistung   als
        \SI{200}{\milli\watt} auf.
    \item
        Die Versorgungsspannung kommt ohne Pufferung aus.
    \item
        Die  Bauteile,  die  zur  Einkopplung der  Signale  gebraucht  werden,
        m\"ussen f\"ur einen Strom von bis zu \SI{10}{\ampere} ausgelegt sein.
    \item
        Die   Modulspannung  wird   in   einem   Bereich  von   \SI{12}{\volt}
        bis  \SI{60}{\volt}gemessen  mit   einer  Genauigkeit  von  mindestens
        \SI{0.1}{\volt}.
    \item
        Die  Kommunikation zwischen  Sensoren  und Master  erfolgt \"uber  die
        Powerleitung.
    \item
        Es muss eine Identifizierung der einzelnen Module erfolgen k\"onnen.
    \item
        Die Produktionskosten f\"ur eine Sensorplatine sollen nicht mehr als 5
        Franken betragen bei einer 1000er-Serie.
    \item
        Die  Reichweite  des  Bus-Systems sollte  mindestems  \SI{100}{\meter}
        betragen.
\end{enumerate}

\clearpage
\textbf{Masterger\"at}
\begin{enumerate}
    \item
        Das Ger\"at soll in einen Schaltschrank eingebaut werden k\"onnen.
    \item
        Die Platzierung ist beim Generatoranschlusskasten vorgesehen.
    \item
        Die Versorgung erfolgt durch die Netzspannung.
    \item
        Enth\"alt ein Display, um defekte Module anzuzeigen.
    \item
        Auch ein fehlender Messwert wird als Fehler erkannt.
    \item
        Besitzt einen Relaiskontakt, um Fehlermeldung weiterzugeben.
    \item
        Konfiguration soll einfach sein.
    \item
        Es  k\"onnen  3  Strings  parallel \"uberwachen,  wobei  jeder  String
        maximal 16 Sensorplatinen beinhaltet.~\footnotemark[1]
\end{enumerate}

\footnotetext[1]{
    Die  Gesamtspannung eines  Strings  soll  unter \SI{1000}{\volt}  bleiben,
    damit keine Sonderbewilligungen f\"ur Installation und Unterhalt notwendig
    sind. Bei einer  Maximalspannung von \SI{60}{\volt} pro  Modul ergibt dies
    \SI{60}{\volt} pro Modul $\cdot$ 16 Module in Serie $=$ \SI{960}{\volt}.
}


% ---------------------------------------------------------------------------- %
\subsubsection{Wunschziele}
% ---------------------------------------------------------------------------- %

\begin{enumerate}
    \item
        Status-LEDs auf den Sensorplatinen f\"ur eine einfachere Fehlersuche
    \item
        Strommessung beim Masterger\"at
    \item
        Fehler- und Statusmeldungen via SMS
    \item
        Sensor: Maximaler Leistungsverbrauch von \SI{100}{\milli\watt}
	\item
		Sensor: Maximale Gr\"osse der Platine von $\SI{50}{\milli\meter} \times \SI{50}{\milli\meter}$
\end{enumerate}


% ---------------------------------------------------------------------------- %
\subsubsection{Nichtziele}
% ---------------------------------------------------------------------------- %

\begin{enumerate}
    \item
        Verwendung zus\"atzlicher Kabel f\"ur die Kommunikation
    \item
        Einsatz von Funkmodulen f\"ur die Kommunikation
    \item
        separates Geh\"ause f\"ur die Sensorplatine
\end{enumerate}

% ---------------------------------------------------------------------------- %
\subsection{Lieferobjekte}
% ---------------------------------------------------------------------------- %

Tabelle \ref{tab:lieferobj}  listet die  vom Modul  vorgegebenen Lieferobjekte
und Termine auf.

\begin{table}[h!]
    \centering
    \caption{Terminplan f\"ur Lieferobjekte}
    \label{tab:lieferobj}
    \begin{tabular}{rp{80mm}r}
        \toprule
        \textsc{Datum} & \textsc{Lieferobjekt} & \textsc{Form} \\
        \midrule
        \texttt{24.03.2016} & Pflichtenheft, organisatorischer Teil         & E-Mail, pdf \\
        \texttt{24.03.2016} & Pflichtenheft, fachlicher Teil, provisorisch  & E-Mail, pdf \\
        \texttt{14.04.2016} & Pflichtenheft, fachlicher Teil, definitiv     & E-Mail, pdf \\
        \texttt{14.04.2016} & Statusbericht 1                               & E-Mail \\
        \texttt{21.04.2016} & Zwischenpr\"asentation                        & Vortrag \\
        \texttt{28.04.2016} & Statusbericht 2                               & E-Mail \\
        \texttt{05.05.2016} & Einleitung/Disposition                        & E-Mail \\
        \texttt{19.05.2016} & Statusbericht 3                               & E-Mail \\
        \texttt{09.06.2016} & Statusbericht 4                               & E-Mail \\
        \texttt{16.06.2016} & Schlusspr\"asentation                         & Vortrag \\
        \texttt{16.06.2016} & Fachbericht                                   & USB-Stick \\
        \texttt{16.06.2016} & PMA-Bericht                                   & USB-Stick \\
        \bottomrule
    \end{tabular}
\end{table}

% TODO
% ---------------------------------------------------------------------------- %
%\subsection{Rahmenbedingungen}
% ---------------------------------------------------------------------------- %
