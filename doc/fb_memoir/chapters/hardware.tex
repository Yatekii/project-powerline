% **************************************************************************** %
\chapter{Hardware}
\label{chap:hardware}
% **************************************************************************** %

In  diesem  Abschnitt  wird  genauer auf  die  Hardware  eingegangen. Es  wird
erl\"autert, welche Probleme zu l\"osen und welche Anforderungen zu erf\"ullen
sind  (und wie  diese ermittelt  und  definiert werden),  und welche  Hardware
aus  welchen Gr\"unden  ausgew\"ahlt  worden ist,  um  diese Anforderungen  zu
erf\"ullen.

Der Abschnitt  ist dabei  aufgeteilt gem\"ass  unserem System  selbst: Es wird
separat auf Sensor  und Master-Ger\"at und ihre  jeweiligen Subsysteme (soweit
sinnvoll) eingegangen.

\anweisung Verifizierbare  Angaben machen, nicht  allgemeine Floskeln. Zahlen,
Fakten,  theoretische Hintergrundinformationen,  daraus gezogene  Konsequenzen
etc. Bezug auf das Pflichtenheft, wo m\"oglich.

\anweisung Bilder, Diagramme, Formeln etc.


% ---------------------------------------------------------------------------- %
\section{Sensorplatine}
\label{sec:hw:sensorplatine}
% ---------------------------------------------------------------------------- %

Was  ist der  Zweck  der Sensorplatine? Welche  Aufgaben  muss sie  erf\"ullen
k\"onnen? Wie ist sie aufgebaut? PCB-Layout? Energieversorgung?

\anweisung Dimensionierung der Bauteile: Berechnungen

\begin{minipage}{\textwidth}
    \centering
    \resizebox{0.5\textwidth}{!}{\begin{tikzpicture}[scale=1.5]
        \begin{axis}
            \addplot[-,color=blue] table {data/fetsim.dat};
        \end{axis}
    \end{tikzpicture}}
    \captionof{figure}{test}
    \label{fig:test}
\end{minipage}

% ---------------------------------------------------------------------------- %
\subsection{Energieversorgung}
\label{subsec:sensor:pcb}
% ---------------------------------------------------------------------------- %

Energiebezug,  Leistungsanforderungen, Standby,  Verhalten bei  ungen\"ugender
Leistungszufuhr, ...


% ---------------------------------------------------------------------------- %
\subsection{PCB}
\label{subsec:sensor:pcb}
% ---------------------------------------------------------------------------- %

Wie sieht das PCB aus, und weshalb?

% ---------------------------------------------------------------------------- %
\section{Master-Ger\"at}
\label{sec:hw:mastergerat}
% ---------------------------------------------------------------------------- %

Was ist der Zweck des Master-Ger\"ats? Aufgaben? Aufbau?

\anweisung Dimensionierung der Bauteile: Berechnungen


% ---------------------------------------------------------------------------- %
\subsection{Speisung}
\label{subsec:mastergerat:speisung}
% ---------------------------------------------------------------------------- %

Woher bezieht das Master-Ger\"at  seine Energie, weshalb? Wieviel Energie wird
ben\"otigt?


% ---------------------------------------------------------------------------- %
\subsection{Benutzer-Interface}
\label{subsec:mastergerat:interface}
% ---------------------------------------------------------------------------- %

\anweisung Das Benutzerinterface ist schon in der Anleitung im vorigen Kapitel
beschrieben worden. Es soll  hier um die Komponentenwahl gehen,  nicht um eine
Rekapitulation bereits gemachter Informationen.

% ---------------------------------------------------------------------------- %
\subsection{PCB}
\label{subsec:mastergerat:pcb}
% ---------------------------------------------------------------------------- %

Wie sieht das PCB aus, und weshalb?

% ---------------------------------------------------------------------------- %
\subsection{Montage/Geh\"ause}
\label{subsec:mastergerat:pcb}
% ---------------------------------------------------------------------------- %

Einbindung in die Umgebung, Wahl des Geh\"auses.


% ---------------------------------------------------------------------------- %
\section{Kommunikation}
\label{sec:kommunikation}
% ---------------------------------------------------------------------------- %

Die   Kommunikation  zwischen   Sensor  und   Master-Ger\"at  ist   eines  der
Herzst\"ucke des ganzen Systems und betrifft beide Sub-Systeme. Daher in einem
eigenen Abschnitt.

\begin{itemize}
    \item
        Grunds\"atzliches Problem
    \item
        Verfolgte L\"osungsans\"atze
    \item
        Ausgew\"ahlter L\"osungsansatz (mit Begr\"undung(en))
    \item
        Genauere Beschreibung dieses L\"osungsansatzes
\end{itemize}
