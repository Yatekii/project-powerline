% **************************************************************************** %
\chapter{\code{LTspice} Schaltung eines Photovoltaik-Moduls}
\label{app:simu:module}
% **************************************************************************** %

Abbildung \ref{fig:ltspice:solarCell} zeigt die Implementation unseres Modells
einer   Solarzelle  gem\"ass   Abbildung  \ref{fig:circuit:solarCell}   (Seite
\pageref{fig:circuit:solarCell} in Abschnitt \ref{subsubsec:hw:ask:modell}) in
\code{LTspice}.

Das in Abbildung  \ref{fig:ltspice:solarmodul:cellBased} gezeigte Modell eines
Solarmoduls  besteht  aus  72   solchen  Zellen,  zusammengeschaltet  zu  zwei
parallelen Str\"angen mit je 36 Zellen in Serieschaltung.

Ein MOSFET  wird benutzt, um  die Zelle  gesteuert (in unserer  Simulation bei
einer Frequenz von \SI{10}{\kilo\hertz})  kurzschliessen zu k\"onnen, wie dies
zur Kommunikation in kurzen Intervallen geschehen w\"urde.

Die zu  diesen Schaltungen geh\"orenden \code{.asc}-Dateien  sind elektronisch
verf\"ugbar  (Datentr\"ager  siehe Anhang  \ref{app:electronicStorage},  Seite
\pageref{app:electronicStorage}).


\begin{figure}[h!tb]
    \centering
    \includegraphics[width=\textwidth]{images/ltspice/singlecell.eps}
    \caption{%
        Schaltung     zur     Simulation     einer     Solarzelle     gem\"ass
        Abbildung         \ref{fig:circuit:solarCell}        von         Seite
        \pageref{fig:circuit:solarCell}. Das    Solarmodul    aus    Abbildung
        \ref{fig:ltspice:solarmodul:cellBased} besteht aus 72 solchen Zellen.
    }
    \label{fig:ltspice:solarCell}
\end{figure}

\begin{figure}[h!tb]
    \centering
    \includegraphics[width=\textwidth]{images/ltspice/module-72cells.eps}
    \caption{%
        Solarmodul,  modelliert durch  2 parallele  Strings mit  je 36  Zellen
        gem\"ass Abbildung \ref{fig:ltspice:solarCell} in Serie.
    }
    \label{fig:ltspice:solarmodul:cellBased}
\end{figure}
