% **************************************************************************** %
\chapter{Einleitung}
\label{chap:einleitung}
% **************************************************************************** %

Photovoltaikanlagen   sind  heutzutage   kein   Nischenprodukt  mehr. Um   die
Abh\"angigkeit vom Erd\"ol zu verringern,  werden vielerorts kleine, aber auch
grosse  Anlagen gebaut. Die  W\"arme-Energie, welche  kostenlos von  der Sonne
kommt, wird in elektrische Energie umgewandelt und kann gleich vor Ort genutzt
werden. Anlagenbesitzer  investieren meistens  einen  grossen  Betrag in  eine
neue  Anlage und  sind  darauf  angewiesen, dass  diese  den maximalen  Ertrag
liefert. Das  ist  in  der  Regel  ohne  grossen  Aufwand  der  Fall. Doch  es
gibt  Umst\"ande,  welche  die Effizienz  einer  Photovoltaikanlage  erheblich
verringern  k\"onnen  und  dies  meist  ohne,  dass  es  jemand  bemerkt.   In
einer  Photovoltaikanlage werden  \"ublicherweise mehrere  PV-Module zu  einem
String zusammengefasst, indem  sie in Serie geschaltet  werden. Dabei kann ein
abgeschattetes,  verschmutztes  oder  gar  defektes  Modul  den  Strom  dieser
Serieschaltung und somit auch die Leistung des gesamten Strings und der Anlage
stark  beeintr\"achtigen. Was grosse  finanzielle  Einbussen  zur Folge  haben
kann.

Das Ziel  des Projektes P4  war es, ein PV-\"Uberwachungssystem  bestehend aus
einer  Sensorplatine  f\"ur  den  Einbau  in  die  Anschlussbox  jedes  Moduls
und  einem  zentralen Meldeger\"at  f\"ur  den  Einbau im  Schaltschrank  beim
Wechselrichter zu entwickeln, aufzubauen und zu testen. Die Sensorplatine soll
die Spannung  des jeweiligen  PV-Modules messen und  sie an  das Masterger\"at
\"uber die  bestehende DC-Leitung  der Anlage  \"ubermitteln. Im Masterger\"at
werden  die  gemessenen Spannungen  der  einzelnen  PV-Module gespeichert  und
ausgewertet. Erkennt das  Masterger\"at ein  fehlerhaftes PV-Modul,  soll eine
Alarmierung am Ger\"at selbst und per SMS ausgegeben werden. Zus\"atzlich wird
ein Relais zur externen Signalisation bet\"atigt.  Das System soll m\"oglichst
energieeffizient  und kosteng\"unstig  sein, um  die wirtschaftlichkeit  einer
Photovoltaikanlage nicht zu verschlechtern.

Das Hauptproblem liegt  bei der Signal\"ubertragung \"uber  die DC-Leitung der
Photovoltaikanlage. Denn die Spannung darauf schwankt  zwischen 12 und 60 Volt
an  der Sensorplatine  und betr\"agt  am Masterger\"at  bis zu  1000 Volt. Auf
dieser Leitung  ein Signal zu  \"ubertragen ist schwierig und  wird heutzutage
kaum gemacht.   Zudem muss auf  kleine Leistung beim gesamten  System geachtet
werden, um keine wertfolle Energie zu verschwenden.

Anhand zahlreicher  Simulationen, L\"osungsentw\"urfen  und Tests,  konnte ein
System  entwickelt werden,  das  allen Anforderungen  an  den Auftrag  gerecht
wird. Mit  dem  Master-Ger\"at  als  zentrales  Gehirn  des  Systems  und  den
Sensoren  als \"Uberwachungskomponente  der  einzelnen  PV-Modulen, wurde  ein
\"Uberwachungssystem  konzipiert,   das  problemlos  in  neue,   wie  auch  in
bestehende Photovoltaikanlagen eingebaut werden kann.


Der vorliegende  Bericht stellt  die technische Dokumentation  unseres Systems
dar.  Zuerst wird das Konzept unserer L\"osung beschrieben, zusammen mit einer
Benutzerf\"uhrung.  Anschliessend  wird auf  das Hardware-  und Firmwaredesign
eingegangen,  und   zuletzt  werden  die  am   System  durchgef\"uhrten  Tests
dokumentiert.
